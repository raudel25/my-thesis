\documentclass[12pt]{article}

\usepackage[utf8]{inputenc} % Permite escribir caracteres especiales directamente
\usepackage[spanish]{babel} % Configura el idioma a español

\usepackage{amsmath}
\usepackage{tikz}
\usepackage{xcolor}
\usepackage[lmargin=2cm,rmargin=5cm]{geometry}

%%%{{{ Comments and the like
\usepackage[textwidth=4cm]{todonotes}
\usepackage{soul}
\usepackage{xcolor}
\newcounter{todocounter}
\newcommand{\comment}[2]{\stepcounter{todocounter}
  {\color{green!50!blue}{(#1$^{{\color{black}\textbf{\thetodocounter}}}$)}}
  \todo[color=green,noline,size=\tiny]{\textbf{\thetodocounter:} #2

  }}
\newcommand{\quitaesto}[1]{{\color{red}(\st{#1})}}

\newcommand{\cambio}[2]{{\color{cyan}{{#2}}}{\color{red}{(\st{#1})}}}

\newcommand{\agregaesto}[1]{{\color{cyan}{{#1}}}}

\newcommand{\notaparaelautor}[1]{{\color{brown}{\textbf{#1}}}}

\newcommand{\errorortografico}[1]{{\fcolorbox{gray}{magenta}{\textcolor{yellow}{\bf #1}}}}
    
%%%}}}


\title{Introducción}
\author{Raudel Alejandro Gómez Molina}

\begin{document}

\maketitle

El problema de satisfacibilidad booleana (\textit{SAT}) es uno de los problemas más estudiados en la teoría de la computación y la lógica.
Consiste en determinar si existe una asignación de valores verdaderos o falsos que satisfaga una fórmula booleana dada, compuesta
por variables y operadores lógicos como conjunciones, disyunciones y negaciones. SAT surge en 1971 como el primer problema NP-completo demostrado por
Stephen Cook,
lo que significa que, en el peor de los casos, su resolución requiere tiempo exponencial respecto al
tamaño de la entrada, pero también que muchos otros problemas pueden reducirse a él. Esto
implica un especial interés por parte de la comunidad científica en la búsqueda de métodos eficientes para la solución
del SAT.

La teoría de lenguajes es una rama fundamental de la Ciencia de la Computación y la matemática que se
enfoca en el estudio de los lenguajes formales. Estos lenguajes, definidos a través de gramáticas,
autómatas y expresiones regulares, permiten modelar y analizar la estructura de los lenguajes naturales y
artificiales. Su aplicación es amplia y abarca desde el diseño de compiladores y procesadores de lenguaje
natural hasta la verificación de sistemas y la teoría de la computabilidad.

Los lenguajes formales se
clasifican en jerarquías, como la jerarquía de Chomsky, que los organiza según su complejidad y poder
expresivo. Esta teoría proporciona las bases para entender cómo se pueden reconocer, generar y transformar
cadenas de símbolos, lo que resulta esencial en el desarrollo de herramientas computacionales para
el procesamiento de información. Además, la teoría de lenguajes constituye la base de los problemas de la Ciencia
de la Computación, ya que cualquier problema puede ser interpretado como un problema de la teoría de lenguajes.

En este trabajo se vinculan las dos ramas de la computación descritas anteriormente, presentando un enfoque para resolver el SAT
utilizando formalismos de teoría de lenguajes. Dicho enfoque resulta un tema no evidenciado en la literatura consultada
y permite demostrar que una serie de problemas relacionados a la teoría de lenguajes pertenecen a la clase NP-completo.

En estudios anteriores, que siguen la idea presentada en esta investigación, se han mostrado estrategias para la solución
de instancias específicas del SAT, usando formalismos de teoría de lenguajes, lo que constituye una solución limitada en su alcance.
En cambio, en este trabajo se presenta una alternativa que resuelve cualquier instancia del mismo, lo que, a criterio del autor, resulta una solución
cualitativamente superior. Esta sigue siendo una estrategia no eficiente, pero que
muestra un nuevo enfoque para resolver el SAT de forma general, y permite abrir nuevas líneas de investigación en este tema.

Para resolver cualquier instancia de SAT empleando formalismos de teoría de lenguajes se propone definir una codificación
de una fórmula booleana en una cadena que se pueda interpretar por algún formalismo de la teoría de lenguajes
y usando dicha codificación se define el lenguaje de todas las fórmulas booleanas satisfacibles. Entonces si se desea
determinar si una fórmula booleana es satisfacible es necesario determinar si la cadena asociada a la  fórmula booleana pertenece o no al lenguaje de todas las fórmulas booleanas satisfacibles.

Para construir el lenguaje
de las fórmulas booleanas satisfacibles se propone utilizar 2 métodos: el primero utiliza un transductor finito y el segundo
utiliza una gramática de concatenación de rango.

A partir de lo expuesto anteriormente se formula como objetivo general de de este trabajo: definir y construir el lenguaje de todas las fórmulas booleanas satisfacibles.

Para cumplir el objetivo general se definen los siguientes objetivos específicos:

\begin{itemize}
      \item Estudiar el estado del arte referido a los formalismos de teoría de lenguajes y el SAT.
      \item Establecer una representación del SAT como una cadena que pueda ser interpretada por un formalismo de la teoría de lenguajes.
      \item Definir el lenguaje de todas las fórmulas booleanas satisfacibles.
      \item Construir el lenguaje de todas las fórmulas booleanas satisfacibles utilizando un transductor finito.
      \item Construir el lenguaje de todas las fórmulas booleanas satisfacibles utilizando gramáticas de concatenación de rango.
\end{itemize}

Para dar cumplimiento a los objetivos trazados, el trabajo se ha estructurado en 4 capítulos: en los 2 primeros se presentan los principales conceptos y definiciones
que serán utilizados en el resto de la investigación y en los restantes 2 capítulos se define y construye el lenguaje de todas las fórmulas booleanas satisfacibles.

En el capítulo \ref{chap:preliminaries} se presentan los principales conceptos y definiciones de la teoría de lenguajes y el SAT, los cuales
son necesarios para la comprensión de los restantes capítulos. Además, se realiza un análisis de 2 trabajos anteriores
que muestran cómo solucionar instancias específicas del SAT utilizando un algoritmo polinomial.

En el capítulo \ref{chap:RCG} se realiza un análisis detallado de las gramáticas de concatenación de rango, presentando las principales
definiciones, proceso de derivación y análisis de la complejidad del algoritmo de reconocimiento.

En el capítulo \ref{chap:LSATFT} se muestra cómo codificar una fórmula booleana mediante una cadena de símbolos y luego
se analiza cómo interpretar una cadena como la asignación de valores para las variables de una fórmula booleana.
Posteriormente, se define el lenguaje de todas las fórmulas booleanas satisfacibles y se muestra cómo construir dicho
lenguaje mediante un transductor finito. Para finalizar, se demuestra que el problema de la palabra, para todos los formalismo que cumplan ciertas propiedades,
las cuales se definen en el capítulo \ref{chap:LSATFT}, es NP-Duro.

En el capítulo \ref{chap:LSATRCG} se demuestra que no es necesario construir el lenguaje de todas las fórmulas
booleanas satisfacibles mediante transducción finita, ya que existe una gramática de concatenación de rango que reconoce
este lenguaje. Por otro lado, se demuestra que las gramáticas de concatenación de rango cubren todos los problemas de la clase NP-Completo.


\begin{thebibliography}{99}

      \bibitem{mainRCGBib}
      Boullier, Pierre.
      \textit{Proposal for a Natural Language Processing Syntactic Backbone}.
      Research Report RR-3342, INRIA, 1998.

      \bibitem{propertiesRCGBib}
      Boullier, Pierre.
      \textit{A Cubic Time Extension of Context-Free Grammars}.
      Research Report RR-3611, INRIA, 1999.

      \bibitem{simpleMatrixLanguages}
      Ibarra, Oscar H.
      \textit{Simple matrix languages}.
      \textit{Information and Control}, Vol. 17, No. 4, pp. 359-394, 1970.

      \bibitem{globalIndexLanguages}
      Castaño, José M.
      \textit{Global Index Languages}.
      Ph.D. Thesis, The Faculty of the Graduate School of Arts and Sciences, Brandeis University, 2004.

      \bibitem{authomataTheory}
      Hopcroft, John E., Motwani, Rajeev, y Ullman, Jeffrey D.
      \textit{Introduction to Automata Theory, Languages, and Computation}.
      3ª edición, Addison-Wesley, 2006. ISBN: 9780321455369.

      \bibitem{aCFSAT}
      Fernández Arias, Alina.
      \textit{El problema de la satisfacibilidad booleana libre del contexto}.
      Facultad de Matemática y Computación, Universidad de La Habana, 2007.

      \bibitem{aSRCSAT}
      Aguilera López, Manuel.
      \textit{Problema de la Satisfacibilidad Booleana de Concatenación de Rango Simple}.
      Facultad de Matemática y Computación, Universidad de La Habana, 2016.

      \bibitem{aSMSAT}
      Rodríguez Salgado, José Jorge.
      \textit{Gramáticas Matriciales Simples. Primera aproximación para una solución al problema SAT}.
      Facultad de Matemática y Computación, Universidad de La Habana, 2019.

\end{thebibliography}


% Posibles conclusiones
% - teorica
% - como las gramáticas de concatenacion de rango constituyen un nuevo enfoque en la solucion del satisfacibilidad

% Posibles recomendaciones
% - por que via del transductor Full-SAT pueden desarrollarse nuevas investigaciones


\end{document}