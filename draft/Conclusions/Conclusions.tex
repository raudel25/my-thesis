\documentclass[12pt]{article}

\usepackage[utf8]{inputenc} % Permite escribir caracteres especiales directamente
\usepackage[spanish]{babel} % Configura el idioma a español

\usepackage{amsmath}
\usepackage{tikz}
\usepackage{xcolor}
\usepackage[lmargin=2cm,rmargin=5cm]{geometry}

%%%{{{ Comments and the like
\usepackage[textwidth=4cm]{todonotes}
\usepackage{soul}
\usepackage{xcolor}
\newcounter{todocounter}
\newcommand{\comment}[2]{\stepcounter{todocounter}
  {\color{green!50!blue}{(#1$^{{\color{black}\textbf{\thetodocounter}}}$)}}
  \todo[color=green,noline,size=\tiny]{\textbf{\thetodocounter:} #2

  }}
\newcommand{\quitaesto}[1]{{\color{red}(\st{#1})}}

\newcommand{\cambio}[2]{{\color{cyan}{{#2}}}{\color{red}{(\st{#1})}}}

\newcommand{\agregaesto}[1]{{\color{cyan}{{#1}}}}

\newcommand{\notaparaelautor}[1]{{\color{brown}{\textbf{#1}}}}

\newcommand{\errorortografico}[1]{{\fcolorbox{gray}{magenta}{\textcolor{yellow}{\bf #1}}}}
    
%%%}}}


\title{Conclusiones}
\author{Raudel Alejandro Gómez Molina}

\begin{document}

\maketitle

En este trabajo se presentó una estrategia para resolver el SAT usando teoría de lenguajes, la cual se basa en definir
una codificación de una fórmula booleana en una cadena y definir y construir el lenguaje de todas las fórmulas booleanas
satisfacibles $L_{S-SAT}$. Luego para determinar si una fórmula booleana es satisfacible es necesario verificar si la cadena asociada
a dicha fórmula pertenece a $L_{S-SAT}$.

En el capítulo \ref{chap:LSATFT}, se construyó $L_{S-SAT}$ mediante el transductor finito $T_{SAT}$ que recibe
cadenas del lenguaje $L_{0,1,d}$, las cuales representan todas las posibles interpretaciones de las fórmulas
booleanas en CNF y genera cadenas del lenguaje $L_{FULL-SAT}$, tales que la fórmula booleana asociada a estas
cadenas es satisfacible.

El problema de la palabra para todo formalismo que genere
el lenguaje $L_{0,1,d}$ y sea cerrado bajo transducción finita, es NP-Duro, teniendo en cuenta la conjetura
de que cualquier formalismo que genere el lenguaje $L_{0,1,d}$, tiene tamaño $O(1)$ en su representación.

En el capítulo \ref{chap:LSATRCG}, se presentó una RCG que reconoce el lenguaje $L_{0,1,d}$ y se argumentó por qué no es posible
usar esta gramática para construir $L_{S-SAT}$ mediante transducción finita, ya que las RCG no son cerradas bajo transducción finita.

Se construyó una RCG que reconoce el lenguaje $L_{S-SAT}$, lo que permitió demostrar
que no es necesario construir $L_{S-SAT}$ mediante transducción finita. La gramática que se construyó tiene el problema
de la palabra no polinomial, y constituye un ejemplo de una RCG donde el algoritmo de reconocimiento es no polinomial.
Además al obtener una RCG que reconoce $L_{S-SAT}$, se demostró que las RCG cubren todos los problemas de la clase NP,
ya que las RCG cubren todos los problemas en P \cite{mainRCGBib} y existe una reducción polinomial del SAT a todo problema en NP \cite{authomataTheory}.

Las estrategias presentadas constituyen una vía diferente
para resolver el SAT, y aunque el problema de la palabra para el formalismo que se construyó es no polinomial,
este acercamiento puede contribuir a nuevas líneas de investigación para la búsqueda de algoritmos eficientes que permitan
resolver el SAT.

\begin{thebibliography}{99}
      
      \bibitem{mainRCGBib}
      Boullier, Pierre.
      \textit{Proposal for a Natural Language Processing Syntactic Backbone}.
      Research Report RR-3342, INRIA, 1998.
      
      \bibitem{propertiesRCGBib}
      Boullier, Pierre.
      \textit{A Cubic Time Extension of Context-Free Grammars}.
      Research Report RR-3611, INRIA, 1999.
      
      \bibitem{simpleMatrixLanguages}
      Ibarra, Oscar H.
      \textit{Simple matrix languages}.
      \textit{Information and Control}, Vol. 17, No. 4, pp. 359-394, 1970.
      
      \bibitem{globalIndexLanguages}
      Castaño, José M.
      \textit{Global Index Languages}.
      Ph.D. Thesis, The Faculty of the Graduate School of Arts and Sciences, Brandeis University, 2004.
      
      \bibitem{authomataTheory}
      Hopcroft, John E., Motwani, Rajeev, y Ullman, Jeffrey D.
      \textit{Introduction to Automata Theory, Languages, and Computation}.
      3ª edición, Addison-Wesley, 2006. ISBN: 9780321455369.
      
      \bibitem{aCFSAT}
      Fernández Arias, Alina.
      \textit{El problema de la satisfacibilidad booleana libre del contexto}.
      Facultad de Matemática y Computación, Universidad de La Habana, 2007.
      
      \bibitem{aSRCSAT}
      Aguilera López, Manuel.
      \textit{Problema de la Satisfacibilidad Booleana de Concatenación de Rango Simple}.
      Facultad de Matemática y Computación, Universidad de La Habana, 2016.
      
      \bibitem{aSMSAT}
      Rodríguez Salgado, José Jorge.
      \textit{Gramáticas Matriciales Simples. Primera aproximación para una solución al problema SAT}.
      Facultad de Matemática y Computación, Universidad de La Habana, 2019.
      
\end{thebibliography}


% Posibles conclusiones
% - teorica
% - como las gramáticas de concatenacion de rango constituyen un nuevo enfoque en la solucion del satisfacibilidad

% Posibles recomendaciones
% - por que via del transductor Full-SAT pueden desarrollarse nuevas investigaciones


\end{document}