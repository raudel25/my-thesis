\documentclass[12pt]{article}

\usepackage[utf8]{inputenc} % Permite escribir caracteres especiales directamente
\usepackage[spanish]{babel} % Configura el idioma a español

\usepackage{amsmath}
\usepackage{tikz}
\usepackage{xcolor}
\usepackage[lmargin=2cm,rmargin=5cm]{geometry}

%%%{{{ Comments and the like
\usepackage[textwidth=4cm]{todonotes}
\usepackage{soul}
\usepackage{xcolor}
\newcounter{todocounter}
\newcommand{\comment}[2]{\stepcounter{todocounter}
  {\color{green!50!blue}{(#1$^{{\color{black}\textbf{\thetodocounter}}}$)}}
  \todo[color=green,noline,size=\tiny]{\textbf{\thetodocounter:} #2

  }}
\newcommand{\quitaesto}[1]{{\color{red}(\st{#1})}}

\newcommand{\cambio}[2]{{\color{cyan}{{#2}}}{\color{red}{(\st{#1})}}}

\newcommand{\agregaesto}[1]{{\color{cyan}{{#1}}}}

\newcommand{\notaparaelautor}[1]{{\color{brown}{\textbf{#1}}}}

\newcommand{\errorortografico}[1]{{\fcolorbox{gray}{magenta}{\textcolor{yellow}{\bf #1}}}}
    
%%%}}}


\title{Recomendaciones}
\author{Raudel Alejandro Gómez Molina}

\begin{document}

\maketitle

A partir del trabajo realizado se proponen como temas para investigaciones futuras los
siguientes:

\begin{itemize}
    \item Buscar un formalismo que sea capaz de generar el lenguaje $L_{0,1,d}$, el cual representa todas las interpretaciones
          de las fórmulas booleanas en CNF, que sea cerrado bajo transducción finita, y luego analizar el problema de la palabra para
          el formalismo que se obtiene después de aplicarle el transductor $T_{SAT}$.
    \item Demostrar que cualquier formalismo que genere $L_{0,1,d}$ tiene un tamaño $O(1)$ en su representación.
    \item Analizar qué tipo de formalismo se obtiene al aplicarle el transductor $T_{SAT}$ a la RCG que reconoce
          el $L_{0,1,d}$.
    \item  Analizar qué propiedades limitan que las RCG no sean cerradas bajo transducción finita, construir
          un formalismo basado en las RCG que sea cerrado bajo transducción finita y comprobar que este formalismo
          sea capaz de describir el lenguaje $L_{0,1,d}$.
          \item Construir una RCG que reconozca fórmulas booleanas satisfacibles, donde cada cláusula tiene a lo sumo dos literales (2-SAT),
          que tenga el problema de la palabra polinomial.
\end{itemize}




\begin{thebibliography}{99}

    \bibitem{mainRCGBib}
    Boullier, Pierre.
    \textit{Proposal for a Natural Language Processing Syntactic Backbone}.
    Research Report RR-3342, INRIA, 1998.

    \bibitem{propertiesRCGBib}
    Boullier, Pierre.
    \textit{A Cubic Time Extension of Context-Free Grammars}.
    Research Report RR-3611, INRIA, 1999.

    \bibitem{simpleMatrixLanguages}
    Ibarra, Oscar H.
    \textit{Simple matrix languages}.
    \textit{Information and Control}, Vol. 17, No. 4, pp. 359-394, 1970.

    \bibitem{globalIndexLanguages}
    Castaño, José M.
    \textit{Global Index Languages}.
    Ph.D. Thesis, The Faculty of the Graduate School of Arts and Sciences, Brandeis University, 2004.

    \bibitem{authomataTheory}
    Hopcroft, John E., Motwani, Rajeev, y Ullman, Jeffrey D.
    \textit{Introduction to Automata Theory, Languages, and Computation}.
    3ª edición, Addison-Wesley, 2006. ISBN: 9780321455369.

    \bibitem{aCFSAT}
    Fernández Arias, Alina.
    \textit{El problema de la satisfacibilidad booleana libre del contexto}.
    Facultad de Matemática y Computación, Universidad de La Habana, 2007.

    \bibitem{aSRCSAT}
    Aguilera López, Manuel.
    \textit{Problema de la Satisfacibilidad Booleana de Concatenación de Rango Simple}.
    Facultad de Matemática y Computación, Universidad de La Habana, 2016.

    \bibitem{aSMSAT}
    Rodríguez Salgado, José Jorge.
    \textit{Gramáticas Matriciales Simples. Primera aproximación para una solución al problema SAT}.
    Facultad de Matemática y Computación, Universidad de La Habana, 2019.

\end{thebibliography}


% Posibles conclusiones
% - teorica
% - como las gramáticas de concatenacion de rango constituyen un nuevo enfoque en la solucion del satisfacibilidad

% Posibles recomendaciones
% - por que via del transductor Full-SAT pueden desarrollarse nuevas investigaciones


\end{document}