\documentclass[12pt]{article}

\usepackage[utf8]{inputenc} % Permite escribir caracteres especiales directamente
\usepackage[spanish]{babel} % Configura el idioma a español


\usepackage{amsmath}
\usepackage{multicol}
\usepackage{tikz}
\usetikzlibrary{automata, positioning}

\usepackage[lmargin=2cm, rmargin=5cm]{geometry}
%%%{{{ Comments and the like
\usepackage[textwidth=4cm]{todonotes}
\usepackage{soul}
\usepackage{xcolor}
\newcounter{todocounter}
\newcommand{\comment}[2]{\stepcounter{todocounter}
  {\color{green!50!blue}{(#1$^{{\color{black}\textbf{\thetodocounter}}}$)}}
  \todo[color=green,noline,size=\tiny]{\textbf{\thetodocounter:} #2

  }}
\newcommand{\quitaesto}[1]{{\color{red}(\st{#1})}}

\newcommand{\cambio}[2]{{\color{cyan}{{#2}}}{\color{red}{(\st{#1})}}}

\newcommand{\agregaesto}[1]{{\color{cyan}{{#1}}}}

\newcommand{\notaparaelautor}[1]{{\color{brown}{\textbf{#1}}}}

\newcommand{\errorortografico}[1]{{\fcolorbox{gray}{magenta}{\textcolor{yellow}{\bf #1}}}}
    
%%%}}}

\title{Estrategia para la solución del SAT usando el problema de la palabra}
\author{Raudel Alejandro Gómez Molina}

\begin{document}

\maketitle

% \chapter{Preliminares}
% \label{chap:preliminaries}

% \chapter{Gramáticas de concatenación de rango}
% \label{chap:RCG}

% \chapter{Lenguaje de las fórmulas booleanas satisfacibles empleando transducción finita}
% \label{chap:LSATFT}

% \chapter{Lenguaje de las fórmulas booleanas satisfacibles empleando gramáticas de concatenación de rango}
% \label{chap:LSATRCG}

En este capítulo se presenta  un lenguaje formal, llamado $L_{S-SAT}$, al cual pertenecen todos los problemas 
SAT que son satisfacibles, y se muestra una forma de construirlo a partir de una transducción finita de una variante
del lenguaje $L_{copy}$ sobre el alfabeto $\{a,b,c,d\}$. Este lenguaje 
permitiría otra vía de solución para el SAT, que consiste en determinar si una fórmula dada pertenece a este 
lenguaje. Esto se puede hacer resolviendo el problema de la palabra.

Para definir el lenguaje $L_{S-SAT}$ se presenta una vía para codificar una fórmula booleana mediante cadenas 
sobre el alfabeto $\Sigma=\{a, b,c,d\}$, y para construirlo se utiliza una transducción finita del 
alfabeto $\{0,1,d\}$ en $\Sigma$. 

La estructura de este capítulo es la siguiente: en la sección \ref{sec:codsat} se muestra como codificar
una fórmula booleana usando el alfabeto $\{a,b,c,d\}$. En la sección \ref{sec:intsat} se muestra como interpretar 
las cadenas sobre el alfabeto $\{0,1,d\}$ como asignaciones de las variables. Finalmente, en la sección \ref{sec:tsat}
se presenta un transductor finito que convierte cadenas del lenguaje $L_{FULL-SAT}$ que se define en la sección \ref{sec:intsat}
en cadenas sobre el alfabeto $\{a,b,c,d\}$ que representan fórmulas booleanas satisfacibles. Seguidamente se 
conjetura por qué la representación del lenguaje de las fórmulas booleanas satisfacibles, en 
cualquier formalismo que lo genere usando la estrategia propuesta en este capítulo, tiene un tamaño $O(1)$, 
lo que implica la que el problema de la palabra para estos formalismos que puedan generar el lenguaje, es NP-Duro.


A continuación se presenta cómo codificar una fórmula booleana mediante una cadena sobre el alfabeto $\{a,b,c,d\}$.

\section{Codificación de una fórmula booleana a una cadena}
\label{sec:codsat}

Una fórmula booleana $F$, con $v$ variables en CNF tiene la siguiente estructura:
$$F=X_1 \wedge X_2 \wedge \ldots \wedge X_n$$
donde cada cláusula $X_i$ es una disyunción de literales
$$X_i=L_{i1} \vee L_{i2} \vee \ldots \vee L_{im},$$
cada literal $L_{ij}$ es una variable booleana o su negación y $m\leq v$. 

Si se tiene una fórmula booleana $F$ en forma normal conjuntiva se puede considerar que cada una de las 
$v$ variables de la fórmula aparece en cada cláusula de $F$ en uno de tres posibles estados: la variable 
aparece sin negar, la variable aparece negada, o la variable no aparece.

Por ejemplo, en la primera cláusula de la siguiente fórmula booleana en $CNF$ con 3 variables:
$$F=(x_1 \vee \neg x_2) \wedge (\neg x_1 \vee x_2 \vee x_3) \wedge (x_1 \vee \neg x_2 \vee x_3)$$

la variable $x_1$ aparece sin negar, la variable $x_2$ aparece negada, y la variable $x_{3}$ no aparece.

El hecho de que se pueda asumir que en todas las cláusulas aparecen todas variables permite representar una cláusula de una una fórmula con $v$ variables como una cadena de $v$ símbolos, donde el símbolo en la posición $i$ indica el estado de la variable $x_i$ en la cláusula.

En este trabajo se propone usar los símbolos $a$, $b$ y $c$ para indicar el estado de una variable en una cláusula:

\begin{itemize}
    \item $a$: la variable aparece sin negar,
    \item $b$: la variable aparece negada, y
    \item $c$: la variable no aparece.
\end{itemize}

Con este convenio, la primera cláusula de $F$ se puede representar mediante la cadena $abc$.

Si se agrega un nuevo símbolo (por ejemplo $d$) que indique el final de una cláusula, una fórmula lógica con $v$ variables y $k$ cláusulas se puede representar mediante $k$ bloques de longitud $v$, donde cada bloque está formado por los símbolos $a$, $b$, o $c$, y cada bloque se separa del siguiente por el símbolo $d$.

Con este convenio, la fórmula $F$ se representa mediante la cadena:
$$abc\mathbf{d}baa\mathbf{d}aba\mathbf{d}$$
donde los símbolos $\mathbf{d}$ aparecen en negrita para facilitar la interpretación de la cadena como fórmula en forma normal conjuntiva.

Solo resta definir el proceso inverso, es decir, dada una cadena determinar la fórmula booleana en CNF que representa dicha cadena.
Para que una cadena $w$ pueda ser interpretada como una fórmula booleana debe cumplir con las siguientes condiciones:
tener $n$ bloques separados por $d$, cada bloque de las misma longitud $v$ y en cada bloque solo pueden ser usados los caracteres
$a$, $b$, $c$. Entonces la cadena $w$ puede ser interpretada por como una fórmula booleana con $n$ cláusulas y $v$ variables y la estructura
de las cláusula depende de los caracteres correspondiente al bloque de $a$, $b$ y $c$ asociado a dicha cláusula.

Por ejemplo la cadena $acc\mathbf{d}aba\mathbf{d}cba\mathbf{d}$, se puede interpretar como la siguiente fórmula booleana:
$$(x_1)\wedge(x_1\vee \neg x_2 \vee x_3) \wedge (\neg x_2\vee x_3)$$

Entonces una vez definida la transformación de una fórmula booleana en una cadena se puede definir el lenguaje
de todas las fórmulas booleanas en CNF:
$$L_{FULL-SAT}=\{w\,|\,\exists F : t_{SAT}(F)=w\},$$
donde $F$ es una fórmula booleana en CNF y $t_{SAT}(x)$ es una función que recibe una fórmula booleana y devuelve
su representación mediante los estados $a$, $b$ y $c$ y el separador $d$.

Seguidamente se muestra como interpretar una cadena como la asignación de los valores de las variables de una fórmula
booleana.

\section{Interpretar una cadena como asignaciones de variables para una fórmula booleana}
\label{sec:intsat}

En esta sección se muestra como dada una cadena $q$ y una cadena $f\in L_{FULL-SAT}$, interpretar 
la cadena $w$ como la asignación de los valores de las variables para la fórmula booleana que representa
la cadena $f$.

Para que la cadena $e$ pueda ser interpretada como una asignación de valores para $f$, es necesario al igual
que en la codificación anterior poder distinguir los valores asignados a cada cláusula, esto se logra mediante
el caracter $d$, y para representar los valores de las variables asignados a cada cláusula se utilizan 
cadenas binarias. Para cada cadena binaria se le asocia el valor del $i$-ésimo caracter al valor de la $i$-ésima
variable de la cláusula, si este caracter es un 1 se le asocia un valor de \textit{true} (verdadero) y si el 
caracter es un 0 se le asocia el valor de \textit{false} (falso).

Otra restricción importante es que la cadena
binaria debe ser la misma para cada cláusula lo que mantiene la invariante de que a todas las instancias de la misma
variable le corresponde el mismo valor y la longitud de la cadena binaria debe ser exactamente igual al número de 
variables de la fórmula booleana.

Por ejemplo, si $e=101\mathbf{d}101\mathbf{d}101\mathbf{d}$ y $f=abc\mathbf{d}cbb\mathbf{d}acc\mathbf{d}$, fórmula booleana:
$$(x_1\vee\neg x_2)\wedge (\neg x_2 \vee \neg x_3)\wedge (x_1),$$
se evalúa de la siguiente manera:
$$(true\vee\neg false)\wedge (\neg false \vee \neg true)\wedge (true).$$

Entonces, se puede definir el lenguaje de todas las interpretaciones de las fórmulas booleanas representas en la codificación
de $L_{FULL-SAT}$ como:
$$L_{0,1}=\{(wd)^+\,|\,w\in\{0,1\}^+\}.$$

En la próxima sección se define el lenguaje $L_{S-SAT}$.

\section{Definición de $L_{S-SAT}$}

El lenguaje de todas las fórmulas booleanas en CNF que son satisfacibles se define como $L_{S-SAT}=\{w\,|\,w \in L_{FULL-SAT} \}$, 
donde $L_{FULL-SAT}$ representa el lenguaje de todas las fórmulas booleanas en CNF y $w$ representa una fórmula booleana satisfacible.

En la próxima sección se presenta una vía para construir $L_{S-SAT}$ usando transducción finita.

En la próxima sección se demuestra que $L_{S-SAT}$ no es un lenguaje libre del contexto, por lo que el formalismo que lo
genere necesariamente debe pertenecer a las gramáticas dependientes del contexto o las gramáticas irrestrictas.

\subsection{Demostración de que $L_{S-SAT}$ no es un lenguaje libre del contexto}

En esta sección se demuestra que el lenguaje $L_{S-SAT}$ no es libre del contexto. Una técnica que se emplea para demostrar que un lenguaje no es libre del contexto, es el lema del bombeo para los lenguajes
libres del contexto \cite{authomataTheory}. 

El lema del bombeo establece que si $L$ es un lenguaje libre del contexto
existe una constante $n$ tal que para toda cadena $q\in L$, $w$ puede escribirse de la forma $q=uvwxy$ tal que:
$|vwx|\leq n$, $vx\neq \varepsilon$ y $\forall i\geq 0\,uv^iwx^iy\in L$ \cite{authomataTheory}.

Suponga que $L_{S-SAT}$ es un lenguaje libre del contexto y la constante es $n$, entonces se cumple que 
$$q=a\underbrace{c\ldots c}_{n+1}\mathbf{d}ca\underbrace{c\ldots c}_{n}\mathbf{d}\ldots \underbrace{c\ldots c}_{n}ac\mathbf{d}\underbrace{c\ldots c}_{n+1}a\mathbf{d},$$
pertenece a $L_{S-SAT}$, ya que la fórmula 
$$(x_1)\wedge(x_2)\wedge\ldots\wedge(x_{n+2})$$ es satisfacible. Por tanto existen $u$, $v$, $w$, $x$ y $y$ con $vx\neq \varepsilon$ y $|vwx|\leq n$
tales que $q=uvwxy$. 

Como $|vwx|\leq n$ y cada cláusula tiene $n+2$ caracteres se cumple que $vwx$ tiene caracteres a lo sumo en 2 cláusulas,
por tanto la cadena $uv^2wx^2y$ tiene una o dos cláusulas con más caracteres que las restantes y esto contradice el hecho 
de que $uv^2wx^2y\in L_{S-SAT}$. Entonces, se cumple que $L_{S-SAT}$ no es un lenguaje libre del contexto.

Seguidamente se muestra como construir $L_{S-SAT}$ mediante una transducción finita.

\subsection{Construcción del $L_{S-SAT}$ usando transducción finita}

La idea para definir $L_{S-SAT}$ es construir un transductor finito, denominado $T_{SAT}$, que acepte como 
entrada cadenas $w\in L_{0,1}$ y devuelva cadenas de $f\in L_{FULL-SAT}$ tales que al evaluar $w$ en $f$, $f$
es verdadera. 

Se define $L_{S-SAT}$ como el lenguaje de todas las transducciones $e$ que se obtienen del transductor 
$T_{SAT}$, a partir del lenguaje de cadenas de entrada $L_{0,1}$. 

$$L_{S-SAT} = \{e\,|\,\exists w \in L_{0,1} \wedge e \in T_{SAT}(w) \}.$$

En la siguiente sección se define el transductor $T_{SAT}$.

\section{Transductor $T_{SAT}$}
\label{sec:tsat}

En esta sección se define el transductor finito $T_{SAT}$ (Figura \ref{fig:transducer_sat}), el cual se usa para construir $L_{S-SAT}$, mediante 
una transducción finita.

Para ello se construye el transductor $T_{CLAUSE}$ (Figura \ref{fig:transducer_clause}) que dada una cadena binaria $w$
genera todas las posibles cláusulas sobre el alfabeto $\{a,b,c,d\}$ que sean satisfacibles por $w$. 

La idea detrás de $T_{CLAUSE}$ es construir un transductor que genere todas las posibles cláusulas satisfacibles 
por los valores de las variables que determina la cadena de entrada. Este transductor tiene 3 estados: el 
estado inicial, el estado positivo (representa que la cláusula generada ya es satisfacible) y el estado 
negativo (representa que la cláusula generada aún no es satisfacible). Las transiciones entre los estados 
se realizan dependiendo del caracter que se lee y el caracter que se escribe y además dependiendo si la asignación
que se realiza en el momento de leer o de escribir satisface la cláusula o no.

A continuación de describe el funcionamiento de cada estado del transductor $T_{CLAUSE}$:
\begin{itemize}
    \item Estado $q_0$: representa el estado inicial. Si la entrada es un 1 el transductor puede escribir $a$,
          $b$ y $c$, si escribe $a$ pasa al estado positivo, si escribe $b$ pasa al estado negativo y si escribe $c$
          permanece en el mismo estado. Por otro lado si la entrada es un 0 cuando se escribe $a$ pasa al estado 
          negativo y cuando se escribe una $b$ pasa al estado positivo, mientras que cuando se escribe $c$
          permanece en el mismo estado.
          
    \item Estado $q_p$ (estado positivo de $T_{CLAUSE}$): representa que para los valores asignados a las
          variables se obtiene un valor de verdad positivo.  Como la fórmula se encuentra ya en un estado positivo 
          lo que significa que al menos un literal se evaluó positivo, no importa la entrada y lo que el transductor 
          escriba se mantiene en el mismo estado. Este estado es el estado de aceptación para el transductor lo cual 
          significa que la cláusula se evalúa con un valor de verdad positivo.
          
    \item Estado $q_n$ (estado negativo de $T_{CLAUSE}$): representa que para los valores asignados a las variables
          hasta el momento se obtiene un valor de verdad negativo. Si la entrada es un 1 el transductor puede 
          escribir $a$, $b$ y $c$. Si escribe $a$ pasa al estado positivo, si escribe $b$ pasa al estado negativo y 
          si escribe $c$ permanece en el mismo estado. Por otro lado si la entrada es un 0 cuando se escribe $a$ pasa 
          al estado negativo y cuando se escribe una $b$ pasa al estado positivo, mientras que cuando se escribe $c$
          permanece en el mismo estado.
          
\end{itemize}

Seguidamente se define $T_{CLAUSE}$.

\[
    T_{CLAUSE} = (Q, {\Sigma}, \Gamma, \delta, q_{0}, F),
\]
donde:
\begin{itemize}
    \item \(Q\) = ${q_0,q_p,q_n}$.
    \item \(\Sigma\) = ${0,1}$.
    \item \(\Gamma\) = ${a,b,c}$.
    \item \(\delta: Q \times \Sigma \to Q \times \Gamma^*\) función de transición.
    \item \(q_{0} = q_0\) estado inicial.
    \item \(F={q_p}\) conjunto de estados finales.
\end{itemize}

Se define la función de transición $\delta$ de la siguiente manera:

\begin{itemize}
    \item Transiciones para el estado $q_0$:
          \begin{multicols}{2}
              \begin{itemize}
                  \item $\delta_{SAT}(q_0,1)=(q_p,a)$
                  \item $\delta_{SAT}(q_0,0)=(q_n,a)$
                  \item $\delta_{SAT}(q_0,1)=(q_n,b)$
                  \item $\delta_{SAT}(q_0,0)=(q_p,b)$
                  \item $\delta_{SAT}(q_0,1)=(q_0,c)$
                  \item $\delta_{SAT}(q_0,0)=(q_0,c)$
              \end{itemize}
          \end{multicols}
          
    \item Transiciones para el estado $q_p$ (estado positivo de $T_{CLAUSE}$):
          \begin{multicols}{2}
              \begin{itemize}
                  \item $\delta_{SAT}(q_{p},1)=(q_{p},a)$
                  \item $\delta_{SAT}(q_{p},0)=(q_{p},a)$
                  \item $\delta_{SAT}(q_{p},1)=(q_{p},b)$
                  \item $\delta_{SAT}(q_{p},0)=(q_{p},b)$
                  \item $\delta_{SAT}(q_{p},1)=(q_{p},c)$
                  \item $\delta_{SAT}(q_{p},0)=(q_{p},c)$
              \end{itemize}
          \end{multicols}
          
    \item Transiciones para el estado $q_n$ (estado negativo de $T_{CLAUSE}$):
          \begin{multicols}{2}
              \begin{itemize}
                  \item $\delta_{SAT}(q_{n},1)=(q_{p},a)$
                  \item $\delta_{SAT}(q_{n},0)=(q_{n},a)$
                  \item $\delta_{SAT}(q_{n},1)=(q_{n},b)$
                  \item $\delta_{SAT}(q_{n},0)=(q_{p},b)$
                  \item $\delta_{SAT}(q_{n},1)=(q_{n},c)$
                  \item $\delta_{SAT}(q_{n},0)=(q_{n},c)$
              \end{itemize}
          \end{multicols}
\end{itemize}

\begin{figure}[h]
    \centering  \begin{otherlanguage}{english}
        \begin{tikzpicture}[shorten >=1pt, node distance=3cm, on grid, auto]
            
            % Nodos
            \node[state, initial] (q0)   {$q_0$};
            \node[state] (qn) [above right=of q0] {$q_n$};
            \node[state, accepting] (qp) [below right=of q0] {$q_p$};
            
            % Transiciones
            \path[->]
            (q0) edge [bend left] node {0/a,1/b} (qn)
            (q0) edge [bend right] node {1/a,0/b} (qp)
            (q0) edge [loop right] node {0/c,1/c} (q0)
            
            (qn) edge [bend left] node {1/a,0/b} (qp)
            (qn) edge [loop above] node {0/a,1/b,0/c,1/c} (qn)
            
            (qp) edge [loop below] node {1/a,0/b,0/a,1/b,0/c,1/c} (qp);
            
        \end{tikzpicture}
    \end{otherlanguage}
    \caption{Transductor $T_{CLAUSE}$.}
    \label{fig:transducer_clause} % Esto es para referenciar la figura en el texto
\end{figure}

Para definir $T_{SAT}$ mediante $T_{CALUSE}$, la idea es modificar $T_{CALUSE}$ para que cuando se genere 
una cláusula y termine en el estado positivo se continue leyendo los valores de la cadena de entrada que 
son necesarios para generar la próxima cláusula. Lo anterior se logra definiendo $q_0$ como el estado de 
aceptación y agregando una transición del estado $q_p$ al estado $q_0$ que lea una $d$ y escriba una $d$, 
de esta manera cuando una cláusula se genera con un valor de verdad positivo se pasa a generar la siguiente 
cláusula desde el estado inicial. 

La construcción anterior tiene una dificultad y es que el transductor genera la cadena vacía, y la cadena
vacía representa una fórmula booleana con 0 variables, por lo que no tiene sentido que se considere en $L_{S-SAT}$. Para 
solucionar lo anterior se pueden tomar 2 transductores $T_{CLAUSE}$ y unirlos mediante una transición, 
esta idea se expone a continuación.

Para definir el transductor $T_{SAT}$ (Figura \ref{fig:transducer_sat}) se toman 2 transductores 
$T_{CLAUSE}$ ($T_1$ y $T_2$ respectivamente) y se concatenan añadiendo una transición del estado 
estado positivo de $T_1$ $q_{p_1}$ al estado inicial de $T_2$ $q_{0_2}$ con el símbolo $d$
(tanto de lectura como de escritura) y además se agrega una transición del estado del estado 
estado positivo de $T_2$ $q_{p_2}$ al estado $q_{0_2}$ con el símbolo $d$ (tanto de lectura como de escritura).
Para terminar se define el estado inicial y el estado final de $T_{SAT}$, los cuales serían el estado inicial de $T_1$
$q_{0_1}$ y el estado inicial de $T_2$ $q_{0_2}$, respectivamente.



\begin{figure}[h]
    \begin{otherlanguage}{english}
        \centering \begin{tikzpicture}[shorten >=1pt, node distance=3cm, on grid, auto]
            
            % Nodos
            \node[state, initial] (q01)   {$q_{0_1}$};
            \node[state] (qn1) [above right=of q01] {$q_{n_1}$};
            \node[state] (qp1) [below right=of q0] {$q_{p_1}$};
            \node[state, accepting] (q02) [right=6cm of q01] {$q_{0_2}$};
            \node[state] (qn2) [above right=of q02] {$q_{n_2}$};
            \node[state] (qp2) [below right=of q02] {$q_{p_2}$};
            
            
            % Transiciones
            \path[->]
            (q01) edge [bend left] node {0/a,1/b} (qn1)
            (q01) edge [bend right] node {1/a,0/b} (qp1)
            (q01) edge [loop right] node {0/c,1/c} (q01)
            
            (qn1) edge [bend left] node {1/a,0/b} (qp1)
            (qn1) edge [loop above] node {0/a,1/b,0/c,1/c} (qn1)
            
            (qp1) edge [loop below] node {1/a,0/b,0/a,1/b,0/c,1/c} (qp1)
            
            (q02) edge [bend left] node {0/a,1/b} (qn2)
            (q02) edge [bend right] node {1/a,0/b} (qp2)
            (q02) edge [loop right] node {0/c,1/c} (q02)
            
            (qn2) edge [bend left] node {1/a,0/b} (qp2)
            (qn2) edge [loop above] node {0/a,1/b,0/c,1/c} (qn2)
            
            (qp2) edge [loop below] node {1/a,0/b,0/a,1/b,0/c,1/c} (qp2)
            
            (qp1) edge [bend right] node {d/d} (q02)
            (qp2) edge [bend left=75] node {d/d} (q02);
            
        \end{tikzpicture}
    \end{otherlanguage}
    \caption{Transductor $T_{SAT}$.}
    \label{fig:transducer_sat} % Esto es para referenciar la figura en el texto
\end{figure}

A continuación se presenta la demostración de que la construcción de $L_{S-SAT}$ mediante una transducción finita reconoce todas las fórmulas satisfacibles.

\section{Demostración de que la construcción de $L_{S-SAT}$ mediante una transducción finita
  reconoce todas las fórmulas satisfacibles}

La idea de la demostración es probar que para cualquier cadena binaria $w$, $T_{CLAUSE}(w)$ es el conjunto de 
todas las cláusulas que son satisfacibles por $w$. Después demostrar que si $e=(wd)^n \wedge w\in\{0,1\}^*$, $T_{SAT}(e)$ contiene todas 
las fórmulas de $n$ cláusulas satisfacibles por la cadena $w$ donde  y por último demostrar que:
$$L_{S-SAT} = \{e\,|\,\exists w \in L_{0,1} \wedge e \in T_{SAT}(w) \}.$$

Para demostrar que dada una cadena binaria $w$, $T_{CLAUSE}(w)$ es el conjunto de todas las 
cláusulas que son satisfacibles por $w$ primero suponga que $q\in T_{CLAUSE}(w)$. 
Esto significa que el transductor terminó en el estado $q_p$ en el proceso que generó $q$ y como empezó 
en el estado $q_0$ ocurrió una transición desde $q_0$ a $q_p$ o desde $q_n$ a $q_p$. Esto solo es posible 
si el transductor leyó un 1 y escribió una a o si leyó un 0 y escribió una b. Lo cual significa que en la fórmula booleana que representa $q$
hay una variable sin negar a la cual se le asigna un 1 o una variable negada a la 
que se le asigna un 0, por lo tanto se cumple que $w$ satisface la fórmula booleana que representa $q$.

Sea una fórmula booleana $F$ satisfacible por $w$, cuya representacion en $L_{FULL-SAT}$ es $q$,  se cumple que 
en $F$ hay una variable sin negar a la cual se le asigna un 1 en $w$
o una variable negada a la cual se le asigna un 0 en $w$. Sin pérdida de la generalidad se asume que la primera variable que cumple lo anterior es la $i$-ésima.  

Si se hace el reconocimiento de los primeros $i-1$ caracteres de la cadena de entrada por $T_{CLAUSE}$ se pueden
tomar las transiciones de tal manera que los primeros $i-1$ caracteres de la cadena generada sean iguales a los primeros $i-1$ caracteres de $q$. 
Dado este punto solo es posible que el transductor esté en el estado $q_0$ o $q_n$, pero como se cumple que la 
$i$-ésima variable está sin negar y se le asigna un 1 en $w$ o está negada y se le asigna un 0 en $w$, entonces se puede 
tomar la opción de leer un 1 y escribir una a o leer un 0 y escribir una b según corresponda. 
De esta manera, según en el estado en que se encuentre el autómata, se pasa al estado $q_p$.

Luego se toman las restantes transiciones de manera que la cadena generada sea $q$ y se mantiene en el mismo 
estado ya que $q_p$ solo tiene transiciones hacia sí mismo. De esta manera se demuestra que $T_{CLAUSE}(w)$
es el conjunto de todas las cláusulas que son satisfacibles por $w$.

Para demostrar que $T_{SAT}(e)$ contiene todas las fórmulas satisfacibles por la cadena $e=(wd)^n$ donde 
$w\in\{0,1\}^*$ se hará una inducción sobre $n$. Se define los conjuntos $A_{w,n}$ como el conjunto 
formado por todas las cadenas que representan fórmulas booleanas de $n$ cláusulas satisfacibles por 
$w$ y $B_w$ el conjunto formado por todas las cadenas que representan las cláusulas que son satisfacibles 
por $w$. El caso base $n=1$ se demuestra porque la transducción se realiza solo sobre el primer transductor 
$T_{CLAUSE}$ que conforma a $T_{SAT}$ y como se demostró anteriormente $T_{CLAUSE}(w)$ contiene todas cláusulas satisfacibles por $w$.  

Una vez demostrado el caso base corresponde asumir para $n=k$ y se demuestra para $n=k+1$. 

Se cumple que 
el conjunto de todas las fórmulas booleanas satisfacibles por $w$ con $k+1$ cláusulas es equivalente 
al conjunto que forman todas las fórmulas booleanas satisfacibles por $w$ con $k$ cláusulas concatenadas con todas 
las cláusulas satisfacibles por $w$:
$$A_{w,k+1}=\{xyd\,|\,x\in A_{w,k} \wedge y\in B_w\}.$$
Además por la estructura de $T_{SAT}$ se cumple que el conjunto de todas las cadenas
generadas por $T_{SAT}((wd)^{k+1})$ es igual al conjunto de todas las cadenas 
generadas por $T_{SAT}((wd)^{k})$ concatenadas con todas las cadenas generadas por $T_{CLAUSE}(w)$:  
$$T_{SAT}((wd)^{k+1})=\{xyd\,|\,x\in T_{SAT}((wd)^{k}) \wedge y\in T_{CLAUSE}(w)\}.$$

Por hipótesis de inducción se cumple que $A_{w,k}=T_{SAT}((wd)^{k})$ y además se cumple que $B_w=T_{CLAUSE}(w)$, lo cual implica que $A_{w,k+1}=T_{SAT}((wd)^{k+1})$.

Para concluir el tercer paso de la demostración, sea una cadena $e$ tal que existe $w$, donde $w \in L_{0,1} \wedge e \in T_{SAT}(w)$ entonces 
se cumple que la fórmula booleana asociada a $e$ es satisfacible por $w$. Ahora sea $F$ una fórmula 
booleana satisfacible y sea $e$ su cadena asociada, por tanto existe $w\in L_{0,1}$ tal que $w$ satisface a $F$, luego se cumple que $e\in T_{SAT}(w)$. Por tanto se cumple que:
$$L_{S-SAT} = \{e\,|\,\exists w \in L_{0,1} \wedge e \in T_{SAT}(w) \}.$$

A continuación de demuestra que el problema de la palabra de cualquier formalismo que genere el lenguaje $L_{0,1}$, sea cerrado bajo transducción finita y su representación luego de la transducción sea $O(1)$, es NP-Duro


\subsection{Demostración de que el problema de la palabra de cualquier formalismo que genere el lenguaje $L_{0,1}$, sea cerrado bajo transducción finita y su representación luego de la transducción sea $O(1)$, es NP-Duro}

Para realizar la demostración suponga que existe un formalismo $G$ que genera $L_{0,1}$, es cerrado bajo transducción finita y su representación 
luego de la transducción es $O(1)$. 
Sea $G'$ el formalismo que resulta de aplicarle el transductor $T_{SAT}$ a $G$, entonces determinar 
si una cadena $w$ pertenece al lenguaje generado por $G'$, es equivalente a saber si la fórmula
booleana a la cual representa $w$ es satisfacible. Por tanto como la representación de $G'$ es $O(1)$
entonces el problema de la palabra para $G'$ es NP-Duro, porque tiene una reducción directa al problema
de la satisfacibilidad booleana.

En este capítulo se presentó una estrategia para resolver el SAT usando teoría de lenguajes que se basa en 
definir y construir el lenguaje de todas las fórmulas satisfacibles. Además se presentó un primer acercamiento 
para construir $L_{S-SAT}$, mediante transducción finita lo cual demuestra que el problema de la palabra para 
todos los formalismos que generen $L_{0,1}$ y sean cerrados bajo transducción finita, es NP-duro. 
En el próximo capítulo se presenta otra estrategia distinta para definir $L_{S-SAT}$, la cual se 
basa en definir una RCG que reconozca las fórmulas booleanas satisfacibles. 


\begin{thebibliography}{99}
    
    \bibitem{mainRCGBib}
    Boullier, Pierre.
    \textit{Proposal for a Natural Language Processing Syntactic Backbone}.
    Research Report RR-3342, INRIA, 1998.
    
    \bibitem{propertiesRCGBib}
    Boullier, Pierre.
    \textit{A Cubic Time Extension of Context-Free Grammars}.
    Research Report RR-3611, INRIA, 1999.
    
    \bibitem{simpleMatrixLanguages}
    Ibarra, Oscar H.
    \textit{Simple matrix languages}.
    \textit{Information and Control}, Vol. 17, No. 4, pp. 359-394, 1970.
    
    \bibitem{globalIndexLanguages}
    Castaño, José M.
    \textit{Global Index Languages}.
    Ph.D. Thesis, The Faculty of the Graduate School of Arts and Sciences, Brandeis University, 2004.
    
    \bibitem{authomataTheory}
    Hopcroft, John E., Motwani, Rajeev, y Ullman, Jeffrey D.
    \textit{Introduction to Automata Theory, Languages, and Computation}.
    3ª edición, Addison-Wesley, 2006. ISBN: 9780321455369.
    
    \bibitem{aCFSAT}
    Fernández Arias, Alina.
    \textit{El problema de la satisfacibilidad booleana libre del contexto}.
    Facultad de Matemática y Computación, Universidad de La Habana, 2007.
    
    \bibitem{aSRCSAT}
    Aguilera López, Manuel.
    \textit{Problema de la Satisfacibilidad Booleana de Concatenación de Rango Simple}.
    Facultad de Matemática y Computación, Universidad de La Habana, 2016.
    
    \bibitem{aSMSAT}
    Rodríguez Salgado, José Jorge.
    \textit{Gramáticas Matriciales Simples. Primera aproximación para una solución al problema SAT}.
    Facultad de Matemática y Computación, Universidad de La Habana, 2019.
    
\end{thebibliography}


\end{document}