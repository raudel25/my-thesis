\documentclass{article}

\title{Solución polinomial de instancias del SAT usando Teoría de Lenguajes}
\author{Raudel Alejandro Gómez Molina}

\begin{document}

\maketitle

% \section{SAT y la teoría de lenguajes}

% Como parte del estudió del problema SAT, se ha desarrollado una línea de investigación en la facultad utilizando un enfoque
% novedoso basado en formalismos de la teoría de lenguajes, buscando resolver instancias específicas del problema.

% \subsection{Problema satisfacibilidad booleana libre del contexto}

% El primer trabajo desarrollado como parte de esta línea de investigación [1] consiste en resolver el problema satisfacibilidad booleana
% libre del contexto (\textit{CF-SAT}), el cual es una instancia específica del SAT donde las variables donde la fórmula booleana es una fórmula booleana
% libre del contexto.

% Una fórmula booleana se considera libre del contexto si para cualquier par de instancias de una variable $x_i$ y $x_j$ con $i<j$ se
% cumple que si existe otra variable con instancia $x_k$ con $i<k<j$ entonces todas las instancias de esta nueva
% variable ocurren entre $x_i$ y $x_j$.

% El CF-SAT consiste en transformar una fórmula booleana en una lista de instancias de variables,
% donde se asume que 2 instancias de una variable no tienen por qué tener el mismo valor de verdad. Luego
% se define un autómata que dada una cadena de 0 y 1 y una fórmula booleana, determina si se obtiene un valor
% de verdad para la fórmula booleana donde cada instancia de una variable toma el valor de verdad que se
% corresponde con la cadena de 0 y 1, dicho autómata se denomina autómata booleano.

% Entonces para verificar que 2 instancias de una variable tengan el mismo valor de verdad se intersecta
% dicho autómata con una gramática libre del contexto \textit{CFG} obteniendo un autómata de pila (esto es posible por la estructura
% de la fórmula booleana libre del contexto planteada anteriormente).
% Después de esto se plantea un algoritmo para dado este autómata de pila, generar todas las posibles
% cadenas de 0 y 1 que se corresponden con una asignación de valores de verdad y como consecuencia se obtiene
% un generador de todas las posibles asignaciones de valores de verdad para una fórmula booleana.
% Luego para determinar si la fórmula es satisfacible solo queda verificar si este conjunto de soluciones es no vacío.

% \subsection{Problema de la satisfacibilidad para gramáticas de concatenación de rango simple}

% El próximo trabajo relacionado con este tema realizado en la facultad [2] consistió en definir y analizar el problema de la
% satisfacibilidad para gramáticas de concatenación de rango simple. En este trabajo se toma el mismo enfoque que el anterior,
% pero en vez de intersectar el autómata booleano con una gramática libre del contexto se intersecta con una gramática de
% concatenación de rango simple dando lugar a la resolución de un conjunto más amplio de problemas SAT que el problema anterior.

% \subsection{Problema de la satisfacibilidad para gramáticas matriciales}

% Continuando la línea del autómata booleano empleado en la resolución de instancias del SAT el próximo trabajo desarrollado [3]
% consistió en analizar las gramáticas matriciales. Nuevamente intersectando el autómata booleano con un formalismo que
% cuente con un algoritmo para comprobar el problema del vacío en tiempo polinomial en este caso se eligieron las
% gramáticas matriciales que nuevamente ofrecen un conjunto más amplio de problemas que el CF-SAT.


\end{document}