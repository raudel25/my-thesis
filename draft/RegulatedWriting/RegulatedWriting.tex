\documentclass[12pt]{article}

\usepackage[utf8]{inputenc} % Permite escribir caracteres especiales directamente
\usepackage[spanish]{babel} % Configura el idioma a español

\usepackage{amsmath}
\usepackage{tikz}
\usepackage{xcolor}
\usepackage[lmargin=2cm,rmargin=5cm]{geometry}

%%%{{{ Comments and the like
\usepackage[textwidth=4cm]{todonotes}
\usepackage{soul}
\usepackage{xcolor}
\newcounter{todocounter}
\newcommand{\comment}[2]{\stepcounter{todocounter}
  {\color{green!50!blue}{(#1$^{{\color{black}\textbf{\thetodocounter}}}$)}}
  \todo[color=green,noline,size=\tiny]{\textbf{\thetodocounter:} #2

  }}
\newcommand{\quitaesto}[1]{{\color{red}(\st{#1})}}

\newcommand{\cambio}[2]{{\color{cyan}{{#2}}}{\color{red}{(\st{#1})}}}

\newcommand{\agregaesto}[1]{{\color{cyan}{{#1}}}}

\newcommand{\notaparaelautor}[1]{{\color{brown}{\textbf{#1}}}}

\newcommand{\errorortografico}[1]{{\fcolorbox{gray}{magenta}{\textcolor{yellow}{\bf #1}}}}
    
%%%}}}


\title{Formalismos de escritura regulada}
\author{Raudel Alejandro Gómez Molina}

\begin{document}

\maketitle

En este capítulo se presentan los formalismos de escritura regulada que se utilizaran en capítulos posteriores.

\section{Gramáticas Matriciales}

Una gramática matricial \cite{simpleMatrixLanguages} de grado $n$ \textit{$n$-MG} es una 4-tupla:

\[
    G_n = (V, P, S,\Sigma)
\]

donde:
\begin{itemize}
    \item \( V \) es un conjunto finito de \textbf{símbolos no terminales}.
    \item \( \Sigma \) es un conjunto finito de \textbf{símbolos terminales}, con \( V \cap \Sigma = \emptyset \).
    \item \( P \) es un conjunto finito de matrices. Cada matriz es una secuencia ordenada de \textbf{producciones} de la forma:
          \[
              [P_1, P_2, \dots, P_k]
          \]
          donde cada \( P_i \) es una regla \( A \to \alpha \), con $1\leq k\leq n$, \( A \in N \) y \( \alpha \in (N \cup T)^* \).
    \item \( S  \) es el \textbf{símbolo inicial}.
\end{itemize}

Observe que das CFG son gramáticas matriciales de grado 1, es decir $1$-MG.

\subsection{Proceso de derivación}

El proceso de derivación en una gramática matricial se realiza de la siguiente manera:
\begin{enumerate}
    \item Se selecciona una matriz \( [P_1, P_2, \dots, P_k] \in P \).
    \item Las reglas \( P_1, P_2, \dots, P_k \) se aplican de manera secuencial a la cadena actual.
    \item La derivación continúa hasta que la cadena derivada contenga solo símbolos terminales, es decir, pertenezca a \( T^* \).
\end{enumerate}

Se presenta un ejemplo a continuación:

\[
    G = (V, P, S,\Sigma)
\]

donde:
\begin{itemize}
    \item \( V = \{A,B,C\} \).
    \item \( \Sigma=\{a,b\} \).
    \item \( P \) es un conjunto finito de matrices de producciones:
    \item \begin{itemize}
              \item \([S\to AB,A\to aA]\)
              \item \([B\to bB]\)
              \item \([A\to \varepsilon]\)
              \item \([B\to \varepsilon]\)
          \end{itemize}
    \item \( S  \) es el \textbf{símbolo inicial}.
\end{itemize}

Observe que la gramática anterior genera cadenas sobre el alfabeto $\{a,b\}$ con al menos una $a$ seguida 
de 0 o varias $b$.

Ahora se presentará un concepto más específico de las gramáticas matriciales.

\subsection{Gramáticas Matriciales Simples}

Una gramática matricial simple de grado $n$ \textit{$n$-SMG} es una ($n$+3)-tupla:
$$
    G_n=(V_1,V_2,\ldots,V_n,P,S,\Sigma)
$$
donde:
\begin{itemize}
    \item \( V_1, V_2, \ldots, V_n \) son conjuntos finitos de \textbf{símbolos no terminales} disjuntos 2 a 2.
    \item \( \Sigma \) es un conjunto finito de \textbf{símbolos terminales}, con \( V_i \cap \Sigma = \emptyset\,\forall\,1\leq i\leq n \).
    \item \( P \) es un conjunto finito de matrices. Cada matriz es una secuencia ordenada \textbf{producciones} que cumplan con una de las siguientes reglas:
          \begin{itemize}
              \item $[S\to w]$, donde $w\in \Sigma ^*$.
              \item $[S\to a_{11}A_{11}\ldots A_{1k}a_{21}A_{21}\ldots A_{2k}\ldots A_{n1}a_{n1}\ldots A_{nk}b]$,
                    donde $\forall i,j$ con $1\leq i\leq n\wedge 1\leq j\leq k$ se cumple que
                    $A_{ij}\in V_i$, $a_{ij}\in \Sigma ^*$ y $b\in \Sigma ^*$.
              \item $[A_1\to w_1,\ldots, A_n\to w_n]$, donde $A_i\in V_i\wedge w_i\in \Sigma ^*$ $\forall i\, 1\leq i\leq n$.
              \item $[A_1 \to a_{11}A_{11}\ldots a_{1k}A_{1k}b_1,\ldots,A_n \to a_{n1}A_{n1}\ldots a_{nk}A_{nk}b_n]$, donde $\forall i,j$
                    con $1\leq i\leq n\wedge 1\leq j\leq k$ se cumple que
                    $A_{ij}\in V_i$, $a_{ij}\in \Sigma ^*$ y $b_{i}\in \Sigma ^*$.
          \end{itemize}
    \item \( S \) es el \textbf{símbolo inicial}.
\end{itemize}


Observe que la restricción impuesta sobre las $n$-MG es cada matriz de producciones debe contener exactamente $n$ reglas de producción
donde cada regla de producción utiliza no terminales de conjuntos distintos o puede contener una única producción cuya secuencia de no terminales
esta compuesta por una secuencia de subsecuentes de terminales de conjuntos distintos.

Se presenta un ejemplo a continuación:

\[
    G = (V, P, S,\Sigma)
\]

donde:
\begin{itemize}
    \item \( V = \{A,B,C\} \).
    \item \( \Sigma=\{a,b\} \).
    \item \( P \) es un conjunto finito de matrices de producciones:
    \item \begin{itemize}
              \item \([S\to AB]\)
              \item \([A\to aA, B\to bB]\)
              \item \([A\to \varepsilon]\)
              \item \([B\to \varepsilon]\)
          \end{itemize}
    \item \( S  \) es el \textbf{símbolo inicial}.
\end{itemize}

Observe que la gramática anterior genera cadenas sobre el alfabeto $\{a,b\}$ con una cantidad de $a$, seguido
de la misma cantidad de $b$.

\subsection{Propiedades de las Gramáticas Matriciales}

A continuación se presentan las principales propiedades de las gramáticas matriciales.

\begin{itemize}
    \item \textbf{Cerradura bajo unión:} Dadas dos gramáticas matriciales $G_1$ y $G_2$,
          la unión de los lenguajes reconocidos por $G_1$ y $G_2$ es reconocida por una gramática matricial.
          $$G=(V_1\cup V_2\cup\{S\} , P_1\cup P_2\cup\{[S\to S_1],[S\to S_2]\}, S,\Sigma)$$
    \item \textbf{Cerradas bajo concatenación:} Dadas 2 gramáticas matriciales $G_1$ y $G_2$, la concatenación de
          los lenguajes reconocidos por $G_1$ y $G_2$ es reconocida por una gramática matricial.
          $$G=(V_1\cup V_2\cup\{S\} , P_1\cup P_2\cup\{[S\to S_1 S_2]\}, S,\Sigma)$$
    \item \textbf{Cerradas bajo clausura de Kleene:} Dada una gramática matricial $G$, el lenguaje reconocido por
          $G$ elevado a la clausura de Kleene es reconocido por una gramática matricial.
          $$G=(V\cup\{S'\} , P\cup\{[S'\to S'S],[S'\to \varepsilon]\}, S',\Sigma)$$
\end{itemize}

\section{Gramáticas de Índice Global}

Una gramática de índice global \textit{GIG}, es una extensión de las CFG, que añaden un mecanismo
de memoria al proceso de derivación, esta característica permite la generación de lenguajes más generales que los generados
por las CFG \cite{globalIndexLanguages}.
El mecanismo de memorización consiste en una pila en la cual se pueden almacenar símbolos que pertenecen a un conjunto predeterminado,
en cada producción se puede realizar una operación de insertar o eliminar de la pila o dejarla en su estado actual.

Una GIG es una 6-tupla:
$$
    G = (N, \Sigma, I, S, \#, P)
$$
donde:

\begin{itemize}
    \item $N$ es un conjunto finito de \textbf{símbolos no terminales}.
    \item \( \Sigma \) es un conjunto finito de \textbf{símbolos terminales}, $\Sigma \cap N=\emptyset$.
    \item $I$ es un conjunto finito de \textbf{índices de pila}, $\Sigma \cap I=\emptyset \wedge I \cap I=\emptyset$.
    \item $S\in N$ es el \textbf{símbolo inicial}.
    \item $\#$ es el \textbf{símbolo inicial de la pila}, $\# \notin \Sigma \cup N \cup I$.
    \item $P$ es un conjunto finito de \textbf{producciones} que tienen la siguiente forma, donde $x\in I\cup N\cup \Sigma$ y $y\in I\cup N$:
          \begin{itemize}
              \item $A \underset{\varepsilon}{\to} \alpha$ o $A \to \alpha$ (reglas epsilon o reglas libres del contexto).
              \item $A \underset{[y]}{\to}  \alpha$ o $[..]A \to [..]\alpha$ (reglas epsilon o reglas con restricciones).
              \item $A \underset{x}{\to} a \beta$ o $[..]A \to  [x..]a\beta$ (reglas de push o apertura de paréntesis).
              \item $A \underset{\overline{x}}{\to} \alpha$ o $[x..]A \to [..]\alpha$ (reglas de pop o cierre de paréntesis).
          \end{itemize}
\end{itemize}

Como se puede observar la primera regla de producción consiste en dejar la pila intacta y puede ser interpretada como una regla de derivación
libre del contexto, la segunda regla consiste en dejar la pila intacta pero solo se puede realizar si el caracter en el tope de la pila es el
especificado, la tercera regla consiste en añadir un caracter a la pila y la cuarta regla consiste en eliminar un caracter de la pila.
Una gramática GIG solo con producciones de la primera regla de producción es equivalente a una CFG.

\subsection{Proceso de derivación}

Como se mencionó anteriormente el proceso de derivación en las GIG es idéntico al proceso de derivación de las CFG, con la diferencia que
en cada paso de la derivación se puede realizar una operación de insertar o eliminar de la pila, además de las operaciones de sustitución de símbolos.
Otra restricción adicional es que el proceso de derivación en las GIG debe ser siempre de extrema izquierda.

Entonces una cadena es reconocida por una GIG si existe una secuencia de derivaciones desde $S$ que genere la cadena y que además la pila
termine vacía al final de la derivación (con el símbolo $\#$ en el tope de la pila). Luego se puede definir el lenguaje generado por una GIG, $G$
como $L(G)=\{w\,|\,\#S\overset{*}{\to}\#w \wedge w\in \Sigma^* \}$.

\subsection{Propiedades de las GIG}

A continuación se describen las principales propiedades de las GIG:
\begin{itemize}
    \item \textbf{Cerradura bajo unión:} Dadas dos GIG $G_1$ y $G_2$, la unión de los lenguajes reconocidos por $G_1$ y $G_2$ es reconocida por una GIG
          $$G=(N_1\cup N_2\cup \{S\},\Sigma_1\cup \Sigma_2,I_1\cup I_2,S,\#,P_1\cup P_2\cup \{S \underset{\varepsilon}{\to} S_1|S_2\})$$
    \item \textbf{Cerradas bajo concatenación:} Dadas 2 GIG $G_1$ y $G_2$, la concatenación de los lenguajes reconocidos por $G_1$ y $G_2$ es reconocida por una GIG
          $$G=(N_1\cup N_2\cup \{S\},\Sigma_1\cup \Sigma_2,I_1\cup I_2,S,\#,P_1\cup P_2\cup \{S \underset{\varepsilon}{\to} S_1S_2\})$$
    \item \textbf{Cerradas bajo clausura de Kleene:} Dada una GIG $G$, el lenguaje reconocido por $G$ elevado a la clausura de Kleene es reconocido por una GIG
          $$G=(N\cup \{S'\},\Sigma,I\cup \{S'\},S,\#,P\cup \{S'\underset{\varepsilon}{\to} S'S|\varepsilon\})$$
    \item  \textbf{Cerradas bajo homomorfismo} Dada una GIG $G$, el homomorfismo de un lenguaje reconocido por $G$ es un lenguaje de índice global \cite{globalIndexLanguages}.
    \item  \textbf{Cerradas bajo transducción finita:} Dada una GIG $G$, la transducción finita de un lenguaje reconocido por $G$ es un lenguaje de índice global \cite{globalIndexLanguages}.
\end{itemize}

\section{Gramáticas de Concatenación de Rango}

Las gramáticas de concatenación de rango (\textit{RCG}) \cite{mainRCGBib} son un formalismo de gramáticas desarrollado
para describir lenguajes más generales que los libres del contexto.

\subsection{Definiciones}

\paragraph{Rango:} un rango es una tupla $(i, j)$ que representa un intervalo de posiciones en la cadena, donde $i$ y $j$ son enteros no negativos tales que $i \leq j$.

\paragraph{Gramática de Concatenación de Rango Positiva:} una gramática de concatenación de rango positiva (\textit{PRCG}) se define como una 5-tupla:

\[
    G = (N, T, V, P, S),
\]
donde:

\begin{itemize}
    \item $N$: Es un conjunto finito de \textbf{predicados o símbolos no terminales}: Cada predicado tiene una \textbf{aridad}, que indica el número de argumentos que toma.
    \item $T$: Es un conjunto finito de \textbf{símbolos terminales}.
    \item $V$: Es un conjunto finito de \textbf{variables}.
    \item $P$: Es un conjunto finito de \textbf{cláusulas}, de la forma:
          \[
              A(x_1, x_2, \ldots, x_k) \to B_1(y_{1,1}, y_{1,2}, \ldots, y_{1,m_1}) \ldots B_n(y_{n,1}, y_{n,2}, \ldots, y_{n,m_n}),
          \]
          donde $A, B_i \in N$, $x_i, y_{i,j} \in (V \cup T)^*$, y $k$ es la aridad de $A$.
    \item $S \in N$: Es el \textbf{predicado inicial} de la gramática.
\end{itemize}

\paragraph{Gramática de Concatenación de Rango Negativa:} una gramática de concatenación de rango negativa (\textit{NRCG}) es similar a una PRCG, pero predicados o no terminales negativos que se denotan de la siguiente manera: $\overline{A}$.

\paragraph{Gramática de Concatenación de Rango Simple:} las gramáticas de concatenación de rango simple (\textit{SRCG}) son un subconjunto de las RCG que restringen la forma de las cláusulas de producción.
Una RCG $G$ es \textbf{simple} si los argumentos en el lado derecho de una cláusula son variables distintas, y todas estas variables (y no otras) aparecen una sola vez en los argumentos del lado izquierdo.
Un resultado interesante es que para cada CFG existe una SRCG equivalente que genera el mismo lenguaje.

\paragraph{Sustiución de rango:} una sustitución de rango es un mecanismo que reemplaza una variable por un rango de la cadena.
Por ejemplo dado el predicado $A(Xa)$ donde $X \in V \wedge a \in T$, si se instancia la cadena $baa$ en $A$, $X$ puede
ser asociada con el rango $ba$ de la cadena original.

\subsection{Proceso de derivación}

La principal idea detrás de las RCG, para realizar una derivación, se basa en encontrar para cada argumento del predicado izquierdo de una cláusula todas las
posibles sustituciones en rango de la cadena, asociar los valores de las variables a los argumentos de los predicados derechos y continuar
el proceso de derivación en los predicados derechos.

Por ejemplo, dada la cláusula $A(X,aYb)\to B(aXb,Y)$ , donde $X$ y $Y$ son símbolos variables y $a$ y $b$
son símbolos terminales, la cadena predicado $A(a,abb)$ deriva como $B(aab,b)$, porque $A(a,abb)$
coincide con $A(X,aYb)$ cuando $ X=a \wedge Y=b$. De forma similar, si existiera una regla

Una secuencia de argumentos son reconocidos por un predicado si existe una secuencia de derivaciones que comienza
en dicho predicado y termina en la cadena vacía, si el predicado es negativo en el caso de las NRCG ocurre lo contrario
la secuencia de argumentos no es reconocida por el predicado. Una RCG reconoce una cadena si dicha cadena es reconocida
por el predicado inicial.

Ejemplo dada la siguiente RCG:

\[
    G = (N, T, V, P, S),
\]
donde:

\begin{itemize}
    \item  N=$\{A,S\}$.
    \item T=$\{a,b,c\}$.
    \item V=$\{X,Y,Z\}$.
    \item El conjunto de cláusulas $P$ es el siguiente:
          $$S(XYZ)\to A(X,Y,Z)$$
          $$A(aX,aY,aZ)\to A(X,Y,Z)$$
          $$A(bX,bY,bZ)\to A(X,Y,Z)$$
          $$A(cX,cY,cZ)\to A(X,Y,Z)$$
          $$A(\varepsilon,\varepsilon,\varepsilon)\to \varepsilon$$
    \item El símbolo inicial es $S$.
\end{itemize}
La cadena $aaabbbccc$ es reconocida por la RCG anterior, ya que se puede derivar de la siguiente manera:
$$S(abcabcabc)\to A(abc,abc,abc)\to A(bc,bc,bc)\to A(c,c,c)\to A(\varepsilon,\varepsilon,\varepsilon)\to \varepsilon$$

De manera general el lenguaje reconocido por la RCG anterior es $L=\{www\,|\,w\in \{a,b,c\}^*\}$.

\subsection{Propiedades de las RCG}

A continuación se describen las principales propiedades de las RCG \cite{propertiesRCGBib}:
\begin{itemize}
    \item \textbf{Cerradura bajo unión:} Dadas dos RCG $G_1$ y $G_2$, la unión de los lenguajes reconocidos por $G_1$ y $G_2$ es reconocida por una RCG
          $$G=(N_1\cup N_2\cup \{S\},T_1\cup T_2,V_1\cup V_2,P_1\cup P_2\cup \{S(X)\to S_1(X)|S_2(X)\},S)$$
    \item \textbf{Cerradas bajo intersección:} Dadas dos RCG $G_1$ y $G_2$, la intersección de los lenguajes reconocidos por $G_1$ y $G_2$ es reconocida por una RCG
          $$G=(N_1\cup N_2\cup \{S\},T_1\cup T_2,V_1\cup V_2,P_1\cup P_2\cup \{S(X)\to S_1(X)S_2(X)\},S)$$
    \item \textbf{Cerradas bajo complemento:} Dada una RCG $G$, el complemento del lenguaje reconocido por $G$ es reconocido por una RCG
          $$G'=(N\cup \{\overline{S}\},T,V,P\cup \{S'(X)\to \overline{S}(X)\},S')$$
    \item \textbf{Cerradas bajo concatenación:} Dadas dos RCG $G_1$ y $G_2$, la concatenación de los lenguajes reconocidos por $G_1$ y $G_2$ es reconocida por una RCG
          $$G=(N_1\cup N_2\cup \{S\},T_1\cup T_2,V_1\cup V_2,P_1\cup P_2\cup \{S(XY)\to S_1(X)S_2(Y)\},S)$$
    \item \textbf{Cerradas bajo clausura de Kleene:} Dada una RCG $G$, la clausura de Kleene del lenguaje reconocido por $G$ es reconocida por una RCG
          $$G'=(N\cup \{S'\},T,V,P\cup \{S'(XY)\to S(X)S'(Y)|\varepsilon\},S')$$
    \item  \textbf{No cerradas bajo homomorfismo:} Dada una RCG $G$, el homomorfismo de un lenguaje reconocido por $G$ no es necesariamente reconocido por una RCG \cite{propertiesRCGBib}.
    \item  \textbf{No cerradas bajo transducción finita:} Dada una RCG $G$, la transducción finita de un lenguaje reconocido por $G$ no es necesariamente reconocida por una RCG.
          Esto es una consecuencia de la no cerradura bajo homomorfismo.
\end{itemize}


\subsection{Problema de la palabra, problema del vacío y equivalencia de 2 RCG}

\paragraph{Problema de la palabra:} En general en la mayoría de los casos este problema es polinomial y pasa por
un algoritmo de memorización sobre las cadenas que son instanciadas en los rangos de los predicados de la RCG \cite{mainRCGBib} (como la cantidad
máxima de rangos de la cadena es $n^2$ y la máxima aridad de un predicado es constante, este proceso de memorización cuenta
con cantidad polinomial de estados), en
una complejidad de $O(|P|n^{2h(l+1)})$ donde $h$ es la máxima aridad en un predicado, $l$ es la máxima cantidad de predicados
en el lado derecho de una cláusula y $n$ es la longitud de la cadena a ser reconocida.

Pero existen casos en los que el problema de la palabra no
es polinomial \cite{propertiesRCGBib}, por ejemplo puede pasar que se instancien argumentos de en los predicados con rangos que no pertenezcan
a la propia cadena de entrada y sean generados durante el proceso de reconocimiento.

\paragraph{Problema del vacío:} El problema del vacío para una RCG es indecidible \cite{propertiesRCGBib}, la razón principal para esto es que como se mencionó anteriormente
para toda CFG existe una RCG equivalente y como las RCG son cerradas bajo intersección existen RCG que describen
la intersección de 2 lenguajes libres del contexto y determinar si dicha intersección es vacía es un problema indecidible.

En el caso de las SRCG este problema es polinomial \cite{mainRCGBib}.

\paragraph{Problema de la equivalencia:} El problema de la equivalencia para 2 RCG es indecidible, la demostración es muy sencilla dadas 2 RCG $G_1$ y $G_2$ el problema
de saber si $G_1$ es equivalente a $G_2$ es equivalente a saber si $G_1\cap \overline{G_2}=\emptyset$ y como se mencionó anteriormente el problema del vacío para una RCG
es indecidible.


\begin{thebibliography}{99}

    \bibitem{mainRCGBib}
    Boullier, Pierre.
    \textit{Proposal for a Natural Language Processing Syntactic Backbone}.
    Research Report RR-3342, INRIA, 1998.

    \bibitem{propertiesRCGBib}
    Boullier, Pierre.
    \textit{A Cubic Time Extension of Context-Free Grammars}.
    Research Report RR-3611, INRIA, 1999.

    \bibitem{simpleMatrixLanguages}
    Ibarra, Oscar H.
    \textit{Simple matrix languages}.
    \textit{Information and Control}, Vol. 17, No. 4, pp. 359-394, 1970.

    \bibitem{globalIndexLanguages}
    Castaño, José M.
    \textit{Global Index Languages}.
    Ph.D. Thesis, The Faculty of the Graduate School of Arts and Sciences, Brandeis University, 2004.

    \bibitem{authomataTheory}
    Hopcroft, John E., Motwani, Rajeev, y Ullman, Jeffrey D.
    \textit{Introduction to Automata Theory, Languages, and Computation}.
    3ª edición, Addison-Wesley, 2006. ISBN: 9780321455369.

    \bibitem{aCFSAT}
    Fernández Arias, Alina.
    \textit{El problema de la satisfacibilidad booleana libre del contexto}.
    Facultad de Matemática y Computación, Universidad de La Habana, 2007.

    \bibitem{aSRCSAT}
    Aguilera López, Manuel.
    \textit{Problema de la Satisfacibilidad Booleana de Concatenación de Rango Simple}.
    Facultad de Matemática y Computación, Universidad de La Habana, 2016.

    \bibitem{aSMSAT}
    Rodríguez Salgado, José Jorge.
    \textit{Gramáticas Matriciales Simples. Primera aproximación para una solución al problema SAT}.
    Facultad de Matemática y Computación, Universidad de La Habana, 2019.

\end{thebibliography}


\end{document}