\begin{conclusionsAndRecomendations}

    En este trabajo se presentó una estrategia para resolver el SAT usando teoría de lenguajes que se basa en 
    codificar una fórmula booleana mediante una cadena sobre el alfabeto $\{a,b,d,c\}$ y definir el lenguaje de 
    todas las fórmulas booleanas satisfacibles $L_{S-SAT}$. A partir de este lenguaje, para determinar si una 
    fórmula booleana es satisfacible solo es necesario verificar si la cadena asociada a dicha fórmula pertenece 
    a $L_{S-SAT}$.
    
    Además de definir el lenguaje $L_{S-SAT}$ se propuso una forma para construirlo utilizando una transducción 
    finita de una variante del lenguaje $Copy$. 
    
    En el capítulo \ref{chap:LSATFT}, se construyó $L_{S-SAT}$ mediante el transductor finito $T_{SAT}$ que 
    recibe cadenas del lenguaje $L_{0,1,d}$, las cuales representan todas las posibles interpretaciones para 
    fórmulas booleanas en CNF y genera cadenas sobre el alfabeto $\{a,b,c,d\}$, tales que cuando se interpretan 
    como fórmulas booleanas son satisfacibles por la cadena de $L_{0,1,d}$ que la generó. 
    
    Por la forma en que se construyó el lenguaje $L_{S-SAT}$ se tiene que el problema de la palabra para todo 
    formalismo que genere el lenguaje $L_{0,1,d}$ y sea cerrado bajo transducción finita, es NP-Duro, asumiendo como válida 
    la conjetura de que cualquier formalismo que genere el lenguaje $L_{0,1,d}$, tiene tamaño $O(1)$ en su 
    representación.
    
    También se construyó una RCG que reconoce el lenguaje $L_{S-SAT}$, lo que permitió demostrar que no es 
    necesario construir $L_{S-SAT}$ mediante transducción finita. La gramática que se construyó tiene el 
    problema de la palabra no polinomial, y constituye un ejemplo de una RCG donde el algoritmo de reconocimiento 
    es no polinomial.  Además, al obtener una RCG que reconoce $L_{S-SAT}$, se demostró que las 
    RCG reconocen todos los problemas de la clase NP.
    
    Las estrategias presentadas constituyen una vía diferente para resolver el SAT, y aunque el problema de 
    la palabra para el formalismo que se construyó es no polinomial, este acercamiento puede contribuir a nuevas 
    líneas de investigación para la búsqueda de algoritmos eficientes que permitan resolver el SAT.
    
    A partir del trabajo realizado se proponen como temas para investigaciones futuras los
    siguientes:
    
    \begin{itemize}
        \item Buscar un formalismo que genere el lenguaje $L_{0,1,d}$, que sea cerrado bajo transducción finita, y analizar el problema de la palabra para el formalismo que se obtiene después de aplicarle el transductor $T_{SAT}$. En la literatura
              consultada \cite{globalIndexLanguages} para la realización de este trabajo se encontró un formalismo que cumple las propiedades anteriores.
        \item Demostrar que cualquier formalismo que genere $L_{0,1,d}$ tiene un tamaño $O(1)$ en su representación.
        \item Analizar qué tipo de formalismo se obtiene al aplicarle el transductor $T_{SAT}$ a la RCG que reconoce el lenguaje $L_{0,1,d}$.
        \item Analizar qué propiedades limitan que las RCG no sean cerradas bajo transducción finita, construir un formalismo basado en las RCG que sea cerrado bajo transducción finita y comprobar si este formalismo es capaz de describir el lenguaje $L_{0,1,d}$.
        \item Construir una RCG que reconozca fórmulas booleanas satisfacibles, donde cada cláusula tiene a lo sumo dos literales (2-SAT), y que tenga el problema de la palabra polinomial.
    \end{itemize}
    
\end{conclusionsAndRecomendations}
