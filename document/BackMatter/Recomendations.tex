\begin{recomendations}
    A partir del trabajo realizado se proponen como temas para investigaciones futuras los
    siguientes:
    
    \begin{itemize}
        \item Buscar un formalismo que sea capaz de generar el lenguaje $L_{0,1,d}$, el cual representa todas las interpretaciones
              de las fórmulas booleanas en CNF, que sea cerrado bajo transducción finita, y luego analizar el problema de la palabra para
              el formalismo que se obtiene después de aplicarle el transductor $T_{SAT}$.
        \item Demostrar que cualquier formalismo que genere $L_{0,1,d}$ tiene un tamaño O(1) en su representación.
        \item Analizar que tipo de formalismo se obtiene al aplicarle el transductor $T_{SAT}$ a la RCG que reconoce
              el $L_{0,1,d}$.
        \item  Analizar que propiedades limitan que las RCG no sean cerradas bajo transducción finita, construir
              un formalismo basado en las RCG que sea cerrado bajo transducción finita y comprobar que este formalismo
              sea capaz de describir el lenguaje $L_{0,1,d}$.
        \item Construir una RCG que reconozca fórmulas booleanas satisfacibles, donde cada cláusula tiene a lo sumo dos literales (2-SAT),
              que tenga el problema de la palabra polinomial.
    \end{itemize}
    
\end{recomendations}
