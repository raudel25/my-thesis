\begin{conclusions}
    En este trabajo se presentó una estrategia para resolver el SAT usando teoría de lenguajes, la cual se basa en definir
    una codificación de una fórmula booleana en una cadena y definir y construir el lenguaje de todas las fórmulas booleanas
    satisfacibles $L_{S-SAT}$. Luego para determinar si una fórmula booleana es satisfacible es necesario verificar si la cadena asociada
    a dicha fórmula pertenece a $L_{S-SAT}$.
    
    En el capítulo \ref{chap:LSATFT}, se construyó $L_{S-SAT}$ mediante el transductor finito $T_{SAT}$ que recibe
    cadenas del lenguaje $L_{0,1,d}$, las cuales representan todas las posibles interpretaciones de las fórmulas
    booleanas en CNF y genera cadenas del lenguaje $L_{FULL-SAT}$, tales que la fórmula booleana asociada a estas
    cadenas es satisfacible.
    
    Posteriormente se demostró que el problema de la palabra para todo formalismo que genere
    el lenguaje $L_{0,1,d}$ y sea cerrado bajo transducción finita, es NP-Duro. Esta demostración se apoya en la conjetura
    de que cualquier formalismo que genere el lenguaje $L_{0,1,d}$, tiene tamaño $O(1)$ en su representación.
    
    En el capítulo \ref{chap:LSATRCG}, se presentó una RCG que reconoce el lenguaje $L_{0,1,d}$ y se argumentó porque no es posible
    usar esta gramática para construir $L_{S-SAT}$ mediante transducción finita, ya que las RCG no son cerradas bajo transducción finita.
    
    Siguiendo la idea de las RCG y el SAT, se construyó una RCG que reconoce el lenguaje $L_{S-SAT}$, esto permitió demostrar
    que no es necesario construir $L_{S-SAT}$ mediante transducción finita. La gramática que se construyó tiene el problema
    de la palabra no polinomial, y constituye un ejemplo de una RCG donde el algoritmo de reconocimiento es no polinomial.
    Además al obtener una RCG que reconoce $L_{S-SAT}$, se demostró que las RCG cubren todos los problemas de la clase NP,
    ya que las RCG cubren todos los problemas en P \cite{mainRCGBib} y existe una reducción polinomial del SAT a todo problema en NP \cite{authomataTheory}.
    
    Las estrategias presentadas en los capítulos \ref{chap:LSATFT} y \ref{chap:LSATRCG} constituyen una vía diferente
    para resolver el SAT y aunque el problema de la palabra para el formalismo que se construyó es no polinomial
    este acercamiento puede contribuir a nuevas líneas de investigación para la búsqueda de algoritmos eficientes que permitan
    resolver el SAT.   
    
\end{conclusions}
