% \chapter{Preliminares}
% \label{chap:preliminaries}

% \chapter{Gramáticas de concatenación de rango}
% \label{chap:RCG}

\chapter{Lenguaje de las fórmulas booleanas satisfacibles empleando transducción finita}
\label{chap:LSATFT}

% \chapter{Lenguaje de las fórmulas booleanas satisfacibles empleando gramáticas de concatenación de rango}
% \label{chap:LSATRCG}

En este capítulo se presenta el lenguaje $L_{S-SAT}$, al cual pertenecen todos los problemas SAT que son 
satisfacibles, y se muestra una forma de construirlo a partir de una transducción finita de una variante 
del lenguaje $L_{copy}$ sobre el alfabeto $\{0,1,d\}$. Este lenguaje permitiría resolver instancias del SAT resolviendo el problema de la palabra.  

Para definir el lenguaje $L_{S-SAT}$ se presenta una vía para codificar una fórmula booleana mediante cadenas sobre el alfabeto $\Sigma=\{a, b,c,d\}$, y para construirlo se utiliza una transducción finita del lenguaje $L_{0,1,d}$, que es una variante del lenguaje $L_{copy}$ sobre el alfabeto $\{0,1,d\}$.

La estructura de este capítulo es la siguiente: en la sección \ref{sec:codsat} se muestra como codificar 
una fórmula booleana cualquiera usando el alfabeto $\{a,b,c,d\}$. 
En al sección \ref{sec:LSATCFL} se demuestra que $L_{S-SAT}$ no es un lenguaje libre del contexto.
En la sección \ref{sec:intsat} se muestra cómo interpretar las cadenas sobre el alfabeto $\{0,1,d\}$ como asignaciones de las variables. Finalmente, en la sección \ref{sec:tsat} se presenta un transductor finito que convierte cadenas del lenguaje $L_{0,1,d}$ en cadenas sobre el alfabeto $\{a,b,c,d\}$ que representan fórmulas booleanas satisfacibles. Seguidamente se conjetura por qué la representación del lenguaje de las fórmulas booleanas satisfacibles, en cualquier formalismo que lo genere usando la estrategia propuesta en este capítulo, tiene un tamaño $O(1)$. Esto implica que el problema de la palabra para todos estos formalismos es NP-Duro.


A continuación se presenta cómo codificar una fórmula booleana cualquiera mediante una cadena sobre el alfabeto $\{a,b,c,d\}$.

\section{Codificación de una fórmula booleana a una cadena}
\label{sec:codsat}

Una fórmula booleana $F$, con $v$ variables en CNF tiene la siguiente estructura:
$$F=X_1 \wedge X_2 \wedge \ldots \wedge X_n$$
donde cada cláusula $X_i$ es una disyunción de literales
$$X_i=L_{i1} \vee L_{i2} \vee \ldots \vee L_{im},$$
cada literal $L_{ij}$ es una variable booleana o su negación. También se asume que $m\leq v$. 

Si se tiene una fórmula booleana $F$ en forma normal conjuntiva se puede considerar que cada una de las $v$ variables aparece en cada cláusula en uno de tres posibles estados: sin negar, negada, o no aparece.

Por ejemplo, en la primera cláusula de la siguiente fórmula booleana en CNF con 3 variables:
$$F=(x_1 \vee \neg x_2) \wedge (\neg x_1 \vee x_2 \vee x_3) \wedge (x_1 \vee \neg x_2 \vee x_3)$$

la variable $x_1$ aparece sin negar, la variable $x_2$ aparece negada, y la variable $x_{3}$ no aparece.

El hecho de que se pueda asumir que en todas las cláusulas aparecen todas variables permite representar una cláusula de una fórmula con $v$ variables como una cadena de $v$ símbolos, donde el símbolo en la posición $i$ indica el estado de la variable $x_i$ en la cláusula.

En este trabajo se propone usar los símbolos $a$, $b$ y $c$ para indicar el estado de una variable en una cláusula, usando el siguiente convenio:

\begin{itemize}
    \item $a$: indica que la variable aparece sin negar,
    \item $b$: indica que la variable aparece negada,
    \item $c$: indica que la variable no aparece.
\end{itemize}

Con este convenio, la primera cláusula de $F$ se puede representar mediante la cadena $abc$.

Una vez que se tiene cómo representar una cláusula es posible representar varias cláusulas usando otro símbolo como separador.
Por ejemplo se puede usar $d$, para indicar el final de una cláusula. De esta forma una fórmula lógica con $v$ variables y $k$ cláusulas se puede representar mediante $k$ bloques de longitud $v$, donde cada bloque está formado por los símbolos $a$, $b$, o $c$, y cada bloque se separa del siguiente por el símbolo $d$.

Con este convenio, la fórmula
$$ F=(x_1 \vee \neg x_2) \wedge (\neg x_1 \vee x_2 \vee x_3) \wedge (x_1 \vee \neg x_2 \vee x_3)$$
se representa mediante la cadena:
$$abc\mathbf{d}baa\mathbf{d}aba\mathbf{d},$$
donde los símbolos $\mathbf{d}$ aparecen en negrita para facilitar la interpretación de la cadena como fórmula 
en forma normal conjuntiva.

Para que una cadena $e$ se pueda interpretar como una fórmula booleana debe cumplir con las siguientes condiciones:
tener $n$ bloques separados por $d$, cada bloque de la misma longitud $v$ y cada bloque solo debe estar formado por los caracteres
$a$, $b$ y $c$. 

Una cadena $e$, si cumple con estas características,  se interpreta como una fórmula booleana con $n$ cláusulas y $v$ variables, donde la estructura de cada cláusula depende de los caracteres correspondiente al bloque de $a$, $b$ y $c$ que se asocia a dicha cláusula.

Por ejemplo la cadena $w=acc\mathbf{d}aba\mathbf{d}cba\mathbf{d}$, tiene 3 bloques separados por $d$, los cuales son $acc$, $aba$ y $cba$, los 3 tienen tamaño 3 y solo tienen los caracteres $a$, $b$ y $c$.
Por tanto  $w$ se puede interpretar como la siguiente fórmula booleana:
$$(x_1)\wedge(x_1\vee \neg x_2 \vee x_3) \wedge (\neg x_2\vee x_3)$$

Una vez definida la transformación de una fórmula booleana en una cadena, se puede definir el lenguaje de todas las fórmulas booleanas en CNF.

\begin{definition}
    El \textbf{lenguaje de todas las fórmulas booleanas en CNF} se define como:
    \[
        L_{FULL-SAT} = \{ q_1dq_2d\dots q_nd \mid q_i \in \{a, b,c\}^+\text{, }
        |q_i| = |q_j| \, \forall i, j =1\dots n, \text{ y } n\in \mathbb{N}\}.
    \]
\end{definition}

Si se tomo un subconjunto de $L_{FULL-SAT}$ se puede definir el lenguaje de todas las fórmulas booleanas satisfacibles
en CNF, el cual se define a continuación.

\begin{definition}
    El \textbf{lenguaje de todas las fórmulas booleanas en CNF que son satisfacibles} $L_{S-SAT}$ se define como todas las cadenas $e\in L_{FULL-SAT}$,
    tales que la fórmula booleana que representa $e$, sea satisfacible.
\end{definition}

Por ejemplo, la fórmula $$x_1\wedge x_2 \wedge x_3,$$ es satisfacible por los valores $x_1=true$, $x_2=true$ y $x_3=true$, por lo que la 
cadena $acc\mathbf{d}cac\mathbf{d}cca\mathbf{d}$ pertenece a $L_{S-SAT}$. Por otro lado la fórmula $$x_1\wedge x_2 \wedge \neg x_1,$$
no es satisfacible para ninguna asignación de los valores de sus variables, por lo que la cadena $ac\mathbf{d}ca\mathbf{d}bc\mathbf{d}$
no pertenece a $L_{S-SAT}$.

En la próxima sección se demuestra que $L_{S-SAT}$ no es un lenguaje libre del contexto, por lo que el formalismo que lo genere necesariamente debe pertenecer a las gramáticas dependientes del contexto o a las gramáticas irrestrictas.

\section{$L_{S-SAT}$ no es un lenguaje libre del contexto}
\label{sec:LSATCFL}

En esta sección se demuestra que el lenguaje $L_{S-SAT}$ no es libre del contexto. Una técnica que se emplea para demostrar que un lenguaje no es libre del contexto, es el lema del bombeo para este tipo de lenguajes \cite{authomataTheory}.

\begin{theorem}
    \label{teo:LSATCFL}
    $L_{S-SAT}$ no es un lenguaje libre del contexto.
\end{theorem}

Para la demostración del teorema \ref{teo:LSATCFL} se presentan los siguientes lemas:

\begin{lemma}
    \label{lem:Lh}
    Sea un homomorfismo $h: \{a,b,c,d\}\to \{1,d\}^*$, tal 
    que $h(a)=1$, $h(b)=1$, $h(c)=1$ y $h(d)=d$. Si se define el lenguaje $L_h=\{h(e)\mid e\in L_{S-SAT}\}$,
    entonces $L_h=\{(1^nd)^+\mid n\in \mathbb{N}\}$.
\end{lemma}

\begin{lemma}
    \label{lem:LhCFL}
    $L_h$ no es un lenguaje libre del contexto.
\end{lemma}

La idea de la demostración del Teorema \ref{teo:LSATCFL}, es definir un homomorfismo sobre 
$L_{S-SAT}$ y luego probar que el lenguaje resultante de evaluar todas las cadenas de $L_{S-SAT}$ en el homomorfismo es igual a $\{(1^nd)^+\mid n\in \mathbb{N}\}$, esto se plantea
en el Lema \ref{lem:Lh}. Seguidamente, se demuestra que $L_h$ no es libre del contexto, usando el lema del bombeo,
esto se plantea en el Lema \ref{lem:LhCFL}. Para finalizar la demostración se prueba que el hecho de que $L_h$ no sea
libre del contexto implica que $L_{S-SAT}$ no es libre del contexto.

A continuación se demuestra el Lema \ref{lem:Lh}.

\begin{proof}[Demostración del Lema \ref{lem:Lh}]
    Para demostrar que $L_h=\{(1^nd)^+\mid n\in \mathbb{N}\}$ se debe demostrar que $L_h$ es subconjunto de $\{(1^nd)^+\mid n\in \mathbb{N}\}$ y 
    que $\{(1^nd)^+\mid n\in \mathbb{N}\}$ es subconjunto de $L_h$.
    
    Para demostrar que $L_h\subseteq \{(1^nd)^+\mid n\in \mathbb{N}\}$, sea una cadena $t\in L_h$, entonces se cumple que existe una cadena
    $e\in L_{S-SAT}$ tal que $h(e)=t$. Luego como $e$ está formada por varios bloques del $a$, $b$ y $c$ del mismo tamaño
    unidos por $d$, entonces $t$ esta formada por varios bloques de 1 del mismo tamaño unidos por $d$, por tanto $t\in \{(1^nd)^+\mid n\in \mathbb{N}\}$.
    
    Para demostrar que $\{(1^nd)^+\mid n\in \mathbb{N}\}\subseteq L_h$, sea una cadena $t\in \{(1^nd)^+\mid n\in \mathbb{N}\}$, sin pérdida
    de la generalidad $t=(1^nd)^k$, con $n\in \mathbb{N}$ y $k\in \mathbb{N}$. Se cumple que la fórmula 
    $$F=\underbrace{(x_1\vee x_2 \ldots x_n) \wedge (x_1\vee x_2 \ldots x_n) \ldots (x_1\vee x_2 \ldots x_n)}_{k},$$
    es satisfacible por los valores $x_1=true$, $x_2=true$, $\ldots$, $x_n=true$, y además la cadena $e=(a^n)^k$ es la codificación
    de $F$ en $L_{FULL-SAT}$. Por tanto  $e\in L_{S-SAT}$ y $h(e)=t$, lo cual implica que $t\in L_h$. Luego se cumple
    que $\{(1^nd)^+\mid n\in \mathbb{N}\}\subseteq L_h$ y por tanto $\{(1^nd)^+\mid n\in \mathbb{N}\}= L_h$.
\end{proof}

Seguidamente se demuestra el Lema \ref{lem:LhCFL}.

\begin{proof}[Demostración del Lema \ref{lem:LhCFL}]
    Para demostrar que $L_h$ no es libre del contexto sea $n$ la constante asociada a $L_h$ en el lema del bombeo.
    
    Sea $t=1^n\mathbf{d}1^n\mathbf{d}1^n\mathbf{d}$, como $|t|\geq n$ existen $u$, $v$, $w$, $x$ y $y$ con $vx\neq \varepsilon$ y $|vwx|\leq n$
    tales que $h(e)=uvwxy$.
    
    Se cumple que entre las dos $d$ hay exactamente $n$ 1 y como $|vwx|\leq n$ existen 2 casos: existe una $d$ que pertenece a
    $v$ o a $x$ pero no a las 2 a la vez o ninguna $d$ pertenece a $v$ y a $x$.
    
    En el primer caso cuando se bombea $v$ y $x$ en la cadena $uv^2wx^2y$ se agrega exactamente un bloque de 1 más, porque $v$ o $x$ contienen una $d$, pero no las dos a la vez, y el bloque de 1
    que se agrega tiene un tamaño menor o igual a $n$, por lo que $uv^2wx^2y\notin L_h$, ya que hay un bloque de 1
    con menos caracteres que las demás.
    
    En el segundo caso, cuando se bombea $v$ y $x$ en la cadena $uv^2wx^2y$ se agrega al menos un caracter al bloque de 1 al que pertenecía 
    $v$ o al menos un caracter al bloque de 1 al que pertenecía $x$.  Si $v$ y $x$ pertenecían al mismo bloque 
    de 1 en $uvwxy$ entonces hay un bloque de 1 en $uv^2wx^2y$ que tiene más caracteres que los restantes 
    bloques, en caso contrario hay uno o dos bloques en $uv^2wx^2y$ que tienen más caracteres que los restantes 
    bloques, entonces $uv^2wx^2y\notin L_{h}$.
    
    En los dos casos se cumple que $uv^2wx^2y\notin L_h$ por lo tanto se cumple que $L_h$ no es un lenguaje libre del contexto.
\end{proof}

A continuación se demuestra el Teorema \ref{teo:LSATCFL}.

\begin{proof}[Demostración del Teorema \ref{teo:LSATCFL}]
    Para demostrar que $L_{S-SAT}$ no es libre del contexto, suponga lo contrario. 
    
    Como los lenguajes libres del contexto
    son cerrados bajo homomorfismo y $L_h$ es el lenguaje generado al evaluar todas las cadenas de $L_{S-SAT}$ en $h$, se cumple
    que $L_h$ es libre del contexto. Pero por el Lema \ref{lem:LhCFL}, $L_h$ no es libre del contexto, lo cual es una contradicción.
    Por tanto se cumple que $L_{S-SAT}$ no es libre del contexto.
\end{proof}

Seguidamente, se muestra cómo interpretar determinados tipos de cadenas de 0 y 1 como la asignación de los valores de las variables de una fórmula booleana.

\section{Definición del lenguaje que representa la asignación de los valores de las variables de una fórmula booleana}
\label{sec:intsat}

En esta sección se muestra cómo interpretar una cadena $r\in \{1,d\}^+$ como la asignación de valores para las variables de una fórmula booleana.
A continuación se define el lenguaje $L_{0,1,d}$.


\begin{definition}
    El lenguaje $L_{0,1,d}$ se define como:
    $$L_{0,1,d}=\{(wd)^+\,|\,w\in\{0,1\}^+\}.$$
    
    $L_{0,1,d}$ contiene cadenas sobre el alfabeto $\{0,1,d\}$, que representan una cadena binaria concatenada con una $d$,
    repetida varias veces.
\end{definition}

Por ejemplo, $011\mathbf{d}011\mathbf{d}011\mathbf{d}$ y $01001\mathbf{d}01001\mathbf{d}01001\mathbf{d}$ son cadenas que pertenecen a
$L_{0,1,d}$.

Para que una cadena $r\in L_{0,1,d}$ se pueda interpretar como una asignación de variables a una fórmula booleana $F$
que representa la cadena $e\in L_{FULL-SAT}$, se deben cumplir 2 condiciones: $|e|=|r|$ y la cantidad de caracteres $d$ en
$e$ debe ser igual a la cantidad de caracteres $d$ en $r$. Luego 
a cada subcadena binaria de $r$ se le asocia un bloque de 
$a$, $b$ y $c$ en $e$, el cual representa una cláusula de $F$, lo anterior se hace en correspondencia de los caracteres $d$
que determinan la subcadena binaria y la cláusula respectivamente. 

Después para cada subcadena binaria el valor del $i$-ésimo caracter se le asocia al valor de 
la $i$-ésima variable de la cláusula correspondiente, si este caracter es un 1 se le asocia un valor de \true{} (verdadero) y 
si el caracter es un 0 se le asocia el valor de \false{} (falso). Como todas las subcadenas binarias de $r$ son iguales 
se garantiza que a dos instancias de la misma variable en dos cláusulas distintas se les asigne el mismo valor. 

Si se tiene una cadena $e\in L_{FULL-SAT}$, la cual representa una fórmula booleana $F$ en CNF con $v$ variables, todas las cadenas $(wd)^n$, tales que $|w|=v$ y $|(wd)^n|=|e|$, representan todas las posibles interpretaciones de $F$.

Por ejemplo, si $r=101\mathbf{d}101\mathbf{d}101\mathbf{d}$ y $e=abc\mathbf{d}cbb\mathbf{d}acc\mathbf{d}$, es una cadena que representa una fórmula booleana:
$$(x_1\vee\neg x_2)\wedge (\neg x_2 \vee \neg x_3)\wedge (x_1),$$
$r$ representa la asignación de valores a las variables de la fórmula booleana que representa $e$ de la 
siguiente manera: $x_1=true$, $x_2=false$ y $x_3=true$, y la fórmula booleana se evalúa con valor \true{}.

Seguidamente se muestra cómo construir $L_{S-SAT}$ mediante una transducción finita del lenguaje $L_{0,1,d}$.

\section{Construcción del $L_{S-SAT}$ usando transducción finita}

En esta sección se construye el lenguaje $L_{S-SAT}$ mediante un transductor finito.

La idea para construir $L_{S-SAT}$ es construir un transductor finito, denominado $T_{SAT}$, que acepte 
como entrada cadenas $r\in L_{0,1,d}$ y devuelva cadenas de $e\in L_{FULL-SAT}$ tales que al evaluar $r$
en $e$ (como se describió en la sección \ref{sec:intsat}), $e$ es verdadera.

Se construye $L_{S-SAT}$ como el lenguaje de todas las transducciones $e$ que se obtienen del transductor $T_{SAT}$, a partir del lenguaje de cadenas de entrada $L_{0,1,d}$.
$$L_{S-SAT} = \{e\,|\,\exists r \in L_{0,1,d} \text{ y } e \in T_{SAT}(r) \}.$$

A continuación se define el transductor $T_{SAT}$.

\subsection{Transductor $T_{SAT}$}
\label{sec:tsat}

En esta sección se define el transductor finito $T_{SAT}$ (Figura \ref{fig:transducer_sat}), el cual se usa para construir $L_{S-SAT}$, mediante una transducción finita del lenguaje $L_{0,1,d}$.

Para definir $T_{SAT}$, se construye el transductor $T_{CLAUSE}$ (Figura \ref{fig:transducer_clause}) que
dada una cadena binaria $w$, genera todas las posibles cláusulas satisfacibles por los valores de las
variables que determina la cadena de entrada. 

La idea detrás del transductor es ir leyendo los caracteres de la cadena de entrada y por cada caracter
que se lee se genera un literal de una cláusula, el caracter que se lee se le asigna como valor a la variable
correspondiente del literal generado (si es un 1, se asigna \true{} y si es un 0 se asigna \false{}).

Este transductor tiene 3 estados: el estado inicial, el estado positivo, y el estado negativo.  
El estado positivo representa que la cláusula generada ya se evalúa con un valor de verdad positivo y 
el estado negativo representa que la cláusula generada aún no se evalúa con un valor de verdad positivo 
para los caracteres que se leyeron hasta el momento. Las transiciones entre los estados se realizan 
dependiendo de si la asignación que se realiza, en el momento de leer y de escribir satisface la cláusula 
generada o no.

A continuación de describe el funcionamiento de cada estado del transductor $T_{CLAUSE}$:
\begin{itemize}
    \item Estado $q_0$: representa el estado inicial.  Si se lee un 1 y se escribe una $a$, se pasa al
          estado positivo, ya que se genera una variable sin negar a la cual se le asigna el valor \true{}.  
          Si se lee un 1 y se escribe una $b$, se pasa al estado negativo, ya que se genera una variable negada 
          a la cual se le asigna el valor \true{}.  Si se lee un 1 y se escribe una $c$, se mantiene en el 
          mismo estado, ya que se genera una variable que no está en la cláusula a la cual se le asigna el 
          valor \true{}.  Si se lee un 0 y se escribe una $a$, se pasa al estado negativo, ya que se genera 
          una variable sin negar a la cual se le asigna el valor \false{}.  Si se lee un 0 y se escribe una 
          $b$, se pasa al estado positivo, ya que se genera una variable negada a la cual se le asigna el valor 
          \false{}.  Si se lee un 0 y se escribe una $c$, se mantiene en el mismo estado, ya que se genera una 
          variable que no está en la cláusula a la cual se le asigna el valor \false{}.
          
    \item Estado $q_p$ (estado positivo de $T_{CLAUSE}$): representa que para los valores ya asignados
          a las variables se obtiene un valor de verdad positivo.  Como la fórmula se encuentra ya en un estado 
          positivo, no importa la entrada ni lo que el transductor escriba, se mantiene en el mismo estado. 
          Este estado es el estado de aceptación para el transductor y significa que la cláusula se evalúa con un 
          valor de verdad positivo.
          
    \item Estado $q_n$ (estado negativo de $T_{CLAUSE}$): representa que para los valores ya asignados a las
          variables se obtiene un valor de verdad negativo.  Si se lee un 1 y se escribe una $a$, se pasa al estado positivo, ya que se genera una variable sin negar a la cual se le asigna el valor \true{}.  Si se lee un 1 y se escribe una $b$, se mantiene en el mismo estado, ya que se genera una variable negada a la cual se le asigna el valor \true{}.  Si se lee un 1 y se escribe una $c$, se mantiene en el mismo estado, ya que se genera una variable que no está en la cláusula a la cual se le asigna el valor \true{}.  Si se lee un 0 y se escribe una $a$, se mantiene en el mismo estado, ya que se genera una variable sin negar a la cual se le asigna el valor \false{}.  Si se lee un 0 y se escribe una $b$, se pasa al estado positivo, ya que se genera una variable negada a la cual se le asigna el valor \false{}.  Si se lee un 0 y se escribe una $c$, se mantiene en el mismo estado, ya que se genera una variable que no está en la cláusula a la cual se le asigna el valor \false{}.
\end{itemize}

Seguidamente se define $T_{CLAUSE}$.

\[
    T_{CLAUSE} = (Q, {\Sigma}, \Gamma, \delta, q_{0}, F),
\]
donde:
\begin{itemize}
    \item \(Q\) = ${q_0,q_p,q_n}$.
    \item \(\Sigma\) = $\{0,1\}$.
    \item \(\Gamma\) = ${a,b,c}$.
    \item \(\delta: Q \times \Sigma \to Q \times \Gamma^*\) función de transición.
    \item \(q_{0} = q_0\) estado inicial.
    \item \(F={q_p}\) conjunto de estados finales.
\end{itemize}

Seguidamente  se define la función de transición $\delta$, con cada una de las transiciones para cada estado.

Las transiciones para el estado $q_0$ son las siguientes:
\begin{multicols}{2}
    \begin{itemize}
        \item $\delta_{SAT}(q_0,1)=(q_p,a)$
        \item $\delta_{SAT}(q_0,0)=(q_n,a)$
        \item $\delta_{SAT}(q_0,1)=(q_n,b)$
        \item $\delta_{SAT}(q_0,0)=(q_p,b)$
        \item $\delta_{SAT}(q_0,1)=(q_0,c)$
        \item $\delta_{SAT}(q_0,0)=(q_0,c)$
    \end{itemize}
\end{multicols}

Las transiciones para el estado $q_p$ (estado positivo de $T_{CLAUSE}$) son las siguientes:
\begin{multicols}{2}
    \begin{itemize}
        \item $\delta_{SAT}(q_{p},1)=(q_{p},a)$
        \item $\delta_{SAT}(q_{p},0)=(q_{p},a)$
        \item $\delta_{SAT}(q_{p},1)=(q_{p},b)$
        \item $\delta_{SAT}(q_{p},0)=(q_{p},b)$
        \item $\delta_{SAT}(q_{p},1)=(q_{p},c)$
        \item $\delta_{SAT}(q_{p},0)=(q_{p},c)$
    \end{itemize}
\end{multicols}

Las transiciones para el estado $q_n$ (estado negativo de $T_{CLAUSE}$) son las siguientes:
\begin{multicols}{2}
    \begin{itemize}
        \item $\delta_{SAT}(q_{n},1)=(q_{p},a)$
        \item $\delta_{SAT}(q_{n},0)=(q_{n},a)$
        \item $\delta_{SAT}(q_{n},1)=(q_{n},b)$
        \item $\delta_{SAT}(q_{n},0)=(q_{p},b)$
        \item $\delta_{SAT}(q_{n},1)=(q_{n},c)$
        \item $\delta_{SAT}(q_{n},0)=(q_{n},c)$
    \end{itemize}
\end{multicols}

\begin{figure}[h]
    \centering  \begin{otherlanguage}{english}
        \begin{tikzpicture}[shorten >=1pt, node distance=3cm, on grid, auto]
            
            % Nodos
            \node[state, initial] (q0)   {$q_0$};
            \node[state] (qn) [above right=of q0] {$q_n$};
            \node[state, accepting] (qp) [below right=of q0] {$q_p$};
            
            % Transiciones
            \path[->]
            (q0) edge [bend left] node {0/a,1/b} (qn)
            (q0) edge [bend right] node {1/a,0/b} (qp)
            (q0) edge [loop right] node {0/c,1/c} (q0)
            
            (qn) edge [bend left] node {1/a,0/b} (qp)
            (qn) edge [loop above] node {0/a,1/b,0/c,1/c} (qn)
            
            (qp) edge [loop below] node {1/a,0/b,0/a,1/b,0/c,1/c} (qp);
            
        \end{tikzpicture}
    \end{otherlanguage}
    \caption{Representación gráfica del Transductor $T_{CLAUSE}$.}
    \label{fig:transducer_clause} % Esto es para referenciar la figura en el texto
\end{figure}

Si se tiene una cadena binaria $w$ se pueden generar todas las fórmulas booleanas de una cláusula satisfacibles por la asignación de valores que representa 
$w$ (si el $i$-ésimo caracter es un 1, la $i$-ésima variable de la cláusula tiene valor \true{} y 
si el $i$-ésimo caracter es un 0, la $i$-ésima variable de la cláusula tiene valor \false{}). Si se concatena 2 veces $w$ se pueden 
generar todas las fórmulas booleanas de 2 cláusulas satisfacibles por la asignación de valores que representa
$w$. Por último, si se concatena $n$ veces $w$ se pueden 
generar todas las fórmulas booleanas de $n$ cláusulas satisfacibles por la asignación de valores que representa
$w$. Pero si se concatenan $n$ cadenas $w$ con un caracter $d$ intermedio, la cadena formada pertenece a $L_{0,1,d}$, 
entonces se puede modificar $T_{CALUSE}$, para que lea cadenas de $L_{0,1,d}$ y genere cadenas que pertenecen a $L_{S-SAT}$.

Esta modificación funciona de la siguiente manera: 
cuando se genera una cláusula y el transductor termina en el estado positivo se continúa leyendo los valores de la cadena de entrada, ya que la cadena de entrada está conformada por varias cláusulas separadas mediante el caracter $d$ que son necesarios para generar la próxima cláusula.

La modificación de $T_{CALUSE}$ se logra definiendo $q_0$ como el estado de aceptación y agregando una 
transición del estado $q_p$ al estado $q_0$ que lea una $d$ y escriba una $d$. De esta manera, cuando se lee una $d$ y el transductor se encuentra en el estado positivo, significa
que se generó una cláusula con un valor de verdad positivo, y entonces se comienza a generar la siguiente cláusula desde el estado inicial. 

Esta modificación de $T_{CALUSE}$ tiene un inconveniente y es que el transductor genera la cadena vacía, y la cadena vacía representa una fórmula booleana con 0 variables, por lo que no tiene sentido que se considere en $L_{S-SAT}$. Para solucionar esto se pueden concatenar 2 transductores $T_{CLAUSE}$ y unirlos mediante una transición, esta idea se expone a continuación.

Para definir el transductor $T_{SAT}$ (Figura \ref{fig:transducer_sat}) se concatenan 2 transductores $T_{CLAUSE}$ ($T_1$ y $T_2$ respectivamente). Sean los estados: $q_{0_1}$, $q_{p_1}$ y $q_{n_1}$ estado inicial, positivo y negativo de $T_1$, respectivamente, y los estados $q_{0_2}$, $q_{p_2}$ y $q_{n_2}$ estado inicial, positivo y negativo de $T_2$, respectivamente. $T_1$ y $T_2$ se concatenan añadiendo una transición de $q_{p_1}$ a $q_{0_2}$ con el símbolo $d$ (tanto de lectura como de escritura) y además se agrega una transición de $q_{p_2}$ a $q_{0_2}$ con el símbolo $d$ (tanto de lectura como de escritura).  Para terminar se definen el estado inicial y el estado final de $T_{SAT}$, los cuales serían $q_{0_1}$ y $q_{0_2}$, respectivamente.



\begin{figure}[h]
    \begin{otherlanguage}{english}
        \centering \begin{tikzpicture}[shorten >=1pt, node distance=3cm, on grid, auto]
            
            % Nodos
            \node[state, initial] (q01)   {$q_{0_1}$};
            \node[state] (qn1) [above right=of q01] {$q_{n_1}$};
            \node[state] (qp1) [below right=of q0] {$q_{p_1}$};
            \node[state, accepting] (q02) [right=6cm of q01] {$q_{0_2}$};
            \node[state] (qn2) [above right=of q02] {$q_{n_2}$};
            \node[state] (qp2) [below right=of q02] {$q_{p_2}$};
            
            
            % Transiciones
            \path[->]
            (q01) edge [bend left] node {0/a,1/b} (qn1)
            (q01) edge [bend right] node {1/a,0/b} (qp1)
            (q01) edge [loop right] node {0/c,1/c} (q01)
            
            (qn1) edge [bend left] node {1/a,0/b} (qp1)
            (qn1) edge [loop above] node {0/a,1/b,0/c,1/c} (qn1)
            
            (qp1) edge [loop below] node {1/a,0/b,0/a,1/b,0/c,1/c} (qp1)
            
            (q02) edge [bend left] node {0/a,1/b} (qn2)
            (q02) edge [bend right] node {1/a,0/b} (qp2)
            (q02) edge [loop right] node {0/c,1/c} (q02)
            
            (qn2) edge [bend left] node {1/a,0/b} (qp2)
            (qn2) edge [loop above] node {0/a,1/b,0/c,1/c} (qn2)
            
            (qp2) edge [loop below] node {1/a,0/b,0/a,1/b,0/c,1/c} (qp2)
            
            (qp1) edge [bend right] node {d/d} (q02)
            (qp2) edge [bend left=75] node {d/d} (q02);
            
        \end{tikzpicture}
    \end{otherlanguage}
    \caption{Representación gráfica del Transductor $T_{SAT}$.}
    \label{fig:transducer_sat} % Esto es para referenciar la figura en el texto
\end{figure}

A continuación se presenta la demostración de que la construcción de $L_{S-SAT}$ mediante una transducción finita genera todas las fórmulas booleanas satisfacibles.

\section{La construcción de $L_{S-SAT}$ mediante una transducción finita
  genera todas las fórmulas satisfacibles}


En esta sección se demuestra que la construcción de $L_{S-SAT}$, mediante la transducción del lenguaje $L_{0,1,d}$ mediante $T_{SAT}$, genera todas las fórmulas booleanas satisfacibles.

\begin{theorem}
    \label{teo:tsat}
    Una cadena $e$ pertenece al lenguaje generado por la transducción finita del lenguaje $L_{0,1,d}$ mediante el transductor $T_{SAT}$, si y solo si la fórmula booleana asociada a la cadena $e$ es satisfacible. Esto significa que:
    $$L_{S-SAT} = \{e\,|\,\exists r \in L_{0,1,d} \text{ y } e \in T_{SAT}(r) \}.$$
\end{theorem}

A continuación, se presentan algunas definiciones que serán usadas en la demostración del Teorema \ref{teo:tsat}.


\begin{definition}
    Sea una cadena $w\in \{0,1\}^+$ y una fórmula booleana $F$ con la misma cantidad de variables que la longitud de 
    $w$. Cuando se le asignan los valores de $w$ las variables de $F$, si el $i$-ésimo caracter de $w$ es un 1 a la $i$-ésima variable de $F$ se le asigna
    el valor \true{} y si el $i$-ésimo caracter de $w$ es un 0 a la $i$-ésima variable de $F$ se le asigna
    el valor \false{}.
\end{definition}

\begin{definition}
    Una cadena  $w\in\{0,1\}^+$ satisface una fórmula booleana $F$ si al asignarle los valores de $w$ a $F$, se obtiene un valor de verdad positivo.
\end{definition}

\begin{definition}
    Una cadena $w\in\{0,1\}^+$ satisface a una cadena $e\in L_{FULL-SAT}$ si $w$ satisface la fórmula booleana asociada a $e$. 
\end{definition}

Para la demostración del Teorema \ref{teo:tsat} se presentan los siguientes lemas:

\begin{lemma}
    \label{lem:clause}
    Dada una cadena $w\in\{0,1\}^+$, $T_{CLAUSE}(w)$ es el conjunto de todas las cadenas que representan cláusulas que son satisfacibles por $w$.
\end{lemma}

\begin{lemma}
    \label{lem:sat}
    Dada una cadena $r=(wd)^n$, con $w\in\{0,1\}^+$, $T_{SAT}(r)$ contiene todas las cadenas que representan fórmulas de $n$ cláusulas satisfacibles por la cadena $w$.
\end{lemma}

La idea para la demostración del Teorema \ref{teo:tsat}, es probar que dada una cadena $w\in \{0,1\}^+$, 
$T_{CLAUSE}(w)$ es el conjunto de todas las cadenas que representan cláusulas que son satisfacibles por $w$, 
esto se plantea en el Lema \ref{lem:clause}. Después, sea una cadena $r=(wd)^n$ se realiza una inducción 
sobre $n$, para demostrar que $T_{SAT}(r)$ contiene todas las cadenas que representan fórmulas de $n$
cláusulas satisfacibles por la cadena $w$, esto se plantea en el Lema \ref{lem:clause}.  
Para finalizar la demostración se prueba que la transducción de $L_{0,1,d}$ mediante $T_{SAT}$, 
genera todas las cadenas $e\in L_{FULL-SAT}$, tales que $e$ representa una fórmula booleana 
satisfacible.

Seguidamente, se demuestra el Lema \ref{lem:clause}.

\begin{proof}[Demostración del Lema \ref{lem:clause}]
    Para demostrar que dada una cadena binaria $w$, $T_{CLAUSE}(w)$ es el conjunto de todas las cláusulas que 
    son satisfacibles por $w$, primero suponga que $q\in T_{CLAUSE}(w)$.  Esto significa que el transductor 
    terminó en el estado $q_p$ en el proceso que generó $q$ y como empezó en el estado $q_0$ ocurrió una 
    transición desde $q_0$ a $q_p$ o desde $q_n$ a $q_p$. 
    
    Una transición de $q_0$ a $q_p$ o de $q_n$ a $q_p$ solo es posible si el transductor leyó un 1 y 
    escribió una $a$ o si leyó un 0 y escribió una $b$. Entonces, cuando se le asignan los valores de 
    $w$ a las variables de la fórmula booleana que representa $q$ hay una variable sin negar con valor 
    \true{} o una variable negada con valor \false{}, por lo tanto, se cumple que $w$ satisface la cláusula 
    que representa $q$. 
    
    Para demostrar que todas las cláusulas satisfacibles por $w\in\{0,1\}^+$ pertenecen a $T_{CLAUSE}(w)$, 
    sea una cláusula $F$ satisfacible por una cadena $w\in\{0,1\}^+$, cuya representación sobre $\{a,b,c\}$ es $q$, 
    entonces se cumple que cuando se le asignan los valores de $w$ a las variables de $F$, hay una variable 
    sin negar con valor \true{} o una variable negada con valor \false{}. Sin pérdida de la generalidad se 
    asume que la primera variable que cumple lo anterior es la $i$-ésima.
    
    Como en cada estado de $T_{CLAUSE}$ para cada símbolo que se lee existe una transición que escribe 
    una $a$, una $b$ o una $c$; si se hace el reconocimiento de los primeros $i-1$ caracteres de la 
    cadena de entrada por $T_{CLAUSE}$ se pueden seguir las transiciones entre los estados del transductor 
    de tal manera que los primeros $i-1$ caracteres de la cadena generada sean iguales a los primeros 
    $i-1$ caracteres de $q$.
    
    Luego de reconocer los primeros $i-1$ caracteres solo es posible que el transductor esté en el estado $q_0$ o $q_n$, pero como se cumple que la $i$-ésima variable está sin negar y con valor \true{} o está negada y con valor \false{}, entonces se puede tomar la opción de leer un 1 y escribir una a o leer un 0 y escribir una b según corresponda.  De esta manera, según en el estado en que se encuentre el autómata, se pasa al estado $q_p$.
    
    Después, se toman las restantes transiciones de manera que la cadena generada sea $q$ y se mantiene en el mismo estado ya que $q_p$ solo tiene transiciones hacia sí mismo. De esta manera se demuestra el Lema \ref{lem:clause}. 
\end{proof}

Seguidamente se demuestra el Lema \ref{lem:sat}.

\begin{proof}[Demostración del Lema \ref{lem:sat}]
    Para demostrar que $T_{SAT}(r)$ contiene todas las fórmulas satisfacibles por la cadena $r=(wd)^n$ donde $w\in\{0,1\}^+$ se hará una inducción sobre $n$, en el caso base y el paso inductivo
    se prueban ambos sentidos de la demostración.  
    
    Para esto, se definen los conjuntos $A_{w,n}$ como el conjunto formado por todas las cadenas del lenguaje $L_{FULL-SAT}$, que representan fórmulas booleanas de $n$ cláusulas satisfacibles por $w$ y $B_w$ como el conjunto formado por todas las cadenas que representan las cláusulas que son satisfacibles por $w$. Las cadenas de $A_{w,n}$ están formadas por $n$ concatenaciones de cadenas de $B_w$ separadas por $d$.
    
    El caso base $n=1$ se demuestra porque la transducción se realiza solo sobre el primer transductor $T_{CLAUSE}$ que conforma a $T_{SAT}$, ya que en la cadena de entrada solo hay una $d$ y como se demostró en el Lema \ref{lem:clause}, $T_{CLAUSE}(w)$ contiene todas cláusulas satisfacibles por $w$. Por tanto $T_{SAT}(wd)$, donde $w\in \{0,1\}^+$, contiene todas las cadenas que representan fórmulas booleanas de una cláusula satisfacibles por $wd$.
    
    Una vez demostrado el caso base se asume que el Lema \ref{lem:sat} es cierto para $n=k$ y se demuestra para $n=k+1$.
    
    El conjunto de todas las fórmulas booleanas con $k+1$ cláusulas satisfacibles por $w$ es equivalente al conjunto que forman todas las fórmulas booleanas con $k$ cláusulas, satisfacibles por $w$, concatenadas con todas las cláusulas satisfacibles por $w$: $$A_{w,k+1}=\{xzd\,|\,x\in A_{w,k} \text{ y } z\in B_w\}.$$ Además, por la estructura de $T_{SAT}$ se cumple que el conjunto de todas las cadenas que pertenecen al lenguaje generado por $T_{SAT}((wd)^{k+1})$ es igual al conjunto de todas las cadenas que pertenecen al lenguaje generado por $T_{SAT}((wd)^{k})$ concatenadas con todas las cadenas que pertenecen al lenguaje generado por $T_{CLAUSE}(w)$:
    $$T_{SAT}((wd)^{k+1})=\{xzd\,|\,x\in T_{SAT}((wd)^{k}) \text{ y } z\in T_{CLAUSE}(w)\}.$$
    
    Por hipótesis de inducción se cumple que $A_{w,k}=T_{SAT}((wd)^{k})$ y además se cumple que $B_w=T_{CLAUSE}(w)$, lo cual implica que $A_{w,k+1}=T_{SAT}((wd)^{k+1})$, por lo tanto se demuestra el Lema \ref{lem:sat}.
\end{proof}

Luego de demostrados los Lemas \ref{lem:clause} y \ref{lem:sat} se demuestra el Teorema \ref{teo:tsat}.

\begin{proof}[Demostración del Teorema \ref{teo:tsat}]
    sea una cadena $r \in L_{0,1,d}$ y sea una cadena $e\in T_{SAT}(r)$, para la demostrar que el 
    lenguaje generado por la transducción de $L_{0,1,d}$ mediante $T_{SAT}$ es igual a $L_{S-SAT}$, 
    entonces se cumple que la fórmula booleana asociada a $e$ es satisfacible por $r$, por el Lema 
    \ref{lem:sat}. Ahora sea $F$ una fórmula booleana satisfacible y sea $e$ su representación en $L_{FULL-SAT}$.
    Por tanto existe una asignación de valores de las variables que satisface $F$ que satisface a $F$, lo cual implica que existe 
    $r\in L_{0,1,d}$ tal que $r$ satisface a $F$, luego se cumple que $e\in T_{SAT}(r)$, por el Lema \ref{lem:sat}. 
    Por tanto se cumple que:
    $$L_{S-SAT} = \{e\,|\,\exists r \in L_{0,1,d} \text{ y } e \in T_{SAT}(r) \}.$$
\end{proof}

Una consecuencia directa del Teorema \ref{teo:tsat} es el siguiente Teorema. 

\begin{theorem}
    \label{teo:gnp-hard}
    El problema de la palabra de cualquier formalismo que genere el lenguaje $L_{0,1,d}$ y sea cerrado bajo transducción finita, es NP-Duro.
\end{theorem}

\begin{proof}[Demostración del Teorema \ref{teo:gnp-hard}]
    Suponga que existe un formalismo $G$ que genera $L_{0,1,d}$, que es cerrado bajo transducción finita.
    
    Sea $G'$ el formalismo que resulta de aplicarle el transductor $T_{SAT}$ a $G$. Entonces determinar si una cadena $e$ pertenece al lenguaje generado por $G'$, es equivalente a saber si la fórmula booleana a la cual representa $e$ es satisfacible. Por tanto el problema de la palabra para $G'$ es NP-Duro, porque tiene una reducción directa al problema de la satisfacibilidad booleana.
\end{proof}

Una restricción importante en la demostración anterior es que la representación de $G'$ tiene que tener tamaño $O(1)$,
porque si no el problema de la palabra de $G'$ puede depender además de la cantidad de estados (en el caso de
la representación mediante una máquina abstracta) o de la cantidad de producciones, símbolos terminales y no terminales
(en el caso de la representación mediante una  gramática).

En este trabajo se conjetura que cualquier formalismo $G$ que genere el lenguaje $L_{0,1,d}$, tiene tamaño $O(1)$ en su representación y como el transductor $T_{SAT}$ tiene una cantidad de estados $O(1)$, entonces $G'$ tiene que tener tamaño $O(1)$ en su representación.

En la literatura consultada \cite{globalIndexLanguages,  propertiesRCGBib} todo formalismo que genera el lenguaje $L_{copy}$ tiene tamaño $O(1)$ en su representación. El lenguaje $L_{0,1,d}$ tiene características similares al lenguaje $L_{copy}$, razón que apoya la conjetura anterior.

En este capítulo se presentó una estrategia para resolver el SAT usando teoría de lenguajes que se basa en 
definir y construir el lenguaje de todas las fórmulas satisfacibles. Además se presentó un primer acercamiento 
para construir este lenguaje, mediante una transducción finita lo cual demuestra que el problema de la palabra 
para todos los formalismos que generen $L_{0,1,d}$ y sean cerrados bajo transducción finita, es NP-Duro.
En el próximo capítulo se argumenta que la estrategia presentada en este para construir $L_{S-SAT}$ no es la única,
porque se construye una RCG que reconoce el lenguaje $L_{S-SAT}$, y los lenguajes de concatenación de rango 
no son cerrados bajo transducción finita. 
