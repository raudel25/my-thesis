\chapter{Estrategia para la solución del SAT usando el problema de la palabra}
\label{chap:LSATFT}

En este capítulo se presenta un enfoque distinto a los expuestos en \cite{aCFSAT} y \cite{aSRCSAT}, que se basa 
en definir un lenguaje al cual pertenecen todos los problemas SAT que son satisfacibles y al cual se le 
denomina $L_{S-SAT}$. Una vez definido este lenguaje, para determinar si un SAT es satisfacible solo es 
necesario verificar si la cadena que lo representa pertenece a $L_{S-SAT}$.

Para definir el lenguaje $L_{S-SAT}$ primeramente es necesario buscar una transformación de una fórmula booleana
que pueda ser interpretada por un formalismo de la teoría de lenguajes, luego se necesita un mecanismo que sea capaz
de determinar si una cadena que representa la asignación de los valores de las variables es capaz de satisfacer dicha 
fórmula booleana.

En este capítulo se presenta una transformación de una fórmula booleana a una cadena. Posteriormente se define 
el lenguaje $L_{0,1}=\{wd\}^+$, donde $w\in\{0,1\}^*$, la cadena $w$ representa la asignación de valores para cada
variable. Entonces se define un transductor finito que toma una cadena del lenguaje $L_{0,1}$ y genera todas las 
fórmulas booleanas que son satisfacibles por dicha cadena, mediante este mecanismo se puede definir el lenguaje
$L_{S-SAT}$. Para finalizar se demuestra que para cualquier formalismo que genere $L_{0,1}$, sea cerrado bajo transducción
finita y su representación luego de la transducción sea $O(1)$, el problema de la palabra es NP-Duro.

A continuación se presenta una transformación de una fórmula booleana en una cadena.

\section{Transformación de una fórmula booleana a una cadena}

Dada una fórmula booleana $F$ en CNF se puede definir la siguiente estructura:
$$F=X_1 \wedge X_2 \wedge \ldots \wedge X_n$$
donde cada cláusula $X_i$ es una disyunción de literales:
$$X_i=L_{i1} \vee L_{i2} \vee \ldots \vee L_{im}$$
y cada literal $L_{ij}$ es una variable booleana o su negación. En cada cláusula $X_i$ las variables que aparecen en $F$, puede tener cada una 3 estados posibles: $a$ si la variable aparece positiva, $b$ si la variable aparece negada y $c$ si la variable no pertenece a ninguno de los literales de la cláusula.

Por ejemplo la siguiente fórmula booleana en $CNF$:

$$F=(x_1 \vee x_2) \wedge (\neg x_1 \vee x_2 \vee x_3) \wedge (x_1 \vee \neg x_2 \vee x_3)$$

para la primera cláusula $x_1$ aparece positiva ($a$), $x_2$ aparece positiva ($a$) y $x_3$ no aparece $(c)$, para la segunda
$x_1$ aparece negada ($b$), $x_2$ aparece positiva ($a$) y $x_3$ aparece positiva $(a)$ y para la tercera 
$x_1$ aparece positiva ($a$), $x_2$ aparece negada ($b$) y $x_3$ aparece positiva $(a)$.

A partir de la afirmación anterior, se puede definir una cadena de símbolos $w$ que representa a la cláusula $X_i$ sobre una secuencia de variables $v_1,v_2,\ldots,v_p$ de la siguiente manera:

\begin{itemize}
    \item $w$ cuenta con exactamente $p$ símbolos.
    \item Si la variable $v_j$ aparece positiva en $X_i$, entonces el $j$-ésimo símbolo es $a$.
    \item Si la variable $v_j$ aparece negada en $X_i$, entonces el $j$-ésimo símbolo es $b$.
    \item Si la variable $v_j$ no aparece en $X_i$, entonces el $j$-ésimo símbolo es $c$.
\end{itemize}
Si se toma la secuencia de variables correspondiente a $F$, y se le aplica el procedimiento anterior a cada cláusula
se obtiene una cadena de símbolos que representa a dicha cláusula en $F$.

Si ya se tiene una representación para cada cláusula de $F$ solo resta obtener una cadena de símbolos que represente a $F$,
esto se puede lograr concatenando las cadenas de símbolos de cada cláusula de $F$ en el orden que aparecen con un separador
en este caso se eligió el símbolo $d$.

Por ejemplo la siguiente fórmula booleana en \textit{CNF}:
$$F=(x_1 \vee x_2) \wedge (\neg x_1 \vee x_2 \vee x_3) \wedge (x_1 \vee \neg x_2 \vee x_3)$$
puede ser expresada como la cadena de símbolos:
$$w=aacdbaadabad$$
tomando como secuencia de variables $x_1, x_2, x_3$.

Entonces una vez definida la transformación de una fórmula booleana en una cadena se puede definir el lenguaje
de todas las fórmulas booleanas en CNF:
$$L_{FULL-SAT}=\{w\,|\,\exists F : t_{SAT}(F)=w\},$$
donde $F$ es una fórmula booleana en CNF y $t_{SAT}(x)$ es una función que recibe una fórmula booleana y devuelve
su representación mediante los estados $a$, $b$ y $c$ y el separador $d$.

En la próxima sección se define el lenguaje $L_{S-SAT}$.

\section{Definición de $L_{S-SAT}$}

El lenguaje de todas las fórmulas booleanas en CNF que son satisfacibles se define como $L_{S-SAT}=\{w\,|\,w \in L_{FULL-SAT} \wedge f_{SAT}(w)\}$, 
donde $L_{FULL-SAT}$ representa el lenguaje de todas las fórmulas booleanas en CNF y $f_{SAT}(w)$
es una función que determina si $w$ es satisfacible.

En la próxima sección se presenta la definición de $L_{S-SAT}$ usando transducción finita.

\subsection{Definición del $L_{S-SAT}$ usando transducción finita}

La idea para definir $L_{S-SAT}$ es construir un transductor finito, denominado $T_{SAT}$, que acepte como entrada cadenas del lenguaje $L_{0,1}$
y devuelva como salida cadenas que representan una fórmula booleana en \textit{CNF}, detrás de esta construcción se busca asociar cada carácter 0 ó 1 en la cadena de entrada al valor de la variable booleana correspondiente en la cadena de salida y verificar que para dichos valores al evaluar la fórmula booleana se obtenga un valor de verdad positivo.

Se define $L_{S-SAT}$ como el lenguaje de todas las transducciones $e$ que se obtienen del transductor $T_{SAT}$, a partir del lenguaje de cadenas de entrada $L_{0,1}=\{wd\}^+$ donde $w\in \{0,1\}^*$.

$$L_{S-SAT} = \{e\,|\,\exists w \in L_{0,1} \wedge e \in T_{SAT}(w) \}.$$

Aunque $L_{S-SAT}$ contiene todas las fórmulas booleanas satisfacibles, este conjunto por si solo no sirve de 
mucho sin un mecanismo que permita conocer si una cadena que representa una fórmula booleana pertenece al 
lenguaje o no. Una forma de obtener ese mecanismo es encontrar un formalismo que sea capaz de generar el 
lenguaje $L_{0,1}$ y que al aplicarle el transductor $T_{SAT}$ a dicho formalismo se obtenga otro que 
cuente con un algoritmo de reconocimiento para determinar si una cadena pertenece a este o no.

Encontrar un formalismo que genere el lenguaje $L_{0,1}$ ($G_{0,1}$), y que sea cerrado bajo transducción finita es 
suficiente para generar el lenguaje $L_{S-SAT}$. Una pregunta relacionada con $L_{S-SAT}$ sería saber si la 
existencia de $G_{0,1}$ es una condición necesaria para construir dicho lenguaje.  Otra cuestión a analizar
sería el problema de la palabra para $G_{0,1}$, observe que si este problema es polinomial con respecto al tamaño
de la fórmula booleana se estaría resolviendo el problema \textbf{P vs NP}.

Mediante la transformación $L_{0,1}$ por el transductor finito se mantiene la invariante del SAT que a dos 
instancias de la misma variable se les asocia el mismo valor de verdad, porque a las variables se
le asignan los mismos valores en todas las cláusulas.

Dada la transformación de $L_{0,1}$ por el transductor finito se puede demostrar que el problema de la 
palabra de cualquier formalismo que genere el lenguaje $L_{0,1}$, sea cerrado bajo transducción finita 
su representación luego de la transducción sea $O(1)$, es NP-Duro, ya que puede ser reducido al SAT y por tanto a cualquier problema en NP. 

En la siguiente sección se define el transductor $T_{SAT}$.

\section{Transductor $T_{SAT}$}

En esta sección se define el transductor finito $T_{SAT}$ (Figura \ref{fig:transducer_sat}) que sigue la construcción definida en la sección anterior.

Para ello se construye el transductor $T_{CLAUSE}$ (Figura \ref{fig:transducer_clause}) que hace el proceso de 
transducción para los valores de las variables de una cláusula $wd$, donde $w\in \{0,1\}$. La idea detrás de $T_{CLAUSE}$
es construir un transductor que genere todas las posibles cláusulas satisfacibles por los valores de las variables
que determina la cadena de entrada, este transductor tiene 3 estados: el estado inicial, el estado positivo (representa que la
cláusula generada ya es satisfacible) y el estado negativo (representa que la cláusula generada aún no es satisfacible).

A continuación se define $T_{CLAUSE}$:

\[
    T_{CLAUSE} = (Q, {\Sigma}, \Gamma, \delta, q_{0}, F),
\]
donde:
\begin{itemize}
    \item \(Q\) = ${q_0,q_p,q_n}$.
    \item \(\Sigma\) = ${0,1}$.
    \item \(\Gamma\) = ${a,b,c}$.
    \item \(\delta: Q \times \Sigma \to Q \times \Gamma^*\) función de transición.
    \item \(q_{0} = q_0\) estado inicial.
    \item \(F={q_p}\) conjunto de estados finales.
\end{itemize}
Se define la función de transición $\delta$ de la siguiente manera:

\begin{itemize}
    \item Transiciones para el estado $q_0$: representa el estado inicial. Si la entrada es un 1 el
          transductor puede escribir $a$, $b$ y $c$, si escribe $a$ pasa al estado positivo, si escribe $b$
          pasa al estado negativo y si escribe $c$ permanece en el mismo estado. Por otro lado si la entrada 
          es un 0 se intercambian los estados cuando se escribe $a$ y $b$, es decir cuando se escribe $a$
          pasa al estado negativo y cuando se escribe una $b$ pasa al estado positivo, mientras que cuando 
          se escribe $c$ permanece en el mismo estado.
          
          \begin{multicols}{2}
              \begin{itemize}
                  \item $\delta_{SAT}(q_0,1)=(q_p,a)$
                  \item $\delta_{SAT}(q_0,0)=(q_n,a)$
                  \item $\delta_{SAT}(q_0,1)=(q_n,b)$
                  \item $\delta_{SAT}(q_0,0)=(q_p,b)$
                  \item $\delta_{SAT}(q_0,1)=(q_0,c)$
                  \item $\delta_{SAT}(q_0,0)=(q_0,c)$
              \end{itemize}
          \end{multicols}
          
    \item Transiciones para el estado $q_p$ (estado positivo de $T_{CLAUSE}$): representa que para los valores
          asignados a las variables se obtiene un valor de verdad positivo.  Como la fórmula se encuentra ya en un 
          estado positivo lo que significa que al menos un literal se evaluó positivo no importa la 
          entrada y lo que el transductor escriba se mantiene en el mismo estado. Este es el estado de 
          aceptación para el transductor lo cual significa que la cláusula se evalúa con un valor de verdad positivo.
          \begin{multicols}{2}
              \begin{itemize}
                  \item $\delta_{SAT}(q_{p},1)=(q_{p},a)$
                  \item $\delta_{SAT}(q_{p},0)=(q_{p},a)$
                  \item $\delta_{SAT}(q_{p},1)=(q_{p},b)$
                  \item $\delta_{SAT}(q_{p},0)=(q_{p},b)$
                  \item $\delta_{SAT}(q_{p},1)=(q_{p},c)$
                  \item $\delta_{SAT}(q_{p},0)=(q_{p},c)$
              \end{itemize}
          \end{multicols}
          
    \item Transiciones para el estado $q_n$ (estado negativo de $T_{CLAUSE}$): representa que para los valores
          asignados a las variables se obtiene un valor de verdad negativo. Si la entrada es un 1 el 
          transductor puede escribir $a$, $b$ y $c$, si escribe $a$ pasa al estado positivo, si escribe $b$
          pasa al estado negativo y si escribe $c$ permanece en el mismo estado. Por otro lado si la entrada 
          es un 0 se intercambian los estados cuando se escribe $a$ y $b$, es decir cuando se escribe $a$
          pasa al estado negativo y cuando se escribe una $b$ pasa al estado positivo, mientras que cuando 
          se escribe $c$ permanece en el mismo estado.
          \begin{multicols}{2}
              \begin{itemize}
                  \item $\delta_{SAT}(q_{n},1)=(q_{p},a)$
                  \item $\delta_{SAT}(q_{n},0)=(q_{n},a)$
                  \item $\delta_{SAT}(q_{n},1)=(q_{n},b)$
                  \item $\delta_{SAT}(q_{n},0)=(q_{p},b)$
                  \item $\delta_{SAT}(q_{n},1)=(q_{n},c)$
                  \item $\delta_{SAT}(q_{n},0)=(q_{n},c)$
              \end{itemize}
          \end{multicols}
\end{itemize}

\begin{figure}[h]
    \centering  \begin{otherlanguage}{english}
        \begin{tikzpicture}[shorten >=1pt, node distance=3cm, on grid, auto]
            
            % Nodos
            \node[state, initial] (q0)   {$q_0$};
            \node[state] (qn) [above right=of q0] {$q_n$};
            \node[state, accepting] (qp) [below right=of q0] {$q_p$};
            
            % Transiciones
            \path[->]
            (q0) edge [bend left] node {0/a,1/b} (qn)
            (q0) edge [bend right] node {1/a,0/b} (qp)
            (q0) edge [loop right] node {0/c,1/c} (q0)
            
            (qn) edge [bend left] node {1/a,0/b} (qp)
            (qn) edge [loop above] node {0/a,1/b,0/c,1/c} (qn)
            
            (qp) edge [loop below] node {1/a,0/b,0/a,1/b,0/c,1/c} (qp);
            
        \end{tikzpicture}
    \end{otherlanguage}
    \caption{Transductor $T_{CLAUSE}$.}
    \label{fig:transducer_clause} % Esto es para referenciar la figura en el texto
\end{figure}

Entonces para definir $T_{SAT}$ mediante $T_{CALUSE}$, la idea es modificar $T_{CALUSE}$ para que cuando se genere
una cláusula y termine en el estado positivo se continue leyendo los valores de la cadena de entrada que son necesarios para generar
la próxima cláusula. Esto se logra definiendo $q_0$ como el estado de aceptación y agregando una transición del estado 
$q_p$ al estado $q_0$ que lea una $d$ y escriba una $d$, de esta manera cuando una cláusula se genera con un valor de 
verdad positivo se pasa a generar la siguiente cláusula desde el estado inicial. La construcción anterior tiene 
una dificultad y es que el transductor genera la cadena vacía, para solucionar esto se pueden tomar 2 instancias de 
$T_{CLAUSE}$ y unirlas mediante una transición, esta idea se expone a continuación.

Para definir el transductor $T_{SAT}$ (Figura \ref{fig:transducer_sat}) se toman dos instancias del transductor $T_{CLAUSE}$ ($T_1$ y $T_2$ respectivamente) y se concatenan añadiendo una transición del estado $q_{p_1}$ (estado positivo de $T_1$) al estado $q_{0_2}$ (estado inicial de $T_2$) con el símbolo $d$ (tanto de lectura como de escritura) y además se agrega una clausura a $T_2$ con una transición del estado $q_{p_2}$ (estado positivo de $T_2$) al estado $q_{0_2}$ con el símbolo $d$ (tanto de lectura como de escritura). Entonces solo resta definir el estado inicial y el estado final de $T_{SAT}$, los cuales serían $q_{0_1}$ (estado inicial de $T_1$) y $q_{0_2}$ (estado inicial de $T_2$), respectivamente.



\begin{figure}[h]
    \begin{otherlanguage}{english}
        \centering \begin{tikzpicture}[shorten >=1pt, node distance=3cm, on grid, auto]
            
            % Nodos
            \node[state, initial] (q01)   {$q_{0_1}$};
            \node[state] (qn1) [above right=of q01] {$q_{n_1}$};
            \node[state] (qp1) [below right=of q0] {$q_{p_1}$};
            \node[state, accepting] (q02) [right=6cm of q01] {$q_{0_2}$};
            \node[state] (qn2) [above right=of q02] {$q_{n_2}$};
            \node[state] (qp2) [below right=of q02] {$q_{p_2}$};
            
            
            % Transiciones
            \path[->]
            (q01) edge [bend left] node {0/a,1/b} (qn1)
            (q01) edge [bend right] node {1/a,0/b} (qp1)
            (q01) edge [loop right] node {0/c,1/c} (q01)
            
            (qn1) edge [bend left] node {1/a,0/b} (qp1)
            (qn1) edge [loop above] node {0/a,1/b,0/c,1/c} (qn1)
            
            (qp1) edge [loop below] node {1/a,0/b,0/a,1/b,0/c,1/c} (qp1)
            
            (q02) edge [bend left] node {0/a,1/b} (qn2)
            (q02) edge [bend right] node {1/a,0/b} (qp2)
            (q02) edge [loop right] node {0/c,1/c} (q02)
            
            (qn2) edge [bend left] node {1/a,0/b} (qp2)
            (qn2) edge [loop above] node {0/a,1/b,0/c,1/c} (qn2)
            
            (qp2) edge [loop below] node {1/a,0/b,0/a,1/b,0/c,1/c} (qp2)
            
            (qp1) edge [bend right] node {d/d} (q02)
            (qp2) edge [bend left=75] node {d/d} (q02);
            
        \end{tikzpicture}
    \end{otherlanguage}
    \caption{Transductor $T_{SAT}$.}
    \label{fig:transducer_sat} % Esto es para referenciar la figura en el texto
\end{figure}

A continuación se presenta la demostración de que le lenguaje $L_{S-SAT}$ definido mediante una transducción finita 
reconoce todas las fórmulas satisfacibles.

\subsection{Demostración de que $L_{S-SAT}$ definido mediante una transducción finita
    reconoce todas las fórmulas satisfacibles}

En esta sección se demuestra que el problema de la palabra de cualquier formalismo que genere el lenguaje 
$L_{0,1}$, sea cerrado bajo transducción finita y su representación luego de la transducción sea $O(1)$, es NP-Duro. Para ello se debe demostrar que el lenguaje 
$L_{S-SAT}$ definido mediante $T_{SAT}$ realmente es igual al lenguaje de todas las fórmulas booleanas satisfacibles.

La idea de la demostración es probar que para cualquier cadena binaria $w$, $T_{CLAUSE}(w)$ es el conjunto de todas las cláusulas que son satisfacibles por $w$.
Después demostrar que $T_{SAT}(e)$ contiene todas las fórmulas de $n$ cláusulas satisfacibles por la cadena $w$ donde $e=(wd)^n \wedge w\in\{0,1\}^*$ y por último demostrar que:
$$L_{S-SAT} = \{e\,|\,\exists w \in L_{0,1} \wedge e \in T_{SAT}(w) \}.$$

Para demostrar que dada una cadena binaria $w$, $T_{CLAUSE}(w)$ es el conjunto de todas las cláusulas que son 
satisfacibles por $w$ primero suponga que $q\in T_{CLAUSE}(w)$. Entonces en el proceso que generó $q$ el 
transductor terminó en el estado $q_p$ y como empezó en el estado $q_0$ ocurrió una transición desde $q_0$
a $q_p$ o desde $q_n$ a $q_p$ y esto solo es posible si el transductor leyó un 1 y escribió una a o si leyó un 
0 y escribió una b. Esto significa que en la fórmula booleana que representa $q$ hay una variable sin negar a 
la cual se le asigna un 1 o una variable negada a la cual se le asigna un 0, por lo tanto se cumple que $w$
satisface la fórmula booleana que representa $q$.

Sea una fórmula booleana $F$ satisfacible por $w$, representada por la cadena $q$, entonces se cumple que en 
$F$ hay una variable sin negar a la cual se le asigna un 1 (lo cual significa que la variable toma
un valor de verdad positivo) o una variable 
negada a la cual se le asigna un 0 (lo cual significa que la variable toma
un valor de verdad negativo). Sin pérdida de la generalidad se asume que la primera variable que cumple esto es la $i$-ésima.  
Entonces observe que si se hace el reconocimiento de los primeros $i-1$ caracteres por $T_{CLAUSE}$ se pueden 
seguir las transiciones de tal manera que los primeros $i-1$ caracteres de la cadena generada sean iguales a 
los primeros $i-1$ caracteres de $q$. Dado este punto solo es posible que el transductor esté en el estado 
$q_0$ o $q_n$, pero como se cumple que la $i$-ésima variable está sin negar y se le asigna un 1 o está negada 
y se le asigna un 0, entonces se puede tomar la opción de leer un 1 y escribir una a o leer un 0 y escribir 
una b según corresponda. De esta manera según en el estado en que se encuentre el autómata se pasa al estado $q_p$. 
Luego se toman las restantes transiciones de manera que la cadena generada sea $q$ y se mantiene en el mismo estado ya que $q_p$ solo tiene transiciones hacia sí mismo. De esta manera se demuestra que $T_{CLAUSE}(w)$ es el conjunto de todas las cláusulas que son satisfacibles por $w$.

Para demostrar que $T_{SAT}(e)$ contiene todas las fórmulas satisfacibles por la cadena $e=(wd)^n$ donde $w\in\{0,1\}^*$ se hará una inducción sobre $n$. Se define los conjuntos $A_{w,n}$ como el conjunto formado por todas las cadenas que representan fórmulas booleanas de $n$ cláusulas satisfacibles por $w$ y $B_w$ el conjunto formado por todas las cadenas que representan las cláusulas que son satisfacibles por $w$. El caso base $n=1$ se demuestra porque la transducción se realiza solo sobre el primer transductor $T_{CLAUSE}$ que conforma a $T_{SAT}$ y como se demostró anteriormente $T_{CLAUSE}(w)$ contiene todas cláusulas satisfacibles por $w$.  Se supone para $n=k$ y se demuestra para $n=k+1$, se cumple que $A_{w,n}=\{xyd\,|\,x\in A_{w,n-1} \wedge y\in B_w\}$ y $T_{SAT}((wd)^n)=\{xyd\,|\,x\in T_{SAT}((wd)^{n-1}) \wedge y\in T_{CLAUSE}(w)\}$. Por hipótesis de inducción se cumple que $A_{w,n-1}=T_{SAT}((wd)^{n-1})$ y además se cumple que $B_w=T_{CLAUSE}(w)$, lo cual implica que $A_{w,n}=T_{SAT}((wd)^n)$.

Para concluir el tercer paso de la demostración sea $e$ tal que existe $w$, donde $w \in L_{0,1} \wedge e \in T_{SAT}(w)$ entonces se cumple que la fórmula booleana asociada a $e$ es satisfacible por $w$. Ahora sea $F$ una fórmula booleana satisfacible con $n$ cláusulas y sea $e$ su cadena asociada, por tanto existe $w$ tal que $w$ satisface a $F$, luego se cumple que $e\in T_{SAT}((wd)^n)$. Por tanto se cumple que:
$$L_{S-SAT} = \{e\,|\,\exists w \in L_{0,1} \wedge e \in T_{SAT}(w) \}.$$

Para demostrar que el problema de la palabra de cualquier formalismo que genere el lenguaje $L_{0,1}$, sea 
cerrado bajo transducción finita y su representación luego de la transducción sea $O(1)$, es NP-Duro suponga 
que existe un formalismo $G$ que genera $L_{0,1}$ y es cerrado bajo transducción finita.

En este capítulo se presentó una nueva estrategia para resolver el SAT usando teoría de lenguajes, la cual se basa en
definir el lenguaje de todas las fórmulas satisfacibles. Además se presentó un primer acercamiento para definir $L_{S-SAT}$, mediante
transducción finita lo cual demuestra que  el problema de la palabra para todos los formalismos que generen $L_{0,1}$ y sean cerrados bajo transducción
finita, es NP-duro. En el próximo capítulo se presenta otra estrategia distinta para definir $L_{S-SAT}$, la cual se basa
en definir una RCG que reconozca las fórmulas booleanas satisfacibles.
