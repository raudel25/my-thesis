\chapter*{Introducción}\label{chapter:introduction}
\addcontentsline{toc}{chapter}{Introducción}

El problema de la satisfacibilidad booleana (\textit{SAT}) surge en 1971 como el primer problema NP-completo demostrado por
Stephen Cook y sirve como base para el resto de los problemas de esta clase. En la actualidad se desconoce una solución eficiente
para todos los problemas de la clase NP-completo que pueda ser interpretada por un modelo de cómputo,
de ahí su especial importancia para las Ciencias la Computación.

Por otra parte, la teoría de lenguajes sirve de base a todos los problemas planteados en Ciencias de la Computación y muchos
de sus conceptos y algoritmos se emplean directamente en los lenguajes de programación, los cuales sirven actualmente
para modelar los algoritmos y modelos que se utilizan para resolver diversos problemas de la vida real. La teoría de lenguajes comenzó
en el siglo XX con los trabajos de figuras como Noam Chomsky, quien introdujo una jerarquía para clasificar estos lenguajes.

En la presente investigación se vinculan las dos ramas de la computación descritas anteriormente, presentando un enfoque para resolver el SAT
utilizando formalismos de teoría de lenguajes. Dicho enfoque resulta un tema novedoso no evidenciado en la literatura consultada
y permite demostrar que una serie de problemas relacionados a la teoría de lenguajes pertenecen a la clase NP-completo. Por otro lado
se presenta una solución a todas las instancias de SAT polinomiales consultadas en la literatura usando un formalismo de teoría de
lenguajes.

En trabajos anteriores que siguen la idea presentada en esta investigación se han mostrado estrategias para la solución
de instancias específicas del SAT usando formalismos de teoría de lenguajes, lo que constituye una solución limitada en su alcance,
aquí en cambio se presenta una alternativa que resuelve cualquier instancia del mismo, lo que constituye una solución
cualitativamente superior empleando gramáticas de concatenación de rango. Esta sigue siendo una estrategia no eficiente pero que
muestra un nuevo enfoque para resolver el SAT de forma general que puede abrir nuevas líneas de investigación en este tema.

Lo anterior permite la formulación del siguiente problema: Como contribuir a alcance de un nuevo enfoque en la solución general del SAT.

La hipótesis que constituyó la guía orientadora para la realización de la investigación quedó enunciada como sigue: Si se diseña un nuevo algoritmo que emplee gramáticas de concatenación de rango se aportará un nuevo enfoque a la solución del
SAT.

El objetivo general a cumplir en la investigación es diseñar un algoritmo que emplee gramáticas de concatenación de rango para la contribución
de un nuevo enfoque en la solución del SAT.

Dado que la búsqueda
de un algoritmo eficiente para este problema constituye un gran reto, este trabajo solo centra en buscar un algoritmo con igual
complejidad computacional que el resto de algoritmos conocidos para este problema y tratar de llevar esta idea a todas las instancias
del SAT que se conoce que pueden ser resueltas eficientemente.

Para cumplir el objetivo general se definen los siguientes objetivos específicos:

\begin{itemize}
      \item Estudiar el estado del arte referido a los formalismos de teoría de lenguajes y el SAT.
      \item Establecer una representación del SAT como una cadena que pueda ser interpretada por un formalismo de la teoría de lenguajes.
      \item Definir el lenguaje de todas las instancias de SAT satisfacibles.
      \item Describir el lenguaje generado mediante una gramática de concatenación de rango.
      \item Solucionar a las instancias de SAT polinomiales consultadas en la literatura usando el método descrito anteriormente.
\end{itemize}

El presente trabajo quedo estructurado en 5 capítulos. Los 2 primeros capítulos contienen los conceptos y definiciones
necesarios, en el primero se describe todo lo relacionado con los elementos básicos
de la teoría de lenguajes y el SAT y en el segundo se presentan los formalismos de escritura regulada que se utilizarán en el resto
del documento.

Un tercer momento consta de una recapitulación de los trabajos anteriores relacionados con este tema, su propuesta
de solución, acompañado de las restricciones de las instancias del SAT que se asumieron en cada caso.

En los próximos 2 capítulos se presenta el aporte del presente trabajo relacionado con el área del SAT y la teoría de lenguajes.
En el cuarto se le da respuesta al segundo, tercer y cuarto objetivo, presentando el enfoque para la representación de una instancia
del SAT como un elemento de la teoría de lenguajes, la formulación de un lenguaje que permita reconocer todas las instancias
de SAT satisfacibles y proponer formalismos que permitan describir dicho lenguaje.
En el quinto se le da respuesta al último objetivo presentando la solución a las instancias del SAT polinomiales reportadas en la literatura
usando el enfoque presentado en los capítulos anteriores.

