\chapter*{Introducción}

El problema de satisfacibilidad booleana (\textit{SAT}) \cite{authomataTheory} es uno de los problemas más estudiados en 
la teoría de la computación y la lógica \cite{biere2021handbook}.  Consiste en determinar si existe una asignación 
de valores verdaderos o falsos que satisfaga una fórmula booleana dada, compuesta por variables y operadores 
lógicos como conjunciones, disyunciones y negaciones. SAT surge en 1971 como el primer problema NP-Completo 
demostrado por Stephen Cook \cite{Cook1971}, lo que significa que, en el peor de los casos, su resolución 
requiere tiempo exponencial respecto al tamaño de la entrada, pero también que muchos otros problemas pueden 
reducirse a él. Esto implica un especial interés por parte de la comunidad científica en la búsqueda de 
métodos eficientes para la solución del SAT \cite{biere2021handbook}.

La teoría de lenguajes es una rama fundamental de la Ciencia de la Computación y la matemática que se enfoca 
en el estudio de los lenguajes formales \cite{authomataTheory}. Estos lenguajes, definidos a través de 
gramáticas, autómatas y expresiones regulares, permiten modelar y analizar la estructura de los lenguajes 
naturales y artificiales. Su aplicación es amplia y abarca desde el diseño de compiladores y procesadores de 
lenguaje natural hasta la verificación de sistemas y la teoría de la computabilidad \cite{authomataTheory}.

Los lenguajes formales se clasifican en jerarquías, como la jerarquía de Chomsky \cite{hunter2020chomsky}, 
que los organiza según su complejidad y poder expresivo. Esta teoría proporciona las bases para 
entender cómo se pueden reconocer, generar y transformar cadenas de símbolos, lo que resulta esencial 
en el desarrollo de herramientas computacionales para el procesamiento de información. Además, 
la teoría de lenguajes constituye la base de los problemas de la Ciencia de la Computación, ya que 
cualquier problema puede ser interpretado como un problema de la teoría de lenguajes \cite{authomataTheory}.

En este trabajo se vinculan las dos ramas de la computación descritas anteriormente, presentando 
un enfoque para resolver el SAT utilizando formalismos de teoría de lenguajes. Dicho enfoque resulta un 
tema no evidenciado en la literatura consultada y permite demostrar que una serie de problemas 
relacionados a la teoría de lenguajes pertenecen a la clase NP-Completo.

En estudios anteriores realizados en la facultad de Matemática y Computación de la Universidad de La Habana \cite{aCFSAT,aSRCSAT}, 
se propone resolver variantes específicas del SAT utilizando el problema del vacío de varios tipos de 
gramáticas. En cambio, en este trabajo se propone resolver variantes del SAT mediante el problema de la 
palabra. 

Para ello se propone una codificación de una fórmula booleana cualquiera en forma normal conjuntiva usando 
cadenas sobre el alfabeto $\{a,b,c,d\}$ y usando dicha codificación se define el lenguaje de todas las fórmulas 
booleanas satisfacibles. Entonces si se desea determinar si una fórmula booleana es satisfacible 
solo hay que determinar si la cadena asociada a la fórmula booleana pertenece o no a este lenguaje.

Las gramáticas de concatenación de rango (RCG) \cite{mainRCGBib} son un formalismo de
gramáticas desarrollado en 1988 como una propuesta de Pierre Boullier, su objetivo principal era
proporcionar un modelo más general y expresivo que las gramáticas libres del contexto para describir lenguajes.

Un transductor finito es un autómata finito que, además de reconocer cadenas de entrada, produce una salida asociada a cada transición \cite{finite_transducer}.

Además de definir el lenguaje de todas las fórmulas booleanas satisfacibles, se proponen dos vías para 
construirlo. La primera utiliza un transductor finito y la segunda utiliza una gramática de concatenación de rango.

El objetivo general de este trabajo es definir y construir el lenguaje de todas las fórmulas booleanas satisfacibles.

Para cumplir el objetivo general se definen los siguientes objetivos específicos:

\begin{itemize}
      \item Estudiar el estado del arte referido a los formalismos de teoría de lenguajes y el SAT.
      \item Establecer una representación de cualquier SAT como una cadena que pueda ser interpretada por un formalismo de la teoría de lenguajes.
      \item Definir el lenguaje de todas las fórmulas booleanas satisfacibles.
      \item Construir el lenguaje de todas las fórmulas booleanas satisfacibles utilizando una transducción finita de una variante del lenguaje $Copy$.
      \item Construir el lenguaje de todas las fórmulas booleanas satisfacibles  utilizando gramáticas de concatenación de rango y sin usar la transducción finita.
      \item Analizar las implicaciones computacionales.
\end{itemize}

Las implicaciones computacionales del último punto son: que para todo problema en la clase NP, existe una gramática
de concatenación de rango que lo reconoce como lenguaje formal, pero el algoritmo de reconocimiento es no 
polinomial, lo cual a efectos prácticos no constituye una mejora para la solución de los problemas.

Este trabajo se ha estructurado en 4 capítulos: en los 2 primeros se presentan los principales conceptos y 
definiciones que serán utilizados en el resto de la investigación y en los dos últimos se define y construye 
el lenguaje de todas las fórmulas booleanas satisfacibles, y se analizan algunas de sus propiedades.

En el capítulo \ref{chap:preliminaries} se presentan los conceptos y definiciones de la teoría de lenguajes y 
el SAT, que serán usados los restantes capítulos. Además, se realiza un análisis de 2 trabajos anteriores 
que muestran cómo solucionar instancias específicas del SAT utilizando el problema del vacío para gramáticas 
libres del contexto y para gramáticas de concatenación de rango simple.

En el capítulo \ref{chap:RCG} se presentan las gramáticas de concatenación de rango: las principales definiciones, 
su proceso de derivación y un análisis de la complejidad del algoritmo de reconocimiento.

En el capítulo \ref{chap:LSATFT} se muestra cómo codificar una fórmula booleana mediante una cadena de símbolos 
y luego se analiza cómo interpretar una cadena como la asignación de valores para las variables de una fórmula 
booleana.  Posteriormente, se define el lenguaje de todas las fórmulas booleanas satisfacibles y se muestra 
cómo construir dicho lenguaje mediante un transductor finito. Para finalizar, se demuestra que el problema de 
la palabra para todos los formalismo que generen una variante del lenguaje \textit{Copy} y sean cerrados bajo
transducción finita, es NP-Duro.

En el capítulo \ref{chap:LSATRCG} se demuestra que no es necesario construir el lenguaje de todas las 
fórmulas booleanas satisfacibles mediante transducción finita, ya que existe una gramática de concatenación 
de rango que reconoce este lenguaje. Por otro lado, se demuestra que las gramáticas de concatenación de rango 
reconocen todos los problemas de la clase NP-Completo.