% \chapter{Preliminares}
% \label{chap:preliminaries}

\chapter{Gramáticas de concatenación de rango}
\label{chap:RCG}

% \chapter{Lenguaje de las fórmulas booleanas satisfacibles empleando transducción finita}
% \label{chap:LSATFT}

% \chapter{Lenguaje de las fórmulas booleanas satisfacibles empleando gramáticas de concatenación de rango}
% \label{chap:LSATRCG}


Las gramáticas de concatenación de rango (\textit{RCG}) \cite{mainRCGBib} son un formalismo de gramáticas desarrollado 
en 1988 como una propuesta de Pierre Boullier, un investigador en el campo de la lingüística computacional. Su objetivo 
principal era proporcionar un modelo más general y expresivo que las gramáticas libres del contexto para describir 
lenguajes.  Las RCG fueron diseñadas con el fin de analizar propiedades y características del lenguaje natural, como los números
chinos y el orden aleatorio de algunas palabras alemanas \cite{boullier1999chinese}.

Las gramáticas de concatenación de rango se emplean en el capítulo \ref{chap:LSATRCG} para construir una gramática que reconozca las fórmulas booleanas satisfacibles.

En la próxima sección se presentan algunas nociones que sirven de introducción para las principales definiciones y conceptos de las gramáticas de concatenación de rango.

\section{Presentación de los elementos de las gramáticas de concatenación de rango}

En esta sección se presentan nociones sobre la sustitución de rango y las derivaciones de las RCG, aspectos 
que sirven de base para los conceptos y definiciones relacionados con las gramáticas de concatenación de rango.

A los no terminales de esta gramática se les llama predicados y cada predicado tiene un conjunto de argumentos. 
A la cantidad de argumentos de un predicado se le denomina aridad.

Por ejemplo, $A(X,Y)$ representa el no terminal (predicado) $A$, que tiene como argumentos $X$ e $Y$. En este caso, la aridad de $A$ es 2.

Cada argumento de los predicados puede estar formado por variables y terminales. En el caso anterior, $X$ e $Y$ son variables.

Por convenio, las variables se denotan por letras mayúsculas del final del alfabeto, y a los terminales, como es 
usual, con letras minúsculas. Con este convenio, el siguiente predicado: $B(aX, XY, abZ)$, tiene aridad 3. Su 
primer argumento está formado por el terminal $a$ y la variable $X$. El segundo argumento por la concatenación 
de las variables $X$ e $Y$. El tercer argumento está formado por la concatenación de los terminales $a$, $b$ y 
la variable $Z$. 

Cada predicado reconoce un vector de cadenas que tiene como dimensión la aridad del predicado y cada cadena del vector se asocia a un argumento del predicado.

Por ejemplo, si al predicado $A(X,Y)$ se le asocia el vector $[abc,d]$, al primer argumento de $A$ se le asigna
la cadena $abc$ y al segundo la cadena $d$. Esto significa que $X=abc$ y que $Y=d$.

Por otro lado, si al predicado $B(aX, XY, abZ)$ se le asigna el vector $[aa,de,abcc]$,
al primer argumento de $B$ se le asigna la cadena $aa$, al segundo $de$ y al tercero $abcc$. Es decir, $aX=aa$, $XY=de$, y $abZ=abcc$.

El predicado $B(aX, XY, abZ)$ se puede usar para ilustrar como funciona la asignación de valores a las variables.
Por ejemplo, como $X$ es una variable y se tiene que $aX=aa$, entonces el único valor que puede tomar $X$ es $X=a$. 
Algo similar ocurre en el caso del tercer elemento del vector: como $abZ=abcc$, para que se cumpla esa igualdad, 
el único valor posible para Z es $Z=cc$. En el caso del segundo argumento, las variables $X$ e $Y$ deben tomar valores 
de forma que se cumpla la igualdad $XY=de$, y eso se puede hacer de tres formas: $X=d$, $Y=e$; $X=de$, $Y=\varepsilon$; 
y $X=\varepsilon$, $Y=de$. 

Por otro lado, si el predicado $B(aX, XY, abZ)$ recibe el vector $[bb,de,abcc]$, entonces cuando se tiene que $aX=bb$,
$X$ no puede tomar ningún valor porque el no terminal de $b$ no coincide con el no terminal $a$ que se encuentra como 
prefijo del argumento $aX$.

A las producciones de esta gramática se les denomina cláusulas. La parte izquierda de la producción 
siempre está formada por un único no terminal y en la parte derecha, puede existir cualquier cantidad de no terminales 
con sus respectivos argumentos, o la cadena vacía. Un ejemplo de cláusula puede ser la siguiente:
$$A(XYZ,W)\to B(X)C(XY,Z)D(W).$$

La regla anterior tiene el siguiente significado: el no terminal $A$ recibe un vector de dimensión 2. A partir 
de las cadenas del vector, se le asigna valores a las variables $X$, $Y$, $Z$ y $W$, y con esos valores 
construye los vectores que recibirán los no terminales $B$, $C$ y $D$.

Este proceso se puede ilustrar con un ejemplo en el que el no terminal $A$ reciba el vector de cadenas $[abc, d]$.

El primer paso es asignar las cadenas del vector a los argumentos del no terminal. El primer argumento de $A$ sería $abc$ y el segundo, $d$.

Como el primer argumento de $A$ está definido como $XYZ$ y el segundo como $W$, debe cumplirse que $XYZ=abc$ y $W=d$. Esto significa que las variables $X$, $Y$ y $Z$ deben tomar todos los posibles valores de forma que su concatenación sea $abc$, y que $W$ solo puede tomar el valor $d$.

En la Figura \ref{fig:xyz_eaxmple} se muestran los posibles valores que pueden tomar las variables $X$, $Y$, $Z$ y $W$ a partir del vector de entrada. Para seguir con el ejemplo, suponga que los valores de $X$, $Y$ y $Z$ son $X=ab$, $Y=\varepsilon$ y $Z=c$.


\begin{figure}
    \centering
    \begin{tabular}{|c|c|c|c|}
        \hline
        X             & Y             & Z             & W \\
        \hline
        a             & b             & c             & d \\
        \hline
        ab            & $\varepsilon$ & c             & d \\
        \hline
        ab            & c             & $\varepsilon$ & d \\
        \hline
        abc           & $\varepsilon$ & $\varepsilon$ & d \\
        \hline
        $\varepsilon$ & ab            & c             & d \\
        \hline
        $\varepsilon$ & abc           & $\varepsilon$ & d \\
        \hline
        $\varepsilon$ & $\varepsilon$ & abc           & d \\
        \hline
        a             & $\varepsilon$ & bc            & d \\
        \hline
        $\vdots$      & \vdots        & \vdots        & d \\
        \hline
        a             & bc            & $\varepsilon$ & d \\
        \hline
    \end{tabular}
    \caption{Posibles valores de las variables $X$, $Y$, $Z$ y $W$}
    \label{fig:xyz_eaxmple}
\end{figure}

A partir de esta  asignación se construyen los vectores con los que se evalúan los no terminales 
de su parte derecha: $B(X)C(Y,ZX)D(W)$. Como $X=ab$, $Y=\varepsilon$, $Z=c$ y $W=d$, la parte derecha se evaluaría con los vectores: $[ab]$, $[\varepsilon, cab]$, $[d]$ y como resultado se tiene la derivación $B(ab)C(\varepsilon,cab)D(d)$.

El proceso de asignar los vectores a los argumentos, identificar los valores de las variables e instanciar las partes derechas se repite en cada uno de los predicados del lado derecho de la cláusula.

Existe otro tipo de producciones en las que, para determinados valores de sus argumentos, un no terminal deriva en $\varepsilon$. Por ejemplo: 
$$B(ab)\to \varepsilon,$$
$$C(\varepsilon,cab)\to \varepsilon,$$
$$D(d)\to \varepsilon.$$

La forma general de estas producciones es $A(X_1,\ldots, X_n)\rightarrow \varepsilon$.

Un no terminal reconoce un vector de cadenas si existe una asignación de las variables 
en sus argumentos que en algún momento derivan en la cadena vacía. En el caso de $B(ab)\to \varepsilon$, 
el no terminal $B$ reconoce la cadena $ab$.

Si se tiene un conjunto de producciones de la forma:
\begin{enumerate}
    \item $A(XYZ,W)\to B(X)C(XY,Z)D(W)$
    \item $B(ab)\to \varepsilon$
    \item $C(\varepsilon,cab)\to \varepsilon$
    \item $D(d)\to \varepsilon$
\end{enumerate}
Entonces el no terminal A reconoce el vector $[abc,d]$, porque con la asignación de valores $X=ab$, $Y=\varepsilon$, $Z=c$ y $W=d$, $A$ derivaría en $B(ab)C(\varepsilon,cab)D(d)$,
y cada uno de los predicados $B(ab)$, $C(\varepsilon,cab)$ y $D(d)$ deriva en la cadena vacía.

Un predicado no reconoce un vector de cadenas cuando para ninguna asignación de variables se deriva en la cadena vacía.

Dadas estas nociones, a continuación se presentan las principales definiciones de las gramáticas de concatenación de rango.

\section{Definiciones}

En esta sección se define el concepto de rango, sustitución de rango, gramática de concatenación de rango \footnote{En la literatura este tipo de RCG se toma como gramática de concatenación de rango positiva, pero como es la única que
    se usa en este trabajo se le llama solo gramática de concatenación de rango.} y gramática de concatenación de rango simple.

\begin{definition}
    Un \textbf{rango} es una tupla $(i, j)$ que representa un intervalo de posiciones en una cadena, donde $i$ y $j$ son enteros no negativos tales que $i \leq j$.    
\end{definition}

Por ejemplo, si los índices son indexados en 0, para la cadena $abcd$, el rango $(1,2)$, representa la subcadena $bc$.

\begin{definition}
    Una \textbf{gramática de concatenación de rango} se define como una 5-tupla:
    \[
        G = (N, T, V, P, S),
    \]
    donde:
    
    \begin{itemize}
        \item $N$: Es un conjunto finito de \textbf{predicados o símbolos no terminales}: Cada predicado tiene una \textbf{aridad}, que indica la dimensión del vector de cadenas que reconoce y cada cadena del vector se asocia a un argumento del predicado.
        \item $T$: Es un conjunto finito de \textbf{símbolos terminales}.
        \item $V$: Es un conjunto finito de \textbf{variables}.
        \item $P$: Es un conjunto finito de \textbf{cláusulas}, de la forma:
              \[
                  A(x_1, x_2, \ldots, x_k) \to B_1(y_{1,1}, y_{1,2}, \ldots, y_{1,m_1}) \ldots B_n(y_{n,1}, y_{n,2}, \ldots, y_{n,m_n}),
              \]
              donde $A, B_i \in N$, $x_i, y_{i,j} \in (V \cup T)^*$, y $k$ es la aridad de $A$.
        \item $S \in N$: Es el \textbf{predicado inicial} de la gramática, que siempre tiene \textbf{aridad} 1.
    \end{itemize}
\end{definition}


Por ejemplo, a continuación se muestra una gramática de concatenación de rango:
\label{g_3copy}
\[
    G^3_{copy} = (N, T, V, P, S),
\]
donde:

\begin{itemize}
    \item  N=$\{A,S\}$.
    \item T=$\{a,b,c\}$.
    \item V=$\{X,Y,Z\}$.
    \item El conjunto de cláusulas $P$ es el siguiente:
          \begin{enumerate}
              \item $S(XYZ)\to A(X,Y,Z)$
              \item $A(aX,aY,aZ)\to A(X,Y,Z)$
              \item $A(bX,bY,bZ)\to A(X,Y,Z)$
              \item $A(cX,cY,cZ)\to A(X,Y,Z)$
              \item $A(\varepsilon,\varepsilon,\varepsilon)\to \varepsilon$
          \end{enumerate}
    \item El símbolo inicial es $S$.
\end{itemize}


Las RCG, a diferencia de las gramáticas definidas en la sección \ref{sec:grammars} del capítulo \ref{chap:preliminaries} no generan cadenas, su funcionamiento se basa en reconocer si una cadena pertenece o no al lenguaje.


\begin{definition}
    Una \textbf{sustitución de rango} es un mecanismo que reemplaza una variable por un 
    rango de la cadena, respetando la estructura del argumento que se asocia a la cadena que se reconoce. 
\end{definition}

Por ejemplo, dado el predicado $A(Xa)$ donde $X \in V$ y $a \in T$, la estructura del argumento de $A$ es una variable $X$ seguida del terminal $a$. Si el no terminal $A$ recibe la cadena $baa$, $X$ se puede asociar con el rango $ba$ de la cadena original porque si $X=ba$, entonces $Xa=baa$.

Por otro lado, la variable $X$ no puede tomar el valor $baa$, porque ningún caracter de la cadena de entrada coincidiría 
con el terminal $a$. De manera similar, $X$ tampoco puede tomar el valor $b$ porque el valor que se asigna a $X$ no permite que el argumento $Xa$ cubra la cadena completa.

\begin{definition}
    Las \textbf{gramáticas de concatenación de rango simple}
    (\textit{SRCG}) son un subconjunto de las RCG que restringen la forma de las cláusulas de producción.  
    Una RCG $G$ es \textbf{simple} si los argumentos en el lado derecho de una cláusula son variables distintas, 
    y todas estas variables (y no otras) aparecen una sola vez en los argumentos del lado izquierdo.  
\end{definition}

Este es un caso particular de las RCG el cual se usa en \cite{aSRCSAT} para describir el orden de las variables de una fórmula booleana. 

En la próxima sección se describe el proceso de derivación de las RCG.

\section{Proceso de derivación}

La idea principal para realizar una derivación en la cláusula 
\[
    A(x_1, x_2, \ldots, x_k) \to B_1(y_{1,1}, y_{1,2}, \ldots, y_{1,m_1}) \ldots B_n(y_{n,1}, y_{n,2}, \ldots, y_{n,m_n}),
\]
de una RCG, se basa en tomar el vector de cadenas $[w_1, w_2,\ldots, w_k]$ que recibe el predicado $A$ y asociar cada elemento del vector al argumento 
correspondiente: $w_1$ se asocia al argumento $x_1$, $w_2$ se asocia al argumento $x_2$ y así hasta que 
$w_k$ se asocia a $x_k$. 

Después de asociar los elementos del vector a los argumentos, se realizan todas las posibles sustituciones de rango para cada argumento y se asocia un rango a 
cada variable del predicado izquierdo.

A partir de los valores de las variables obtenidos en el paso anterior se construyen vectores de cadenas con los que se 
instancian las variables de los predicados del lado derecho de la cláusula.

A modo de ejemplo se puede considerar la producción $A(X,aYb)\to B(aXb,Y)$, donde $X$ e $Y$ son variables y $a$ y $b$ son símbolos terminales.

Cuando $A$ recibe el vector $[a,abb]$, el primer argumento de $A$ recibe $a$ y el segundo recibe $abb$.  El primer argumento de $A$ es $X$ y el segundo es $aYb$, por lo que $X=a$ y $aYb=abb$. En este caso la única sustitución de rango posible es $X=a$ y $Y=b$. Con estos valores se construyen los vectores con los que se instancia la parte derecha, que serían $aXb$ = $aab$ y $Y$=b. Con este vector el predicado $B$ se instancia como $B(aab,b)$, y por tanto, el predicado $A(a,abb)$ deriva como $B(aab,b)$.


Un vector de cadenas se reconoce por un predicado $A$ si existe una secuencia de derivaciones que comienza en $A$ y termina en la cadena vacía.

Por ejemplo, dada la cláusula $A(X_1,X_2,X_3)\to B_1(X_1)B_2(X_2)B_3(X_3)$, el vector $[w_1,w_2,w_3]$ se reconoce por $A$, si existe una secuencia de derivaciones para cada uno de los predicados $B_1(w_1)$, $B_2(w_2)$, $B_3(w_3)$ que derive en la cadena vacía.

A continuación se presenta un ejemplo de reconocimiento de la cadena $abcabcabc$ por la gramática $G^3_{copy}$
presentada en la página \pageref{g_3copy}.

La cadena $abcabcabc$ se reconoce por $G^3_{copy}$, ya que $S(abcabcabc)$ se puede derivar de la siguiente manera:
$$S(abcabcabc)\to A(abc,abc,abc)\to A(bc,bc,bc)\to A(c,c,c)\to A(\varepsilon,\varepsilon,\varepsilon)\to \varepsilon.$$

A continuación se muestran estas derivaciones paso a paso.

\begin{itemize}
    \item \textbf{Primer paso:} En la primera cláusula $S(XYZ)\to A(X,Y,Z)$, existe una sustitución de rango que asocia las variables $X$, $Y$, $Z$ a los valores $X=abc$, $Y=abc$ y $Z=abc$. De esta forma se deriva en el predicado $A(abc,abc,abc)$.
    \item \textbf{Segundo paso:} En la segunda cláusula $A(aX,aY,aZ)\to A(X,Y,Z)$, existe una sustitución de rango que asocia las
          variables $X$, $Y$, $Z$ a los valores $X=bc$, $Y=bc$ y $Z=bc$. Con estos valores se deriva en el predicado $A(bc,bc,bc)$.
    \item \textbf{Tercer paso:} En la tercera cláusula $A(bX,bY,bZ)\to A(X,Y,Z)$, existe una sustitución de rango que asocia las
          variables $X$, $Y$, $Z$ a los valores $X=c$, $Y=c$ y $Z=c$. Con estos valores se deriva en el predicado $A(c,c,c)$.
    \item \textbf{Cuarto paso:} En la cuarta cláusula $A(cX,cY,cZ)\to A(\varepsilon,\varepsilon,\varepsilon)$, existe una sustitución de rango que asocia las variables $X$, $Y$, $Z$ a los valores $X=\varepsilon$, $Y=\varepsilon$ y $Z=\varepsilon$. Con estos valores se deriva en el predicado $A(\varepsilon,\varepsilon,\varepsilon)$.
    \item \textbf{Quinto paso:} Finalmente en el último paso se toma la última cláusula $A(\varepsilon,\varepsilon,\varepsilon)\to \varepsilon$ que deriva en la cadena vacía, por lo que de esta manera se reconoce la cadena $abcabcabc$.
\end{itemize}


A continuación se presentan algunas propiedades de las RCG relevantes para este trabajo.

\section{Propiedades de las RCG}

La motivación fundamental detrás de la creación de las RCG fue crear un formalismo modular. Esto significa que las principales operaciones sobre conjuntos: unión, intersección y complemento son cerradas para dicho formalismo \cite{mainRCGBib}.  Esta es precisamente la limitación de las CFG que se propone suplir con las RCG para el procesamiento del lenguaje natural \cite{mainRCGBib}.

En esta sección se describen las principales propiedades que demuestran que las RCG son un formalismo
modular, además de las propiedades que demuestran que las RCG no son cerradas 
bajo transducción finita. En esta sección también se presenta el problema de la palabra 
para las RCG, el cual se emplea en el capítulo \ref{chap:LSATRCG} para determinar si una fórmula booleana es satisfacible.

\begin{theorem}
    \label{teo:RCGset}
    Las RCG \textbf{son cerradas bajo unión, intersección y complemento}, la unión y la intersección de 2 RCG da como resultado un formalismo que pertenece a las RCG, mientras
    que el formalismo resultante del complemento de una RCG es también una RCG.
\end{theorem}

La demostración del Teorema \ref{teo:RCGset} se realiza en \cite{mainRCGBib}.

\begin{theorem}
    \label{teo:RCGh}
    Las RCG \textbf{no son cerradas bajo homomorfismo}, dada una RCG $G$, el homomorfismo de un lenguaje que se reconoce por $G$ necesariamente no se reconoce por una RCG.    
\end{theorem}

La demostración del Teorema \ref{teo:RCGh} se realiza en \cite{propertiesRCGBib}.

\begin{theorem}
    \label{teo:RCGt}
    Las RCG \textbf{no son cerradas bajo transducción finita}, dada una RCG $G$, la transducción finita de un lenguaje que se reconoce por $G$ necesariamente no se reconoce por una RCG.    
\end{theorem}

La demostración del Teorema \ref{teo:RCGt} es una consecuencia del Teorema \ref{teo:RCGh} porque un homomorfismo es un transductor finito de un solo estado y tantas transiciones hacia el mismo estado como transformaciones de símbolos en el homomorfismo.

En este trabajo se propone una forma para construir el lenguaje de todos los SAT satisfacibles usando un formalismo que sea capaz de generar una variante de $L_{copy}$, que se llamará $L_{01d}$, y sea cerrado bajo transducción finita. Sin embargo, esta forma de construir el lenguaje es solo suficiente y no necesaria para construir el lenguaje de todas las fórmulas booleanas satisfacibles porque existen formalismos que no son cerrados bajo transducción finita (como las RCG) que también describen el lenguaje.


A continuación se describe el problema de la palabra para las RCG.


En \cite{mainRCGBib} se menciona que en la mayoría de los casos el problema de la palabra para las RCG es polinomial y se resuelve mediante un algoritmo de memorización sobre las cadenas asignadas a los argumentos de los predicados de la RCG.  Como la cantidad máxima de rangos de la cadena es $n^2$ y la máxima aridad de un predicado es constante, este proceso de memorización cuenta con una cantidad polinomial de estados, y tiene una complejidad de $O(|P|n^{2h(l+1)})$ donde $h$ es la máxima aridad en un predicado, $l$ es la máxima cantidad de predicados en el lado derecho de una cláusula y $n$ es la longitud de la cadena que se reconoce.

Sin embargo, existen casos en los que el problema de la palabra no es polinomial. En la siguiente sección se analiza un caso en el que este problema no es polinomial.

\subsection{Problema de la palabra no polinomial para las RCG}

El algoritmo de reconocimiento que se menciona en la sección anterior utiliza un proceso de memorización sobre los rangos de la cadena.  La idea fundamental para esto y lo que acota la complejidad del algoritmo es que la cantidad de estados asociados a la memorización es igual a la cantidad de rangos de la cadena, el cual es polinomial con respecto a la longitud de la cadena. Esto se cumple siempre que todos los argumentos que reciben todos los no terminales de la gramática sean subcadenas de la cadena original que se está analizando. Existen gramáticas de concatenación de rango en que esto no ocurre, como en la que se muestra a continuación.

La siguiente RCG reconoce el lenguaje $L=\{w\,|\,w\in\{0,1\}^*\}$. Esta gramática de concatenación de rango no tiene uso real porque existen otras RCG que reconocen el mismo lenguaje, pero ilustra una RCG donde se generan cadenas que no son subcadenas de la cadena de entrada durante el proceso de reconocimiento.

\[
    G_e = (N, T, V, P, S),
\]
donde:

\begin{itemize}
    \item  N=$\{A,B,Eq,S\}$.
    \item T=$\{0,1\}$.
    \item V=$\{X,Y\}$.
    \item El conjunto de cláusulas $P$ es el siguiente:
          \begin{enumerate}
              \item $S(X)\to A(X,X)$
              \item $A(1X,Y)\to B(X,0,Y)$
              \item $A(1X,Y)\to B(X,1,Y)$
              \item $A(0X,Y)\to B(X,1,Y)$
              \item $A(0X,Y)\to B(X,0,Y)$
              \item $B(1X,Y,Z)\to B(X,1Y,Z)$
              \item $B(1X,Y,Z)\to B(X,0Y,Z)$
              \item $B(0X,Y,Z)\to B(X,0Y,Z)$
              \item $B(0X,Y,Z)\to B(X,1Y,Z)$
              \item $B(\varepsilon,Y,Z)\to Eq(Y,Z)$
          \end{enumerate}
          
    \item El símbolo inicial es $S$.
\end{itemize}

Para procesar una cadena $w$, la gramática anterior genera todas las posibles cadenas $q$, tales que $|w|=|q|$ y luego comprueba si $w = q$.

Esta gramática no tiene caso de uso, ya que para toda cadena $w$ siempre va a existir una cadena $q$ tal que $w=q$, 
por lo que se puede modelar con solamente la cláusula $S(X)\to \varepsilon$. Sin embargo, la complejidad del 
reconocimiento de $G$ es mayor que $2^n$ (con $n$ igual al tamaño de la cadena de entrada), ya que esta es la 
cantidad de cadenas posibles que puede recibir el segundo argumento del predicado $B$, porque la gramática es 
ambigua y en cada derivación de $B$ existen 2 posibles decisiones, se añade un 1 delante al valor de la $Y$ o 
se añade un $0$.

En el capítulo \ref{chap:LSATRCG} se presenta una RCG que reconoce fórmulas booleanas satisfacibles, que al ser ambigua la complejidad del problema de la palabra es no polinomial.

En este capítulo se analizaron las principales definiciones y propiedades de las RCG, que se usan en el capítulo \ref{chap:LSATRCG} para definir una gramática que reconozca las fórmulas booleanas satisfacibles.  En el próximo capítulo se presenta un primer enfoque para definir el lenguaje de todas las fórmulas booleanas satisfacibles y a esta idea se le da continuidad en el capítulo \ref{chap:LSATRCG}, mediante las RCG.
