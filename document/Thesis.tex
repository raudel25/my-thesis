\documentclass[12pt,oneside]{uhthesis}
\usepackage{subfigure}
\usepackage[ruled,lined,linesnumbered,titlenumbered,algochapter,spanish,onelanguage]{algorithm2e}
\usepackage{amsmath}
\usepackage{amssymb}
\usepackage{amsbsy}
\usepackage{caption,booktabs}
\captionsetup{ justification = centering }
%\usepackage{mathpazo}
\usepackage{float}
\setlength{\marginparwidth}{2cm}
\usepackage{todonotes}
\usepackage{listings}
\usepackage{xcolor}
\usepackage{multicol}
\usepackage{graphicx}
\floatstyle{plaintop}
\restylefloat{table}
\addbibresource{Bibliography.bib}
% \setlength{\parskip}{\baselineskip}%
\renewcommand{\tablename}{Tabla}
\renewcommand{\listalgorithmcfname}{Índice de Algoritmos}
%\dontprintsemicolon
\SetAlgoNoEnd

\definecolor{codegreen}{rgb}{0,0.6,0}
\definecolor{codegray}{rgb}{0.5,0.5,0.5}
\definecolor{codepurple}{rgb}{0.58,0,0.82}
\definecolor{backcolour}{rgb}{0.95,0.95,0.92}

\lstdefinestyle{mystyle}{
    backgroundcolor=\color{backcolour},   
    commentstyle=\color{codegreen},
    keywordstyle=\color{purple},
    numberstyle=\tiny\color{codegray},
    stringstyle=\color{codepurple},
    basicstyle=\ttfamily\footnotesize,
    breakatwhitespace=false,         
    breaklines=true,                 
    captionpos=b,                    
    keepspaces=true,                 
    numbers=left,                    
    numbersep=5pt,                  
    showspaces=false,                
    showstringspaces=false,
    showtabs=false,                  
    tabsize=4
}

\lstset{style=mystyle}

\title{Título de la tesis}
\author{\\\vspace{0.25cm}Nombre del autor}
\advisor{\\\vspace{0.25cm}Nombre del primer tutor\\\vspace{0.2cm}Nombre del segundo tutor}
\degree{Licenciado en (Matemática o Ciencia de la Computación)}
\faculty{Facultad de Matemática y Computación}
\date{Fecha\\\vspace{0.25cm}\href{https://github.com/username/repo}{github.com/username/repo}}
\logo{Graphics/uhlogo}
\makenomenclature

\renewcommand{\vec}[1]{\boldsymbol{#1}}
\newcommand{\diff}[1]{\ensuremath{\mathrm{d}#1}}
\newcommand{\me}[1]{\mathrm{e}^{#1}}
\newcommand{\pf}{\mathfrak{p}}
\newcommand{\qf}{\mathfrak{q}}
%\newcommand{\kf}{\mathfrak{k}}
\newcommand{\kt}{\mathtt{k}}
\newcommand{\mf}{\mathfrak{m}}
\newcommand{\hf}{\mathfrak{h}}
\newcommand{\fac}{\mathrm{fac}}
\newcommand{\maxx}[1]{\max\left\{ #1 \right\} }
\newcommand{\minn}[1]{\min\left\{ #1 \right\} }
\newcommand{\lldpcf}{1.25}
\newcommand{\nnorm}[1]{\left\lvert #1 \right\rvert }
\renewcommand{\lstlistingname}{Ejemplo de código}
\renewcommand{\lstlistlistingname}{Ejemplos de código}

\begin{document}

\frontmatter
\maketitle

\begin{dedication}
    A mi familia
\end{dedication}
\begin{acknowledgements}
    A mi familia, por acompañarme en este camino, lleno de retos y desafíos, pero que hoy recoge el fruto de tanto esfuerzo.
    A mi mamá, por estar a mi lado en todo momento y por ser mi mayor apoyo. A mi hermana y mi primito por ser mi mayor motivación y compartir
    momentos de diversión y alegría. A mis abuelos por ser mi ejemplo a seguir y por enseñarme a ser mejor persona.
    A mis tíos por ser casi casi como mis segundos padres y estar en todo momento.
    
    A mis compañeros de la universidad, por ser parte de esta gran aventura y por compartir momentos inolvidables, por 
    estar siempre ahí para apoyarme, por compartir todas las horas de intenso estudio y el estrés de tantos proyectos y pruebas.
    A Anabel, Daniel, Alex, Omar, Javier, Juan Carlos y a todos los demás con los que compartí durante estos años.
    
    A la gente de la beca, mi casa durante estos años, por ser mi familia universitaria y estar siempre.
    
    A la gente del concurso: Gaby, Adilen, Arianna, Ailec, Marquito, Tito, Jocdan, Rey, Leo,  por ir juntos en el viaje por el mundo de las ciencias, que comenzó en el pre y 
    se mantiene hasta nuestros días. 
    
    A Alex y a Rafael, por tantos rounds de 5 horas dándonos cabezazos contra los ejercicios en ancias del tan esperado 
    \textit{ACCEPTED}.
    
    A los profes Manzano y Donet que introdujeron en mí la pasión por la matemática, lo que posteriormente se convirtió
    en amor por la programación y los algoritmos.
    
    A los profes de la universidad, por su guía y su apoyo, por la formación y la paciencia, en especial a los profes (colegas) de EDA,
    a Cartaya por sus consejos y enseñanzas, a mi tutor por su patrocinio y exigencia.
    
    A todos los que de una manera u otra aportaron su granito de arena para que hoy pueda estar aquí.
\end{acknowledgements}
\begin{opinion}
    En este trabajo se propone una vía de solución para el SAT utilizando elementos de la teoría de lenguajes formales.  Además, se define y construye el lenguaje de todas las fórmulas lógicas satisfacibles y se analizan algunas de las implicaciones que se derivan de este lenguaje.   Los resultados que se recogen en el documento abren nuevas e interesantes líneas de trabajo.

   Para esta tesis, Raudel tuvo que estudiar, de manera independiente, contenidos que no forman parte de su plan de estudios y usarlos de manera creativa y original.

   Considero que estamos en presencia de un trabajo excelente, desarrollado por un excelente científico de la computación.

   \vspace{1cm}


 \begin{flushright}
   \underline{\hspace{6.5cm}}\\
   MSc. Fernando Raul Rodriguez Flores

   Facultad de Matemática y Computación
  
   Universidad de la Habana

   Febrero, 2025
 \end{flushright}
\end{opinion}
\begin{resumen}
	El problema de la satisfacibilidad booleana es un problema NP-Completo y consiste en determinar si existe
	alguna interpretación verdadera de una fórmula booleana dada. La teoría de lenguajes es una rama fundamental de la Ciencia de la Computación y la matemática que se enfoca 
	en el estudio de los lenguajes formales. El objetivo de este trabajo es vincular el problema de la satisfacibilidad
	booleana y la teoría de lenguajes, para construir el lenguaje de todas las fórmulas booleanas satisfacibles.
	Para esta construcción se emplean 2 estrategias, la primera utiliza una transducción de una variante del lenguaje
	\textit{Copy} para construir dicho lenguaje y la segunda utiliza una gramática de concatenación de rango 
	que reconoce este lenguaje. Como resultado de la primera estrategia se demuestra que el problema de la palabra 
	para todos los formalismos
	que generen la variante del lenguaje \textit{Copy} y sean cerrados bajo transducción finita, es NP-Duro.
	Por otro lado, la gramática de concatenación de rango que se obtiene en la segunda estrategia permite 
	demostrar que las gramáticas de concatenación de rango reconocen todos los problemas de la clase NP, en 
	su representación como lenguaje formal.
\end{resumen}

\begin{abstract}
	The boolean satisfiability problem is an NP-Complete problem and consists in determining whether there exists
	any true interpretation of a given boolean formula. Language theory is a fundamental branch of Computer Science and Mathematics that focuses
	on the study of formal languages. The objective of this work is to link the problem of boolean
	satisfiability and language theory, to build the language of all satisfiable boolean formulas.
	For this construction, 2 strategies are used, the first uses a transduction of a variant of the language
	\textit{Copy} to build said language and the second uses a range concatenation grammar
	that recognizes this language. As a result of the first strategy, it is shown that the word problem
	for all formalisms
	that generate the variant of the language \textit{Copy} and are closed under finite transduction, is NP-Hard.
	On the other hand, the rank concatenation grammar obtained in the second strategy allows
	to demonstrate that rank concatenation grammars recognize all the problems of the NP class, in
	their representation as a formal language.
\end{abstract}
\tableofcontents
\listoffigures
% \listoftables
% \listofalgorithms
% \lstlistoflistings

\mainmatter

\documentclass[12pt]{article}

\usepackage[utf8]{inputenc} % Permite escribir caracteres especiales directamente
\usepackage[spanish]{babel} % Configura el idioma a español

\usepackage{amsmath}
\usepackage{tikz}
\usepackage{xcolor}
\usepackage[lmargin=2cm,rmargin=5cm]{geometry}

%%%{{{ Comments and the like
\usepackage[textwidth=4cm]{todonotes}
\usepackage{soul}
\usepackage{xcolor}
\newcounter{todocounter}
\newcommand{\comment}[2]{\stepcounter{todocounter}
  {\color{green!50!blue}{(#1$^{{\color{black}\textbf{\thetodocounter}}}$)}}
  \todo[color=green,noline,size=\tiny]{\textbf{\thetodocounter:} #2

  }}
\newcommand{\quitaesto}[1]{{\color{red}(\st{#1})}}

\newcommand{\cambio}[2]{{\color{cyan}{{#2}}}{\color{red}{(\st{#1})}}}

\newcommand{\agregaesto}[1]{{\color{cyan}{{#1}}}}

\newcommand{\notaparaelautor}[1]{{\color{brown}{\textbf{#1}}}}

\newcommand{\errorortografico}[1]{{\fcolorbox{gray}{magenta}{\textcolor{yellow}{\bf #1}}}}
    
%%%}}}


\title{Introducción}
\author{Raudel Alejandro Gómez Molina}

\begin{document}

\maketitle

El problema de satisfacibilidad booleana (\textit{SAT}) es uno de los problemas más estudiados en la teoría de la computación y la lógica.
Consiste en determinar si existe una asignación de valores verdaderos o falsos que satisfaga una fórmula booleana dada, compuesta
por variables y operadores lógicos como conjunciones, disyunciones y negaciones. SAT surge en 1971 como el primer problema NP-completo demostrado por
Stephen Cook,
lo que significa que, en el peor de los casos, su resolución requiere tiempo exponencial respecto al
tamaño de la entrada, pero también que muchos otros problemas pueden reducirse a él. Esto
implica un especial interés por parte de la comunidad científica en la búsqueda de métodos eficientes para la solución
del SAT.

La teoría de lenguajes es una rama fundamental de la Ciencia de la Computación y la matemática que se
enfoca en el estudio de los lenguajes formales. Estos lenguajes, definidos a través de gramáticas,
autómatas y expresiones regulares, permiten modelar y analizar la estructura de los lenguajes naturales y
artificiales. Su aplicación es amplia y abarca desde el diseño de compiladores y procesadores de lenguaje
natural hasta la verificación de sistemas y la teoría de la computabilidad.

Los lenguajes formales se
clasifican en jerarquías, como la jerarquía de Chomsky, que los organiza según su complejidad y poder
expresivo. Esta teoría proporciona las bases para entender cómo se pueden reconocer, generar y transformar
cadenas de símbolos, lo que resulta esencial en el desarrollo de herramientas computacionales para
el procesamiento de información. Además, la teoría de lenguajes constituye la base de los problemas de la Ciencia
de la Computación, ya que cualquier problema puede ser interpretado como un problema de la teoría de lenguajes.

En este trabajo se vinculan las dos ramas de la computación descritas anteriormente, presentando un enfoque para resolver el SAT
utilizando formalismos de teoría de lenguajes. Dicho enfoque resulta un tema no evidenciado en la literatura consultada
y permite demostrar que una serie de problemas relacionados a la teoría de lenguajes pertenecen a la clase NP-completo.

En estudios anteriores, que siguen la idea presentada en esta investigación, se han mostrado estrategias para la solución
de instancias específicas del SAT, usando formalismos de teoría de lenguajes, lo que constituye una solución limitada en su alcance.
En cambio, en este trabajo se presenta una alternativa que resuelve cualquier instancia del mismo, lo que, a criterio del autor, resulta una solución
cualitativamente superior. Esta sigue siendo una estrategia no eficiente, pero que
muestra un nuevo enfoque para resolver el SAT de forma general, y permite abrir nuevas líneas de investigación en este tema.

Para resolver cualquier instancia de SAT empleando formalismos de teoría de lenguajes se propone definir una codificación
de una fórmula booleana en una cadena que se pueda interpretar por algún formalismo de la teoría de lenguajes
y usando dicha codificación se define el lenguaje de todas las fórmulas booleanas satisfacibles. Entonces si se desea
determinar si una fórmula booleana es satisfacible es necesario determinar si la cadena asociada a la  fórmula booleana pertenece o no al lenguaje de todas las fórmulas booleanas satisfacibles.

Para construir el lenguaje
de las fórmulas booleanas satisfacibles se propone utilizar 2 métodos: el primero utiliza un transductor finito y el segundo
utiliza una gramática de concatenación de rango.

A partir de lo expuesto anteriormente se formula como objetivo general de de este trabajo: definir y construir el lenguaje de todas las fórmulas booleanas satisfacibles.

Para cumplir el objetivo general se definen los siguientes objetivos específicos:

\begin{itemize}
      \item Estudiar el estado del arte referido a los formalismos de teoría de lenguajes y el SAT.
      \item Establecer una representación del SAT como una cadena que pueda ser interpretada por un formalismo de la teoría de lenguajes.
      \item Definir el lenguaje de todas las fórmulas booleanas satisfacibles.
      \item Construir el lenguaje de todas las fórmulas booleanas satisfacibles utilizando un transductor finito.
      \item Construir el lenguaje de todas las fórmulas booleanas satisfacibles utilizando gramáticas de concatenación de rango.
\end{itemize}

Para dar cumplimiento a los objetivos trazados, el trabajo se ha estructurado en 4 capítulos: en los 2 primeros se presentan los principales conceptos y definiciones
que serán utilizados en el resto de la investigación y en los restantes 2 capítulos se define y construye el lenguaje de todas las fórmulas booleanas satisfacibles.

En el capítulo \ref{chap:preliminaries} se presentan los principales conceptos y definiciones de la teoría de lenguajes y el SAT, los cuales
son necesarios para la comprensión de los restantes capítulos. Además, se realiza un análisis de 2 trabajos anteriores
que muestran cómo solucionar instancias específicas del SAT utilizando un algoritmo polinomial.

En el capítulo \ref{chap:RCG} se realiza un análisis detallado de las gramáticas de concatenación de rango, presentando las principales
definiciones, proceso de derivación y análisis de la complejidad del algoritmo de reconocimiento.

En el capítulo \ref{chap:LSATFT} se muestra cómo codificar una fórmula booleana mediante una cadena de símbolos y luego
se analiza cómo interpretar una cadena como la asignación de valores para las variables de una fórmula booleana.
Posteriormente, se define el lenguaje de todas las fórmulas booleanas satisfacibles y se muestra cómo construir dicho
lenguaje mediante un transductor finito. Para finalizar, se demuestra que el problema de la palabra, para todos los formalismo que cumplan ciertas propiedades,
las cuales se definen en el capítulo \ref{chap:LSATFT}, es NP-Duro.

En el capítulo \ref{chap:LSATRCG} se demuestra que no es necesario construir el lenguaje de todas las fórmulas
booleanas satisfacibles mediante transducción finita, ya que existe una gramática de concatenación de rango que reconoce
este lenguaje. Por otro lado, se demuestra que las gramáticas de concatenación de rango cubren todos los problemas de la clase NP-Completo.


\begin{thebibliography}{99}

      \bibitem{mainRCGBib}
      Boullier, Pierre.
      \textit{Proposal for a Natural Language Processing Syntactic Backbone}.
      Research Report RR-3342, INRIA, 1998.

      \bibitem{propertiesRCGBib}
      Boullier, Pierre.
      \textit{A Cubic Time Extension of Context-Free Grammars}.
      Research Report RR-3611, INRIA, 1999.

      \bibitem{simpleMatrixLanguages}
      Ibarra, Oscar H.
      \textit{Simple matrix languages}.
      \textit{Information and Control}, Vol. 17, No. 4, pp. 359-394, 1970.

      \bibitem{globalIndexLanguages}
      Castaño, José M.
      \textit{Global Index Languages}.
      Ph.D. Thesis, The Faculty of the Graduate School of Arts and Sciences, Brandeis University, 2004.

      \bibitem{authomataTheory}
      Hopcroft, John E., Motwani, Rajeev, y Ullman, Jeffrey D.
      \textit{Introduction to Automata Theory, Languages, and Computation}.
      3ª edición, Addison-Wesley, 2006. ISBN: 9780321455369.

      \bibitem{aCFSAT}
      Fernández Arias, Alina.
      \textit{El problema de la satisfacibilidad booleana libre del contexto}.
      Facultad de Matemática y Computación, Universidad de La Habana, 2007.

      \bibitem{aSRCSAT}
      Aguilera López, Manuel.
      \textit{Problema de la Satisfacibilidad Booleana de Concatenación de Rango Simple}.
      Facultad de Matemática y Computación, Universidad de La Habana, 2016.

      \bibitem{aSMSAT}
      Rodríguez Salgado, José Jorge.
      \textit{Gramáticas Matriciales Simples. Primera aproximación para una solución al problema SAT}.
      Facultad de Matemática y Computación, Universidad de La Habana, 2019.

\end{thebibliography}


% Posibles conclusiones
% - teorica
% - como las gramáticas de concatenacion de rango constituyen un nuevo enfoque en la solucion del satisfacibilidad

% Posibles recomendaciones
% - por que via del transductor Full-SAT pueden desarrollarse nuevas investigaciones


\end{document}
\chapter{Estado del Arte}\label{chapter:state-of-the-art}

\chapter{Propuesta}\label{chapter:proposal}

\chapter{Detalles de Implementación y Experimentos}\label{chapter:implementation}


\backmatter

\begin{conclusions}

    En este trabajo se presentó una estrategia para resolver el SAT usando teoría de lenguajes, la cual se basa en definir
    una codificación de una fórmula booleana en una cadena y definir y construir el lenguaje de todas las fórmulas booleanas
    satisfacibles $L_{S-SAT}$. Luego para determinar si una fórmula booleana es satisfacible es necesario verificar si la cadena asociada
    a dicha fórmula pertenece a $L_{S-SAT}$.
    
    En el capítulo \ref{chap:LSATFT}, se construyó $L_{S-SAT}$ mediante el transductor finito $T_{SAT}$ que recibe
    cadenas del lenguaje $L_{0,1,d}$, las cuales representan todas las posibles interpretaciones de las fórmulas
    booleanas en CNF y genera cadenas del lenguaje $L_{FULL-SAT}$, tales que la fórmula booleana asociada a estas
    cadenas es satisfacible.
    
    El problema de la palabra para todo formalismo que genere
    el lenguaje $L_{0,1,d}$ y sea cerrado bajo transducción finita, es NP-Duro, teniendo en cuenta la conjetura
    de que cualquier formalismo que genere el lenguaje $L_{0,1,d}$, tiene tamaño $O(1)$ en su representación.
    
    En el capítulo \ref{chap:LSATRCG}, se presentó una RCG que reconoce el lenguaje $L_{0,1,d}$ y se argumentó por qué no es posible
    usar esta gramática para construir $L_{S-SAT}$ mediante transducción finita, ya que las RCG no son cerradas bajo transducción finita.
    
    Se construyó una RCG que reconoce el lenguaje $L_{S-SAT}$, lo que permitió demostrar
    que no es necesario construir $L_{S-SAT}$ mediante transducción finita. La gramática que se construyó tiene el problema
    de la palabra no polinomial, y constituye un ejemplo de una RCG donde el algoritmo de reconocimiento es no polinomial.
    Además al obtener una RCG que reconoce $L_{S-SAT}$, se demostró que las RCG cubren todos los problemas de la clase NP,
    ya que las RCG cubren todos los problemas en P \cite{mainRCGBib} y existe una reducción polinomial del SAT a todo problema en NP \cite{authomataTheory}.
    
    Las estrategias presentadas constituyen una vía diferente
    para resolver el SAT, y aunque el problema de la palabra para el formalismo que se construyó es no polinomial,
    este acercamiento puede contribuir a nuevas líneas de investigación para la búsqueda de algoritmos eficientes que permitan
    resolver el SAT.
    
\end{conclusions}

\documentclass[12pt]{article}

\usepackage[utf8]{inputenc} % Permite escribir caracteres especiales directamente
\usepackage[spanish]{babel} % Configura el idioma a español

\usepackage{amsmath}
\usepackage{tikz}
\usepackage{xcolor}
\usepackage[lmargin=2cm,rmargin=5cm]{geometry}

%%%{{{ Comments and the like
\usepackage[textwidth=4cm]{todonotes}
\usepackage{soul}
\usepackage{xcolor}
\newcounter{todocounter}
\newcommand{\comment}[2]{\stepcounter{todocounter}
  {\color{green!50!blue}{(#1$^{{\color{black}\textbf{\thetodocounter}}}$)}}
  \todo[color=green,noline,size=\tiny]{\textbf{\thetodocounter:} #2

  }}
\newcommand{\quitaesto}[1]{{\color{red}(\st{#1})}}

\newcommand{\cambio}[2]{{\color{cyan}{{#2}}}{\color{red}{(\st{#1})}}}

\newcommand{\agregaesto}[1]{{\color{cyan}{{#1}}}}

\newcommand{\notaparaelautor}[1]{{\color{brown}{\textbf{#1}}}}

\newcommand{\errorortografico}[1]{{\fcolorbox{gray}{magenta}{\textcolor{yellow}{\bf #1}}}}
    
%%%}}}


\title{Recomendaciones}
\author{Raudel Alejandro Gómez Molina}

\begin{document}

\maketitle

A partir del trabajo realizado se proponen como temas para investigaciones futuras los
siguientes:

\begin{itemize}
    \item Buscar un formalismo que sea capaz de generar el lenguaje $L_{0,1,d}$, el cual representa todas las interpretaciones
          de las fórmulas booleanas en CNF, que sea cerrado bajo transducción finita, y luego analizar el problema de la palabra para
          el formalismo que se obtiene después de aplicarle el transductor $T_{SAT}$.
    \item Demostrar que cualquier formalismo que genere $L_{0,1,d}$ tiene un tamaño $O(1)$ en su representación.
    \item Analizar qué tipo de formalismo se obtiene al aplicarle el transductor $T_{SAT}$ a la RCG que reconoce
          el $L_{0,1,d}$.
    \item  Analizar qué propiedades limitan que las RCG no sean cerradas bajo transducción finita, construir
          un formalismo basado en las RCG que sea cerrado bajo transducción finita y comprobar que este formalismo
          sea capaz de describir el lenguaje $L_{0,1,d}$.
          \item Construir una RCG que reconozca fórmulas booleanas satisfacibles, donde cada cláusula tiene a lo sumo dos literales (2-SAT),
          que tenga el problema de la palabra polinomial.
\end{itemize}




\begin{thebibliography}{99}

    \bibitem{mainRCGBib}
    Boullier, Pierre.
    \textit{Proposal for a Natural Language Processing Syntactic Backbone}.
    Research Report RR-3342, INRIA, 1998.

    \bibitem{propertiesRCGBib}
    Boullier, Pierre.
    \textit{A Cubic Time Extension of Context-Free Grammars}.
    Research Report RR-3611, INRIA, 1999.

    \bibitem{simpleMatrixLanguages}
    Ibarra, Oscar H.
    \textit{Simple matrix languages}.
    \textit{Information and Control}, Vol. 17, No. 4, pp. 359-394, 1970.

    \bibitem{globalIndexLanguages}
    Castaño, José M.
    \textit{Global Index Languages}.
    Ph.D. Thesis, The Faculty of the Graduate School of Arts and Sciences, Brandeis University, 2004.

    \bibitem{authomataTheory}
    Hopcroft, John E., Motwani, Rajeev, y Ullman, Jeffrey D.
    \textit{Introduction to Automata Theory, Languages, and Computation}.
    3ª edición, Addison-Wesley, 2006. ISBN: 9780321455369.

    \bibitem{aCFSAT}
    Fernández Arias, Alina.
    \textit{El problema de la satisfacibilidad booleana libre del contexto}.
    Facultad de Matemática y Computación, Universidad de La Habana, 2007.

    \bibitem{aSRCSAT}
    Aguilera López, Manuel.
    \textit{Problema de la Satisfacibilidad Booleana de Concatenación de Rango Simple}.
    Facultad de Matemática y Computación, Universidad de La Habana, 2016.

    \bibitem{aSMSAT}
    Rodríguez Salgado, José Jorge.
    \textit{Gramáticas Matriciales Simples. Primera aproximación para una solución al problema SAT}.
    Facultad de Matemática y Computación, Universidad de La Habana, 2019.

\end{thebibliography}


% Posibles conclusiones
% - teorica
% - como las gramáticas de concatenacion de rango constituyen un nuevo enfoque en la solucion del satisfacibilidad

% Posibles recomendaciones
% - por que via del transductor Full-SAT pueden desarrollarse nuevas investigaciones


\end{document}
\printbibliography[heading=bibintoc]


\end{document}