\documentclass[12pt,oneside]{uhthesis}

\usepackage{subfigure}
\usepackage[ruled,lined,linesnumbered,titlenumbered,algochapter,spanish,onelanguage]{algorithm2e}
\usepackage{amsmath}
\usepackage{amssymb}
\usepackage{amsbsy}
\usepackage{caption,booktabs}
\captionsetup{ justification = centering }
%\usepackage{mathpazo}
\usepackage{float}
\setlength{\marginparwidth}{2cm}
\usepackage{todonotes}
\usepackage{listings}
\usepackage{xcolor}
\usepackage{multicol}
\usepackage{graphicx}
\floatstyle{plaintop}
\restylefloat{table}
\addbibresource{Bibliography.bib}
% \setlength{\parskip}{\baselineskip}%
\renewcommand{\tablename}{Tabla}
\renewcommand{\listalgorithmcfname}{Índice de Algoritmos}
%\dontprintsemicolon
\SetAlgoNoEnd

\usepackage{tikz}
\usepackage{enumitem}
\usetikzlibrary{automata, positioning, arrows}

\definecolor{codegreen}{rgb}{0,0.6,0}
\definecolor{codegray}{rgb}{0.5,0.5,0.5}
\definecolor{codepurple}{rgb}{0.58,0,0.82}
\definecolor{backcolour}{rgb}{0.95,0.95,0.92}

\lstdefinestyle{mystyle}{
    backgroundcolor=\color{backcolour},   
    commentstyle=\color{codegreen},
    keywordstyle=\color{purple},
    numberstyle=\tiny\color{codegray},
    stringstyle=\color{codepurple},
    basicstyle=\ttfamily\footnotesize,
    breakatwhitespace=false,         
    breaklines=true,                 
    captionpos=b,                    
    keepspaces=true,                 
    numbers=left,                    
    numbersep=5pt,                  
    showspaces=false,                
    showstringspaces=false,
    showtabs=false,                  
    tabsize=4
}

\lstset{style=mystyle}

\title{Una aproximación al lenguaje de todas las fórmulas booleanas satisfacibles}
\author{\\\vspace{0.25cm}Raudel Alejandro Gómez Molina}
\advisor{\\\vspace{0.25cm}Mcs. Fernando Rodríguez Flores}
\degree{Licenciado en Ciencia de la Computación}
\faculty{Facultad de Matemática y Computación}
\date{Febrero de 2025}
\logo{Graphics/uhlogo}
\makenomenclature

\newcommand{\true}{\textit{true}}
\newcommand{\false}{\textit{false}}

\usepackage{amsthm}
\newtheorem{theorem}{Teorema}[chapter]
\newtheorem{definition}{Definición}[chapter]
\newtheorem{lemma}{Lema}[chapter]


\renewcommand{\vec}[1]{\boldsymbol{#1}}
\newcommand{\diff}[1]{\ensuremath{\mathrm{d}#1}}
\newcommand{\me}[1]{\mathrm{e}^{#1}}
\newcommand{\pf}{\mathfrak{p}}
\newcommand{\qf}{\mathfrak{q}}
%\newcommand{\kf}{\mathfrak{k}}
\newcommand{\kt}{\mathtt{k}}
\newcommand{\mf}{\mathfrak{m}}
\newcommand{\hf}{\mathfrak{h}}
\newcommand{\fac}{\mathrm{fac}}
\newcommand{\maxx}[1]{\max\left\{ #1 \right\} }
\newcommand{\minn}[1]{\min\left\{ #1 \right\} }
\newcommand{\lldpcf}{1.25}
\newcommand{\nnorm}[1]{\left\lvert #1 \right\rvert }
\renewcommand{\lstlistingname}{Ejemplo de código}
\renewcommand{\lstlistlistingname}{Ejemplos de código}



\begin{document}

\frontmatter
\maketitle

\begin{dedication}
    A mi familia
\end{dedication}
\begin{acknowledgements}
    A mi familia, por acompañarme en este camino, lleno de retos y desafíos, pero que hoy recoge el fruto de tanto esfuerzo.
    A mi mamá, por estar a mi lado en todo momento y por ser mi mayor apoyo. A mi hermana y mi primito por ser mi mayor motivación y compartir
    momentos de diversión y alegría. A mis abuelos por ser mi ejemplo a seguir y por enseñarme a ser mejor persona.
    A mis tíos por ser casi casi como mis segundos padres y estar en todo momento.
    
    A mis compañeros de la universidad, por ser parte de esta gran aventura y por compartir momentos inolvidables, por 
    estar siempre ahí para apoyarme, por compartir todas las horas de intenso estudio y el estrés de tantos proyectos y pruebas.
    A Anabel, Daniel, Alex, Omar, Javier, Juan Carlos y a todos los demás con los que compartí durante estos años.
    
    A la gente de la beca, mi casa durante estos años, por ser mi familia universitaria y estar siempre.
    
    A la gente del concurso: Gaby, Adilen, Arianna, Ailec, Marquito, Tito, Jocdan, Rey, Leo,  por ir juntos en el viaje por el mundo de las ciencias, que comenzó en el pre y 
    se mantiene hasta nuestros días. 
    
    A Alex y a Rafael, por tantos rounds de 5 horas dándonos cabezazos contra los ejercicios en ancias del tan esperado 
    \textit{ACCEPTED}.
    
    A los profes Manzano y Donet que introdujeron en mí la pasión por la matemática, lo que posteriormente se convirtió
    en amor por la programación y los algoritmos.
    
    A los profes de la universidad, por su guía y su apoyo, por la formación y la paciencia, en especial a los profes (colegas) de EDA,
    a Cartaya por sus consejos y enseñanzas, a mi tutor por su patrocinio y exigencia.
    
    A todos los que de una manera u otra aportaron su granito de arena para que hoy pueda estar aquí.
\end{acknowledgements}
\begin{opinion}
    En este trabajo se propone una vía de solución para el SAT utilizando elementos de la teoría de lenguajes formales.  Además, se define y construye el lenguaje de todas las fórmulas lógicas satisfacibles y se analizan algunas de las implicaciones que se derivan de este lenguaje.   Los resultados que se recogen en el documento abren nuevas e interesantes líneas de trabajo.

   Para esta tesis, Raudel tuvo que estudiar, de manera independiente, contenidos que no forman parte de su plan de estudios y usarlos de manera creativa y original.

   Considero que estamos en presencia de un trabajo excelente, desarrollado por un excelente científico de la computación.

   \vspace{1cm}


 \begin{flushright}
   \underline{\hspace{6.5cm}}\\
   MSc. Fernando Raul Rodriguez Flores

   Facultad de Matemática y Computación
  
   Universidad de la Habana

   Febrero, 2025
 \end{flushright}
\end{opinion}
\begin{resumen}
	El problema de la satisfacibilidad booleana es un problema NP-Completo y consiste en determinar si existe
	alguna interpretación verdadera de una fórmula booleana dada. La teoría de lenguajes es una rama fundamental de la Ciencia de la Computación y la matemática que se enfoca 
	en el estudio de los lenguajes formales. El objetivo de este trabajo es vincular el problema de la satisfacibilidad
	booleana y la teoría de lenguajes, para construir el lenguaje de todas las fórmulas booleanas satisfacibles.
	Para esta construcción se emplean 2 estrategias, la primera utiliza una transducción de una variante del lenguaje
	\textit{Copy} para construir dicho lenguaje y la segunda utiliza una gramática de concatenación de rango 
	que reconoce este lenguaje. Como resultado de la primera estrategia se demuestra que el problema de la palabra 
	para todos los formalismos
	que generen la variante del lenguaje \textit{Copy} y sean cerrados bajo transducción finita, es NP-Duro.
	Por otro lado, la gramática de concatenación de rango que se obtiene en la segunda estrategia permite 
	demostrar que las gramáticas de concatenación de rango reconocen todos los problemas de la clase NP, en 
	su representación como lenguaje formal.
\end{resumen}

\begin{abstract}
	The boolean satisfiability problem is an NP-Complete problem and consists in determining whether there exists
	any true interpretation of a given boolean formula. Language theory is a fundamental branch of Computer Science and Mathematics that focuses
	on the study of formal languages. The objective of this work is to link the problem of boolean
	satisfiability and language theory, to build the language of all satisfiable boolean formulas.
	For this construction, 2 strategies are used, the first uses a transduction of a variant of the language
	\textit{Copy} to build said language and the second uses a range concatenation grammar
	that recognizes this language. As a result of the first strategy, it is shown that the word problem
	for all formalisms
	that generate the variant of the language \textit{Copy} and are closed under finite transduction, is NP-Hard.
	On the other hand, the rank concatenation grammar obtained in the second strategy allows
	to demonstrate that rank concatenation grammars recognize all the problems of the NP class, in
	their representation as a formal language.
\end{abstract}
\tableofcontents
\listoffigures
% \listoftables
% \listofalgorithms
% \lstlistoflistings

\mainmatter

\documentclass[12pt]{article}

\usepackage[utf8]{inputenc} % Permite escribir caracteres especiales directamente
\usepackage[spanish]{babel} % Configura el idioma a español

\usepackage{amsmath}
\usepackage{tikz}
\usepackage{xcolor}
\usepackage[lmargin=2cm,rmargin=5cm]{geometry}

%%%{{{ Comments and the like
\usepackage[textwidth=4cm]{todonotes}
\usepackage{soul}
\usepackage{xcolor}
\newcounter{todocounter}
\newcommand{\comment}[2]{\stepcounter{todocounter}
  {\color{green!50!blue}{(#1$^{{\color{black}\textbf{\thetodocounter}}}$)}}
  \todo[color=green,noline,size=\tiny]{\textbf{\thetodocounter:} #2

  }}
\newcommand{\quitaesto}[1]{{\color{red}(\st{#1})}}

\newcommand{\cambio}[2]{{\color{cyan}{{#2}}}{\color{red}{(\st{#1})}}}

\newcommand{\agregaesto}[1]{{\color{cyan}{{#1}}}}

\newcommand{\notaparaelautor}[1]{{\color{brown}{\textbf{#1}}}}

\newcommand{\errorortografico}[1]{{\fcolorbox{gray}{magenta}{\textcolor{yellow}{\bf #1}}}}
    
%%%}}}


\title{Introducción}
\author{Raudel Alejandro Gómez Molina}

\begin{document}

\maketitle

El problema de satisfacibilidad booleana (\textit{SAT}) es uno de los problemas más estudiados en la teoría de la computación y la lógica.
Consiste en determinar si existe una asignación de valores verdaderos o falsos que satisfaga una fórmula booleana dada, compuesta
por variables y operadores lógicos como conjunciones, disyunciones y negaciones. SAT surge en 1971 como el primer problema NP-completo demostrado por
Stephen Cook,
lo que significa que, en el peor de los casos, su resolución requiere tiempo exponencial respecto al
tamaño de la entrada, pero también que muchos otros problemas pueden reducirse a él. Esto
implica un especial interés por parte de la comunidad científica en la búsqueda de métodos eficientes para la solución
del SAT.

La teoría de lenguajes es una rama fundamental de la Ciencia de la Computación y la matemática que se
enfoca en el estudio de los lenguajes formales. Estos lenguajes, definidos a través de gramáticas,
autómatas y expresiones regulares, permiten modelar y analizar la estructura de los lenguajes naturales y
artificiales. Su aplicación es amplia y abarca desde el diseño de compiladores y procesadores de lenguaje
natural hasta la verificación de sistemas y la teoría de la computabilidad.

Los lenguajes formales se
clasifican en jerarquías, como la jerarquía de Chomsky, que los organiza según su complejidad y poder
expresivo. Esta teoría proporciona las bases para entender cómo se pueden reconocer, generar y transformar
cadenas de símbolos, lo que resulta esencial en el desarrollo de herramientas computacionales para
el procesamiento de información. Además, la teoría de lenguajes constituye la base de los problemas de la Ciencia
de la Computación, ya que cualquier problema puede ser interpretado como un problema de la teoría de lenguajes.

En este trabajo se vinculan las dos ramas de la computación descritas anteriormente, presentando un enfoque para resolver el SAT
utilizando formalismos de teoría de lenguajes. Dicho enfoque resulta un tema no evidenciado en la literatura consultada
y permite demostrar que una serie de problemas relacionados a la teoría de lenguajes pertenecen a la clase NP-completo.

En estudios anteriores, que siguen la idea presentada en esta investigación, se han mostrado estrategias para la solución
de instancias específicas del SAT, usando formalismos de teoría de lenguajes, lo que constituye una solución limitada en su alcance.
En cambio, en este trabajo se presenta una alternativa que resuelve cualquier instancia del mismo, lo que, a criterio del autor, resulta una solución
cualitativamente superior. Esta sigue siendo una estrategia no eficiente, pero que
muestra un nuevo enfoque para resolver el SAT de forma general, y permite abrir nuevas líneas de investigación en este tema.

Para resolver cualquier instancia de SAT empleando formalismos de teoría de lenguajes se propone definir una codificación
de una fórmula booleana en una cadena que se pueda interpretar por algún formalismo de la teoría de lenguajes
y usando dicha codificación se define el lenguaje de todas las fórmulas booleanas satisfacibles. Entonces si se desea
determinar si una fórmula booleana es satisfacible es necesario determinar si la cadena asociada a la  fórmula booleana pertenece o no al lenguaje de todas las fórmulas booleanas satisfacibles.

Para construir el lenguaje
de las fórmulas booleanas satisfacibles se propone utilizar 2 métodos: el primero utiliza un transductor finito y el segundo
utiliza una gramática de concatenación de rango.

A partir de lo expuesto anteriormente se formula como objetivo general de de este trabajo: definir y construir el lenguaje de todas las fórmulas booleanas satisfacibles.

Para cumplir el objetivo general se definen los siguientes objetivos específicos:

\begin{itemize}
      \item Estudiar el estado del arte referido a los formalismos de teoría de lenguajes y el SAT.
      \item Establecer una representación del SAT como una cadena que pueda ser interpretada por un formalismo de la teoría de lenguajes.
      \item Definir el lenguaje de todas las fórmulas booleanas satisfacibles.
      \item Construir el lenguaje de todas las fórmulas booleanas satisfacibles utilizando un transductor finito.
      \item Construir el lenguaje de todas las fórmulas booleanas satisfacibles utilizando gramáticas de concatenación de rango.
\end{itemize}

Para dar cumplimiento a los objetivos trazados, el trabajo se ha estructurado en 4 capítulos: en los 2 primeros se presentan los principales conceptos y definiciones
que serán utilizados en el resto de la investigación y en los restantes 2 capítulos se define y construye el lenguaje de todas las fórmulas booleanas satisfacibles.

En el capítulo \ref{chap:preliminaries} se presentan los principales conceptos y definiciones de la teoría de lenguajes y el SAT, los cuales
son necesarios para la comprensión de los restantes capítulos. Además, se realiza un análisis de 2 trabajos anteriores
que muestran cómo solucionar instancias específicas del SAT utilizando un algoritmo polinomial.

En el capítulo \ref{chap:RCG} se realiza un análisis detallado de las gramáticas de concatenación de rango, presentando las principales
definiciones, proceso de derivación y análisis de la complejidad del algoritmo de reconocimiento.

En el capítulo \ref{chap:LSATFT} se muestra cómo codificar una fórmula booleana mediante una cadena de símbolos y luego
se analiza cómo interpretar una cadena como la asignación de valores para las variables de una fórmula booleana.
Posteriormente, se define el lenguaje de todas las fórmulas booleanas satisfacibles y se muestra cómo construir dicho
lenguaje mediante un transductor finito. Para finalizar, se demuestra que el problema de la palabra, para todos los formalismo que cumplan ciertas propiedades,
las cuales se definen en el capítulo \ref{chap:LSATFT}, es NP-Duro.

En el capítulo \ref{chap:LSATRCG} se demuestra que no es necesario construir el lenguaje de todas las fórmulas
booleanas satisfacibles mediante transducción finita, ya que existe una gramática de concatenación de rango que reconoce
este lenguaje. Por otro lado, se demuestra que las gramáticas de concatenación de rango cubren todos los problemas de la clase NP-Completo.


\begin{thebibliography}{99}

      \bibitem{mainRCGBib}
      Boullier, Pierre.
      \textit{Proposal for a Natural Language Processing Syntactic Backbone}.
      Research Report RR-3342, INRIA, 1998.

      \bibitem{propertiesRCGBib}
      Boullier, Pierre.
      \textit{A Cubic Time Extension of Context-Free Grammars}.
      Research Report RR-3611, INRIA, 1999.

      \bibitem{simpleMatrixLanguages}
      Ibarra, Oscar H.
      \textit{Simple matrix languages}.
      \textit{Information and Control}, Vol. 17, No. 4, pp. 359-394, 1970.

      \bibitem{globalIndexLanguages}
      Castaño, José M.
      \textit{Global Index Languages}.
      Ph.D. Thesis, The Faculty of the Graduate School of Arts and Sciences, Brandeis University, 2004.

      \bibitem{authomataTheory}
      Hopcroft, John E., Motwani, Rajeev, y Ullman, Jeffrey D.
      \textit{Introduction to Automata Theory, Languages, and Computation}.
      3ª edición, Addison-Wesley, 2006. ISBN: 9780321455369.

      \bibitem{aCFSAT}
      Fernández Arias, Alina.
      \textit{El problema de la satisfacibilidad booleana libre del contexto}.
      Facultad de Matemática y Computación, Universidad de La Habana, 2007.

      \bibitem{aSRCSAT}
      Aguilera López, Manuel.
      \textit{Problema de la Satisfacibilidad Booleana de Concatenación de Rango Simple}.
      Facultad de Matemática y Computación, Universidad de La Habana, 2016.

      \bibitem{aSMSAT}
      Rodríguez Salgado, José Jorge.
      \textit{Gramáticas Matriciales Simples. Primera aproximación para una solución al problema SAT}.
      Facultad de Matemática y Computación, Universidad de La Habana, 2019.

\end{thebibliography}


% Posibles conclusiones
% - teorica
% - como las gramáticas de concatenacion de rango constituyen un nuevo enfoque en la solucion del satisfacibilidad

% Posibles recomendaciones
% - por que via del transductor Full-SAT pueden desarrollarse nuevas investigaciones


\end{document}
\chapter{Preliminares}

\section{Teoría de Lenguajes}

\subsection{Conceptos básicos}

\paragraph{Alfabeto:} Un alfabeto, denotado como $\Sigma$, es un conjunto finito y no vacío de símbolos, ejemplo:
$$\Sigma=\{1,0\}$$
\paragraph{Cadena:} Una cadena es una sucesión finita de símbolos del alfabeto, ejemplo: la representación binaria de
los números $3=11$ y $5=101$ puede ser un ejemplo de cadena sobre el alfabeto $\Sigma$ anteriormente
definido.
\paragraph{Lenguaje:} Un lenguaje es un conjunto de cadenas definido sobre un alfabeto, ejemplo: el lenguaje de la
representación binaria de todos los números pares $L=\{w\,|\,\text{last}(w)=0\}$, $\text{last}(w)$
representa el último caracter de la cadena $w$.

\subsection{Operaciones con Lenguajes}

\paragraph{Unión:} La unión de dos lenguajes $L_1$ y $L_2$ se define como el conjunto de cadenas que
pertenecen a $L_1$ o a $L_2$:
$$L_1\cup L_2=\{w\,|\,w\in L_1\,\vee\,w\in L_2\}$$
\paragraph{Intersección:} La intersección de dos lenguajes $L_1$ y $L_2$ se define como el conjunto de
cadenas que pertenecen a $L_1$ y a $L_2$:
$$L_1\cap L_2=\{w\,|\,w\in L_1\,\wedge\,w\in L_2\}$$
\paragraph{Concatenación:} La concatenación de dos lenguajes $L_1$ y $L_2$ se define como el conjunto
de cadenas que resultan de concatenar una cadena de $L_1$ con una cadena de $L_2$:
$$L_1\circ L_2=\{w_1w_2\,|\,w_1\in L_1\,\wedge\,w_2\in L_2\}$$
\paragraph{Complemento:} El complemento de un lenguaje $L$ se define como el conjunto de cadenas que no
pertenecen a $L$:
$$\overline{L}=\{w\,|\,w\notin L\}$$
\paragraph{Clausura de Kleene:} La clausura de Kleene de un lenguaje $L$ se define como el conjunto de
cadenas que resultan de concatenar cero o más cadenas de $L$:
$$L^*=\{w_1w_2\ldots w_n\,|\,n\geq 0\,\text{y}\,w_i\in L\}$$

\subsection{Problemas relacionados con Lenguajes}

\paragraph{Problema de la palabra:} Consiste en determinar si una cadena pertenece a un lenguaje dado. Todo problema en Ciencias de la Computación puede ser reducido a un problema de la palabra, ya que cualquier problema
puede ser codificado como un lenguaje formal.
\paragraph{Problema del vacío:} Consiste en determinar si un lenguaje es vacío.
\paragraph{Problema de la finitud:} Consiste en determinar si un lenguaje es finito.
\paragraph{Problema de la equivalencia:} Consiste en determinar si dos $L_1$ y $L_2$ lenguajes son iguales (es decir si se cumple que
$L_1\subseteq L_2 \wedge L_2\subseteq L_1$).

\subsection{Gramáticas}

Una \textbf{gramática} es un sistema matemático utilizado para describir lenguajes formales. Se define como una 4-tupla:
\[
      G = (N, \Sigma, P, S),
\]
donde:
\begin{itemize}
      \item \(N\): Es un conjunto finito de \textbf{símbolos no terminales}, que representan variables o categorías intermedias.
      \item \(\Sigma\): Es un conjunto finito de \textbf{símbolos terminales}, que constituyen el alfabeto del lenguaje. Se cumple que \(N \cap \Sigma = \emptyset\).
      \item \(P\): Es un conjunto finito de \textbf{reglas de producción}, cada una de la forma:
            \[
                  \alpha \to \beta, \quad \text{donde } \alpha \in (N \cup \Sigma)^* \wedge \beta \in (N \cup \Sigma)^*.
            \]
      \item \(S\): Es el \textbf{símbolo inicial}, \(S \in N\), que define el punto de partida para derivar cadenas del lenguaje.
\end{itemize}

El lenguaje generado por una gramática \(G\) se denota como:
\[
      L(G) = \{ w \in \Sigma^* \mid S \overset{*}{\Rightarrow} w \},
\]
donde \(\overset{*}{\Rightarrow}\) indica una derivación en cero o más pasos.

\subsection{Jerarquía de Chomsky}

La \textbf{Jerarquía de Chomsky} (Figura~\ref{fig:ChomskySchema}) clasifica las gramáticas en cuatro tipos, según las restricciones en sus reglas de producción y la capacidad expresiva de los lenguajes que generan.

\begin{enumerate}
      \item \textbf{Tipo 0: Gramáticas irrestrictas}
            \begin{itemize}
                  \item No tienen restricciones en las reglas de producción.
                  \item Cada regla tiene la forma: \(\alpha \to \beta\), donde \(\alpha, \beta \in (N \cup \Sigma)^*\) y \(\alpha \neq \varepsilon\).
                  \item Generan los \textbf{lenguajes recursivamente enumerables}.
            \end{itemize}
            
      \item \textbf{Tipo 1: Gramáticas sensibles al contexto}
            \begin{itemize}
                  \item Cada regla tiene la forma: \(\alpha A \gamma \to \alpha \beta \gamma\), donde \(A \in N\), \(\alpha, \beta, \gamma \in (N \cup \Sigma)^*\), y \(|\beta| \geq 1\).
                  \item Generan los \textbf{lenguajes sensibles al contexto}.
            \end{itemize}
            
      \item \textbf{Tipo 2: Gramáticas libres de contexto}
            \begin{itemize}
                  \item Cada regla tiene la forma: \(A \to \beta\), donde \(A \in N\) y \(\beta \in (N \cup \Sigma)^*\).
                  \item Generan los \textbf{lenguajes libres de contexto}.
            \end{itemize}
            
      \item \textbf{Tipo 3: Gramáticas regulares}
            \begin{itemize}
                  \item Las reglas tienen la forma:
                        \[
                              A \to aB \quad \text{o} \quad A \to a,
                        \]
                        donde \(A, B \in N\) y \(a \in \Sigma\).
                  \item Generan los \textbf{lenguajes regulares}.
            \end{itemize}
\end{enumerate}


\begin{figure}
      \centering
      \begin{tikzpicture}[scale=1]
            \draw[thick, fill=red!20] (0,0) ellipse (4cm and 2cm);
            \node at (0,1.7) {\textbf{\textcolor{red!70!black}{Tipo 0}}};
            
            \draw[thick, fill=blue!20] (0,0) ellipse (3cm and 1.5cm);
            \node at (0,1.2) {\textbf{\textcolor{blue!70!black}{Tipo 1}}};
            
            \draw[thick, fill=green!20] (0,0) ellipse (2cm and 1cm);
            \node at (0,0.7) {\textbf{\textcolor{green!70!black}{Tipo 2}}};
            
            \draw[thick, fill=yellow!20] (0,0) ellipse (1cm and 0.5cm);
            \node at (0,0) {\textbf{\textcolor{yellow!70!black}{Tipo 3}}};
      \end{tikzpicture}
      \caption{Esquema de la Jerarquía de Chomsky}
      \label{fig:ChomskySchema} %
\end{figure}

\section{Complejidad computacional}

\subsection{Máquina de Turing}

Una Máquina de Turing \cite{authomataTheory} es un modelo abstracto de computación universal introducido por Alan Turing en 1936. Este modelo consiste en los siguientes componentes:

\begin{itemize}
      \item \textbf{Cinta}: Un medio de almacenamiento infinito dividido en celdas, donde cada celda contiene un símbolo de un alfabeto finito.
      \item \textbf{Cabezal de lectura/escritura}: Un dispositivo que puede leer el contenido de una celda, escribir un nuevo símbolo y moverse a la izquierda o derecha.
      \item \textbf{Conjunto de estados}: Una colección finita de estados internos que describen la configuración actual de la máquina.
      \item \textbf{Función de transición}: Un conjunto de reglas que determinan cómo la máquina cambia de estado, escribe en la cinta y mueve el cabezal en función del estado actual y el símbolo leído.
\end{itemize}

\paragraph{Máquina de Turing determinista (\textit{DTM}):}
En una Máquina de Turing determinista, para cada estado y cada símbolo leído, existe como máximo una transición
definida.
\paragraph{Máquina de Turing no determinista (\textit{NTM}):}
En una Máquina de Turing no determinista, para cada estado y símbolo leído, pueden existir múltiples
transiciones posibles.

Una máquina de Turing consiste la definición formal de algoritmo en Ciencias de la Computación y es el eje central para la resolución de problemas.

\subsection{Notación asintótica}

La notación asintótica se utiliza para describir el comportamiento de una función $f(n)$ a medida que $n$ crece hacia el infinito. A continuación se definen las notaciones más comunes:

\begin{itemize}
      \item \textbf{Notación $O(f(n))$}: Una función $g(n)$ pertenece a $O(f(n))$ si existen constantes positivas $c$ y $n_0$ tales que:
            \[
                  g(n) \leq c \cdot f(n) \quad \text{para todo } n \geq n_0.
            \]
            Esta notación proporciona un límite superior asintótico para $g(n)$.
            
      \item \textbf{Notación $\Omega(f(n))$}: Una función $g(n)$ pertenece a $\Omega(f(n))$ si existen constantes positivas $c$ y $n_0$ tales que:
            \[
                  g(n) \geq c \cdot f(n) \quad \text{para todo } n \geq n_0.
            \]
            Esta notación proporciona un límite inferior asintótico para $g(n)$.
            
      \item \textbf{Notación $\Theta(f(n))$}: Una función $g(n)$ pertenece a $\Theta(f(n))$ si:
            \[
                  g(n) \in O(f(n)) \quad \text{y} \quad g(n) \in \Omega(f(n)).
            \]
            Es decir, $f(n)$ acota $g(n)$ tanto superior como inferiormente.
\end{itemize}

La notación asintótica permite describir el tiempo de ejecución un algoritmo en cuanto al número de operaciones básicos realizadas
por un modelo formal de cómputo (por ejemplo una máquina de Turing). Algoritmos como determinar el mínimo y el máximo de
un array son $\Theta(n)$, ya que necesitan realizar una cantidad de operaciones de orden $n$ en relación con el tamaño de la entrada.

\subsection{Clases de problemas}

Los problemas computacionales \cite{authomataTheory} se agrupan en diferentes clases según los recursos necesarios para resolverlos.

\paragraph{Problemas en la clase P:}
Un problema pertenece a la clase P si puede resolverse en tiempo polinomial mediante una Máquina de Turing determinista. Es decir, existe un algoritmo determinista que, para una entrada de tamaño $n$, produce la solución en tiempo $O(n^k)$ para alguna constante $k$.

\paragraph{Problemas en la clase NP:}
Un problema pertenece a NP si su solución puede verificarse en tiempo polinomial mediante una Máquina de Turing determinista. Alternativamente, un problema está en NP si puede resolverse en tiempo polinomial mediante una Máquina de Turing no determinista.

\paragraph{Problemas en la clase NP-Completo:}
Un problema pertenece a la clase NP-Completo, si pertenece a NP y además es tan difícil como cualquier otro problema en NP. Esto significa que cualquier problema en NP puede reducirse a este problema en tiempo polinómico.

\paragraph{Problemas no decidibles:}
Un problema es no decidible si no existe una Máquina de Turing que pueda resolverlo correctamente para todas las entradas posibles. Esto significa que no hay algoritmo que garantice una respuesta en tiempo finito en todos los casos. Un ejemplo clásico de problema no decidible es el \textit{Problema de la Parada}, que consiste en determinar si una Máquina de Turing se detendrá para una entrada dada.
\subsection{P vs NP}

La relación entre las clases P y NP es uno de los problemas abiertos más importantes en la teoría de la
computación. Hasta la fecha, se desconoce si $\text{P} = \text{NP}$ o si $\text{P} \neq \text{NP}$. Por otro
lado el conjunto de problemas NP-Completo brinda una base sólida para el problema anterior, ya que dada su
definición cualquier problema perteneciente a este conjunto que sea soluble en tiempo polinomial
implica que todos los problemas en NP lo son. Además, existen problemas que no están en P ni en NP, como
los problemas indecidibles.

\section{Transformaciones en Lenguajes Formales}

\subsection{Homorfismo}

Dado un alfabeto \( \Sigma \) y un alfabeto \( \Gamma \), un homomorfismo es una función:
\[
      h: \Sigma^* \to \Gamma^*
\]
tal que:
\begin{enumerate}
      \item Para cada \( a \in \Sigma \), \( h(a) \) es una cadena en \( \Gamma^* \).
      \item Para cualquier par de cadenas \( u, v \in \Sigma^* \), se cumple que:
            \[
                  h(uv) = h(u) h(v),
            \]
            es decir, el homomorfismo preserva la concatenación.
\end{enumerate}

\subsection{Transductor finito}

Un transductor finito es un modelo computacional que extiende los autómatas finitos al incluir tanto entradas como salidas.
Formalmente, un transductor finito es un autómata finito determinista o no determinista con una función de transición extendida
que asocia una salida a cada transición.

Un transductor finito puede representarse como una 6-tupla:
\[
      T = (Q, \Sigma, \Gamma, \delta, q_0, F),
\]
donde:
\begin{itemize}
      \item \(Q\) es el conjunto finito de estados.
      \item \(\Sigma\) es el alfabeto de entrada.
      \item \(\Gamma\) es el alfabeto de salida.
      \item \(\delta: Q \times \Sigma \to Q \times \Gamma^*\) es la función de transición, que mapea una combinación de estado actual y símbolo de entrada a un nuevo estado y una salida.
      \item \(q_0 \in Q\) es el estado inicial.
      \item \(F \subseteq Q\) es el conjunto de estados finales.
\end{itemize}

Observe que un homorfismo es un transductor finito de un solo estado y tantas transiciones hacia el mismo estado como transformaciones
de símbolos en el homomorfismo.


\section{Formalismos de escritura regulada}

\subsection{Autómata regular}

Un autómata regular \cite{authomataTheory}, también conocido como autómata finito, es un modelo matemático para reconocer lenguajes regulares. Este tipo de autómata se define como una máquina abstracta que procesa cadenas de símbolos de un alfabeto finito y determina si una cadena pertenece a un lenguaje regular.

Un autómata regular se puede representar como una 5-tupla $$\mathcal{A} = (Q, \Sigma, \delta, q_0, F)$$ donde:

\begin{itemize}
      \item $Q$: Es un conjunto finito de \textbf{estados}.
      \item $\Sigma$: Es el \textbf{alfabeto} finito de entrada.
      \item $\delta$: Es la \textbf{función de transición}, $\delta: Q \times \Sigma \to Q$, que define cómo el autómata cambia de estado en función del símbolo leído.
      \item $q_0 \in Q$: Es el \textbf{estado inicial} desde donde comienza la computación.
      \item $F \subseteq Q$: Es el conjunto de \textbf{estados de aceptación o estados finales}.
\end{itemize}

El autómata comienza en el estado inicial $q_0$ y procesa una cadena de entrada símbolo por símbolo. En cada paso, la función de transición $\delta$ determina el siguiente estado del autómata. Si, después de procesar toda la cadena, el autómata termina en un estado de aceptación $q \in F$, entonces la cadena es aceptada; de lo contrario, es rechazada.

Se puede extender el concepto de autómata finito añadiendo un nuevo tipo de transición que no consume ningún caracter, la cual recibe el nombre de transición $\varepsilon$.
Se puede demostrar que el conjunto de lenguajes reconocido por un autómata finito sin transiciones $\varepsilon$ (\textit{autómata finito determinista}) es equivalente al conjunto de lenguajes reconocidos por un autómata finito con transiciones $\varepsilon$ (\textit{autómata finito no determinista}).

\subsection{Autómata de pila y Gramáticas libres del contexto}

Un autómata de pila \cite{authomataTheory} es un modelo matemático de computación que extiende los autómatas finitos al incluir una estructura de datos adicional: una pila. Este modelo es capaz de reconocer lenguajes libres de contexto,
proporcionando una conexión directa con las gramáticas libres de contexto \textit{CFG}, es decir dada una gramática libre del contexto se puede construir un autómata de pila que reconozca el lenguaje generado por la gramática y viceversa.

Formalmente, un autómata de pila se define como una 7-tupla
\[
      \mathcal{A} = (Q, \Sigma, \Gamma, \delta, q_0, Z_0, F),
\]
donde:

\begin{itemize}
      \item $Q$: Es un conjunto finito de \textbf{estados}.
      \item $\Sigma$: Es el \textbf{alfabeto} finito de entrada.
      \item $\Gamma$: Es el \textbf{alfabeto} finito de la pila (conjunto de símbolos que se pueden almacenar en la pila).
      \item $\delta$: Es la función de transición, $\delta: Q \times (\Sigma \cup \{\varepsilon\}) \times \Gamma \to \mathcal{P}(Q \times \Gamma^*)$, que describe cómo cambia el estado, el contenido de la pila y la posición en la entrada.
      \item $q_0 \in Q$: Es el \textbf{estado inicial} desde donde comienza la computación.
      \item $Z_0 \in \Gamma$: Es el símbolo inicial en la pila.
      \item $F \subseteq Q$: Es el conjunto de \textbf{estados de aceptación o estados finales}.
\end{itemize}

Un autómata de pila procesa una cadena de entrada desde el estado inicial $q_0$ y puede utilizar transiciones $\varepsilon$ (sin consumir entrada). En cada paso, la función $\delta$ determina el nuevo estado, los símbolos que se empujan o desapilan, y el avance en la entrada. Si, tras procesar toda la cadena, el autómata termina en un estado de aceptación o con una pila vacía (dependiendo del criterio de aceptación), la cadena es aceptada.


\subsection{Gramáticas Matriciales}

Una gramática matricial \cite{simpleMatrixLanguages} de grado $n$ \textit{$n$-MG} es una 4-tupla:

\[
      G_n = (V, P, S,\Sigma)
\]

donde:
\begin{itemize}
      \item \( V \) es un conjunto finito de \textbf{símbolos no terminales}.
      \item \( \Sigma \) es un conjunto finito de \textbf{símbolos terminales}, con \( V \cap \Sigma = \emptyset \).
      \item \( P \) es un conjunto finito de matrices. Cada matriz es una secuencia ordenada de \textbf{producciones} de la forma:
            \[
                  [P_1, P_2, \dots, P_k]
            \]
            donde cada \( P_i \) es una regla \( A \to \alpha \), con $1\leq k\leq n$, \( A \in N \) y \( \alpha \in (N \cup T)^* \).
      \item \( S  \) es el \textbf{símbolo inicial}.
\end{itemize}

Observe que das CFG son gramáticas matriciales de grado 1, es decir $1-MG$.

\subsubsection{Proceso de derivación}

El proceso de derivación en una gramática matricial se realiza de la siguiente manera:
\begin{enumerate}
      \item Se selecciona una matriz \( [P_1, P_2, \dots, P_k] \in P \).
      \item Las reglas \( P_1, P_2, \dots, P_k \) se aplican de manera secuencial a la cadena actual.
      \item La derivación continúa hasta que la cadena derivada contenga solo símbolos terminales, es decir, pertenezca a \( T^* \).
\end{enumerate}

\subsubsection{Gramáticas Matriciales Simples}

Una gramática matricial simple de grado $n$ \textit{$n$-SMG} es una ($n$+3)-tupla:
$$
      G_n=(V_1,V_2,\ldots,V_n,P,S,\Sigma)
$$
donde:
\begin{itemize}
      \item \( V_1, V_2, \ldots, V_n \) son conjuntos finitos de \textbf{símbolos no terminales} disjuntos 2 a 2.
      \item \( \Sigma \) es un conjunto finito de \textbf{símbolos terminales}, con \( V_i \cap \Sigma = \emptyset\,\forall\,1\leq i\leq n \).
      \item \( P \) es un conjunto finito de matrices. Cada matriz es una secuencia ordenada \textbf{producciones} que cumplan con una de las siguientes reglas:
            \begin{itemize}
                  \item $[S\to w]$, donde $w\in \Sigma ^*$.
                  \item $[S\to a_{11}A_{11}\ldots A_{1k}a_{21}A_{21}\ldots A_{2k}\ldots A_{n1}a_{n1}\ldots A_{nk}b]$,
                        donde $\forall i,j$ con $1\leq i\leq n\wedge 1\leq j\leq k$ se cumple que
                        $A_{ij}\in V_i$, $a_{ij}\in \Sigma ^*$ y $b\in \Sigma ^*$.
                  \item $[A_1\to w_1,\ldots, A_n\to w_n]$, donde $A_i\in V_i\wedge w_i\in \Sigma ^*$ $\forall i\, 1\leq i\leq n$.
                  \item $[A_1 \to a_{11}A_{11}\ldots a_{1k}A_{1k}b_1,\ldots,A_n \to a_{n1}A_{n1}\ldots a_{nk}A_{nk}b_n]$, donde $\forall i,j$
                        con $1\leq i\leq n\wedge 1\leq j\leq k$ se cumple que
                        $A_{ij}\in V_i$, $a_{ij}\in \Sigma ^*$ y $b_{i}\in \Sigma ^*$.
            \end{itemize}
      \item \( S \) es el \textbf{símbolo inicial}.
\end{itemize}


Observe que la restricción impuesta sobre las $n$-MG es cada matriz de producciones debe contener exactamente $n$ reglas de producción
donde cada regla de producción utiliza no terminales de conjuntos distintos o puede contener una única producción cuya secuencia de no terminales
esta compuesta por una secuencia de subsecuentes de terminales de conjuntos distintos.

\subsection{Gramáticas de Índice Global}

Una gramática de índice global \textit{GIG} \cite{globalIndexLanguages}, es una extensión de las CFL, que añaden un mecanismo 
de memoria al proceso de derivación, esta característica permite la generación de lenguajes más generales que los generados por las CFL.
El mecanismo de memorización consiste en una pila en la cual se pueden almacenar símbolos que pertenecen a un conjunto predeterminado,
en cada producción se puede realizar una operación de push o pop en la pila o dejarla en su estado actual.

Una GIG es una 6-tupla:
$$
      G = (N, \Sigma, I, S, \#, P) 
$$
donde:

\begin{itemize}
      \item $N$ es un conjunto finito de \textbf{símbolos no terminales}.
      \item \( \Sigma \) es un conjunto finito de \textbf{símbolos terminales}, $\Sigma \cap N=\emptyset$.
      \item $I$ es un conjunto finito de \textbf{índices de pila}, $\Sigma \cap I=\emptyset \wedge I \cap I=\emptyset$.
      \item $S\in N$ es el \textbf{símbolo inicial}.
      \item $\#$ es el \textbf{símbolo inicial de la pila}, $\# \notin \Sigma \cup N \cup I$.
      \item $P$ es un conjunto finito de \textbf{producciones} que tienen la siguiente forma, donde $x\in I\cup N\cup \Sigma$ y $y\in I\cup N$:
            \begin{itemize}
                  \item $A \underset{\varepsilon}{\to} \alpha$ o $A \to \alpha$ (reglas epsilon o reglas libres del contexto).
                  \item $A \underset{[y]}{\to}  \alpha$ o $[..]A \to [..]\alpha$ (reglas epsilon o reglas con restricciones).
                  \item $A \underset{x}{\to} a \beta$ o $[..]A \to  [x..]a\beta$ (reglas de push o apertura de paréntesis).
                  \item $A \underset{\overline{x}}{\to} \alpha$ o $[x..]A \to [..]\alpha$ (reglas de pop o cierre de paréntesis).
            \end{itemize}
\end{itemize}

Como se puede observar la primera regla de producción consiste en dejar la pila intacta y puede ser interpretada como una regla de derivación
libre del contexto, la segunda regla consiste en dejar la pila intacta pero solo se puede realizar si el caracter en el tope de la pila es el
especificado, la tercera regla consiste en añadir un caracter a la pila y la cuarta regla consiste en eliminar un caracter de la pila.
Una gramática GIG solo con producciones de la primera regla de producción es equivalente a una CFG.

\subsubsection{Proceso de derivación}

Como se mencionó anteriormente el proceso de derivación en las GIG es idéntico al proceso de derivación de las CFG, con la diferencia que
en cada paso de la derivación se puede realizar una operación de push o pop en la pila, además de las operaciones de sustitución de símbolos.
Otra restricción adicional es que el proceso de derivación en las GIG debe ser siempre de extrema izquierda.

Entonces una cadena es reconocida por una GIG si existe una secuencia de derivaciones desde $S$ que genere la cadena y que además la pila
termine vacía al final de la derivación (con el símbolo $\#$ en el tope de la pila). Luego se puede definir el lenguaje generado por una GIG, $G$
como $L(G)=\{w\,|\,\#S\overset{*}{\to}\#w \wedge w\in \Sigma^* \}$.

\subsection{Propiedades de las GIG}

A continuación se describen las principales propiedades de las GIG:
\begin{itemize}
      \item \textbf{Cerradura bajo unión:} Dadas dos GIG $G_1$ y $G_2$, la unión de los lenguajes reconocidos por $G_1$ y $G_2$ es reconocida por una GIG
            $$G=(N_1\cup N_2\cup \{S\},\Sigma_1\cup \Sigma_2,I_1\cup I_2,S,\#,P_1\cup P_2\cup \{S \underset{\varepsilon}{\to} S_1|S_2\})$$
      \item \textbf{Cerradas bajo concatenación:} Dadas 2 GIG $G_1$ y $G_2$, la concatenación de los lenguajes reconocidos por $G_1$ y $G_2$ es reconocida por una GIG
            $$G=(N_1\cup N_2\cup \{S\},\Sigma_1\cup \Sigma_2,I_1\cup I_2,S,\#,P_1\cup P_2\cup \{S \underset{\varepsilon}{\to} S_1S_2\})$$
      \item \textbf{Cerradas bajo clausura de Kleene:} Dada una GIG $G$, el lenguaje reconocido por $G$ elevado a la clausura de Kleene es reconocido por una GIG
            $$G=(N\cup \{S'\},\Sigma,I\cup \{S'\},S,\#,P\cup \{S'\underset{\varepsilon}{\to} S'S|\varepsilon\})$$
      \item  \textbf{Cerradas bajo homomorfismo} Dada una GIG $G$, el homomorfismo de un lenguaje reconocido por $G$ es un lenguaje de índice global \cite{globalIndexLanguages}.
      \item  \textbf{Cerradas bajo transducción finita:} Dada una GIG $G$, la transducción finita de un lenguaje reconocido por $G$ es un lenguaje de índice global \cite{globalIndexLanguages}.
\end{itemize}

\subsection{Gramáticas de Concatenación de Rango}

Las gramáticas de concatenación de rango (\textit{RCG}) \cite{mainRCGBib} son un formalismo de gramáticas desarrollado para describir lenguajes más generales que los generados por gramáticas libres del contexto.
Este formalismo extiende las capacidades descriptivas al incluir relaciones entre rangos de la cadena de una manera más flexible,
permitiendo la generación de lenguajes sensibles al contexto.

\subsubsection{Definiciones}

\paragraph{Rango:} un rango es una tupla $(i, j)$ que representa un intervalo de posiciones en la cadena, donde $i$ y $j$ son enteros no negativos tales que $i \leq j$.

\paragraph{Gramática de Concatenación de Rango Positiva:} una gramática de concatenación de rango positiva (\textit{PRCG}) se define como una 5-tupla:

\[
      G = (N, T, V, P, S),
\]
donde:

\begin{itemize}
      \item $N$: Es un conjunto finito de \textbf{predicados o símbolos no terminales}: Cada predicado tiene una \textbf{aridad}, que indica el número de argumentos que toma.
      \item $T$: Es un conjunto finito de \textbf{símbolos terminales}.
      \item $V$: Es un conjunto finito de \textbf{variables}.
      \item $P$: Es un conjunto finito de \textbf{cláusulas}, de la forma:
            \[
                  A(x_1, x_2, \ldots, x_k) \to B_1(y_{1,1}, y_{1,2}, \ldots, y_{1,m_1}) \ldots B_n(y_{n,1}, y_{n,2}, \ldots, y_{n,m_n}),
            \]
            donde $A, B_i \in N$, $x_i, y_{i,j} \in (V \cup T)^*$, y $k$ es la aridad de $A$.
      \item $S \in N$: Es el \textbf{predicado inicial} de la gramática.
\end{itemize}

\paragraph{Gramática de Concatenación de Rango Negativa:} una gramática de concatenación de rango negativa (\textit{NRCG}) es similar a una PRCG, pero predicados o no terminales negativos que se denotan de la siguiente manera: $\overline{A}$.

\paragraph{Gramática de Concatenación de Rango Simple:} las gramáticas de concatenación de rango simple (\textit{SRCG}) son un subconjunto de las RCG que restringen la forma de las cláusulas de producción.
Una RCG $G$ es \textbf{simple} si los argumentos en el lado derecho de una cláusula son variables distintas, y todas estas variables (y no otras) aparecen una sola vez en los argumentos del lado izquierdo.
Un resultado interesante es que para cada CFL existe una SRCG equivalente que genera el mismo lenguaje.

\paragraph{Sustiución de rango:} una sustitución de rango es un mecanismo que reemplaza una variable por un rango de la cadena.
Por ejemplo dado el predicado $A(Xa)$ donde $X \in V \wedge a \in T$, si se instancia la cadena $baa$ en $A$, $X$ puede
ser asociada con el rango $ba$ de la cadena original.

\subsubsection{Proceso de derivación}

La principal idea detrás de las RCG, para realizar una derivación, se basa en encontrar para cada argumento del predicado izquierdo de una cláusula todas las
posibles sustituciones en rango de la cadena, asociar los valores de las variables a los argumentos de los predicados derechos y continuar
el proceso de derivación en los predicados derechos.

Por ejemplo, dada la cláusula $A(X,aYb)\to B(aXb,Y)$ , donde $X$ y $Y$ son símbolos variables y $a$ y $b$
son símbolos terminales, la cadena predicado $A(a,abb)$ deriva como $B(aab,b)$, porque $A(a,abb)$
coincide con $A(X,aYb)$ cuando $ X=a \wedge Y=b$. De forma similar, si existiera una regla

Una secuencia de argumentos son reconocidos por un predicado si existe una secuencia de derivaciones que comienza
en dicho predicado y termina en la cadena vacía, si el predicado es negativo en el caso de las NRCG ocurre lo contrario
la secuencia de argumentos no es reconocida por el predicado. Una RCG reconoce una cadena si dicha cadena es reconocida
por el predicado inicial.

Ejemplo dada la siguiente RCG:

\[
      G = (N, T, V, P, S),
\]
donde:

\begin{itemize}
      \item  N=$\{A,S\}$.
      \item T=$\{a,b,c\}$.
      \item V=$\{X,Y,Z\}$.
      \item El conjunto de cláusulas $P$ es el siguiente:
            $$S(XYZ)\to A(X,Y,Z)$$
            $$A(aX,aY,aZ)\to A(X,Y,Z)$$
            $$A(bX,bY,bZ)\to A(X,Y,Z)$$
            $$A(cX,cY,cZ)\to A(X,Y,Z)$$
            $$A(\varepsilon,\varepsilon,\varepsilon)\to \varepsilon$$
      \item El símbolo inicial es $S$.
\end{itemize}
La cadena $aaabbbccc$ es reconocida por la RCG anterior, ya que se puede derivar de la siguiente manera:
$$S(abcabcabc)\to A(abc,abc,abc)\to A(bc,bc,bc)\to A(c,c,c)\to A(\varepsilon,\varepsilon,\varepsilon)\to \varepsilon$$

De manera general el lenguaje reconocido por la RCG anterior es $L=\{www\,|\,w\in \{a,b,c\}^*\}$.

\subsubsection{Propiedades de las RCG}

A continuación se describen las principales propiedades de las RCG:
\begin{itemize}
      \item \textbf{Cerradura bajo unión:} Dadas dos RCG $G_1$ y $G_2$, la unión de los lenguajes reconocidos por $G_1$ y $G_2$ es reconocida por una RCG
            $$G=(N_1\cup N_2\cup \{S\},T_1\cup T_2,V_1\cup V_2,P_1\cup P_2\cup \{S(X)\to S_1(X)|S_2(X)\},S)$$
      \item \textbf{Cerradas bajo intersección:} Dadas dos RCG $G_1$ y $G_2$, la intersección de los lenguajes reconocidos por $G_1$ y $G_2$ es reconocida por una RCG
            $$G=(N_1\cup N_2\cup \{S\},T_1\cup T_2,V_1\cup V_2,P_1\cup P_2\cup \{S(X)\to S_1(X)S_2(X)\},S)$$
      \item \textbf{Cerradas bajo complemento:} Dada una RCG $G$, el complemento del lenguaje reconocido por $G$ es reconocido por una RCG
            $$G'=(N\cup \{\overline{S}\},T,V,P\cup \{S'(X)\to \overline{S}(X)\},S')$$
      \item \textbf{Cerradas bajo concatenación:} Dadas dos RCG $G_1$ y $G_2$, la concatenación de los lenguajes reconocidos por $G_1$ y $G_2$ es reconocida por una RCG
            $$G=(N_1\cup N_2\cup \{S\},T_1\cup T_2,V_1\cup V_2,P_1\cup P_2\cup \{S(XY)\to S_1(X)S_2(Y)\},S)$$
      \item \textbf{Cerradas bajo clausura de Kleene:} Dada una RCG $G$, la clausura de Kleene del lenguaje reconocido por $G$ es reconocida por una RCG
            $$G'=(N\cup \{S'\},T,V,P\cup \{S'(XY)\to S(X)S'(Y)|\varepsilon\},S')$$
      \item  \textbf{No cerradas bajo homomorfismo:} Dada una RCG $G$, el homomorfismo de un lenguaje reconocido por $G$ no es necesariamente reconocido por una RCG \cite{propertiesRCGBib}.
      \item  \textbf{No cerradas bajo transducción finita:} Dada una RCG $G$, la transducción finita de un lenguaje reconocido por $G$ no es necesariamente reconocida por una RCG.
            Esto es una consecuencia de la no cerradura bajo homomorfismo.
\end{itemize}


\subsubsection{Problema de la palabra, problema del vacío y equivalencia de 2 RCG}

\paragraph{Problema de la palabra:} En general en la mayoría de los casos este problema es polinomial y pasa por
un algoritmo de memorización sobre las cadenas que son instanciadas en los rangos de los predicados de la RCG \cite{mainRCGBib} (como la cantidad
máxima de rangos de la cadena es $n^2$ y la máxima aridad de un predicado es constante, este proceso de memorización cuenta
con cantidad polinomial de estados), en
una complejidad de $O(|P|n^{2h(l+1)})$ donde $h$ es la máxima aridad en un predicado, $l$ es la máxima cantidad de predicados
en el lado derecho de una cláusula y $n$ es la longitud de la cadena a ser reconocida.

Pero existen casos en los que el problema de la palabra no
es polinomial, por ejemplo puede pasar que se instancien argumentos de en los predicados con rangos que no pertenezcan
a la propia cadena de entrada y sean generados durante el proceso de reconocimiento.

\paragraph{Problema del vacío:} El problema del vacío para una RCG es indecidible \cite{propertiesRCGBib}, la razón principal para esto es que como se mencionó anteriormente
para toda CFL existe una RCG equivalente y como las RCG son cerradas bajo intersección existen RCG que describen
la intersección de 2 lenguajes libres del contexto y determinar si dicha intersección es vacía es un problema indecidible.

En el caso de las SRCG este problema es polinomial \cite{mainRCGBib}.

\paragraph{Problema de la equivalencia:} El problema de la equivalencia para 2 RCG es indecidible, la demostración es muy sencilla dadas 2 RCG $G_1$ y $G_2$ el problema
de saber si $G_1$ es equivalente a $G_2$ es equivalente a saber si $G_1\cap \overline{G_2}=\emptyset$ y como se mencionó anteriormente el problema del vacío para una RCG
es indecidible.

\section{Problema de la satisfacibilidad booleana}

El problema de la satisfacibilidad booleana (\textit{SAT}), es un problema fundamental en la teoría de la computación y la lógica matemática. El objetivo principal del problema es determinar si existe una asignación de valores a las variables de una expresión booleana tal que la expresión sea verdadera.

\subsection{Variables booleanas}

Una variable booleana es una variable que puede tomar uno de dos valores posibles: \textit{true} (verdadero) o \textit{false} (falso). Estas variables se utilizan para construir expresiones lógicas.

\subsection{Literales}

Un literal es una variable booleana o su negación. Formalmente, si \( x \) es una variable booleana, entonces \( x \) y \( \neg x \) (la negación de \( x \)) son literales. Un literal puede tomar los valores \( true \) o \( false \) dependiendo de la asignación de valores a las variables.

\subsection{Cláusulas}

Una cláusula es una disyunción (operador \textbf{OR}) de uno o más literales. Por ejemplo, la cláusula \( (x \vee \neg y \vee z) \) es una disyunción de tres literales: \( x \), \( \neg y \) y \( z \). Una cláusula es verdadera si al menos uno de sus literales es verdadero. Si todos los literales son falsos, la cláusula será falsa.

\subsection{Fórmulas en forma normal conjuntiva}

Una fórmula booleana en forma normal conjuntiva (\textit{CNF}) es una conjunción (operador \textbf{AND}) de cláusulas. En otras palabras, es una expresión booleana que se puede escribir como una serie de cláusulas unidas por el operador \textbf{AND}. Por ejemplo:

\[
      (x \vee \neg y \vee z) \wedge (\neg x \vee y) \wedge (x \vee \neg z)
\]

\subsection{Fórmulas booleanas equivalentes}

Dos fórmulas booleanas se consideran equivalentes si, para cualquier asignación de valores a sus variables, ambas producen el mismo resultado lógico. Por ejemplo, las fórmulas \( x \vee (y \wedge z) \) y \( (x \vee y) \wedge (x \vee z) \) son equivalentes, ya que para cualquier combinación de valores \( x, y, z \), ambas tienen el mismo valor lógico.

Para cualquier fórmula booleana existe una fórmula booleana equivalente en CNF y el algoritmo para encontrarla es polinomial, por lo tanto
de aquí se puede asumir que toda fórmula booleana está en CNF.

\subsection{Definición del problema de la satisfacibilidad booleana}

El problema de la satisfacibilidad booleana, o SAT, consiste en determinar si existe una asignación de valores \( true \) o \( false \) a las variables de una fórmula booleana tal que la fórmula completa sea verdadera. En términos formales, dado un conjunto de cláusulas en CNF, el problema es encontrar una asignación de valores a las variables que haga que la conjunción de las cláusulas sea verdadera.

Formalmente, se dice que una fórmula booleana en CNF es satisfacible si existe una asignación de valores a las variables tal que todas las cláusulas de la fórmula sean verdaderas simultáneamente.

\begin{itemize}
      \item Si existe tal asignación, la fórmula es \textit{satisfacible}.
      \item Si no existe tal asignación, la fórmula es \textit{insatisfacible}.
\end{itemize}

Un SAT con exactamente $n$ variables distintas se denomina $n$-SAT.

\subsection{SAT como Problema NP-Completo}

El SAT es el primer problema demostrado como NP-Completo \cite{authomataTheory} y juega un rol central en la teoría de la complejidad computacional. Se define en la clase NP porque, dada una asignación de valores a las variables de la fórmula booleana, se puede verificar en tiempo polinómico si dicha asignación satisface la fórmula.

Además, la prueba de que SAT es NP-Completo fue una de las contribuciones principales de Stephen Cook en 1971, marcando el inicio de la teoría de la NP-completitud.

\subsection{Equivalencia entre SAT y 3-SAT}

Para el problema 2-SAT existe una solución polinomial que determina si la fórmula booleana es satisfacible o no, pero para el problema 3-SAT no se conoce ningún algoritmo que permita
determinar si una fórmula booleana es satisfacible o no.

Cualquier fórmula booleana del problema $n$-SAT puede ser reducida a una fórmula booleana equivalente del problema 3-SAT, por lo tanto, SAT es equivalente a 3-SAT en términos de complejidad computacional.

\paragraph{Transformación de SAT a 3-SAT:}

Dada una fórmula en CNF con cláusulas de \( k > 3 \) literales, se puede convertir a 3-CNF introduciendo variables adicionales. Por ejemplo, considere una cláusula de cuatro literales:

\[
      (a \vee b \vee c \vee d)
\]

Esta cláusula puede reescribirse como un conjunto de cláusulas en 3-CNF introduciendo una nueva variable \( x \):

\[
      (a \vee b \vee x) \wedge (\neg x \vee c \vee d)
\]

Este proceso se puede aplicar iterativamente para todas las cláusulas con más de tres literales, asegurando que la nueva fórmula sea satisfacible si y solo si la fórmula original también lo es.

\subsection{Problemas SAT solubles en tiempo polinomial}

Como se mencionó anteriormente no se conoce ningún algoritmo polinomial para resolver el problema SAT en general, pero
existen casos particulares del problema que sí pueden ser resueltos en tiempo polinomial. A continuación se presentan los
principales casos:

\begin{enumerate}
      \item \textbf{1-SAT:} El problema 1-SAT es una instancia particular de SAT donde cada cláusula tiene a lo sumo un literal.
            Este problema puede ser resuelto en tiempo polinomial mediante un algoritmo de asignación de valores de verdad.
      \item \textbf{2-SAT:} Como se mencionó anteriormente, el problema 2-SAT puede ser resuelto en tiempo polinomial mediante
            una modelación basada en grafos.
      \item \textbf{Horn-SAT:} El problema Horn-SAT es una generalización del problema 2-SAT, donde cada cláusula tiene a lo sumo
            un literal positivo. Este problema puede ser resuelto en tiempo polinomial mediante el algoritmo de resolución de Horn.
\end{enumerate}

% \chapter{Preliminares}
% \label{chap:preliminaries}

\chapter{Gramáticas de concatenación de rango}
\label{chap:RCG}

% \chapter{Lenguaje de las fórmulas booleanas satisfacibles empleando transducción finita}
% \label{chap:LSATFT}

% \chapter{Lenguaje de las fórmulas booleanas satisfacibles empleando gramáticas de concatenación de rango}
% \label{chap:LSATRCG}


Las gramáticas de concatenación de rango (\textit{RCG}) \cite{mainRCGBib} son un formalismo de gramáticas desarrollado
en 1988 como una propuesta de Pierre Boullier, un investigador en el campo de la lingüística computacional. Su
objetivo principal era proporcionar un modelo más general y expresivo que las CFG para describir lenguajes.
Las RCG fueron diseñadas con el fin de analizar propiedades y características del lenguaje natural.

Las gramáticas de concatenación de rango se emplean en el capítulo \ref{chap:LSATRCG} para construir una gramática que 
reconozca las fórmulas booleanas satisfacibles. 

En la próxima sección se presentan algunas nociones que sirven de introducción para las principales definiciones y conceptos
de las gramáticas de concatenación de rango.

\section{Presentación de los elementos de las gramáticas de concatenación de rango}

En esta sección se presentan nociones sobre las sustitución en rango y las derivaciones de las RCG, aspectos que sirven como base introductoria para comprender los 
conceptos y definiciones relacionados con las gramáticas de concatenación de rango.

A los no terminales de esta gramática se les llama predicados, cada predicado tiene una secuencia de argumentos, a la cantidad 
de argumentos de un predicado se le denomina aridad. Los argumentos de los predicados pueden estar formados por variables y terminales.
Cada predicado reconoce un vector de cadenas con la misma dimensión de la aridad del predicado y cada cadena del vector
se asocia a un argumento del predicado.

A las producciones de esta gramática se les denomina cláusulas y cada cláusula puede tener la siguiente forma:
$$A(XYZ,W)\to B(X)C(Y,Z)D(W).$$

% \comment{Para ello suponga que tiene una gramática con la siguiente regla de derivación:
% $$A(XYZ)\to B(X)C(Y)D(Z).$$}{abusador :-/.}

% \notaparaelautor{Está mucho mejor que antes :-/, pero vamos más despacito. Lo primero que vamos a decir es que las producciones de las RCG pueden ser ¿son? de la forma A(XY) rayita B(X)C(Y,X).}

% \notaparaelautor{quizás otra vía es que los noterminales de esta gramática reciben argumentos. O sea, se puede tener un noterminal A, y que aparezca en una producción como A(XY) rayita ...}

% \notaparaelautor{Al noterminal también se le llama cláusula, y a los argumentos de la cláusula se les llama variables. en el caso del ejemplo, el noterminal A tiene dos variables: X e Y.}

% \notaparaelautor{El otro detalle es que los no terminales reciben una cadena... por ejemmplo, el noterminal A (de arriba) puede recibir la cadena w=0101.}

% \notaparaelautor{Si se tiene una producción de la forma A(XY) rayita ... y A recibe la cadena w=0101, lo que pasa es que las variables X e Y toman como valor todas las posibles subcadenas de w, que respeten la forma en que aparecen en la definición de A. Como en este caso A tiene la forma A(XY), X e Y pueden tomar los siguientes valores: X=0, Y =101, o X=01, Y=01... y así con todas las posibles formas de separar 0101 en dos partes consecutivas.}

% \notaparaelautor{Una vez que las variables X e Y asumieron un valor, esos valores \textit{pasan} hacia la parte derecha de la producción.}

% \notaparaelautor{por ejemplo, en una produccionde la forma A(XY) rayita B(X)C(X,Y),
%     donde A recibe a la cadena w=0101 ocurre lo siguiente: }

% \notaparaelautor{Lo primero es que X e Y toman alguno de los posibles valores de w. por ejemplo, X=0, Y =101. A esto se le llama \textit{hacer una asignación de rangos} (o como se llame). El segundo paso es que con esos rangos asignados, se instancian los no terminales de la parte derecha, con los valores de X e Y.}

% \notaparaelautor{En este caso B(X) sería B(0) y C(X,Y) sería C(0, 101).}

% \notaparaelautor{Llegado este punto, se repite el proceso, ahora con los noterminales B y C.}

% \notaparaelautor{Cuando se analizan todos las posibles derivaciones para B y C con esos valores de X e Y, se le asignan nuevos valores a X e Y y se repite el proceso.}

% \notaparaelautor{Por ejemplo, otra asignación de rangos puede ser X=01 e Y =01. Los noterminales B y C se instanciarían como B(01) y C(01, 01)... y así con todas las posibles asignaciones de rango.}

% \notaparaelautor{Combina lo que te acabo de escribir con lo que tú tienes, para que quede algo masticadito.}

La regla anterior significa que el no terminal $A$ recibe un vector de cadenas de dimensión 2, y las variables que se encuentran como primer argumento
de $A$: $XYZ$, significan todas las formas de dividir la primera cadena del vector que recibe $A$ en 3 subcadenas de la cadena que no se solapen
y que su concatenación forme la cadena original. Por ejemplo, si el no terminal $A$ recibe el vector $(abc,d)$ los valores de $X$, $Y$, $Z$ y $W$
pueden ser interpretados de la siguiente manera (Figura \ref{fig:xyz_eaxmple}):

\begin{figure}
    \centering
    \begin{tabular}{|c|c|c|c|}
        \hline
        X             & Y             & Z             & W \\
        \hline
        a             & b             & c             & d \\
        \hline
        ab            & $\varepsilon$ & c             & d \\
        \hline
        ab            & c             & $\varepsilon$ & d \\
        \hline
        abc           & $\varepsilon$ & $\varepsilon$ & d \\
        \hline
        $\varepsilon$ & ab            & c             & d \\
        \hline
        $\varepsilon$ & abc           & $\varepsilon$ & d \\
        \hline
        $\varepsilon$ & $\varepsilon$ & abc           & d \\
        \hline
        a             & $\varepsilon$ & bc            & d \\
        \hline
        $\vdots$      & \vdots        & \vdots        & d \\
        \hline
        a             & bc            & $\varepsilon$ & d \\
        \hline
    \end{tabular}
    \caption{Posibles valores de las variables $X$, $Y$ y $Z$}
    \label{fig:xyz_eaxmple}
\end{figure}

Entonces el primer paso es asignarle a cada cadena del vector de entrada el argumento correspondiente 
en el predicado. Luego para cada cadena asociada a un argumento, asociar cada variable del argumento
a una subcadena de la cadena de entrada.

Suponga que fue la primera, en la que $X=a$, $Y = b$, $Z = c$, $W=d$, y con esa asignación de variables
se evalúa en los no terminales de su parte derecha: $B(X)C(Y,Z)D(W)$, que en este caso sería $B(a)C(b,c)D(d)$.
Este proceso se repite en cada uno de los predicados del lado derecho de la cláusula.

Las siguientes cláusulas de la gramática son:
$$B(a)\to \varepsilon,$$
$$C(b,c)\to \varepsilon,$$
$$D(d)\to \varepsilon.$$
Si se continúa el proceso de derivación, $B(a)$, $C(b,c)$ y $D(d)$ derivan en la cadena vacía, cuando esto pasa se dice
que $B$, $C$ y $D$ reconocen los vectores de cadenas $(a)$, $(b,c)$ y $(d)$ respectivamente. A su vez, $A$ reconoce el
vector $(abc,d)$ ya que existe una derivación desde $A(abc,d)$ a $B(a)C(b,c)D(d)$ y cada uno de estos predicados deriva en la cadena vacía. 

Con otra asignación de valores a las variables $X$, $Y$, $Z$ y $W$, se tiene por ejemplo que si $X=ab$, $Y = \varepsilon$, $Z=c$, $W=d$
entonces la parte derecha de $A(XYZ,W)$ se evalúa de la siguiente forma: $B(ab)C(\varepsilon,c)D(d)$.

Con la idea anterior se puede hablar del concepto de rango, que no es más que un par de índices $i$ y $j$, tales que $i\leq j$, estos
representan la subcadena de la cadena de entrada que comienza en el $i$-ésimo caracter y termina en el $j$-ésimo caracter.

El concepto de rango se utiliza cuando se evalúa en un no terminal y se le asigna a cada variable un rango de la cadena, tales
que estos no se solapen, como se mostró en el ejemplo anterior.

Dadas estas nociones, a continuación se presentan las principales definiciones de las gramáticas de concatenación de rango.

\section{Definiciones}

En esta sección se definen los principales conceptos de las gramáticas de concatenación de rango.

\paragraph{Rango:} un rango es una tupla $(i, j)$ que representa un intervalo de posiciones en una cadena, donde $i$ y $j$ son enteros no negativos tales que $i \leq j$.

\paragraph{Gramática de Concatenación de Rango:} una gramática de concatenación de rango se define como una 5-tupla
\footnote{En la literatura este tipo de RCG se toma com gramática de concatenación de rango positiva, pero como es la única que
    se usa en este trabajo se le llama solo gramática de concatenación de rango}:
\[
    G = (N, T, V, P, S),
\]
donde:

\begin{itemize}
    \item $N$: Es un conjunto finito de \textbf{predicados o símbolos no terminales}: Cada predicado tiene una \textbf{aridad}, que indica la dimensión del vector de cadenas que reconoce y cada cadena del vector se asocia a un argumento del predicado.
    \item $T$: Es un conjunto finito de \textbf{símbolos terminales}.
    \item $V$: Es un conjunto finito de \textbf{variables}.
    \item $P$: Es un conjunto finito de \textbf{cláusulas}, de la forma:
          \[
              A(x_1, x_2, \ldots, x_k) \to B_1(y_{1,1}, y_{1,2}, \ldots, y_{1,m_1}) \ldots B_n(y_{n,1}, y_{n,2}, \ldots, y_{n,m_n}),
          \]
          donde $A, B_i \in N$, $x_i, y_{i,j} \in (V \cup T)^*$, y $k$ es la aridad de $A$.
    \item $S \in N$: Es el \textbf{predicado inicial} de la gramática, que siempre tiene \textbf{aridad} 1.
\end{itemize}

Por ejemplo, la siguiente gramática reconoce el lenguaje $L^3_{copy}=\{www\,|\,w\in \{a,b,c\}^*\}$:
\label{g_3copy}
\[
    G^3_{copy} = (N, T, V, P, S),
\]
donde:

\begin{itemize}
    \item  N=$\{A,S\}$.
    \item T=$\{a,b,c\}$.
    \item V=$\{X,Y,Z\}$.
    \item El conjunto de cláusulas $P$ es el siguiente:
          \begin{enumerate}
              \item $S(XYZ)\to A(X,Y,Z)$
              \item $A(aX,aY,aZ)\to A(X,Y,Z)$
              \item $A(bX,bY,bZ)\to A(X,Y,Z)$
              \item $A(cX,cY,cZ)\to A(X,Y,Z)$
              \item $A(\varepsilon,\varepsilon,\varepsilon)\to \varepsilon$
          \end{enumerate}
    \item El símbolo inicial es $S$.
\end{itemize}


\paragraph{Gramática de Concatenación de Rango Simple:} las gramáticas de concatenación de rango simple 
(\textit{SRCG}) son un subconjunto de las RCG que restringen la forma de las cláusulas de producción.  
Una RCG $G$ es \textbf{simple} si los argumentos en el lado derecho de una cláusula son variables distintas, 
y todas estas variables (y no otras) aparecen una sola vez en los argumentos del lado izquierdo.  
Este es un caso particular de las RCG el cual se usa en \cite{aSRCSAT} para describir el orden de las variables de una fórmula booleana.

\paragraph{Sustitución de rango:} una sustitución de rango es un mecanismo que reemplaza una variable por un 
rango de la cadena, respetando la estructura del argumento que se asocia a la cadena que se reconoce. 

Por ejemplo, dado el predicado $A(Xa)$ donde $X \in V$ y $a \in T$, si se instancia en $A$ la cadena $baa$, $X$ puede ser asociada con el rango $ba$ de la cadena original. Mientras
que la cadena $X$ no puede ser asociada al rango $baa$, ni al rango $b$ porque en el primer caso no hay ningún rango de la cadena
que coincida con el terminal $a$ y en el segundo caso la sustitución en rango no cubre por completo la cadena.

En la próxima sección se describe el proceso de derivación de las RCG.
\section{Proceso de derivación}

La idea principal para realizar una derivación en las RCG se basa en tomar un vector de cadenas y asociar cada 
cadena al argumento correspondiente del predicado de la parte izquierda de la cláusula. Después se identifican 
todas las posibles sustituciones en rango para cada argumento y se asocia un rango a cada variable del predicado
izquierdo. Los valores de las variables obtenidos en el paso anterior se asocian a las variables de los predicados
del lado derecho de la cláusula y se continúa el proceso de derivación en cada uno de los predicados del lado
derecho.

Por ejemplo, se tiene la cláusula $A(X,aYb)\to B(aXb,Y)$ , donde $X$ e $Y$ son variables y $a$ y $b$ son símbolos terminales, cuando $ X=a$ y $Y=b$, $a$ coincide con $X$ y $abb$ coincide con $aYb$, entonces el predicado $A(a,abb)$ deriva como $B(aab,b)$.

Las RCG, a diferencia de las gramáticas definidas en la sección \ref{sec:grammars} del capítulo \ref{chap:preliminaries} no generan cadenas, su funcionamiento se basa en reconocer si una cadena pertenece o no al lenguaje.

Un vector de cadenas se reconoce por un predicado $A$ si existe una secuencia de derivaciones que comienza en $A$ y termina en la cadena vacía.

Por ejemplo, dada la cláusula $A(X_1,X_2,X_3)\to B_1(X_1)B_2(X_2)B_3(X_3)$, el vector $(w_1,w_2,w_3)$ se reconoce por $A$, si existe una secuencia de derivaciones para cada uno de los predicados $B_1(w_1)$, $B_2(w_2)$, $B_3(w_3)$ que derive en la cadena vacía.

A continuación se presenta un ejemplo de reconocimiento de la cadena $abcabcabc$ por la gramática $G^3_{copy}$
presentada en la página \pageref{g_3copy}.

La cadena $abcabcabc$ se reconoce por $G^3_{copy}$, ya que $S(abcabcabc)$ se puede derivar de la siguiente manera:
$$S(abcabcabc)\to A(abc,abc,abc)\to A(bc,bc,bc)\to A(c,c,c)\to A(\varepsilon,\varepsilon,\varepsilon)\to \varepsilon.$$

Para mostrar un ejemplo de sustitución en rango sobre la cadena $w$ se define $w[i\dots j]$ como el rango que va desde el $i$-ésimo caracter hasta el $j$-ésimo con la cadena indexada en 0.

Entonces, si $w=abcabcabc$, al realizar el reconocimiento en rango sobre el predicado $S$ se pueden asociar las variables $X=ab$, $Y=ca$ $Z=bcabc$ a los rangos $w[0\dots 1]$, $w[1\dots 2]$, $w[3\dots 8]$, respectivamente.

Otra opción es asociar las variables $X=abca$, $Y=bc$ $Z=abc$ a los rangos $w[0\dots 3]$, $w[4\dots 5]$, $w[6\dots 8]$, respectivamente. De manera similar, se pueden hacer $(^8_3)$ sustituciones en rango distintas. 

El proceso de reconocimiento de la cadena $abcabcabc$ por la gramática $G^3_{copy}$ se detalla como sigue.

\begin{itemize}
    \item \textbf{Primer paso:} Se toma la primera cláusula, la sustitución en rango asocia las variables
          $X=abc$, $Y=abc$ y $Z=abc$ a los rangos $w[0\dots 2]$, $w[3\dots 5]$ y $w[6\dots 8]$.
    \item \textbf{Segundo paso:} Se toma la segunda cláusula, la sustitución en rango asocia las variables $X=bc$, $Y=bc$ y $Z=bc$ a los rangos $w[1\dots 2]$,
          $w[4\dots 5]$ y $w[7\dots 8]$ respectivamente, derivando en el predicado $A(bc,bc,bc)$.
    \item \textbf{Tercer paso:} Se toma la tercera cláusula, la sustitución en rango asocia las variables $X=c$, $Y=c$ y $Z=c$ a los
          rangos $w[2\dots 2]$, $w[5\dots 5]$ y $w[8\dots 8]$ respectivamente, derivando en el predicado $A(c,c,c)$.
    \item \textbf{Cuarto paso:} Se toma la cuarta cláusula, la sustitución en rango asocia las variables
          $X=\varepsilon$, $Y=\varepsilon$ y $Z=\varepsilon$ respectivamente, derivando en el predicado
          $A(\varepsilon,\varepsilon,\varepsilon)$.
    \item \textbf{Quinto paso:} Finalmente en el último paso se toma la última
          cláusula que deriva en la cadena vacía, por lo que de esta manera se reconoce la cadena $abcabcabc$.
\end{itemize}

A continuación se presentan las propiedades de las RCG que demuestran que las RCG no son cerradas
bajo transducción finita.

\section{Propiedades de las RCG}

En esta sección se describen las principales propiedades de las RCG que demuestran que las RCG no 
son cerradas bajo transducción finita \cite{propertiesRCGBib}.

\begin{itemize}
    \item  \textbf{No cerradas bajo homomorfismo:} Dada una RCG $G$, el homomorfismo de un lenguaje que se reconoce por $G$ necesariamente no se reconoce por una RCG \cite{propertiesRCGBib}.
    \item \textbf{No cerradas bajo transducción finita:} Dada una RCG $G$, la transducción finita de un lenguaje que se reconoce por $G$ necesariamente no se reconoce por una RCG.  Esto es una consecuencia de la propiedad anterior ya que como se mencionó en el capítulo anterior un homomorfismo es un caso particular de un transductor finito.
\end{itemize}


En la siguiente sección se analiza el problema de la palabra para las RCG, el cual se utiliza en el
capítulo \ref{chap:LSATRCG}, para construir el lenguaje de todas las fórmulas booleanas satisfacibles mediante
una RCG.

\section{Problema de la palabra}

En \cite{mainRCGBib} se menciona que el problema de la palabra para las RCG es polinomial y 
se resuelve mediante un algoritmo de memorización sobre las cadenas asignadas a los argumentos 
de los predicados de la RCG \cite{mainRCGBib}.  Como la cantidad máxima de rangos de la cadena es 
$n^2$ y la máxima aridad de un predicado es constante, 
este proceso de memorización cuenta con una cantidad polinomial de estados, y 
tiene una complejidad de $O(|P|n^{2h(l+1)})$ donde $h$ es la máxima aridad en un predicado, $l$ es 
la máxima cantidad de predicados en el lado derecho de una cláusula y $n$ es la longitud de la cadena que se reconoce.

Existen casos en los que el problema de la palabra no es polinomial \cite{propertiesRCGBib}.  El ejemplo presentado en \cite{propertiesRCGBib} muestra una RCG que reconoce cadenas de unos, donde la cantidad de unos es un cuadrado perfecto, en la siguiente sección se analiza otro caso en el que este problema no es polinomial.

\subsection{Problema de la palabra no polinomial para las RCG}

El algoritmo de reconocimiento que se menciona en la sección anterior utiliza un proceso de memorización sobre los rangos de la cadena.  La idea fundamental para esto y lo que acota la complejidad del algoritmo es que la cantidad de estados asociados a la memorización es igual a la cantidad de rangos de la cadena, el cual es polinomial con respecto a la longitud de la cadena.

¿Qué pasaría si algún predicado de la gramática trabajara con cadenas que no son subcadenas de la cadena de entrada? 
En este caso, si se emplea el algoritmo anterior ya la complejidad no depende de la cantidad de rangos de la cadena de entrada 
porque pueden aparecer otras cadenas que se generan durante el reconocimiento de la cadena de entrada.

Por ejemplo, a continuación se presenta una RCG que reconoce el lenguaje $L=\{w\,|\,w\in\{0,1\}^*\}$. Esta RCG no tiene uso real porque existe otra RCG equivalente que reconoce el
mismo lenguaje, pero ilustra una RCG donde se generan cadenas que no son subcadenas de la cadena de entrada durante el proceso
de reconocimiento:
\[
    G = (N, T, V, P, S),
\]
donde:

\begin{itemize}
    \item  N=$\{A,B,Eq,S\}$.
    \item T=$\{0,1\}$.
    \item V=$\{X,Y\}$.
    \item El conjunto de cláusulas $P$ es el siguiente:
          \begin{enumerate}
              \item $S(X)\to A(X,X)$
              \item $A(1X,Y)\to B(X,0,Y)$
              \item $A(1X,Y)\to B(X,1,Y)$
              \item $A(0X,Y)\to B(X,1,Y)$
              \item $A(0X,Y)\to B(X,0,Y)$
              \item $B(1X,Y,Z)\to B(X,1Y,Z)$
              \item $B(1X,Y,Z)\to B(X,0Y,Z)$
              \item $B(0X,Y,Z)\to B(X,0Y,Z)$
              \item $B(0X,Y,Z)\to B(X,1Y,Z)$
              \item $B(\varepsilon,Y,Z)\to Eq(Y,Z)$
          \end{enumerate}
          
    \item El símbolo inicial es $S$.
\end{itemize}

El funcionamiento de la gramática anterior toma una cadena $w$, genera todas las posibles cadenas $q$, tales que 
$|w|=|q|$ y luego comprueba si $w = q$.

Como se dijo anteriormente, esta gramática no tiene caso de uso ya que para toda cadena $w$ siempre va a existir 
una cadena $q$ tal que $w=q$, por lo que se puede modelar con solamente la cláusula $S(X)\to \varepsilon$. 
Pero la complejidad del reconocimiento de $G$ es mayor que $2^n$ (con $n$ igual al tamaño de la cadena de entrada), 
ya que esta es la cantidad de rangos posibles que puede recibir el segundo argumento  del predicado $B$, porque la gramática es ambigua y en cada derivación de $B$ existen 2 posibles decisiones, se añade un 1 delante al valor de la $Y$ o se añade un $0$. 

En el capítulo \ref{chap:LSATRCG} se presenta una RCG ambigua con el problema de la palabra no polinomial, 
pero que reconoce fórmulas booleanas satisfacibles.

En este capítulo se analizaron las principales definiciones y propiedades de las RCG, que son utilizadas en el capítulo \ref{LSATRCG} para definir una gramática que reconozca las fórmulas booleanas satisfacibles.  En el próximo capítulo se presenta un primer enfoque para definir el lenguaje de todas las fórmulas booleanas satisfacibles y a esta idea se le da continuidad en el capítulo \ref{chap:LSATRCG}, mediante las RCG.


% \chapter{Preliminares}
% \label{chap:preliminaries}

% \chapter{Gramáticas de concatenación de rango}
% \label{chap:RCG}

\chapter{Lenguaje de las fórmulas booleanas satisfacibles empleando transducción finita}
\label{chap:LSATFT}

% \chapter{Lenguaje de las fórmulas booleanas satisfacibles empleando gramáticas de concatenación de rango}
% \label{chap:LSATRCG}

En este capítulo se presenta el lenguaje $L_{S-SAT}$, al cual pertenecen todos los problemas SAT que son 
satisfacibles, y se muestra una forma de construirlo a partir de una transducción finita de una variante 
del lenguaje $L_{copy}$ sobre el alfabeto $\{0,1,d\}$. Este lenguaje permitiría resolver instancias del SAT resolviendo el problema de la palabra.  

Para definir el lenguaje $L_{S-SAT}$ se presenta una vía para codificar una fórmula booleana mediante cadenas sobre el alfabeto $\Sigma=\{a, b,c,d\}$, y para construirlo se utiliza una transducción finita del lenguaje $L_{0,1,d}$, que es una variante del lenguaje $L_{copy}$ sobre el alfabeto $\{0,1,d\}$.

La estructura de este capítulo es la siguiente: en la sección \ref{sec:codsat} se muestra como codificar una fórmula booleana cualquiera usando el alfabeto $\{a,b,c,d\}$ y se define el lenguaje $L_{S-SAT}$.  En la sección \ref{sec:LSATCFL} se demuestra que $L_{S-SAT}$ no es un lenguaje libre del contexto.  En la sección \ref{sec:intsat} se muestra cómo interpretar las cadenas sobre el alfabeto $\{0,1,d\}$ como asignaciones de las variables. Finalmente, en la sección \ref{sec:tsat} se presenta un transductor finito que convierte cadenas del lenguaje $L_{0,1,d}$ en cadenas sobre el alfabeto $\{a,b,c,d\}$ que representan fórmulas booleanas satisfacibles. Seguidamente se conjetura por qué la representación del lenguaje de las fórmulas booleanas satisfacibles, en cualquier formalismo que lo genere usando la estrategia propuesta en este capítulo, tiene un tamaño $O(1)$. Esto implica que el problema de la palabra para todos estos formalismos es NP-Duro.


A continuación se presenta cómo codificar una fórmula booleana cualquiera mediante una cadena sobre el alfabeto $\{a,b,c,d\}$.

\section{Codificación de una fórmula booleana a una cadena}
\label{sec:codsat}

Una fórmula booleana $F$, con $v$ variables en CNF tiene la siguiente estructura:
$$F=X_1 \wedge X_2 \wedge \ldots \wedge X_n$$
donde cada cláusula $X_i$ es una disyunción de literales
$$X_i=L_{i1} \vee L_{i2} \vee \ldots \vee L_{im},$$
cada literal $L_{ij}$ es una variable booleana o su negación. También se asume que $m\leq v$. 

Si se tiene una fórmula booleana $F$ en forma normal conjuntiva se puede considerar que cada una de las $v$ variables aparece en cada cláusula en uno de tres posibles estados: sin negar, negada, o no aparece.

Por ejemplo, en la primera cláusula de la siguiente fórmula booleana en CNF con 3 variables:
$$F=(x_1 \vee \neg x_2) \wedge (\neg x_1 \vee x_2 \vee x_3) \wedge (x_1 \vee \neg x_2 \vee x_3)$$

la variable $x_1$ aparece sin negar, la variable $x_2$ aparece negada, y la variable $x_{3}$ no aparece.

El hecho de que se pueda asumir que en todas las cláusulas aparecen todas variables permite representar una cláusula de una fórmula con $v$ variables como una cadena de $v$ símbolos, donde el símbolo en la posición $i$ indica el estado de la variable $x_i$ en la cláusula.

En este trabajo se propone usar los símbolos $a$, $b$ y $c$ para indicar el estado de una variable en una cláusula, usando el siguiente convenio:

\begin{itemize}
    \item $a$: indica que la variable aparece sin negar,
    \item $b$: indica que la variable aparece negada,
    \item $c$: indica que la variable no aparece.
\end{itemize}

Con este convenio, la primera cláusula de $F$ se puede representar mediante la cadena $abc$.

Una vez que se tiene cómo representar una cláusula es posible representar varias cláusulas usando otro símbolo como separador.
En este trabajo se propone usar $d$ para indicar el final de una cláusula. De esta forma, una fórmula lógica con $v$ variables y $k$ cláusulas se puede representar mediante $k$ bloques de longitud $v$, donde cada bloque está formado por los símbolos $a$, $b$, o $c$, y cada bloque se separa del siguiente por el símbolo $d$.

Con este convenio, la fórmula
$$ F=(x_1 \vee \neg x_2) \wedge (\neg x_1 \vee x_2 \vee x_3) \wedge (x_1 \vee \neg x_2 \vee x_3)$$
se representa mediante la cadena:
$$abc\mathbf{d}baa\mathbf{d}aba\mathbf{d},$$
donde los símbolos $\mathbf{d}$ aparecen en negrita para facilitar la interpretación de la cadena como fórmula 
en forma normal conjuntiva.

Para que una cadena $e$ se pueda interpretar como una fórmula booleana debe cumplir con las siguientes condiciones:
tener $n$ bloques separados por $d$, cada bloque de la misma longitud $v$ y cada bloque solo debe estar formado por los caracteres
$a$, $b$ y $c$. 

Una cadena $e$ que cumpla con estas características  se puede interpretar como una fórmula booleana con $n$ cláusulas y $v$ variables, donde la estructura de cada cláusula depende de los caracteres correspondiente al bloque de $a$, $b$ y $c$ que se asocia a dicha cláusula.

Por ejemplo, la cadena $w=acc\mathbf{d}aba\mathbf{d}cba\mathbf{d}$, tiene 3 bloques separados por $d$, los cuales son $acc$, $aba$ y $cba$, los 3 tienen tamaño 3 y solo tienen los caracteres $a$, $b$ y $c$.
Por tanto  $w$ se puede interpretar como la siguiente fórmula booleana:
$$(x_1)\wedge(x_1\vee \neg x_2 \vee x_3) \wedge (\neg x_2\vee x_3).$$

Una vez definida la transformación de una fórmula booleana en una cadena, se puede definir el lenguaje de todas las fórmulas booleanas en CNF.

\begin{definition}
    El \textbf{lenguaje de todas las fórmulas booleanas en CNF} se define como:
    \[
        L_{FULL-SAT} = \{ q_1dq_2d\dots q_nd \mid q_i \in \{a, b,c\}^+\text{, }
        |q_i| = |q_j| \, \forall i, j =1\dots n, \text{ y } n\in \mathbb{N}\}.
    \]
\end{definition}

A partir de $L_{FULL-SAT}$ se puede definir el lenguaje de todas las fórmulas booleanas satisfacibles
en CNF, el cual se define a continuación.

\begin{definition}
    El \textbf{lenguaje de todas las fórmulas booleanas en CNF que son satisfacibles} $L_{S-SAT}$ se define como todas las cadenas $e\in L_{FULL-SAT}$,
    tales que la fórmula booleana que representa $e$, sea satisfacible.
\end{definition}

Por ejemplo, la fórmula $$x_1\wedge x_2 \wedge x_3,$$ es satisfacible por los valores $x_1=true$, $x_2=true$ y $x_3=true$, por lo que la 
cadena $acc\mathbf{d}cac\mathbf{d}cca\mathbf{d}$ pertenece a $L_{S-SAT}$. Por otro lado la fórmula $$x_1\wedge x_2 \wedge \neg x_1,$$
no es satisfacible para ninguna asignación de los valores de sus variables, por lo que la cadena $ac\mathbf{d}ca\mathbf{d}bc\mathbf{d}$
no pertenece a $L_{S-SAT}$.

En la próxima sección se demuestra que $L_{S-SAT}$ no es un lenguaje libre del contexto, por lo que el formalismo que lo genere necesariamente debe pertenecer a las gramáticas dependientes del contexto o a las gramáticas irrestrictas.

\section{$L_{S-SAT}$ no es un lenguaje libre del contexto}
\label{sec:LSATCFL}

En esta sección se demuestra que el lenguaje $L_{S-SAT}$ no es libre del contexto usando el lema del bombeo para lenguajes libres del contexto.

\begin{theorem}
    \label{teo:LSATCFL}
    $L_{S-SAT}$ no es un lenguaje libre del contexto.
\end{theorem}

Para la demostración del teorema \ref{teo:LSATCFL} se presentan los siguientes lemas:

\begin{lemma}
    \label{lem:Lh}
    Sea un homomorfismo $h: \{a,b,c,d\}\to \{1,d\}^*$, tal 
    que $h(a)=1$, $h(b)=1$, $h(c)=1$ y $h(d)=d$. Si se define el lenguaje $L_h=\{h(e)\mid e\in L_{S-SAT}\}$,
    entonces $L_h=\{(1^nd)^+\mid n\in \mathbb{N}\}$.
\end{lemma}

\begin{lemma}
    \label{lem:LhCFL}
    $L_h$ no es un lenguaje libre del contexto.
\end{lemma}

La idea de la demostración del Teorema \ref{teo:LSATCFL}, es definir un homomorfismo sobre $L_{S-SAT}$ y luego probar que el lenguaje resultante de evaluar todas las cadenas de $L_{S-SAT}$ en el homomorfismo es igual a $\{(1^nd)^+\mid n\in \mathbb{N}\}$, esto se plantea en el Lema \ref{lem:Lh}. Seguidamente, se demuestra que $L_h$ no es libre del contexto, usando el lema del bombeo, esto se plantea en el Lema \ref{lem:LhCFL}. Para finalizar la demostración se prueba que el hecho de que $L_h$ no sea libre del contexto implica que $L_{S-SAT}$ no es libre del contexto.

A continuación se demuestra el Lema \ref{lem:Lh}.

\begin{proof}[Demostración del Lema \ref{lem:Lh}] \
    
    Para demostrar que $L_h=\{(1^nd)^+\mid n\in \mathbb{N}\}$ se debe demostrar que $L_h$ es subconjunto de $\{(1^nd)^+\mid n\in \mathbb{N}\}$ y 
    que $\{(1^nd)^+\mid n\in \mathbb{N}\}$ es subconjunto de $L_h$.
    
    Para demostrar que $L_h\subseteq \{(1^nd)^+\mid n\in \mathbb{N}\}$, sea una cadena $t\in L_h$, entonces se cumple que existe una cadena $e\in L_{S-SAT}$ tal que $h(e)=t$. Luego como $e$ está formada por varios bloques de $a$, $b$ y $c$ del mismo tamaño unidos por $d$. Sea $n$ el tamaño de todos los bloques de $e$. Como $t$ es la imagen de $e$ por $h$, $t$ está formada por varios bloques de 1 de tamaño $n$ unidos por $d$, por tanto $t\in \{(1^nd)^+\mid n\in \mathbb{N}\}$.
    
    Para demostrar que $\{(1^nd)^+\mid n\in \mathbb{N}\}\subseteq L_h$, sea una cadena $t\in \{(1^nd)^+\mid n\in \mathbb{N}\}$, sin pérdida de la generalidad $t=(1^nd)^k$, con $n\in \mathbb{N}$ y $k\in \mathbb{N}$.
    
    La fórmula
    $$F=\underbrace{(x_1\vee x_2 \ldots x_n) \wedge (x_1\vee x_2 \ldots x_n) \ldots (x_1\vee x_2 \ldots x_n)}_{k},$$
    es satisfacible por los valores $x_1=true$, $x_2=true$, $\ldots$, $x_n=true$, y además la cadena $e=(a^nd)^k$ es la codificación de $F$ en $L_{FULL-SAT}$. Por tanto $e\in L_{S-SAT}$ y $h(e)=t$, lo cual implica que $t\in L_h$. Luego se cumple que $\{(1^nd)^+\mid n\in \mathbb{N}\}\subseteq L_h$ y por tanto $\{(1^nd)^+\mid n\in \mathbb{N}\}= L_h$.
\end{proof}

Seguidamente se demuestra el Lema \ref{lem:LhCFL}.

\begin{proof}[Demostración del Lema \ref{lem:LhCFL}] \
    
    Para demostrar que $L_h$ no es libre del contexto sea $n$ la constante asociada a $L_h$ en el lema del bombeo.
    
    Sea $t=1^n\mathbf{d}1^n\mathbf{d}1^n\mathbf{d}$, como $|t|\geq n$ existen $u$, $v$, $w$, $x$ y $y$ con $vx\neq \varepsilon$ y $|vwx|\leq n$
    tales que $h(e)=uvwxy$.
    
    En la cadena $t$ se cumple que entre dos $d$ hay exactamente $n$ caracteres 1, y como $|vwx|\leq n$ existen 2 casos: 
    \begin{itemize}
        \item caso 1: o $v$ o $x$ contienen una d, pero las 2 no pueden contener una $d$,
        \item caso 2: ni $v$ ni $x$ contienen una d.
    \end{itemize}
    
    
    En el primer caso, cuando se bombea $v$ y $x$ en la cadena $uv^2wx^2y$ se agrega exactamente un bloque de 1 más, porque $v$ o $x$ contienen una $d$, pero no las dos a la vez, y el bloque de 1
    que se agrega tiene un tamaño menor o igual a $n$, por lo que $uv^2wx^2y\notin L_h$, ya que hay un bloque de 1
    con menos caracteres que los demás.
    
    En el segundo caso, cuando se bombea $v$ y $x$ en la cadena $uv^2wx^2y$ se agrega al menos un caracter al bloque de 1 al que pertenecía 
    $v$ o al menos un caracter al bloque de 1 al que pertenecía $x$.  Si $v$ y $x$ pertenecían al mismo bloque 
    de 1 en $uvwxy$ entonces hay un bloque de 1 en $uv^2wx^2y$ que tiene más caracteres que los restantes 
    bloques, en caso contrario hay uno o dos bloques en $uv^2wx^2y$ que tienen más caracteres que los restantes 
    bloques, entonces $uv^2wx^2y\notin L_{h}$.
    
    En los dos casos se cumple que $uv^2wx^2y\notin L_h$ por lo tanto se cumple que $L_h$ no es un lenguaje libre del contexto.
\end{proof}

A continuación se demuestra el Teorema \ref{teo:LSATCFL}.

\begin{proof}[Demostración del Teorema \ref{teo:LSATCFL}] \
    
    Para demostrar que $L_{S-SAT}$ no es libre del contexto, suponga lo contrario. 
    
    Como los lenguajes libres del contexto
    son cerrados bajo homomorfismo y $L_h$ es el lenguaje generado al evaluar todas las cadenas de $L_{S-SAT}$ en $h$, se cumple
    que $L_h$ es libre del contexto. Pero por el Lema \ref{lem:LhCFL}, $L_h$ no es libre del contexto, lo cual es una contradicción.
    Por tanto se cumple que $L_{S-SAT}$ no es libre del contexto.
\end{proof}

Seguidamente, se muestra cómo interpretar determinados tipos de cadenas de 0 y 1 como la asignación de los valores de las variables de una fórmula booleana.

\section{Definición del lenguaje que representa la asignación de los valores de las variables de una fórmula booleana}
\label{sec:intsat}

En esta sección se muestra cómo interpretar algunas cadenas $r\in \{1,d\}^+$ como la asignación de valores para las variables de una fórmula booleana.
A continuación se define el lenguaje $L_{0,1,d}$.


\begin{definition}
    El lenguaje $L_{0,1,d}$ se define como:
    $$L_{0,1,d}=\{(wd)^+\,|\,w\in\{0,1\}^+\}.$$
    
    $L_{0,1,d}$ contiene cadenas sobre el alfabeto $\{0,1,d\}$, que representan una cadena binaria concatenada con una $d$,
    repetida varias veces.
\end{definition}

Por ejemplo, $011\mathbf{d}011\mathbf{d}011\mathbf{d}$ y $01001\mathbf{d}01001\mathbf{d}01001\mathbf{d}$ son cadenas que pertenecen a
$L_{0,1,d}$.

Para que una cadena $r\in L_{0,1,d}$ se pueda interpretar como una asignación de variables a una fórmula booleana $F$ que representa la cadena $e\in L_{FULL-SAT}$, se deben cumplir 2 condiciones: la longitud de $e$ es igual a la longitud de $r$, $|e|=|r|$, y la cantidad de caracteres $d$ en $e$ debe ser igual a la cantidad de caracteres $d$ en $r$. 

Si estas condiciones se cumplen, a cada subcadena binaria de $r$ se le asocia un bloque de $a$, $b$ y $c$ en $e$, el cual representa una cláusula de $F$.

Después, para cada subcadena binaria $r$ el valor del $i$-ésimo caracter se le asocia al valor de la $i$-ésima variable de la cláusula correspondiente: si este caracter es un 1 se le asocia un valor de \true{} (verdadero) y si el caracter es un 0 se le asocia el valor de \false{} (falso). Como todas las subcadenas binarias de $r$ son iguales, se garantiza que a dos instancias de la misma variable en dos cláusulas distintas se les asigne el mismo valor.

Por ejemplo, si $r=101\mathbf{d}101\mathbf{d}101\mathbf{d}$ y $e=abc\mathbf{d}cbb\mathbf{d}acc\mathbf{d}$ representa una fórmula $F_e$:
$$F_e=(x_1\vee\neg x_2)\wedge (\neg x_2 \vee \neg x_3)\wedge (x_1),$$
$r$ se puede interpretar como la asignación de valores a las variables de $F_e$ de la 
siguiente manera: $x_1=true$, $x_2=false$ y $x_3=true$, y la fórmula booleana se evalúa con valor \true{}.

Seguidamente se muestra cómo construir $L_{S-SAT}$ mediante una transducción finita del lenguaje $L_{0,1,d}$.

\section{Construcción del $L_{S-SAT}$ usando transducción finita}

En esta sección se construye el lenguaje $L_{S-SAT}$ mediante un transductor finito.

La idea para la construcción de $L_{S-SAT}$ es construir un transductor finito, denominado $T_{SAT}$, que acepte como entrada cadenas $r\in L_{0,1,d}$ y devuelva cadenas $e\in L_{FULL-SAT}$ tales que al evaluar $r$ en $e$ (como se describió en la sección \ref{sec:intsat}), $e$ sea verdadera.

Se construye $L_{S-SAT}$ como el lenguaje de todas las transducciones $e$ que se obtienen del transductor $T_{SAT}$, a partir del lenguaje de cadenas de entrada $L_{0,1,d}$.
$$L_{S-SAT} = \{e\,|\,\exists r \in L_{0,1,d} \text{ y } e \in T_{SAT}(r) \}.$$

A continuación se define el transductor $T_{SAT}$.

\subsection{Transductor $T_{SAT}$}
\label{sec:tsat}

En esta sección se define el transductor finito $T_{SAT}$ (Figura \ref{fig:transducer_sat} de la página \pageref{fig:transducer_sat}), el cual se usa para construir $L_{S-SAT}$, mediante una transducción finita del lenguaje $L_{0,1,d}$.

Para definir $T_{SAT}$, se construye el transductor $T_{CLAUSE}$ (Figura \ref{fig:transducer_clause} de la página \pageref{fig:transducer_clause}) que dada una cadena binaria $w$, genera todas las posibles cláusulas satisfacibles por los valores de las variables que determina la cadena de entrada.

La idea detrás del transductor es ir leyendo los caracteres de la cadena de entrada y, por cada caracter que se lee, se genera un literal de una cláusula. El caracter que se lee se le asigna como valor a la variable correspondiente del literal generado (si es un 1, se asigna \true{} y si es un 0 se asigna \false{}).

Este transductor tiene 3 estados: el estado inicial, el estado positivo, y el estado negativo.  
El estado positivo representa que la cláusula generada ya se evalúa con un valor de verdad positivo y 
el estado negativo representa que la cláusula generada aún no se evalúa con un valor de verdad positivo 
para los caracteres que se leyeron hasta el momento. Las transiciones entre los estados se realizan 
dependiendo de si la asignación que se realiza, en el momento de leer y de escribir satisface la cláusula 
generada o no.

A continuación de describe el funcionamiento de cada estado del transductor $T_{CLAUSE}$:
\begin{itemize}
    \item Estado $q_0$: representa el estado inicial.  Si se lee un 1 y se escribe una $a$, se pasa al estado positivo, ya que se genera una variable sin negar a la cual se le asigna el valor \true{}.  Si se lee un 1 y se escribe una $b$, se pasa al estado negativo, ya que se genera una variable negada a la cual se le asigna el valor \true{}.  Si se lee un 1 y se escribe una $c$, se mantiene en el mismo estado, ya que se genera una variable (con valor \true) que no está en la cláusula.  Si se lee un 0 y se escribe una $a$, se pasa al estado negativo, ya que se genera una variable sin negar a la cual se le asigna el valor \false{}.  Si se lee un 0 y se escribe una $b$, se pasa al estado positivo, ya que se genera una variable negada a la cual se le asigna el valor \false{}.  Si se lee un 0 y se escribe una $c$, se mantiene en el mismo estado, ya que se genera una variable (con valor \false) que no está en la cláusula.
          
    \item Estado $q_p$ (estado positivo de $T_{CLAUSE}$): representa que para los valores ya asignados a las variables se obtiene un valor de verdad positivo.  Como la fórmula se encuentra ya en un estado positivo, no importa la entrada ni lo que el transductor escriba, se mantiene en el mismo estado.  Este estado es el estado de aceptación para el transductor y significa que la cláusula se evalúa con un valor de verdad positivo.
          
    \item Estado $q_n$ (estado negativo de $T_{CLAUSE}$): representa que para los valores ya asignados a las variables se obtiene un valor de verdad negativo.  Si se lee un 1 y se escribe una $a$, se pasa al estado positivo, ya que se genera una variable sin negar a la cual se le asigna el valor \true{}.  Si se lee un 1 y se escribe una $b$, se mantiene en el mismo estado, ya que se genera una variable negada a la cual se le asigna el valor \true{}.  Si se lee un 1 y se escribe una $c$, se mantiene en el mismo estado, ya que se genera una variable (con valor \true) que no está en la cláusula. Si se lee un 0 y se escribe una $a$, se mantiene en el mismo estado, ya que se genera una variable sin negar a la cual se le asigna el valor \false{}.  Si se lee un 0 y se escribe una $b$, se pasa al estado positivo, ya que se genera una variable negada a la cual se le asigna el valor \false{}.  Si se lee un 0 y se escribe una $c$, se mantiene en el mismo estado, ya que se genera una variable (con valor \false) que no está en la cláusula.
\end{itemize}

Seguidamente se define $T_{CLAUSE}$.

\[
    T_{CLAUSE} = (Q, {\Sigma}, \Gamma, \delta, q_{0}, F),
\]
donde:
\begin{itemize}
    \item \(Q\) = ${q_0,q_p,q_n}$.
    \item \(\Sigma\) = $\{0,1\}$.
    \item \(\Gamma\) = ${a,b,c}$.
    \item \(\delta: Q \times \Sigma \to Q \times \Gamma^*\) función de transición.
    \item \(q_{0} = q_0\) estado inicial.
    \item \(F={q_p}\) conjunto de estados finales.
\end{itemize}

Seguidamente se define la función de transición $\delta$, con cada una de las transiciones para cada estado.

Las transiciones para el estado $q_0$ son las siguientes:
\begin{multicols}{2}
    \begin{itemize}
        \item $\delta_{SAT}(q_0,1)=(q_p,a)$
        \item $\delta_{SAT}(q_0,0)=(q_n,a)$
        \item $\delta_{SAT}(q_0,1)=(q_n,b)$
        \item $\delta_{SAT}(q_0,0)=(q_p,b)$
        \item $\delta_{SAT}(q_0,1)=(q_0,c)$
        \item $\delta_{SAT}(q_0,0)=(q_0,c)$
    \end{itemize}
\end{multicols}

Las transiciones para el estado $q_p$ (estado positivo de $T_{CLAUSE}$) son las siguientes:
\begin{multicols}{2}
    \begin{itemize}
        \item $\delta_{SAT}(q_{p},1)=(q_{p},a)$
        \item $\delta_{SAT}(q_{p},0)=(q_{p},a)$
        \item $\delta_{SAT}(q_{p},1)=(q_{p},b)$
        \item $\delta_{SAT}(q_{p},0)=(q_{p},b)$
        \item $\delta_{SAT}(q_{p},1)=(q_{p},c)$
        \item $\delta_{SAT}(q_{p},0)=(q_{p},c)$
    \end{itemize}
\end{multicols}

Las transiciones para el estado $q_n$ (estado negativo de $T_{CLAUSE}$) son las siguientes:
\begin{multicols}{2}
    \begin{itemize}
        \item $\delta_{SAT}(q_{n},1)=(q_{p},a)$
        \item $\delta_{SAT}(q_{n},0)=(q_{n},a)$
        \item $\delta_{SAT}(q_{n},1)=(q_{n},b)$
        \item $\delta_{SAT}(q_{n},0)=(q_{p},b)$
        \item $\delta_{SAT}(q_{n},1)=(q_{n},c)$
        \item $\delta_{SAT}(q_{n},0)=(q_{n},c)$
    \end{itemize}
\end{multicols}

\begin{figure}[h]
    \centering  \begin{otherlanguage}{english}
        \begin{tikzpicture}[shorten >=1pt, node distance=3cm, on grid, auto]
            
            % Nodos
            \node[state, initial] (q0)   {$q_0$};
            \node[state] (qn) [above right=of q0] {$q_n$};
            \node[state, accepting] (qp) [below right=of q0] {$q_p$};
            
            % Transiciones
            \path[->]
            (q0) edge [bend left] node {0/a,1/b} (qn)
            (q0) edge [bend right] node {1/a,0/b} (qp)
            (q0) edge [loop right] node {0/c,1/c} (q0)
            
            (qn) edge [bend left] node {1/a,0/b} (qp)
            (qn) edge [loop above] node {0/a,1/b,0/c,1/c} (qn)
            
            (qp) edge [loop below] node {1/a,0/b,0/a,1/b,0/c,1/c} (qp);
            
        \end{tikzpicture}
    \end{otherlanguage}
    \caption{Representación gráfica del Transductor $T_{CLAUSE}$.}
    \label{fig:transducer_clause} % Esto es para referenciar la figura en el texto
\end{figure}

A continuación se presenta como construir el transductor $T_{SAT}$, mediante una modificación de $T_{CLAUSE}$.

Si se tiene una cadena binaria $w$ se pueden generar todas las fórmulas booleanas de una cláusula 
satisfacibles por la asignación de valores que representa $w$ (si el $i$-ésimo caracter es un 1, la 
$i$-ésima variable de la cláusula tiene valor \true{} y si el $i$-ésimo caracter es un 0, la $i$-ésima 
variable de la cláusula tiene valor \false{}). Si se tienen 2 copias de $w$ se pueden generar todas las fórmulas 
booleanas de 2 cláusulas satisfacibles por la asignación de valores que representa $w$. 
Por último, si se tienen $n$ copias de $w$ se pueden generar todas las fórmulas booleanas de $n$ cláusulas satisfacibles por la asignación de valores que representa $w$.

Pero el problema es que si se concatenan 2 cadenas $w$, se obtiene la cadena $ww$, la cual no representa 2 cláusulas,
sino que representa una cláusula con el doble de variables. Para arreglar este inconveniente se puede agregar el separador
$d$, por lo que la cadena que se obtiene es $wdwd$. De igual forma se puede proceder para $n$ copias de $w$, obteniéndose
la cadena $(wd)^n$. Pero $(wd)^n\in L_{0,1,d}$, entonces se puede modificar $T_{CALUSE}$, para que lea cadenas de $L_{0,1,d}$ y genere cadenas que pertenecen a $L_{S-SAT}$.

Esta modificación funciona de la siguiente manera: 
cuando se genera una cláusula y el transductor termina en el estado positivo se continúa leyendo los valores de la cadena de entrada, ya que la cadena de entrada está conformada por varias cláusulas separadas mediante el caracter $d$ que son necesarios para generar la próxima cláusula.

La modificación de $T_{CALUSE}$ se logra definiendo $q_0$ como el estado de aceptación y agregando una 
transición del estado $q_p$ al estado $q_0$ que lea una $d$ y escriba una $d$. De esta manera, cuando se lee una $d$ y el transductor se encuentra en el estado positivo, significa
que se generó una cláusula con un valor de verdad positivo, y entonces se comienza a generar la siguiente cláusula desde el estado inicial. 

Esta modificación de $T_{CALUSE}$ tiene un inconveniente y es que el transductor genera la cadena vacía, y la cadena vacía representa una fórmula booleana con 0 variables, por lo que no tiene sentido que se considere en $L_{S-SAT}$. Para solucionar esto se pueden concatenar 2 transductores $T_{CLAUSE}$ y unirlos mediante una transición, esta idea se expone a continuación.

Para definir el transductor $T_{SAT}$ (Figura \ref{fig:transducer_sat}) se concatenan 2 transductores $T_{CLAUSE}$ ($T_1$ y $T_2$ respectivamente). Sean los estados: $q_{0_1}$, $q_{p_1}$ y $q_{n_1}$ estado inicial, positivo y negativo de $T_1$, respectivamente, y los estados $q_{0_2}$, $q_{p_2}$ y $q_{n_2}$ estado inicial, positivo y negativo de $T_2$, respectivamente. $T_1$ y $T_2$ se concatenan añadiendo una transición de $q_{p_1}$ a $q_{0_2}$ con el símbolo $d$ (tanto de lectura como de escritura) y además se agrega una transición de $q_{p_2}$ a $q_{0_2}$ con el símbolo $d$ (tanto de lectura como de escritura).  Para terminar se definen el estado inicial y el estado final de $T_{SAT}$, los cuales serían $q_{0_1}$ y $q_{0_2}$, respectivamente.



\begin{figure}[h]
    \begin{otherlanguage}{english}
        \centering \begin{tikzpicture}[shorten >=1pt, node distance=3cm, on grid, auto]
            
            % Nodos
            \node[state, initial] (q01)   {$q_{0_1}$};
            \node[state] (qn1) [above right=of q01] {$q_{n_1}$};
            \node[state] (qp1) [below right=of q0] {$q_{p_1}$};
            \node[state, accepting] (q02) [right=6cm of q01] {$q_{0_2}$};
            \node[state] (qn2) [above right=of q02] {$q_{n_2}$};
            \node[state] (qp2) [below right=of q02] {$q_{p_2}$};
            
            
            % Transiciones
            \path[->]
            (q01) edge [bend left] node {0/a,1/b} (qn1)
            (q01) edge [bend right] node {1/a,0/b} (qp1)
            (q01) edge [loop right] node {0/c,1/c} (q01)
            
            (qn1) edge [bend left] node {1/a,0/b} (qp1)
            (qn1) edge [loop above] node {0/a,1/b,0/c,1/c} (qn1)
            
            (qp1) edge [loop below] node {1/a,0/b,0/a,1/b,0/c,1/c} (qp1)
            
            (q02) edge [bend left] node {0/a,1/b} (qn2)
            (q02) edge [bend right] node {1/a,0/b} (qp2)
            (q02) edge [loop right] node {0/c,1/c} (q02)
            
            (qn2) edge [bend left] node {1/a,0/b} (qp2)
            (qn2) edge [loop above] node {0/a,1/b,0/c,1/c} (qn2)
            
            (qp2) edge [loop below] node {1/a,0/b,0/a,1/b,0/c,1/c} (qp2)
            
            (qp1) edge [bend right] node {d/d} (q02)
            (qp2) edge [bend left=75] node {d/d} (q02);
            
        \end{tikzpicture}
    \end{otherlanguage}
    \caption{Representación gráfica del Transductor $T_{SAT}$.}
    \label{fig:transducer_sat} % Esto es para referenciar la figura en el texto
\end{figure}

A continuación se demuestra que la construcción de $L_{S-SAT}$ mediante una transducción finita genera todas las fórmulas booleanas satisfacibles.

\section{La construcción de $L_{S-SAT}$ mediante una transducción finita
  genera todas las fórmulas satisfacibles}


En esta sección se demuestra que la construcción de $L_{S-SAT}$, a partir de la transducción del lenguaje $L_{0,1,d}$ mediante $T_{SAT}$, genera todas las fórmulas booleanas satisfacibles.

\begin{theorem}
    \label{teo:tsat}
    Una cadena $e$ pertenece al lenguaje generado por la transducción finita del lenguaje $L_{0,1,d}$ mediante el transductor $T_{SAT}$, si y solo si la fórmula booleana asociada a la cadena $e$ es satisfacible. Esto significa que:
    $$L_{S-SAT} = \{e\,|\,\exists r \in L_{0,1,d} \text{ y } e \in T_{SAT}(r) \}.$$
\end{theorem}

A continuación se presentan algunas definiciones que serán usadas en la demostración del Teorema \ref{teo:tsat}.


\begin{definition}
    Sea una cadena $w\in \{0,1\}^+$ y una fórmula booleana $F$ con la misma cantidad de variables que la longitud de 
    $w$. Cuando se le asignan los valores de $w$ las variables de $F$, si el $i$-ésimo caracter de $w$ es un 1 a la $i$-ésima variable de $F$ se le asigna
    el valor \true{} y si el $i$-ésimo caracter de $w$ es un 0 a la $i$-ésima variable de $F$ se le asigna
    el valor \false{}.
\end{definition}

\begin{definition}
    Una cadena  $w\in\{0,1\}^+$ satisface una fórmula booleana $F$ si al asignarle los valores de $w$ a $F$, se obtiene un valor de verdad positivo.
\end{definition}

\begin{definition}
    Una cadena $w\in\{0,1\}^+$ satisface a una cadena $e\in L_{FULL-SAT}$ si $w$ satisface la fórmula booleana asociada a $e$. 
\end{definition}

Para la demostración del Teorema \ref{teo:tsat} se presentan los siguientes lemas:

\begin{lemma}
    \label{lem:clause}
    Dada una cadena $w\in\{0,1\}^+$, $T_{CLAUSE}(w)$ es el conjunto de todas las cadenas que representan cláusulas que son satisfacibles por $w$.
\end{lemma}

\begin{lemma}
    \label{lem:sat}
    Dada una cadena $r=(wd)^n$, con $w\in\{0,1\}^+$, $T_{SAT}(r)$ contiene todas las cadenas que representan fórmulas de $n$ cláusulas satisfacibles por la cadena $w$.
\end{lemma}

La idea para la demostración del Teorema \ref{teo:tsat}, es probar que dada una cadena $w\in \{0,1\}^+$, 
$T_{CLAUSE}(w)$ es el conjunto de todas las cadenas que representan cláusulas que son satisfacibles por $w$, 
esto se plantea en el Lema \ref{lem:clause}. Después, dada una cadena $r=(wd)^n$ se realiza una inducción 
sobre $n$, para demostrar que $T_{SAT}(r)$ contiene todas las cadenas que representan fórmulas de $n$
cláusulas satisfacibles por la cadena $w$, esto se plantea en el Lema \ref{lem:clause}.  
Para finalizar la demostración se prueba que la transducción de $L_{0,1,d}$ mediante $T_{SAT}$, 
genera todas las cadenas $e\in L_{FULL-SAT}$, tales que $e$ representa una fórmula booleana 
satisfacible.

Seguidamente, se demuestra el Lema \ref{lem:clause}.

\begin{proof}[Demostración del Lema \ref{lem:clause}] \
    
    Para demostrar que dada una cadena binaria $w$, $T_{CLAUSE}(w)$ es el conjunto de todas las cláusulas que 
    son satisfacibles por $w$, primero suponga que $q\in T_{CLAUSE}(w)$.  Esto significa que el transductor 
    terminó en el estado $q_p$ en el proceso que generó $q$ y como empezó en el estado $q_0$ ocurrió una 
    transición desde $q_0$ a $q_p$ o desde $q_n$ a $q_p$. 
    
    Una transición de $q_0$ a $q_p$ o de $q_n$ a $q_p$ solo es posible si el transductor leyó un 1 y 
    escribió una $a$ o si leyó un 0 y escribió una $b$. Entonces, cuando se le asignan los valores de 
    $w$ a las variables de la fórmula booleana que representa $q$ hay una variable sin negar con valor 
    \true{} o una variable negada con valor \false{}, por lo tanto, se cumple que $w$ satisface la cláusula 
    que representa $q$. 
    
    Para demostrar que todas las cláusulas satisfacibles por $w\in\{0,1\}^+$ pertenecen a $T_{CLAUSE}(w)$, sea una cláusula $F$ satisfacible por una cadena $w\in\{0,1\}^+$, cuya representación sobre $\{a,b,c\}$ es $q$, entonces se cumple que cuando se le asignan los valores de $w$ a las variables de $F$, hay una variable sin negar con valor \true{} o una variable negada con valor \false{}. Sin pérdida de la generalidad se asume que la primera variable que cumple lo anterior es la $i$-ésima.
    
    Como en cada estado de $T_{CLAUSE}$ para cada símbolo que se lee existe una transición que escribe una $a$, una $b$ o una $c$; si se hace el reconocimiento de los primeros $i-1$ caracteres de la cadena de entrada por $T_{CLAUSE}$ se pueden seguir las transiciones entre los estados del transductor de tal manera que los primeros $i-1$ caracteres de la cadena generada sean iguales a los primeros $i-1$ caracteres de $q$.
    
    Luego de reconocer los primeros $i-1$ caracteres solo es posible que el transductor esté en el estado $q_0$ o $q_n$, pero como se cumple que la $i$-ésima variable está sin negar y con valor \true{} o está negada y con valor \false{}, entonces se puede tomar la opción de leer un 1 y escribir una $a$ o leer un 0 y escribir una $b$ según corresponda.  De esta manera, según el estado en que se encuentre el autómata, se pasa al estado $q_p$.
    
    A partir de este punto, se realizan las restantes transiciones de manera que la cadena generada sea $q$ y el transductor se mantiene en el mismo estado ya que $q_p$ solo tiene transiciones hacia sí mismo. De esta manera se demuestra el Lema \ref{lem:clause}. 
\end{proof}

Seguidamente se demuestra el Lema \ref{lem:sat}.

\begin{proof}[Demostración del Lema \ref{lem:sat}] \
    
    Para demostrar que $T_{SAT}(r)$ contiene todas las fórmulas satisfacibles por la cadena $r=(wd)^n$ donde $w\in\{0,1\}^+$ se hará una inducción sobre $n$. En el caso base y el paso inductivo
    se prueban ambos sentidos de la demostración.  
    
    Para esto, se definen los conjuntos $A_{w,n}$ como el conjunto formado por todas las cadenas del lenguaje $L_{FULL-SAT}$, que representan fórmulas booleanas de $n$ cláusulas satisfacibles por $w$ y $B_w$ como el conjunto formado por todas las cadenas que representan las cláusulas que son satisfacibles por $w$. Las cadenas de $A_{w,n}$ están formadas por $n$ concatenaciones de cadenas de $B_w$ separadas por $d$.
    
    El caso base $n=1$ se demuestra porque la transducción se realiza solo sobre el primer transductor $T_{CLAUSE}$ de $T_{SAT}$, ya que en la cadena de entrada solo hay una $d$, y como se demostró en el Lema \ref{lem:clause}, $T_{CLAUSE}(w)$ contiene todas cláusulas satisfacibles por $w$. Por tanto $T_{SAT}(wd)$, donde $w\in \{0,1\}^+$, contiene todas las cadenas que representan fórmulas booleanas de una cláusula satisfacibles por $wd$.
    
    Una vez demostrado el caso base se asume que el Lema \ref{lem:sat} es cierto para $n=k$ y se demuestra para $n=k+1$.
    
    El conjunto de todas las fórmulas booleanas con $k+1$ cláusulas satisfacibles por $w$ es equivalente al conjunto que forman todas las fórmulas booleanas con $k$ cláusulas, satisfacibles por $w$, concatenadas con todas las cláusulas satisfacibles por $w$: $$A_{w,k+1}=\{xzd\,|\,x\in A_{w,k} \text{ y } z\in B_w\}.$$ Además, por la estructura de $T_{SAT}$ se cumple que el conjunto de todas las cadenas que pertenecen al lenguaje generado por $T_{SAT}((wd)^{k+1})$ es igual al conjunto de todas las cadenas que pertenecen al lenguaje generado por $T_{SAT}((wd)^{k})$ concatenadas con todas las cadenas que pertenecen al lenguaje generado por $T_{CLAUSE}(w)$:
    $$T_{SAT}((wd)^{k+1})=\{xzd\,|\,x\in T_{SAT}((wd)^{k}) \text{ y } z\in T_{CLAUSE}(w)\}.$$
    
    Por hipótesis de inducción se cumple que $A_{w,k}=T_{SAT}((wd)^{k})$ y además se cumple que $B_w=T_{CLAUSE}(w)$, lo cual implica que $A_{w,k+1}=T_{SAT}((wd)^{k+1})$, por lo tanto se demuestra el Lema \ref{lem:sat}.
\end{proof}

Luego de demostrados los Lemas \ref{lem:clause} y \ref{lem:sat} se demuestra el Teorema \ref{teo:tsat}.

\begin{proof}[Demostración del Teorema \ref{teo:tsat}] \
    
    Para demostrar que $L_{S-SAT} = \{e\,|\,\exists r \in L_{0,1,d} \text{ y } e \in T_{SAT}(r) \}$, es necesario demostrar que 
    $L_{S-SAT}$ es subconjunto de $\{e\,|\,\exists r \in L_{0,1,d} \text{ y } e \in T_{SAT}(r) \}$ y que $\{e\,|\,\exists r \in L_{0,1,d} \text{ y } e \in T_{SAT}(r) \}$ es subconjunto de $L_{S-SAT}$.
    
    Para demostrar $\{e\,|\,\exists r \in L_{0,1,d} \text{ y } e \in T_{SAT}(r) \} \subseteq L_{S-SAT}$,
    sea una cadena $r \in L_{0,1,d}$ y sea una cadena $e\in T_{SAT}(r)$, se cumple que $e\in\{e\,|\,\exists r \in L_{0,1,d} \text{ y } e \in T_{SAT}(r) \}$. Por el Lema \ref{lem:sat} se cumple que la fórmula booleana asociada a $e$ es satisfacible por $r$, por tanto $e\in T_{SAT}(r)$. 
    
    Para demostrar $ L_{S-SAT} \subseteq \{e\,|\,\exists r \in L_{0,1,d} \text{ y } e \in T_{SAT}(r) \}$, sea $F$ una fórmula booleana satisfacible y sea $e$ su representación en $L_{FULL-SAT}$, se cumple que $e\in L_{S-SAT}$.  Por tanto existe una asignación de valores de las variables de $F$ que satisface a $F$, lo cual implica que existe $r\in L_{0,1,d}$ tal que $r$ satisface a $F$, luego se cumple que $e\in T_{SAT}(r)$, por el Lema \ref{lem:sat}.  Por tanto se cumple que $e\in \{e\,|\,\exists r \in L_{0,1,d} \text{ y } e \in T_{SAT}(r) \}$, lo cual implica que:
    $$L_{S-SAT} = \{e\,|\,\exists r \in L_{0,1,d} \text{ y } e \in T_{SAT}(r) \}.$$
\end{proof}

Una consecuencia directa del Teorema \ref{teo:tsat} es el siguiente resultado. 

\begin{theorem}
    \label{teo:gnp-hard}
    El problema de la palabra de cualquier formalismo que genere el lenguaje $L_{0,1,d}$ y sea cerrado bajo transducción finita es NP-Duro.
\end{theorem}

\begin{proof}[Demostración del Teorema \ref{teo:gnp-hard}] \
    
    Suponga que existe un formalismo $G$ que genera $L_{0,1,d}$ y que es cerrado bajo transducción finita.
    
    Sea $G'$ el formalismo que resulta de aplicarle el transductor $T_{SAT}$ a $G$. Entonces determinar si una cadena $e$ pertenece al lenguaje generado por $G'$, es equivalente a saber si la fórmula booleana a la cual representa $e$ es satisfacible. Por tanto el problema de la palabra para $G'$ es NP-Duro, porque tiene una reducción directa al problema de la satisfacibilidad booleana.
\end{proof}

Una restricción importante en la demostración anterior es que la representación de $G'$ tiene que tener tamaño $O(1)$, porque si no el problema de la palabra de $G'$ puede depender además de la cantidad de estados (en el caso de la representación mediante una máquina abstracta) o de la cantidad de producciones, símbolos terminales y no terminales (en el caso de la representación mediante una gramática).

En este trabajo se conjetura que cualquier formalismo $G$ que genere el lenguaje $L_{0,1,d}$, tiene tamaño $O(1)$ en cualquiera de sus representaciones, y como el transductor $T_{SAT}$ tiene una cantidad de estados $O(1)$, entonces $G'$ tiene que tener tamaño $O(1)$ en su representación.

En la literatura consultada para este trabajo aparecen dos formalismo que pueden generar el lenguaje $L_{copy}$: las gramáticas de índice global \cite{globalIndexLanguages} y las gramáticas de concatenación de rango \cite{propertiesRCGBib}. En ambos casos, la gramática que genera el lenguaje $L_{copy}$ tiene tamaño $O(1)$. La conjetura está basada en el hecho de que el lenguaje $L_{0,1,d}$ tiene características similares al lenguaje $L_{copy}$.

En este capítulo se presentó una estrategia para resolver el SAT usando teoría de lenguajes que se basa en definir y construir el lenguaje de todas las fórmulas satisfacibles. Además se presentó un primer acercamiento para construir este lenguaje, mediante una transducción finita. Esta forma de construir el lenguaje demuestra que el problema de la palabra para todos los formalismos que generen $L_{0,1,d}$ y sean cerrados bajo transducción finita es NP-Duro.

En el próximo capítulo se argumenta que la estrategia presentada en este para construir $L_{S-SAT}$ no es la única, porque se construye una RCG que reconoce el lenguaje $L_{S-SAT}$, y los lenguajes de concatenación de rango no son cerrados bajo transducción finita.

\documentclass[12pt]{article}

\usepackage[utf8]{inputenc} % Permite escribir caracteres especiales directamente
\usepackage[spanish]{babel} % Configura el idioma a español

\usepackage{amsmath}
\usepackage{tikz}
\usepackage{xcolor}
\usepackage{multicol}
\usepackage[lmargin=2cm,rmargin=5cm]{geometry}

\usepackage{enumitem}

%%%{{{ Comments and the like
\usepackage[textwidth=4cm]{todonotes}
\usepackage{soul}
\usepackage{xcolor}
\newcounter{todocounter}
\newcommand{\comment}[2]{\stepcounter{todocounter}
  {\color{green!50!blue}{(#1$^{{\color{black}\textbf{\thetodocounter}}}$)}}
  \todo[color=green,noline,size=\tiny]{\textbf{\thetodocounter:} #2

  }}
\newcommand{\quitaesto}[1]{{\color{red}(\st{#1})}}

\newcommand{\cambio}[2]{{\color{cyan}{{#2}}}{\color{red}{(\st{#1})}}}

\newcommand{\agregaesto}[1]{{\color{cyan}{{#1}}}}

\newcommand{\notaparaelautor}[1]{{\color{brown}{\textbf{#1}}}}

\newcommand{\errorortografico}[1]{{\fcolorbox{gray}{magenta}{\textcolor{yellow}{\bf #1}}}}
    
%%%}}}


\title{Estrategia para la solución del SAT usando gramáticas de concatenación de rango}
\author{Raudel Alejandro Gómez Molina}

\begin{document}

\maketitle

En este capítulo se abordan las estrategias presentadas en los 2 capítulos anteriores
(resolver el SAT usando el problema del vacío y usando el problema de la palabra)
utilizando gramáticas de concatenación
de rango. Además se obtiene un resultado que permite demostrar que las RCG reconocen
todos los problemas que pertenecen a la clase NP y se deja como problema abierto
obtener una RCG que permita reconocer todas las instancias de SAT que son solubles en
tiempo polinomial.

Para ello primeramente se muestra como reconocer $L_{0,1}$ con una RCG, el cual es la base para las RCG que permiten
resolver el SAT usando el problema del vacío y el problema de la palabra, las cuales se presentan en próximas
secciones.

\section{$L_{0,1}$ como lenguaje de concatenación de rango}

En esta sección se presenta una RCG que reconoce el lenguaje $L_{0,1}=\{wd\}^+$ donde $w\in \{0,1\}^*$, el cual sirve para la asignación de valores a las variables de un SAT. La gramática empleada se basa reconocer
primeramente una cadena $w$, luego un caracter $d$ y después comprobar que las siguientes cadenas sean iguales a $w$ seguidas
del caracter $d$.

Para reconocer $L_{0,1}$ se define la gramática $G_{0,1}$ como sigue:
\[
    G_{0,1} = (N, T, V, P, S),
\]
donde:

\begin{itemize}
    \item $N=\{S,A,B,C,Eq\}$
    \item $T=\{0,1,d\}$.
    \item $V=\{X,Y,P\}$.
    \item El conjunto de cláusulas $P$ es el siguiente:
          \begin{enumerate}
              \item $S(X)\to A(X)$
              \item $A(XdY)\to B(Y,X)C(X)$
              \item $B(XdY,P)\to B(Y,P) C(X) Eq(X,P)$
              \item $B(\varepsilon,P)\to \varepsilon$
              \item $C(0X)\to C(X)$
              \item $C(1X)\to C(X)$
              \item $C(\varepsilon)\to \varepsilon$
          \end{enumerate}
    \item El \textbf{símbolo inicial} es $S$.
\end{itemize}

El predicado $Eq$ se define en \cite{mainRCGBib} y comprueba que dos cadenas sobre un alfabeto sean iguales.
Por otro lado el predicado $B$ se encarga de definir la sustitución en rango de la próxima cadena de 0 y 1 y
comprobar que este sea igual al patrón inicial. De esta manera se pueden reconocer cadenas que pertenezcan al lenguaje $L_{0,1}$.

A continuación se presenta un ejemplo de como se reconoce la cadena $101d101d101d$ en $G_{0,1}$.

\subsection{Ejemplo de reconocimiento de $G_{0,1}$}

En esta sección se describen las derivaciones que permiten reconocer la cadena $101d101d101d$. Las derivaciones
de la gramática son las siguientes (después de cada derivación se especifica la cláusula
usada para la derivación y los rangos asociados a las variables):

\begin{enumerate}
    \item $S(101d101d101d) \to A(101d101d101d)$: c-1, $X=101d101d101d$
    \item $A(101d101d101d) \to B(101d101d,101)C(101)$: c-2, $X=101$ $Y=101d101d$
    \item $B(101d101d,101) \to B(101d,101)C(101)Eq(101,101)$: c-3, $X=101$ $Y=101d$ $P=101$
    \item $B(101d,101) \to B(\varepsilon,101)C(101)Eq(101,101)$: c-3, $X=101$ $Y=\varepsilon$ $P=101$
    \item $B(\varepsilon,101) \to \varepsilon$: c-4, $P=101$
    \item $C(101)\to C(01)$: c-6, $X=01$
    \item $C(01)\to C(1)$: c-5, $X=1$
    \item $C(1)\to C(\varepsilon)$: c-6 $X=\varepsilon$
    \item $C(\varepsilon)\to \varepsilon$: c-7
\end{enumerate}

Como todos los predicados derivan en la cadena vacía entonces $101d101d101d$ es reconocida por $G_{0,1}$.

En las próximas 2 secciones se presentan 2 RCG que siguen la idea descrita en los capítulos 3 y 4 respectivamente.

\section{Solución del SAT usando el problema del vacío}

En esta sección se presentan 2 enfoques, el primero usa el lenguaje $L^n_m=\{w^m\,|\,w\in\{0,1\} \wedge |w|=n\}$ para asignar valores
a las instancias de las variables de una fórmula booleana y el segundo busca obtener una gramática
que describa el orden de las variables en una fórmula booleana.

Estos 2 enfoques siguen la línea de lo expuesto en el capítulo 3:
asumir que todas las variables en
la fórmula son distintas, construir un autómata finito que reconozca cadenas de 0 y 1 hagan verdadera esa
fórmula (asumiendo que todas las variables son distintas), y por último intersectar ese lenguaje con algún
formalismo que garantice que todas las instancias de la misma variable tenga el mismo valor (en ambos enfoques
se emplea una RCG).

En la próxima sección se presentan una RCG que reconoce el lenguaje $L^n_m$.

\subsection{$L^n_m$ y $L_{0,1}$}

En esta sección se presenta una RCG que permite reconocer el lenguaje $L^n_m$ definido en \cite{aSMSAT}.

Según las definiciones planteadas en los capítulos 3 y 4,
$$L^n_m=\{w^m\,|\,w\in\{0,1\}^* \wedge |w|=n\},$$
y
$$L_{0,1}=\{wd \,|\,w\in \{0,1\}^+\},$$
por tanto para un $n$ y un $m$ específico se cumple que $L^n_m \subset L_{0,1}$.

Luego también es posible reconocer el lenguaje $L^n_m$ mediante una RCG haciendo modificaciones en $G_{0,1}$. Para ello
se define la gramática $G^n_m$ como sigue:

\[
    G^n_m = (N, T, V, P, S),
\]
donde:

\begin{itemize}
    \item $N=\{S,A,B_0,\ldots,B_m,C_1,\ldots,C_n,Eq\}$
    \item $T=\{0,1,d\}$.
    \item $V=\{X,Y,P\}$.
    \item El \textbf{símbolo inicial} es $S$.
\end{itemize}

A continuación se desglosa el conjunto de cláusulas P en varias fases agrupando las cláusulas por funcionalidad:
\begin{itemize}
    \item  \textbf{Primera fase:} Representa las cláusulas iniciales de la gramática:
          \begin{enumerate}
              \item $S(X)\to A(X)$
              \item $A(YdX)\to B_0(X,Y)C_1(X)$
          \end{enumerate}

    \item \textbf{Segunda fase:} Los predicados $B_i\,\forall i:\,0\leq i\leq m$ se encargan de definir las transiciones en la gramática
          que permiten contar el número de cadenas $w\in \{0,1\}^*$ que existen en la cadena original, verificar que
          este número es $m$ y comprobar que las cadenas $w$ reconocidas sean iguales:
          \begin{enumerate}[start=3]
              \item $B_0(XdY,P)\to B_1(Y,P) C_1(X) Eq(X,P)$
              \item $B_1(XdY,P)\to B_2(Y,P) C_1(X) Eq(X,P)$
                    $$\vdots$$
              \item $B_{m-1}(XdY,P)\to B_m(Y,P) C_1(X) Eq(X,P)$
              \item $B_m(\varepsilon,Y)\to \varepsilon$
          \end{enumerate}
    \item \textbf{Tercera fase:} Los predicados $C_i\,\forall i:\,1\leq i\leq n$ se encargan de verificar que las
          cadenas $w$ reconocidas en la fase anterior solo estén conformadas por los caracteres 0 ó 1 y tengan exactamente
          longitud $n$:
          \begin{enumerate}[start=7]
              \item $C_1(0X)\to C_2(X)$
              \item $C_1(1X)\to C_2(X)$
              \item $C_2(0X)\to C_3(X)$
              \item $C_2(1X)\to C_3(X)$
                    $$\vdots$$
              \item $C_{n-1}(0X)\to C_n(X)$
              \item $C_{n-1}(1X)\to C_n(X)$
              \item $C_n(\varepsilon)\to \varepsilon$
          \end{enumerate}
\end{itemize}

Observe que las modificaciones que se hicieron a $G_{0,1}$, fueron añadir nuevos predicados y transiciones que permitan
contar la cantidad de cadenas $w$ reconocidas y la longitud de dichas cadenas. Además la nueva gramática obtenida
tiene una cantidad de cláusulas en el orden $O(n+m)$.

Siguiendo la idea de \cite{aSMSAT} ahora solo resta interceptar $G^n_m$ con el autómata booleano correspondiente a dicha
fórmula y comprobar si el lenguaje generado es no vacío. La intercepción de una RCG con un autómata finito se expone en secciones posteriores.

A continuación se presenta una generalización de los trabajos \cite{aCFSAT} y \cite{aSRCSAT}, presentados en el capítulo 3, los cuales
se basan en obtener un formalismo que permita describir el orden de las variables de instancias específicas de un SAT.

\subsection{Orden de las instancias de las variables de un SAT}

En \cite{aCFSAT} y en \cite{aSRCSAT} se proponen 2 formalismos para reconocer el orden de las variables de instancias específicas del SAT que permiten dar una solución en tiempo polinomial. En esta sección se presenta un algoritmo para generar una RCG que permita reconocer el orden de las instancias de las variables de cualquier SAT.

Sea una fórmula booleana:

$$F=L_1\,S_1\,L_2\,S_2\,\ldots\,S_{m-1}\,L_m,$$

donde los $L_i\,\forall i:\,1\leq i\leq m$ son literales que representan una variable o su negación (dos literales
$L_x$ y $L_y$ pueden representar la misma variable) y los $S_i\,\forall i:\,1\leq i\leq m-1$ son los operadores
booleanos de disyunción y conjunción según corresponda. Además se define la función $\texttt{equals}(L_x,L_y)$
que recibe dos literales de $F$ y devuelve verdadero o falso en dependencia de si los literales corresponden
a la misma variable o no.

Ahora dada un fórmula booleana $F$, con $m$ literales y $n$ variables se define la gramática $G_{ord}$
como sigue:

\[
    G_{ord} = (N, T, V, P, S),
\]
donde:

\begin{itemize}
    \item $N=\{S,A,B_1,B_2,\ldots,B_n,C_1,C_2,\ldots,C_n\}$
    \item $T=\{0,1,d\}$.
    \item $V=\{X_1,X_2,\ldots,X_m,Y\}$.
    \item El conjunto de cláusulas $P$ es el siguiente:
          \begin{enumerate}
              \item  $S(X)\to A(X)$
              \item $A(X_1X_2\ldots X_m)\to
                        B_1(X_{\alpha_{1,1}}\ldots X_{\alpha_{1,\beta_1}})
                        B_2(X_{\alpha_{2,1}}\ldots X_{\alpha_{2,\beta_2}})
                        \ldots B_n(X_{\alpha_{n,1}}\ldots X_{\alpha_{n,\beta_n}})$
              \item $B_i(1Y)\to  Eq(Y,\underbrace{111\ldots 1}_{\beta_i})\,\forall i: \,1\leq i\leq n$

              \item $B_i(0Y)\to  Eq(Y,\underbrace{000\ldots 0}_{\beta_i})\,\forall i:\,1\leq i\leq n$
          \end{enumerate}
    \item El \textbf{símbolo inicial} es $S$.
\end{itemize}

Cada una de las variables $X_i\,\forall\,1\leq i\leq n$ en $G_{ord}$ corresponde al literal $L_i$ de $F$.
Además se cumple que el conjunto $\{\alpha_{1,1},\ldots,\alpha_{1,\beta_1},\alpha_{2,1},\ldots,
    \alpha_{2,\beta_2},\ldots,\alpha_{n,1},\ldots,\alpha_{n,\beta_n}\}$ es una permutación de los los números del
$1$ al $m$ y para todos los conjuntos $S_i=\{\alpha_{i,1},\ldots,\alpha_{i,\beta_i}\}\,\forall i:\,1\leq i\leq n$ se cumple
$\texttt{equals}(L_x,L_y)$ es verdadera para todos los pares $x, y$ tales que $x\neq y \wedge x,y\in S_i$.

Observe que en la construcción de $G_{ord}$ a cada variable se le asigna un literal de $F$, lo que significa
para una cadena reconocida por $G_{ord}$ cada caracter asociado a una variable $X_i$ representa el valor de verdad
para la variable asociada a $L_i$ en $F$. El funcionamiento de la gramática se basa en reconocer los rangos de la cadena
de entrada que representa la asignación a los valores de las instancias de las variables en $F$ y a cada rango se le asocia
una variable de $G_{ord}$. Luego las variables que en $G_{ord}$ que corresponden a una misma instancia de una variable en
$F$ se agrupan en un solo predicado comprobando que todas tengan el mismo valor de verdad.

Siguiendo la estrategia de \cite{aCFSAT} y \cite{aSRCSAT} solo resta interceptar la gramática generada con el autómata
booleano correspondiente a $F$ y comprobar si el lenguaje generado es no vacío. En la próxima sección se aborda el problema de interceptar una RCG con un autómata finito.

\subsection{Problema del vacío para la intercepción de una RCG con un autómata finito}

Primeramente como se mencionó en el capítulo 1 toda CFG tiene una RCG equivalente asociada y a su vez todo
autómata finito tiene una CFG equivalente asociada \cite{authomataTheory}, por lo que dado un autómata finito
se puede construir una RCG que sea equivalente.  Entonces la intercepción de una RCG y un autómata finito puede
ser descrita mediante otra RCG, el problema aquí es como se mencionó anteriormente el problema del vacío para
una RCG es indecidible. Por tanto este enfoque no es valido para analizar el problema descrito en esta sección.

En \cite{propertiesRCGBib1} se menciona que el problema del vacío para la intersección de un lenguaje regular
y una RCG es un problema NP-Completo y se demuestra con una reducción al 3-SAT. En el caso que ocupa el estudio
de este trabajo se puede hacer una reducción parecida tomando como referencia los resultados de las 2 secciones
anteriores que permiten reducir este problema al SAT.

En la literatura consultada no se encontró un algoritmo que permitiera resolver este problema, por tanto encontrar
un algoritmo específico para este problema puede constituir una vía diferente para resolver el SAT.

En siguiente sección se presenta otra vía de solución para el SAT que usa el problema de la palabra para las RCG.

\section{Solución del SAT usando el problema de la palabra}

En la presente sección se propone utilizar gramáticas de concatenación de rango para definir
el enfoque descrito en el capítulo 4, el cual se basa en buscar un formalismo que reconozca las
fórmulas booleanas satisfacibles.

\subsection{$L_{S-SAT}$ como lenguaje de concatenación de rango}

Como se mostró en una sección anterior el lenguaje $L_{0,1}$ puede ser reconocido mediante una RCG.
Entonces siguiendo con la idea abordada en las secciones del capítulo 4, que busca definir $L_{S-SAT}$
mediante una transducción finita, se puede tratar de definir $L_{0,1}$ para la transducción finita mediante una RCG.
El problema radica en que las RCG no son cerradas bajo transducción finita como se mencionó en el capítulo 1.
Por ello no se puede asegurar que al aplicarle el transductor $T_{SAT}$ a la RCG que reconoce el lenguaje
$L_{0,1}$ se obtenga una RCG, por lo cual no se puede realizar el análisis del problema de la palabra como
mismo se hizo con las gramáticas de índice global en el capítulo 4.

Ahora se pudiera profundizar en qué tipo de formalismo se obtiene al aplicar el transductor $T_{SAT}$
sobre la RCG que reconoce $L_{0,1}$ y realizar un análisis de la complejidad del problema de la palabra
para dicho formalismo. El cual automáticamente se demuestra que es NP-Duro, porque tiene una reducción
directa al SAT. Otro aspecto a considerar sería investigar qué propiedades de las RCG \comment{limitan que estas no sean cerradas bajo transducción finita y dado esas propiedades identificadas comprobar si todavía es posible reconocer el lenguaje $L_{0,1}$.}{quizás haga falta justificar esto un poquito mejor.}

A continuación se presenta otro enfoque distinto para generar $L_{S-SAT}$ que no emplea la transducción finita,
la cual se basa en una RCG que reconozca los problemas SAT satisfacibles.

\subsection{Otro enfoque para generar $L_{S-SAT}$}

En esta sección se presenta una RCG que es capaz de reconocer las fórmulas booleanas satisfacibles.

La idea detrás de esta gramática es la misma vista en el capítulo 4 cuando se definió $L_{S-SAT}$ mediante
una transducción finita: obtener un formalismo que permita reconocer si un SAT es satisfacible y para comprobar
si un SAT es satisfacibles entonces solo es necesario analizar el problema de la palabra. En la definición
de $L_{S-SAT}$ mediante una transducción finita se genera mediante el lenguaje $L_{0,1}$ todas las posibles interpretaciones
de cualquier fórmula booleana y luego el transductor $T_{S-SAT}$ genera todas las posibles fórmulas booleanas satisfacibles
según las interpretaciones generadas por $L_{0,1}$. Aquí en cambio se presenta un RCG que reconoce las fórmulas booleanas
satisfacibles, la cual se basa en generar mediante el reconocimiento de la cadena original todas las posibles interpretaciones
de la fórmula booleana que satisfacen la primera cláusula y luego comprobar si dicha intercepción satisface el resto
de las cláusulas.


Para reconocer $L_{S-SAT}$ define la siguiente RCG:
\[
    G_{S-SAT} = (N, T, V, P, S),
\]
donde:

\begin{itemize}
    \item $N=\{S,A,B,C,P,N,Cp,Cn\}$
    \item $T=\{a,b,c,d\}$.
    \item $V=\{X,Y,X_1,X_2\}$.
    \item El \textbf{símbolo inicial} es $S$.
\end{itemize}

A continuación se desglosa el conjunto de \textbf{cláusulas} $P$ en varias en 4 fases agrupando las cláusulas por funcionalidad.
La primera representa la derivación inicial de la gramática. La segunda se encarga de generar todas las posibles
interpretaciones de las variables que satisfacen la primera cláusula. La tercera comprueba que la interpretación
definida en la fase anterior sea satisfacible para el resto de las cláusulas. Por último la cuarta fase
define el algoritmo para determinar si una intercepción satisface una cláusula dada. Finalmente
si para una cadena que representa una fórmula booleana existe una secuencia de derivaciones desde el predicado
inicial pasando por cada fase hasta la cadena vacía entonces la cadena se reconoce:

\begin{itemize}
    \item \textbf{Primera fase:} Representa la cláusula de derivación inicial de la gramática:
          \begin{enumerate}
              \item $S(X)\to A(X)$
          \end{enumerate}

    \item \textbf{Segunda fase:} El siguiente conjunto de cláusulas genera la cadena de 0 y 1 que que da valores a las variables de la
          fórmula booleana:
          \begin{multicols}{2}
              \begin{enumerate}[start=2]
                  \item $A(aX)\to P(X,1)$
                  \item $A(aX)\to N(X,0)$
                  \item $A(bX)\to N(X,1)$
                  \item $A(bX)\to P(X,0)$
                  \item $A(cX)\to N(X,1)$
                  \item $A(cX)\to N(X,0)$

                  \item $P(aX,Y)\to P(X,Y1)$
                  \item $P(aX,Y)\to P(X,Y0)$
                  \item $P(bX,Y)\to P(X,Y1)$
                  \item $P(bX,Y)\to P(X,Y0)$
                  \item $P(cX,Y)\to P(X,Y1)$
                  \item $P(cX,Y)\to P(X,Y0)$
                  \item $P(dX,Y)\to B(X,Y)$

                  \item $N(aX,Y)\to P(X,Y1)$
                  \item $N(aX,Y)\to N(X,Y0)$
                  \item $N(bX,Y)\to N(X,Y1)$
                  \item $N(bX,Y)\to P(X,Y0)$
                  \item $N(cX,Y)\to N(X,Y1)$
                  \item $N(cX,Y)\to N(X,Y0)$
              \end{enumerate}
          \end{multicols}

          El predicado $A$ representa el predicado por donde inician las derivaciones de esta fase, de este se
          deriva a los predicados $P$ (representa que la cláusula de la fórmula booleana se encuentra en un
          estado de verdad positivo) y $N$ (representa que la cláusula de la fórmula booleana se encuentra en
          un estado de verdad negativo) en dependencia del valor asignado. El predicado $P$ deriva hacia sí
          mismo independientemente del símbolo, exceptuando el símbolo $d$, caso en el que se procede a la
          siguiente fase.  El funcionamiento de esta fase es prácticamente el mismo que el del transductor
          $T_{SAT}$: a partir de un estado de una variable en la fórmula booleana y una asignación que se le
          haga a la misma pasar a un estado positivo o negativo que representa el valor de verdad actual de la
          cláusula.

    \item \textbf{Tercera fase:} El siguiente conjunto de cláusulas se encarga de un mecanismo de iteración que
          le permite a la gramática reconocer si la asignación realizada en la fase anterior es válida para las restantes
          cláusulas de la fórmula booleana.
          \begin{enumerate}[start=21]
              \item $B(X_1dX_2,Y)\to C(X_1,Y) B(X_2,Y)$
              \item $B(\varepsilon,Y)\to\varepsilon$
          \end{enumerate}

          El predicado $B$ permite realizar la iteración sobre las cláusulas restantes mientras que el
          predicado $C$ comprueba que la cláusula de la fórmula booleana actual sea satisfacible, comportamiento
          que se encuentra definido en la cuarta fase.

    \item \textbf{Cuarta fase:} Solo resta definir el comportamiento de $C$, que recibe una cláusula y una intercepción
          de las variables y comprueba que dicha intercepción sea satisfacible para la cláusula analizada:
          \begin{enumerate}[start=23]
              \begin{multicols}{2}
                  \item $C(X,Y)\to Cn(X,Y)$

                  \item $Cn(aX,1Y) \to Cp(X,Y)$
                  \item $Cn(aX,0Y) \to Cn(X,Y)$
                  \item $Cn(bX,1Y) \to Cn(X,Y)$
                  \item $Cn(bX,0Y) \to Cp(X,Y)$
                  \item $Cn(cX,1Y) \to Cn(X,Y)$
                  \item $Cn(cX,0Y) \to Cn(X,Y)$

                  \item $Cp(aX,1Y) \to Cp(X,Y)$
                  \item $Cp(aX,0Y) \to Cp(X,Y)$
                  \item $Cp(bX,1Y) \to Cp(X,Y)$
                  \item $Cp(bX,0Y) \to Cp(X,Y)$
                  \item $Cp(cX,1Y) \to Cp(X,Y)$
                  \item $Cp(cX,0Y) \to Cp(X,Y)$
                  \item $Cp(\varepsilon,\varepsilon)\to \varepsilon$
              \end{multicols}
          \end{enumerate}

          Observe que este funcionamiento es exactamente igual al de la primera fase con un predicado que
          representa un estado positivo ($Cp$) y un predicado que representa un estado positivo ($Cn$)
          pero esta vez no se genera la cadena sino que se comprueba con un patrón, el cual se predefine
          en la segunda fase y pasa a las siguientes mediante las derivaciones de la gramática.

\end{itemize}

A continuación se demuestra que el lenguaje reconocido por $G_{S-SAT}$ es exactamente igual al lenguaje que representa
todas las fórmulas booleanas satisfacibles.

\subsubsection{Demostración de la gramática $G_{S-SAT}$}

La idea para la demostración de que el lenguaje reconocido por $G_{S-SAT}$ es exactamente igual al lenguaje que representa
todas las fórmulas booleanas satisfacibles se basa en demostrar el correcto funcionamiento de las 4 fases de la gramática. Primeramente
demostrar que el predicado $C$ que pertenece a la cuarta fase reconoce los argumentos $X$ y $Y$ si y solo si
$Y$ satisface la cláusula de la fórmula booleana asociada a $X$. Posteriormente demostrar que el predicado $B$ asociado
a la tercera fase reconoce los argumentos $X$ y $Y$ si y solo si $Y$ satisface todas las cláusulas asociadas a $X$, funcionamiento presente en la segunda cláusula.
Luego demostrar que el conjunto de cadenas $Q$ formado por todas las cadenas $Y$ tales que existe una secuencia de
derivaciones desde del predicado $A(X_1)$ hasta $B(X_2,Y)$ es exactamente igual a al conjunto de todas las interpretaciones
que hacen verdadera la primera cláusula de $X_1$, funcionamiento presente en la primera cláusula. Por último demostrar que el
conjunto de cadenas $X$ que son reconocidas por el predicado inicial $S$ es exactamente igual al conjunto de la
representación de todas las fórmulas booleanas satisfacibles.

A continuación se presenta la demostración para el funcionamiento de cada fase:

\begin{itemize}
    \item \textbf{Cuarta fase:} Dado el predicado $C(X,Y)$.

          Para demostrar el funcionamiento de la cuarta fase primeramente suponga que la fórmula booleana asociada a $X$, la
          cual se denomina $F_x$ es satisfacible por $Y$, entonces se debe demostrar que existe una secuencia de
          derivaciones desde $C(X,Y)$ hasta la cadena vacía. Como $Y$ satisface a $F_x$ debe existir una variable sin negar a
          la cual se le da como valor un 1 o un variable negada con valor 0, sin pérdida de la generalidad se dice que
          la primera variable que cumple esto es la $i$-ésima variable. Por tanto pueden darse 2 casos, el $i$-ésimo caracter de $X$ es $a$ y el $i$-ésimo caracter de
          $Y$ es 1 o el $i$-ésimo caracter de $X$ es $b$ y el $i$-ésimo caracter de $Y$ es 0. Del predicado $C$ automáticamente
          se deriva al predicado $Cn$, la única derivación de la gramática donde se deriva del predica del predicado $Cn$ a $Cp$
          es precisamente la combinación de una $a$ y un 1 o de una $b$ y un 0 y como $Y$ satisface $F_x$ esta combinación existe. Por último
          $Cp$ siempre deriva en sí mismo o en la cadena vacía por lo tanto queda demostrado que existe una secuencia de derivaciones
          desde $C(X,Y)$ hasta la cadena vacía.

          La demostración del funcionamiento de la cuarta fase la completa el hecho de que si $C(X,Y)$ es reconocido entonces
          $Y$ satisface a $X$. Por la estructura de la gramática si existe una secuencia de derivaciones desde $C(X,Y)$ hasta la cadena
          vacía entonces hay una derivación desde $Cn$ hacia $Cp$ y esta derivación solo es posible por una combinación de una $a$ y un 1
          o de una $b$ y un 0, por lo tanto una de estas combinaciones existe. Luego existe en $F_x$ una variable sin negar con valor 1 o una variable
          negada con valor 0 lo cual implica que $Y$ satisface $F_x$.

    \item \textbf{Tercera fase:} Dado el predicado $B(X,Y)$.

          Para demostrar el funcionamiento de la tercera fase se hará una inducción sobre la cantidad de cláusulas $n$ de
          la fórmula booleana asociada a $X$. Para $n=1$ se cumple que los
          rangos asociados a las variables $X_1$ y $X_2$ son $X$ sin su último caracter y la cadena vacía respectivamente, por
          tanto $B(X,Y)$ se reconoce por la gramática si y solo si $C(X_1,Y)$ se reconoce y esto solo es posible si $Y$ satisface a
          $X_1$, por lo que se demuestra el caso base. Se asume para $n=k$ y se demuestra para $k+1$, en todas las posibles sustituciones en
          rango de $X_1$ y $X_2$, $C(X_1,Y)$ solo se reconoce si $|X_1|=|Y|$, entonces el caso de sustitución en rango que ocupa a la demostración
          es si $|X_1|=|Y|$. Luego $Y$ satisface todas las cláusulas de $X$ si y solo si satisface $X_1$ y $X_2$ y precisamente
          $B(X,Y)$ se reconoce si y solo si se reconoce $C(X_1,Y)$ y $B(X_2,Y)$, $C(X_1,Y)$ se demuestra por el funcionamiento de la
          cuarta fase y  $B(X_2,Y)$ se demuestra por hipótesis de inducción.

    \item \textbf{Segunda fase:} Dado el predicado $A(X)$.

          Para la demostración del funcionamiento de la segunda fase se utilizan 2 conjuntos $W$ representa el conjunto de todas
          las cadenas que satisfacen la primera cláusula de la fórmula asociada a $X$ ($F_{1x}$) y $Q$ que representa el conjunto de todas las cadenas $Y$
          tales que existe una secuencia de derivaciones desde $A(X)$ hasta $B(Z_x,Y)$, donde $Z_x$ esta
          conformada por todas las cláusulas de $X$ menos la primera. Dadas estas definiciones es necesario
          demostrar que $W=Q$, para ello se debe demostrar que $W\subseteq Q \wedge Q\subseteq W$.

          Para demostrar que $W\subseteq Q$, sea $w$ una cadena tal que $w\in W$, es decir, $w$ satisface la primera $F_{1x}$. Por tanto en
          $F_{1x}$ existe una variable sin negar con valor 1 en $w$ o existe una variable negada con valor 0 en $w$, lo que representa
          una combinación de una $a$ y un 1 o de una $b$ y un 0 en una de las derivaciones de la segunda fase. Por la estructura
          de la gramática del predicado $A$ solo hay derivaciones a $P$ con una de estas 2 combinaciones, el resto son hacia
          el predicado $N$ y del predicado $N$ solo hay derivaciones a $P$ con una de las combinaciones anteriores. Por tanto como
          existe una combinación de una $a$ y un 1 o de una $b$ y un 0, existe una secuencia de derivaciones que lleva del predicado $A$ al predicado $P$
          pasando por $N$ o sin pasar por $N$. Como el predicado $P$ solo tiene derivaciones hacia sí mismo o hacia $B(Z_x,Y)$ por lo tanto se cumple
          que $w\in Q$.

          Para demostrar que $Q\subseteq W$, sea $q$ una cadena tal que $q\in Q$, es decir, existe una secuencia de derivaciones
          desde $A(X)$ a $B(X,Y)$ con $q=Y$. Por la estructura de la gramática solo se puede derivar al predicado $B$ desde el predicado
          $P$ y a su vez a este predicado solo se puede derivar mediante una combinación de una $a$ y un 1 o de una $b$ y un 0 en la gramática. Por tanto
          $F_{1x}$ tiene una variable sin negar con valor 1 en $q$ o una variable negada con valor 0 en $q$. Entonces se cumple que $q$ satisface
          a $F_{1x}$ por lo que $q\in W$. Por tanto queda demostrado que $Q=W$.

    \item \textbf{Primera fase:} Dado el predicado $S(X)$.

          Para demostrar que el lenguaje reconocido por $G_{S-SAT}$ es exactamente igual al lenguaje que representa
          todas las fórmulas booleanas satisfacibles se define el lenguaje $L_{G_{S-SAT}}$ que representa
          el lenguaje de todas las cadenas reconocidas por $G_{S-SAT}$, es necesario demostrar que $L_{S-SAT}=L_{G_{S-SAT}}$.

          Sea una fórmula booleana satisfacible $F$ y sea $X$ su representación como cadena, entonces existe una cadena binaria
          $w$ que satisface $F$. Como $w$ satisface $F$ entones $w$ pertenece al conjunto de cadenas que satisfacen a la primera cláusula de $F$,
          entonces existe una secuencia de derivaciones desde el predicado $S(X)$ hasta $B(Z_x,w)$ y como $w$ satisface todas las cláusulas de $F$
          entonces $B(Z_x,w)$ deriva en la cadena vacía por lo que $X$ es reconocida por $G_{S-SAT}$, lo cual implica que $L_{S-SAT}\subseteq L_{G_{S-SAT}}$.

          Sea una cadena $X$ reconocida por $G_{S-SAT}$ y sea $F$ la fórmula booleana asociada, entonces existe una cadena binara $q$ tal que existe una secuencia
          de derivaciones desde $A(X)$ a $B(Z_x,q)$ y de $B(Z_x,q)$ a la cadena vacía, por tanto $q$ satisface a la primera cláusula de $F$
          y a las restantes también, luego $q$ satisface a $F$, por lo que $F$ es satisfacible, lo cual implica que $L_{G_{S-SAT}}\subseteq L_{S-SAT}$.
          Por tanto se demuestra que $L_{G_{S-SAT}}= L_{S-SAT}$.

\end{itemize}

En la siguiente sección se presentan un ejemplo del funcionamiento de $G_{S-SAT}$.

\subsubsection{Ejemplo de reconocimiento de $G_{S-SAT}$}

En esta sección se presentan 2 ejemplos del funcionamiento de $G_{S-SAT}$ en el primero se muestra como se reconoce
la cadena asociada a la fórmula booleana $x_1 \vee  x_2 \wedge x_1 \wedge \neg x_2$ y en el segundo se muestra como no
se reconoce la fórmula booleana asociada a $x_1 \wedge \neg x_1$.

La cadena asociada a $x_1 \vee  x_2 \wedge x_1 \wedge \neg x_2$ es $aadacdcbd$, la secuencia de derivaciones asociada
a esta cadena en $G_{S-SAT}$ es la siguiente  (después de
cada derivación se especifica la cláusula usada para la derivación y los rangos
asociados a las variables):

\begin{enumerate}
    \item $S(aadacdcbd)\to A(aadacdcbd)$: c-1, $X=aadacdcbd$
    \item $A(aadacdcbd)\to P(adacdcbd,1)$: c-2, $X=adacdcbd$
    \item $P(adacdcbd,1)\to P(dacdcbd,10)$: c-9, $X=dacdcbd$ $Y=1$
    \item $P(dacdcbd,10)\to B(acdcbd, 10)$: c-14, $X=acdcbd$ $Y=10$
    \item $B(acdcbd, 10)\to C(ac,10) B(cbd,10)$: c-21, $X_1=ac$ $X_2=cbd$ $Y=10$
    \item $B(cbd,10)\to C(cb,10) B(\varepsilon,10)$: c-21, $X_1=cb$ $X_2=\varepsilon$ $Y=10$
    \item $B(\varepsilon,10)\to \varepsilon$: c-22, $Y=10$
    \item $C(ac,10)\to Cn(ac,10)$: c-23, $X=ac$ $Y=10$
    \item $Cn(ac,10)\to Cp(c,0)$: c-24, $X=c$ $Y=0$
    \item $Cp(c,0)\to Cp(\varepsilon,\varepsilon)$: c-35, $X=\varepsilon$ $Y=\varepsilon$
    \item $Cp(\varepsilon, \varepsilon) \to \varepsilon$: c-36
    \item $C(cb,10)\to Cn(cb,10)$: c-23, $X=cb$ $Y=10$
    \item $Cn(cb,10)\to Cn(b,0)$: c-28, $X=b$ $Y=0$
    \item $Cn(b,0)\to Cp(\varepsilon,\varepsilon)$: c-27, $X=\varepsilon$ $Y=\varepsilon$
    \item $Cp(\varepsilon, \varepsilon) \to \varepsilon$: c-36
\end{enumerate}

Como todos los predicados derivan en la cadena vacía entonces $aadacdcbd$ es reconocida por $G_{S-SAT}$, lo
cual coincide con el hecho de que $x_1 \vee  x_2 \wedge x_1 \wedge \neg x_2$, para la asignación de valores
$x_1=1$ y $x_2=0$.

Ahora se presenta un caso asociado a una fórmula que no es satisfacible y por tanto la cadena asociada a dicha
fórmula no es reconocida por $G_{S-SAT}$. La cadena asociada a $x_1  \wedge \neg x_1$ es $adbd$, la secuencia
de derivaciones asociada a esta cadena en $G_{S-SAT}$ es la siguiente  (después de
cada derivación se especifica la cláusula usada para la derivación y los rangos
asociados a las variables):

\begin{enumerate}
    \item $S(adbd)\to A(adbd)$: c-1, $X=abdbd$
    \item $A(adbd)\to P(dbd,1)$: c-2, $X=dbd$
    \item $A(adbd)\to N(dbd,0)$: c-3, $X=dbd$
    \item $P(dbd,1) \to B(bd,1)$: c-14, $X=bd$ $Y=1$
    \item $B(bd,1)\to C(b,1) B(\varepsilon,1)$: c-21, $X_1=b$ $X_2=\varepsilon$ $Y=1$
    \item $C(b,1)\to Cn(b,1)$: c-23, $X=b$ $Y=1$
    \item $Cn(b,1)\to Cn(\varepsilon,\varepsilon)$: c-26, $X=\varepsilon$ $Y=\varepsilon$
\end{enumerate}

El predicado $C(b,1)$ en la quinta derivación no deriva en la cadena vacía por tanto, después de realizar todas las
posibles derivaciones y las sustituciones en rango $G_{S-SAT}$ no reconoce la cadena $adbd$. Esto coincide con el hecho de
que $x_1  \wedge \neg x_1$ para ninguna asignación de valores a sus variables.

Como $G_{S-SAT}$ reconoce las fórmulas booleanas satisfacibles solo se debe analizar el problema de la palabra para determinar si una fórmula es satisfacible,
por lo que el siguiente paso es analizar el la complejidad del problema de la palabra para $G_{S-SAT}$.

\subsubsection{Análisis de la complejidad computacional del reconocimiento en $G_{S-SAT}$}

Como se mencionó en el capítulo 2 no todas las RCG tienen un algoritmo de reconocimiento polinomial y $G_{S-SAT}$
es un ejemplo de ello. Observe que en la primera fase se genera la cadena binaria que representa la asignación
de valores a las variables booleanas y dicha cadena participa en los predicados de fases posteriores, mediante
las derivaciones de la gramática.

Si se analiza el algoritmo de reconocimiento descrito en \cite{mainRCGBib}
un factor en la complejidad del algoritmo de reconocimiento es la cantidad de rangos posibles para una cadena
que se reconoce por un predicado. En este caso la cadena que representa los valores de las variables de la
fórmula booleana puede tomar $2^n$ valores distintos, donde $n$ es la cantidad de variables en la fórmula booleana,
ya que dicha cadena se genera durante la primera fase donde la gramática es ambigua y en cada derivación hay
decisiones que generan valores distintos.
Entonces la cantidad de rangos sería $n^22^n$, pero esta es una cota burda ya que una vez generada
la cadena de asignación por la forma de la gramática solo se utiliza un solo rango que se va construyendo
bajo demanda.

El resto de las fases de la gramática tienen una complejidad de $m^2$ donde $m$ es la cantidad de caracteres
en la cadena de entrada, por lo que la complejidad total sería $O(2^nm^2)$.

Este es resultado demuestra que no es necesario usar el transductor $T_{SAT}$ para definir el lenguaje $L_{S-SAT}$, \notaparaelautor{¿por qué?} mediante un formalismo de escritura regulada. \notaparaelautor{¿qué implicaciones pudiera tener esto?}

En la siguiente sección se analiza una consecuencia directa del resultado de la gramática $G_{S-SAT}$.

\section{Clases de problemas que reconocen las RCG}

En \cite{propertiesRCGBib2} se menciona que para todo problema en P existe una RCG que lo reconoce en su representación como
lenguaje formal. Ahora, como se mostró en la sección anterior con la gramática $G_{S-SAT}$ existe una RCG que permite resolver el SAT, por tanto como el SAT se puede reducir a cualquier problema en NP en una complejidad polinomial, entonces para todo problema en NP también existe una RCG que lo reconoce en su representación como lenguaje formal.

En la próxima sección se presenta un primer acercamiento para resolver las instancias polinomiales del SAT usando gramáticas
de concatenación de rango.

\section{Instancias de SAT polinomiales empleando RCG}

En esta sección se presenta una RCG que es capaz de reconocer problemas SAT, satisfacibles que pertenecen
al 2-SAT, es decir, problemas SAT donde cada cláusula tiene a lo sumo 2 literales. La idea detrás de esta
gramática es obtener una RCG que reconozca cuando la fórmula booleana pertenece al conjunto
de fórmulas booleanas de 2-SAT y luego intersectar dicha gramática con $G_{S-SAT}$.  Para ello se define la siguiente RCG:
\[
    G_{2-SAT} = (N, T, V, P, S),
\]
donde:

\begin{itemize}
    \item $N=\{S,A,A_0,A_1,A_2,A_3\}$
    \item $T=\{a,b,c,d\}$.
    \item $V=\{X,Y,X_1,X_2\}$.
    \item El conjunto de cláusulas $P$ es el siguiente:
          \begin{enumerate}
              \begin{multicols}{2}

                  \item $S(X)\to A(X)$
                  \item $A(X_1dX_2)\to A_0(X_1) A(X_2)$
                  \item $A(\varepsilon)\to \varepsilon$
                  \item $A_0(aX)\to A_1(X)$
                  \item $A_0(bX)\to A_1(X)$
                  \item $A_0(cX)\to A_0(X)$
                  \item $A_1(aX)\to A_2(X)$
                  \item $A_1(bX)\to A_2(X)$
                  \item $A_1(cX)\to A_1(X)$
                  \item $A_2(aX)\to A_3(X)$
                  \item $A_2(bX)\to A_3(X)$
                  \item $A_2(cX)\to A_2(X)$
                  \item $A_2(\varepsilon)\to \varepsilon$
              \end{multicols}
          \end{enumerate}
    \item El \textbf{símbolo inicial} es $S$.
\end{itemize}

El funcionamiento de la gramática anterior es el siguiente: la segunda cláusula permite reconocer todas las
cláusulas asociadas a la cadena original. Mientras que las cláusulas de la 4 a la 13 permiten contar
la cantidad de a o b en la cláusula (osea la cantidad de literales de cada cláusula), para esto se definen
4 estados: se reconocieron 0 a o b, se reconocieron una a o b, se reconocieron 2 a o b y se reconocen
más de 2 a o b, los cuales están representados por los predicados $A_0$, $A_1$, $A_2$ y $A_3$ respectivamente.
Se crean las derivaciones entre los predicados que representan cada estado y se deriva en la cadena vacía
desde el predicado $A_2$ cuando el argumento es la cadena vacía, lo que indica que la cláusula tiene 2 literales.


Como para todo problema en P existe una RCG que lo reconoce en su representación como lenguaje formal,
entonces es posible resolver las todas las instancias polinomiales del SAT usando gramáticas de
concatenación de rango.

Si se observa el enfoque seguido en la construcción de $G_{S-SAT}$, en la representación del SAT como cadena se
trabaja con una instancia del SAT general por lo que no se tienen en cuenta las propiedades específicas
del problema, que en el caso de las instancias polinomiales, es lo que permite que el algoritmo para las
misma sea polinomial.

Entonces la gramática que reconoce los problemas 2-SAT satisfacible sería:
$$G_{S-2-SAT}=G_{S-SAT}\cap G_{2-SAT}.$$
Pero $G_{S-2-SAT}$ el problema de la palabra para $G_{S-2-SAT}$ es exponencial y se conoce que para el 2-SAT
existe un algoritmo polinomial.

Como para todo problema en P existe una RCG que lo reconoce en su representación como lenguaje formal,
entonces es posible resolver el 2-SAT y todas las todas las instancias polinomiales del SAT usando gramáticas de
concatenación de rango. Pero no se ha podido encontrar una RCG que permita reconocer problemas 2-SAT satisfacibles
en un tiempo polinomial, esto se prepone como problema abierto ya que puede conducir a entontrar otras instancias 
de SAT solubles en tiempo polinomial.



% Entonces\agregaesto{COMA} por la anterior observación \comment{el autor}{¿el autor de qué?} considera que esta no es una via factible para atacar los problemas de SAT polinomiales, por lo que se cree que se requiere una interpretación diferente que explote las propiedades del SAT analizado. \notaparaelautor{Esto no me queda claro} \agregaesto{NUEVO PÁRRAFO.}

% \notaparaelautor{De todas formas, y pensando así por arribita... tú sí puedes construir una gramática que reconozca 2 sat... eso es simplemente que solo haya dos «a o b» en cada cláusula... Y eso lo puedes hace añadiendo nuevas cláusulas que cuente: no hay ninguna, ya hay una, y ya haya dos... ¿eso no está hecho? Lo que habría que decir entonces es que como esa gramática se deriva casi textualmente de «la general» sigue teniendo el problema que de que su reconocimiento es exponencial... Y aquí es que vendría entonces... peeeeeeero, como las RCG reconocen todo lo que sea polinomial debería existir OTRA RCG que también reconozca el 2-SAT, y eso es lo que no hemos logrado hacer nosotros y proponemos como problema abierto, proque al estudiar las características de esa gramática se pudieran identificar nuevos SAT con solución polionomial. Creo que esot es lo que deberíamos tener aquí. }

% Por otro lado si se toma \comment{el enfoque de $G_{S-SAT}$}{¿qué esto?} puede ser posible \comment{que estas propiedades}{¿cuáles propiedades?} puedan \comment{ser explotadas durante el reconocimiento de la gramática}{¿qué significa esto?} que como se mencionó en el capítulo \comment{1}{usa un comando ref, no pongas números cableados.} pasa por un mecanismo de memorización\cambio{,}{PUNTO} es decir, \comment{según las propiedades del SAT analizado modificar el mecanismo de memorización}{cambia el orden: modificar el mecanismo de memorización...} para que tenga en cuenta otras propiedades que permita que el mismo sea polinomial.

% \comment{Ahora}{Esto decididamente ya es una muletilla literaria. A partir de ahora te lo voy a marcar como error.} por lo anteriormente planteado y según los algoritmos conocidos para problemas como el 2-SAT que se basan en algoritmos sobre grafos que marcan las dependencias entre las cláusulas del SAT, el autor recomienda que quizás una via factible \comment{para atacar problemas}{¿atacarlos en general o siguiendo estaidea?}, como ejemplo el 2-SAT, \comment{sea tratar de armar transiciones dentro de una RCG que representen los mismos mecanismos de dependencias entre cláusulas que describe el grafo y luego realizar una asignación similar a como se hace en $G_{S-SAT}$.}{Uffff... esto suena interesante pero así no lo entiendo bien :-(.}

% Con las ideas presentadas anteriormente sería \comment{interesante}{Adjetivo} para futuros trabajos \quitaesto{de investigacion} encontrar una RCG para cada instancia polinomial del SAT, lo que completaría el análisis hecho en este trabajo abarcando los principales puntos de vista del SAT desde las gramáticas de concatenación de rango.

\begin{thebibliography}{99}

    \bibitem{mainRCGBib}
    Boullier, Pierre.
    \textit{Proposal for a Natural Language Processing Syntactic Backbone}.
    Research Report RR-3342, INRIA, 1998.

    \bibitem{propertiesRCGBib}
    Boullier, Pierre.
    \textit{A Cubic Time Extension of Context-Free Grammars}.
    Research Report RR-3611, INRIA, 1999.

    \bibitem{propertiesRCGBib1}
    Eberhard Bertsch and Mark-Jan Nederhof
    \textit{On the Complexity of Some Extensions of RCG Parsing}.
    International Workshop/Conference on Parsing Technologies, 2001.

    \bibitem{propertiesRCGBib2}
    Boullier, Pierre.
    \textit{Counting with range concatenation grammar}.
    Theor. Comput. Sci., 2003.

    \bibitem{simpleMatrixLanguages}
    Ibarra, Oscar H.
    \textit{Simple matrix languages}.
    \textit{Information and Control}, Vol. 17, No. 4, pp. 359-394, 1970.

    \bibitem{globalIndexLanguages}
    Castaño, José M.
    \textit{Global Index Languages}.
    Ph.D. Thesis, The Faculty of the Graduate School of Arts and Sciences, Brandeis University, 2004.

    \bibitem{authomataTheory}
    Hopcroft, John E., Motwani, Rajeev, y Ullman, Jeffrey D.
    \textit{Introduction to Automata Theory, Languages, and Computation}.
    3ª edición, Addison-Wesley, 2006. ISBN: 9780321455369.

    \bibitem{aCFSAT}
    Fernández Arias, Alina.
    \textit{El problema de la satisfacibilidad booleana libre del contexto}.
    Facultad de Matemática y Computación, Universidad de La Habana, 2007.

    \bibitem{aSRCSAT}
    Aguilera López, Manuel.
    \textit{Problema de la Satisfacibilidad Booleana de Concatenación de Rango Simple}.
    Facultad de Matemática y Computación, Universidad de La Habana, 2016.

    \bibitem{aSMSAT}
    Rodríguez Salgado, José Jorge.
    \textit{Gramáticas Matriciales Simples. Primera aproximación para una solución al problema SAT}.
    Facultad de Matemática y Computación, Universidad de La Habana, 2019.

\end{thebibliography}


\end{document}

\backmatter

\begin{conclusionsAndRecomendations}

    En este trabajo se presentó una estrategia para resolver el SAT usando teoría de lenguajes que se basa en 
    codificar una fórmula booleana mediante una cadena sobre el alfabeto $\{a,b,d,c\}$ y definir el lenguaje de 
    todas las fórmulas booleanas satisfacibles $L_{S-SAT}$. A partir de este lenguaje, para determinar si una 
    fórmula booleana es satisfacible solo es necesario verificar si la cadena asociada a dicha fórmula pertenece 
    a $L_{S-SAT}$.
    
    Además de definir el lenguaje $L_{S-SAT}$ se propuso una forma para construirlo utilizando una transducción 
    finita de una variante del lenguaje $Copy$. 
    
    En el capítulo \ref{chap:LSATFT}, se construyó $L_{S-SAT}$ mediante el transductor finito $T_{SAT}$ que 
    recibe cadenas del lenguaje $L_{0,1,d}$, las cuales representan todas las posibles interpretaciones para 
    fórmulas booleanas en CNF y genera cadenas sobre el alfabeto $\{a,b,c,d\}$, tales que cuando se interpretan 
    como fórmulas booleanas son satisfacibles por la cadena de $L_{0,1,d}$ que la generó. 
    
    Por la forma en que se construyó el lenguaje $L_{S-SAT}$ se tiene que el problema de la palabra para todo 
    formalismo que genere el lenguaje $L_{0,1,d}$ y sea cerrado bajo transducción finita, es NP-Duro, asumiendo como válida 
    la conjetura de que cualquier formalismo que genere el lenguaje $L_{0,1,d}$, tiene tamaño $O(1)$ en su 
    representación.
    
    En el capítulo \ref{chap:LSATRCG}, se construyó una RCG que reconoce el lenguaje $L_{S-SAT}$, lo que permitió demostrar que no es 
    necesario construir $L_{S-SAT}$ mediante transducción finita. La gramática que se construyó tiene el 
    problema de la palabra no polinomial, y constituye un ejemplo de una RCG donde el algoritmo de reconocimiento 
    es no polinomial.  Además, al obtener una RCG que reconoce $L_{S-SAT}$, se demostró que las 
    RCG reconocen todos los problemas de la clase NP.
    
    Las estrategias presentadas constituyen una vía diferente para resolver el SAT, y aunque el problema de 
    la palabra para el formalismo que se construyó es no polinomial, este acercamiento puede contribuir a nuevas 
    líneas de investigación para la búsqueda de algoritmos eficientes que permitan resolver el SAT.
    
    A partir del trabajo realizado se proponen como temas para investigaciones futuras los
    siguientes:
    
    \begin{itemize}
        \item Buscar otro formalismo que genere el lenguaje $L_{0,1,d}$, que sea cerrado bajo transducción finita, y analizar el problema de la palabra para el formalismo que se obtiene después de aplicarle el transductor $T_{SAT}$. En la literatura
              consultada \cite{globalIndexLanguages} para la realización de este trabajo se encontró un formalismo que cumple las propiedades anteriores.
        \item Demostrar que cualquier formalismo que genere $L_{0,1,d}$ tiene un tamaño $O(1)$ en su representación.
        \item Analizar qué tipo de formalismo se obtiene al aplicarle el transductor $T_{SAT}$ a la RCG que reconoce el lenguaje $L_{0,1,d}$.
        \item Analizar qué propiedades limitan que las RCG no sean cerradas bajo transducción finita, construir un formalismo basado en las RCG que sea cerrado bajo transducción finita y comprobar si este formalismo es capaz de describir el lenguaje $L_{0,1,d}$.
        \item Construir una RCG que reconozca fórmulas booleanas satisfacibles, donde cada cláusula tiene a lo sumo dos literales (2-SAT), y que tenga el problema de la palabra polinomial.
    \end{itemize}
    
\end{conclusionsAndRecomendations}

\printbibliography[heading=bibintoc]


\end{document}