\documentclass[12pt,oneside]{uhthesis}

\usepackage{subfigure}
\usepackage[ruled,lined,linesnumbered,titlenumbered,algochapter,spanish,onelanguage]{algorithm2e}
\usepackage{amsmath}
\usepackage{amssymb}
\usepackage{amsbsy}
\usepackage{caption,booktabs}
\captionsetup{ justification = centering }
%\usepackage{mathpazo}
\usepackage{float}
\setlength{\marginparwidth}{2cm}
\usepackage{todonotes}
\usepackage{listings}
\usepackage{xcolor}
\usepackage{multicol}
\usepackage{graphicx}
\floatstyle{plaintop}
\restylefloat{table}
\addbibresource{Bibliography.bib}
% \setlength{\parskip}{\baselineskip}%
\renewcommand{\tablename}{Tabla}
\renewcommand{\listalgorithmcfname}{Índice de Algoritmos}
%\dontprintsemicolon
\SetAlgoNoEnd

\usepackage{tikz}
\usetikzlibrary{automata, positioning}

\definecolor{codegreen}{rgb}{0,0.6,0}
\definecolor{codegray}{rgb}{0.5,0.5,0.5}
\definecolor{codepurple}{rgb}{0.58,0,0.82}
\definecolor{backcolour}{rgb}{0.95,0.95,0.92}

\lstdefinestyle{mystyle}{
    backgroundcolor=\color{backcolour},   
    commentstyle=\color{codegreen},
    keywordstyle=\color{purple},
    numberstyle=\tiny\color{codegray},
    stringstyle=\color{codepurple},
    basicstyle=\ttfamily\footnotesize,
    breakatwhitespace=false,         
    breaklines=true,                 
    captionpos=b,                    
    keepspaces=true,                 
    numbers=left,                    
    numbersep=5pt,                  
    showspaces=false,                
    showstringspaces=false,
    showtabs=false,                  
    tabsize=4
}

\lstset{style=mystyle}

\title{Título de la tesis}
\author{\\\vspace{0.25cm}Raudel Alejandro Gómez Molina}
\advisor{\\\vspace{0.25cm}Fernando Rodríguez Flores}
\degree{Licenciado en Ciencia de la Computación}
\faculty{Facultad de Matemática y Computación}
\date{Fecha\\\vspace{0.25cm}\href{https://github.com/raudel25/my-thesis}{https://github.com/raudel25/my-thesis}}
\logo{Graphics/uhlogo}
\makenomenclature

\renewcommand{\vec}[1]{\boldsymbol{#1}}
\newcommand{\diff}[1]{\ensuremath{\mathrm{d}#1}}
\newcommand{\me}[1]{\mathrm{e}^{#1}}
\newcommand{\pf}{\mathfrak{p}}
\newcommand{\qf}{\mathfrak{q}}
%\newcommand{\kf}{\mathfrak{k}}
\newcommand{\kt}{\mathtt{k}}
\newcommand{\mf}{\mathfrak{m}}
\newcommand{\hf}{\mathfrak{h}}
\newcommand{\fac}{\mathrm{fac}}
\newcommand{\maxx}[1]{\max\left\{ #1 \right\} }
\newcommand{\minn}[1]{\min\left\{ #1 \right\} }
\newcommand{\lldpcf}{1.25}
\newcommand{\nnorm}[1]{\left\lvert #1 \right\rvert }
\renewcommand{\lstlistingname}{Ejemplo de código}
\renewcommand{\lstlistlistingname}{Ejemplos de código}

\begin{document}

\frontmatter
\maketitle

\begin{dedication}
    A mi familia
\end{dedication}
\begin{acknowledgements}
    A mi familia, por acompañarme en este camino, lleno de retos y desafíos, pero que hoy recoge el fruto de tanto esfuerzo.
    A mi mamá, por estar a mi lado en todo momento y por ser mi mayor apoyo. A mi hermana y mi primito por ser mi mayor motivación y compartir
    momentos de diversión y alegría. A mis abuelos por ser mi ejemplo a seguir y por enseñarme a ser mejor persona.
    A mis tíos por ser casi casi como mis segundos padres y estar en todo momento.
    
    A mis compañeros de la universidad, por ser parte de esta gran aventura y por compartir momentos inolvidables, por 
    estar siempre ahí para apoyarme, por compartir todas las horas de intenso estudio y el estrés de tantos proyectos y pruebas.
    A Anabel, Daniel, Alex, Omar, Javier, Juan Carlos y a todos los demás con los que compartí durante estos años.
    
    A la gente de la beca, mi casa durante estos años, por ser mi familia universitaria y estar siempre.
    
    A la gente del concurso: Gaby, Adilen, Arianna, Ailec, Marquito, Tito, Jocdan, Rey, Leo,  por ir juntos en el viaje por el mundo de las ciencias, que comenzó en el pre y 
    se mantiene hasta nuestros días. 
    
    A Alex y a Rafael, por tantos rounds de 5 horas dándonos cabezazos contra los ejercicios en ancias del tan esperado 
    \textit{ACCEPTED}.
    
    A los profes Manzano y Donet que introdujeron en mí la pasión por la matemática, lo que posteriormente se convirtió
    en amor por la programación y los algoritmos.
    
    A los profes de la universidad, por su guía y su apoyo, por la formación y la paciencia, en especial a los profes (colegas) de EDA,
    a Cartaya por sus consejos y enseñanzas, a mi tutor por su patrocinio y exigencia.
    
    A todos los que de una manera u otra aportaron su granito de arena para que hoy pueda estar aquí.
\end{acknowledgements}
\begin{opinion}
    En este trabajo se propone una vía de solución para el SAT utilizando elementos de la teoría de lenguajes formales.  Además, se define y construye el lenguaje de todas las fórmulas lógicas satisfacibles y se analizan algunas de las implicaciones que se derivan de este lenguaje.   Los resultados que se recogen en el documento abren nuevas e interesantes líneas de trabajo.

   Para esta tesis, Raudel tuvo que estudiar, de manera independiente, contenidos que no forman parte de su plan de estudios y usarlos de manera creativa y original.

   Considero que estamos en presencia de un trabajo excelente, desarrollado por un excelente científico de la computación.

   \vspace{1cm}


 \begin{flushright}
   \underline{\hspace{6.5cm}}\\
   MSc. Fernando Raul Rodriguez Flores

   Facultad de Matemática y Computación
  
   Universidad de la Habana

   Febrero, 2025
 \end{flushright}
\end{opinion}
\begin{resumen}
	El problema de la satisfacibilidad booleana es un problema NP-Completo y consiste en determinar si existe
	alguna interpretación verdadera de una fórmula booleana dada. La teoría de lenguajes es una rama fundamental de la Ciencia de la Computación y la matemática que se enfoca 
	en el estudio de los lenguajes formales. El objetivo de este trabajo es vincular el problema de la satisfacibilidad
	booleana y la teoría de lenguajes, para construir el lenguaje de todas las fórmulas booleanas satisfacibles.
	Para esta construcción se emplean 2 estrategias, la primera utiliza una transducción de una variante del lenguaje
	\textit{Copy} para construir dicho lenguaje y la segunda utiliza una gramática de concatenación de rango 
	que reconoce este lenguaje. Como resultado de la primera estrategia se demuestra que el problema de la palabra 
	para todos los formalismos
	que generen la variante del lenguaje \textit{Copy} y sean cerrados bajo transducción finita, es NP-Duro.
	Por otro lado, la gramática de concatenación de rango que se obtiene en la segunda estrategia permite 
	demostrar que las gramáticas de concatenación de rango reconocen todos los problemas de la clase NP, en 
	su representación como lenguaje formal.
\end{resumen}

\begin{abstract}
	The boolean satisfiability problem is an NP-Complete problem and consists in determining whether there exists
	any true interpretation of a given boolean formula. Language theory is a fundamental branch of Computer Science and Mathematics that focuses
	on the study of formal languages. The objective of this work is to link the problem of boolean
	satisfiability and language theory, to build the language of all satisfiable boolean formulas.
	For this construction, 2 strategies are used, the first uses a transduction of a variant of the language
	\textit{Copy} to build said language and the second uses a range concatenation grammar
	that recognizes this language. As a result of the first strategy, it is shown that the word problem
	for all formalisms
	that generate the variant of the language \textit{Copy} and are closed under finite transduction, is NP-Hard.
	On the other hand, the rank concatenation grammar obtained in the second strategy allows
	to demonstrate that rank concatenation grammars recognize all the problems of the NP class, in
	their representation as a formal language.
\end{abstract}
\tableofcontents
\listoffigures
% \listoftables
% \listofalgorithms
% \lstlistoflistings

\mainmatter

\documentclass[12pt]{article}

\usepackage[utf8]{inputenc} % Permite escribir caracteres especiales directamente
\usepackage[spanish]{babel} % Configura el idioma a español

\usepackage{amsmath}
\usepackage{tikz}
\usepackage{xcolor}
\usepackage[lmargin=2cm,rmargin=5cm]{geometry}

%%%{{{ Comments and the like
\usepackage[textwidth=4cm]{todonotes}
\usepackage{soul}
\usepackage{xcolor}
\newcounter{todocounter}
\newcommand{\comment}[2]{\stepcounter{todocounter}
  {\color{green!50!blue}{(#1$^{{\color{black}\textbf{\thetodocounter}}}$)}}
  \todo[color=green,noline,size=\tiny]{\textbf{\thetodocounter:} #2

  }}
\newcommand{\quitaesto}[1]{{\color{red}(\st{#1})}}

\newcommand{\cambio}[2]{{\color{cyan}{{#2}}}{\color{red}{(\st{#1})}}}

\newcommand{\agregaesto}[1]{{\color{cyan}{{#1}}}}

\newcommand{\notaparaelautor}[1]{{\color{brown}{\textbf{#1}}}}

\newcommand{\errorortografico}[1]{{\fcolorbox{gray}{magenta}{\textcolor{yellow}{\bf #1}}}}
    
%%%}}}


\title{Introducción}
\author{Raudel Alejandro Gómez Molina}

\begin{document}

\maketitle

El problema de satisfacibilidad booleana (\textit{SAT}) es uno de los problemas más estudiados en la teoría de la computación y la lógica.
Consiste en determinar si existe una asignación de valores verdaderos o falsos que satisfaga una fórmula booleana dada, compuesta
por variables y operadores lógicos como conjunciones, disyunciones y negaciones. SAT surge en 1971 como el primer problema NP-completo demostrado por
Stephen Cook,
lo que significa que, en el peor de los casos, su resolución requiere tiempo exponencial respecto al
tamaño de la entrada, pero también que muchos otros problemas pueden reducirse a él. Esto
implica un especial interés por parte de la comunidad científica en la búsqueda de métodos eficientes para la solución
del SAT.

La teoría de lenguajes es una rama fundamental de la Ciencia de la Computación y la matemática que se
enfoca en el estudio de los lenguajes formales. Estos lenguajes, definidos a través de gramáticas,
autómatas y expresiones regulares, permiten modelar y analizar la estructura de los lenguajes naturales y
artificiales. Su aplicación es amplia y abarca desde el diseño de compiladores y procesadores de lenguaje
natural hasta la verificación de sistemas y la teoría de la computabilidad.

Los lenguajes formales se
clasifican en jerarquías, como la jerarquía de Chomsky, que los organiza según su complejidad y poder
expresivo. Esta teoría proporciona las bases para entender cómo se pueden reconocer, generar y transformar
cadenas de símbolos, lo que resulta esencial en el desarrollo de herramientas computacionales para
el procesamiento de información. Además, la teoría de lenguajes constituye la base de los problemas de la Ciencia
de la Computación, ya que cualquier problema puede ser interpretado como un problema de la teoría de lenguajes.

En este trabajo se vinculan las dos ramas de la computación descritas anteriormente, presentando un enfoque para resolver el SAT
utilizando formalismos de teoría de lenguajes. Dicho enfoque resulta un tema no evidenciado en la literatura consultada
y permite demostrar que una serie de problemas relacionados a la teoría de lenguajes pertenecen a la clase NP-completo.

En estudios anteriores, que siguen la idea presentada en esta investigación, se han mostrado estrategias para la solución
de instancias específicas del SAT, usando formalismos de teoría de lenguajes, lo que constituye una solución limitada en su alcance.
En cambio, en este trabajo se presenta una alternativa que resuelve cualquier instancia del mismo, lo que, a criterio del autor, resulta una solución
cualitativamente superior. Esta sigue siendo una estrategia no eficiente, pero que
muestra un nuevo enfoque para resolver el SAT de forma general, y permite abrir nuevas líneas de investigación en este tema.

Para resolver cualquier instancia de SAT empleando formalismos de teoría de lenguajes se propone definir una codificación
de una fórmula booleana en una cadena que se pueda interpretar por algún formalismo de la teoría de lenguajes
y usando dicha codificación se define el lenguaje de todas las fórmulas booleanas satisfacibles. Entonces si se desea
determinar si una fórmula booleana es satisfacible es necesario determinar si la cadena asociada a la  fórmula booleana pertenece o no al lenguaje de todas las fórmulas booleanas satisfacibles.

Para construir el lenguaje
de las fórmulas booleanas satisfacibles se propone utilizar 2 métodos: el primero utiliza un transductor finito y el segundo
utiliza una gramática de concatenación de rango.

A partir de lo expuesto anteriormente se formula como objetivo general de de este trabajo: definir y construir el lenguaje de todas las fórmulas booleanas satisfacibles.

Para cumplir el objetivo general se definen los siguientes objetivos específicos:

\begin{itemize}
      \item Estudiar el estado del arte referido a los formalismos de teoría de lenguajes y el SAT.
      \item Establecer una representación del SAT como una cadena que pueda ser interpretada por un formalismo de la teoría de lenguajes.
      \item Definir el lenguaje de todas las fórmulas booleanas satisfacibles.
      \item Construir el lenguaje de todas las fórmulas booleanas satisfacibles utilizando un transductor finito.
      \item Construir el lenguaje de todas las fórmulas booleanas satisfacibles utilizando gramáticas de concatenación de rango.
\end{itemize}

Para dar cumplimiento a los objetivos trazados, el trabajo se ha estructurado en 4 capítulos: en los 2 primeros se presentan los principales conceptos y definiciones
que serán utilizados en el resto de la investigación y en los restantes 2 capítulos se define y construye el lenguaje de todas las fórmulas booleanas satisfacibles.

En el capítulo \ref{chap:preliminaries} se presentan los principales conceptos y definiciones de la teoría de lenguajes y el SAT, los cuales
son necesarios para la comprensión de los restantes capítulos. Además, se realiza un análisis de 2 trabajos anteriores
que muestran cómo solucionar instancias específicas del SAT utilizando un algoritmo polinomial.

En el capítulo \ref{chap:RCG} se realiza un análisis detallado de las gramáticas de concatenación de rango, presentando las principales
definiciones, proceso de derivación y análisis de la complejidad del algoritmo de reconocimiento.

En el capítulo \ref{chap:LSATFT} se muestra cómo codificar una fórmula booleana mediante una cadena de símbolos y luego
se analiza cómo interpretar una cadena como la asignación de valores para las variables de una fórmula booleana.
Posteriormente, se define el lenguaje de todas las fórmulas booleanas satisfacibles y se muestra cómo construir dicho
lenguaje mediante un transductor finito. Para finalizar, se demuestra que el problema de la palabra, para todos los formalismo que cumplan ciertas propiedades,
las cuales se definen en el capítulo \ref{chap:LSATFT}, es NP-Duro.

En el capítulo \ref{chap:LSATRCG} se demuestra que no es necesario construir el lenguaje de todas las fórmulas
booleanas satisfacibles mediante transducción finita, ya que existe una gramática de concatenación de rango que reconoce
este lenguaje. Por otro lado, se demuestra que las gramáticas de concatenación de rango cubren todos los problemas de la clase NP-Completo.


\begin{thebibliography}{99}

      \bibitem{mainRCGBib}
      Boullier, Pierre.
      \textit{Proposal for a Natural Language Processing Syntactic Backbone}.
      Research Report RR-3342, INRIA, 1998.

      \bibitem{propertiesRCGBib}
      Boullier, Pierre.
      \textit{A Cubic Time Extension of Context-Free Grammars}.
      Research Report RR-3611, INRIA, 1999.

      \bibitem{simpleMatrixLanguages}
      Ibarra, Oscar H.
      \textit{Simple matrix languages}.
      \textit{Information and Control}, Vol. 17, No. 4, pp. 359-394, 1970.

      \bibitem{globalIndexLanguages}
      Castaño, José M.
      \textit{Global Index Languages}.
      Ph.D. Thesis, The Faculty of the Graduate School of Arts and Sciences, Brandeis University, 2004.

      \bibitem{authomataTheory}
      Hopcroft, John E., Motwani, Rajeev, y Ullman, Jeffrey D.
      \textit{Introduction to Automata Theory, Languages, and Computation}.
      3ª edición, Addison-Wesley, 2006. ISBN: 9780321455369.

      \bibitem{aCFSAT}
      Fernández Arias, Alina.
      \textit{El problema de la satisfacibilidad booleana libre del contexto}.
      Facultad de Matemática y Computación, Universidad de La Habana, 2007.

      \bibitem{aSRCSAT}
      Aguilera López, Manuel.
      \textit{Problema de la Satisfacibilidad Booleana de Concatenación de Rango Simple}.
      Facultad de Matemática y Computación, Universidad de La Habana, 2016.

      \bibitem{aSMSAT}
      Rodríguez Salgado, José Jorge.
      \textit{Gramáticas Matriciales Simples. Primera aproximación para una solución al problema SAT}.
      Facultad de Matemática y Computación, Universidad de La Habana, 2019.

\end{thebibliography}


% Posibles conclusiones
% - teorica
% - como las gramáticas de concatenacion de rango constituyen un nuevo enfoque en la solucion del satisfacibilidad

% Posibles recomendaciones
% - por que via del transductor Full-SAT pueden desarrollarse nuevas investigaciones


\end{document}
\chapter{Preliminares}

\section{Teoría de Lenguajes}

\subsection{Conceptos básicos}

\paragraph{Alfabeto:} Un alfabeto, denotado como $\Sigma$, es un conjunto finito y no vacío de símbolos, ejemplo:
$$\Sigma=\{1,0\}$$
\paragraph{Cadena:} Una cadena es una sucesión finita de símbolos del alfabeto, ejemplo: la representación binaria de
los números $3=11$ y $5=101$ puede ser un ejemplo de cadena sobre el alfabeto $\Sigma$ anteriormente
definido.
\paragraph{Lenguaje:} Un lenguaje es un conjunto de cadenas definido sobre un alfabeto, ejemplo: el lenguaje de la
representación binaria de todos los números pares $L=\{w\,|\,\text{last}(w)=0\}$, $\text{last}(w)$
representa el último caracter de la cadena $w$.

\subsection{Operaciones con Lenguajes}

\paragraph{Unión:} La unión de dos lenguajes $L_1$ y $L_2$ se define como el conjunto de cadenas que
pertenecen a $L_1$ o a $L_2$:
$$L_1\cup L_2=\{w\,|\,w\in L_1\,\vee\,w\in L_2\}$$
\paragraph{Intersección:} La intersección de dos lenguajes $L_1$ y $L_2$ se define como el conjunto de
cadenas que pertenecen a $L_1$ y a $L_2$:
$$L_1\cap L_2=\{w\,|\,w\in L_1\,\wedge\,w\in L_2\}$$
\paragraph{Concatenación:} La concatenación de dos lenguajes $L_1$ y $L_2$ se define como el conjunto
de cadenas que resultan de concatenar una cadena de $L_1$ con una cadena de $L_2$:
$$L_1\circ L_2=\{w_1w_2\,|\,w_1\in L_1\,\wedge\,w_2\in L_2\}$$
\paragraph{Complemento:} El complemento de un lenguaje $L$ se define como el conjunto de cadenas que no
pertenecen a $L$:
$$\overline{L}=\{w\,|\,w\notin L\}$$
\paragraph{Clausura de Kleene:} La clausura de Kleene de un lenguaje $L$ se define como el conjunto de
cadenas que resultan de concatenar cero o más cadenas de $L$:
$$L^*=\{w_1w_2\ldots w_n\,|\,n\geq 0\,\text{y}\,w_i\in L\}$$

\subsection{Problemas relacionados con Lenguajes}

\paragraph{Problema de la palabra:} Consiste en determinar si una cadena pertenece a un lenguaje dado. Todo problema en Ciencias de la Computación puede ser reducido a un problema de la palabra, ya que cualquier problema
puede ser codificado como un lenguaje formal.
\paragraph{Problema del vacío:} Consiste en determinar si un lenguaje es vacío.
\paragraph{Problema de la finitud:} Consiste en determinar si un lenguaje es finito.
\paragraph{Problema de la equivalencia:} Consiste en determinar si dos $L_1$ y $L_2$ lenguajes son iguales (es decir si se cumple que
$L_1\subseteq L_2 \wedge L_2\subseteq L_1$).

\subsection{Gramáticas}

Una \textbf{gramática} es un sistema matemático utilizado para describir lenguajes formales. Se define como una 4-tupla:
\[
      G = (N, \Sigma, P, S),
\]
donde:
\begin{itemize}
      \item \(N\): Es un conjunto finito de \textbf{símbolos no terminales}, que representan variables o categorías intermedias.
      \item \(\Sigma\): Es un conjunto finito de \textbf{símbolos terminales}, que constituyen el alfabeto del lenguaje. Se cumple que \(N \cap \Sigma = \emptyset\).
      \item \(P\): Es un conjunto finito de \textbf{reglas de producción}, cada una de la forma:
            \[
                  \alpha \to \beta, \quad \text{donde } \alpha \in (N \cup \Sigma)^* \wedge \beta \in (N \cup \Sigma)^*.
            \]
      \item \(S\): Es el \textbf{símbolo inicial}, \(S \in N\), que define el punto de partida para derivar cadenas del lenguaje.
\end{itemize}

El lenguaje generado por una gramática \(G\) se denota como:
\[
      L(G) = \{ w \in \Sigma^* \mid S \overset{*}{\Rightarrow} w \},
\]
donde \(\overset{*}{\Rightarrow}\) indica una derivación en cero o más pasos.

\subsection{Jerarquía de Chomsky}

La \textbf{Jerarquía de Chomsky} (Figura~\ref{fig:ChomskySchema}) clasifica las gramáticas en cuatro tipos, según las restricciones en sus reglas de producción y la capacidad expresiva de los lenguajes que generan.

\begin{enumerate}
      \item \textbf{Tipo 0: Gramáticas irrestrictas}
            \begin{itemize}
                  \item No tienen restricciones en las reglas de producción.
                  \item Cada regla tiene la forma: \(\alpha \to \beta\), donde \(\alpha, \beta \in (N \cup \Sigma)^*\) y \(\alpha \neq \varepsilon\).
                  \item Generan los \textbf{lenguajes recursivamente enumerables}.
            \end{itemize}
            
      \item \textbf{Tipo 1: Gramáticas sensibles al contexto}
            \begin{itemize}
                  \item Cada regla tiene la forma: \(\alpha A \gamma \to \alpha \beta \gamma\), donde \(A \in N\), \(\alpha, \beta, \gamma \in (N \cup \Sigma)^*\), y \(|\beta| \geq 1\).
                  \item Generan los \textbf{lenguajes sensibles al contexto}.
            \end{itemize}
            
      \item \textbf{Tipo 2: Gramáticas libres de contexto}
            \begin{itemize}
                  \item Cada regla tiene la forma: \(A \to \beta\), donde \(A \in N\) y \(\beta \in (N \cup \Sigma)^*\).
                  \item Generan los \textbf{lenguajes libres de contexto}.
            \end{itemize}
            
      \item \textbf{Tipo 3: Gramáticas regulares}
            \begin{itemize}
                  \item Las reglas tienen la forma:
                        \[
                              A \to aB \quad \text{o} \quad A \to a,
                        \]
                        donde \(A, B \in N\) y \(a \in \Sigma\).
                  \item Generan los \textbf{lenguajes regulares}.
            \end{itemize}
\end{enumerate}


\begin{figure}
      \centering
      \begin{tikzpicture}[scale=1]
            \draw[thick, fill=red!20] (0,0) ellipse (4cm and 2cm);
            \node at (0,1.7) {\textbf{\textcolor{red!70!black}{Tipo 0}}};
            
            \draw[thick, fill=blue!20] (0,0) ellipse (3cm and 1.5cm);
            \node at (0,1.2) {\textbf{\textcolor{blue!70!black}{Tipo 1}}};
            
            \draw[thick, fill=green!20] (0,0) ellipse (2cm and 1cm);
            \node at (0,0.7) {\textbf{\textcolor{green!70!black}{Tipo 2}}};
            
            \draw[thick, fill=yellow!20] (0,0) ellipse (1cm and 0.5cm);
            \node at (0,0) {\textbf{\textcolor{yellow!70!black}{Tipo 3}}};
      \end{tikzpicture}
      \caption{Esquema de la Jerarquía de Chomsky}
      \label{fig:ChomskySchema} %
\end{figure}

\section{Complejidad computacional}

\subsection{Máquina de Turing}

Una Máquina de Turing \cite{authomataTheory} es un modelo abstracto de computación universal introducido por Alan Turing en 1936. Este modelo consiste en los siguientes componentes:

\begin{itemize}
      \item \textbf{Cinta}: Un medio de almacenamiento infinito dividido en celdas, donde cada celda contiene un símbolo de un alfabeto finito.
      \item \textbf{Cabezal de lectura/escritura}: Un dispositivo que puede leer el contenido de una celda, escribir un nuevo símbolo y moverse a la izquierda o derecha.
      \item \textbf{Conjunto de estados}: Una colección finita de estados internos que describen la configuración actual de la máquina.
      \item \textbf{Función de transición}: Un conjunto de reglas que determinan cómo la máquina cambia de estado, escribe en la cinta y mueve el cabezal en función del estado actual y el símbolo leído.
\end{itemize}

\paragraph{Máquina de Turing determinista (\textit{DTM}):}
En una Máquina de Turing determinista, para cada estado y cada símbolo leído, existe como máximo una transición
definida.
\paragraph{Máquina de Turing no determinista (\textit{NTM}):}
En una Máquina de Turing no determinista, para cada estado y símbolo leído, pueden existir múltiples
transiciones posibles.

Una máquina de Turing consiste la definición formal de algoritmo en Ciencias de la Computación y es el eje central para la resolución de problemas.

\subsection{Notación asintótica}

La notación asintótica se utiliza para describir el comportamiento de una función $f(n)$ a medida que $n$ crece hacia el infinito. A continuación se definen las notaciones más comunes:

\begin{itemize}
      \item \textbf{Notación $O(f(n))$}: Una función $g(n)$ pertenece a $O(f(n))$ si existen constantes positivas $c$ y $n_0$ tales que:
            \[
                  g(n) \leq c \cdot f(n) \quad \text{para todo } n \geq n_0.
            \]
            Esta notación proporciona un límite superior asintótico para $g(n)$.
            
      \item \textbf{Notación $\Omega(f(n))$}: Una función $g(n)$ pertenece a $\Omega(f(n))$ si existen constantes positivas $c$ y $n_0$ tales que:
            \[
                  g(n) \geq c \cdot f(n) \quad \text{para todo } n \geq n_0.
            \]
            Esta notación proporciona un límite inferior asintótico para $g(n)$.
            
      \item \textbf{Notación $\Theta(f(n))$}: Una función $g(n)$ pertenece a $\Theta(f(n))$ si:
            \[
                  g(n) \in O(f(n)) \quad \text{y} \quad g(n) \in \Omega(f(n)).
            \]
            Es decir, $f(n)$ acota $g(n)$ tanto superior como inferiormente.
\end{itemize}

La notación asintótica permite describir el tiempo de ejecución un algoritmo en cuanto al número de operaciones básicos realizadas
por un modelo formal de cómputo (por ejemplo una máquina de Turing). Algoritmos como determinar el mínimo y el máximo de
un array son $\Theta(n)$, ya que necesitan realizar una cantidad de operaciones de orden $n$ en relación con el tamaño de la entrada.

\subsection{Clases de problemas}

Los problemas computacionales \cite{authomataTheory} se agrupan en diferentes clases según los recursos necesarios para resolverlos.

\paragraph{Problemas en la clase P:}
Un problema pertenece a la clase P si puede resolverse en tiempo polinomial mediante una Máquina de Turing determinista. Es decir, existe un algoritmo determinista que, para una entrada de tamaño $n$, produce la solución en tiempo $O(n^k)$ para alguna constante $k$.

\paragraph{Problemas en la clase NP:}
Un problema pertenece a NP si su solución puede verificarse en tiempo polinomial mediante una Máquina de Turing determinista. Alternativamente, un problema está en NP si puede resolverse en tiempo polinomial mediante una Máquina de Turing no determinista.

\paragraph{Problemas en la clase NP-Completo:}
Un problema pertenece a la clase NP-Completo, si pertenece a NP y además es tan difícil como cualquier otro problema en NP. Esto significa que cualquier problema en NP puede reducirse a este problema en tiempo polinómico.

\paragraph{Problemas no decidibles:}
Un problema es no decidible si no existe una Máquina de Turing que pueda resolverlo correctamente para todas las entradas posibles. Esto significa que no hay algoritmo que garantice una respuesta en tiempo finito en todos los casos. Un ejemplo clásico de problema no decidible es el \textit{Problema de la Parada}, que consiste en determinar si una Máquina de Turing se detendrá para una entrada dada.
\subsection{P vs NP}

La relación entre las clases P y NP es uno de los problemas abiertos más importantes en la teoría de la
computación. Hasta la fecha, se desconoce si $\text{P} = \text{NP}$ o si $\text{P} \neq \text{NP}$. Por otro
lado el conjunto de problemas NP-Completo brinda una base sólida para el problema anterior, ya que dada su
definición cualquier problema perteneciente a este conjunto que sea soluble en tiempo polinomial
implica que todos los problemas en NP lo son. Además, existen problemas que no están en P ni en NP, como
los problemas indecidibles.

\section{Transformaciones en Lenguajes Formales}

\subsection{Homorfismo}

Dado un alfabeto \( \Sigma \) y un alfabeto \( \Gamma \), un homomorfismo es una función:
\[
      h: \Sigma^* \to \Gamma^*
\]
tal que:
\begin{enumerate}
      \item Para cada \( a \in \Sigma \), \( h(a) \) es una cadena en \( \Gamma^* \).
      \item Para cualquier par de cadenas \( u, v \in \Sigma^* \), se cumple que:
            \[
                  h(uv) = h(u) h(v),
            \]
            es decir, el homomorfismo preserva la concatenación.
\end{enumerate}

\subsection{Transductor finito}

Un transductor finito es un modelo computacional que extiende los autómatas finitos al incluir tanto entradas como salidas.
Formalmente, un transductor finito es un autómata finito determinista o no determinista con una función de transición extendida
que asocia una salida a cada transición.

Un transductor finito puede representarse como una 6-tupla:
\[
      T = (Q, \Sigma, \Gamma, \delta, q_0, F),
\]
donde:
\begin{itemize}
      \item \(Q\) es el conjunto finito de estados.
      \item \(\Sigma\) es el alfabeto de entrada.
      \item \(\Gamma\) es el alfabeto de salida.
      \item \(\delta: Q \times \Sigma \to Q \times \Gamma^*\) es la función de transición, que mapea una combinación de estado actual y símbolo de entrada a un nuevo estado y una salida.
      \item \(q_0 \in Q\) es el estado inicial.
      \item \(F \subseteq Q\) es el conjunto de estados finales.
\end{itemize}

Observe que un homorfismo es un transductor finito de un solo estado y tantas transiciones hacia el mismo estado como transformaciones
de símbolos en el homomorfismo.


\section{Formalismos de escritura regulada}

\subsection{Autómata regular}

Un autómata regular \cite{authomataTheory}, también conocido como autómata finito, es un modelo matemático para reconocer lenguajes regulares. Este tipo de autómata se define como una máquina abstracta que procesa cadenas de símbolos de un alfabeto finito y determina si una cadena pertenece a un lenguaje regular.

Un autómata regular se puede representar como una 5-tupla $$\mathcal{A} = (Q, \Sigma, \delta, q_0, F)$$ donde:

\begin{itemize}
      \item $Q$: Es un conjunto finito de \textbf{estados}.
      \item $\Sigma$: Es el \textbf{alfabeto} finito de entrada.
      \item $\delta$: Es la \textbf{función de transición}, $\delta: Q \times \Sigma \to Q$, que define cómo el autómata cambia de estado en función del símbolo leído.
      \item $q_0 \in Q$: Es el \textbf{estado inicial} desde donde comienza la computación.
      \item $F \subseteq Q$: Es el conjunto de \textbf{estados de aceptación o estados finales}.
\end{itemize}

El autómata comienza en el estado inicial $q_0$ y procesa una cadena de entrada símbolo por símbolo. En cada paso, la función de transición $\delta$ determina el siguiente estado del autómata. Si, después de procesar toda la cadena, el autómata termina en un estado de aceptación $q \in F$, entonces la cadena es aceptada; de lo contrario, es rechazada.

Se puede extender el concepto de autómata finito añadiendo un nuevo tipo de transición que no consume ningún caracter, la cual recibe el nombre de transición $\varepsilon$.
Se puede demostrar que el conjunto de lenguajes reconocido por un autómata finito sin transiciones $\varepsilon$ (\textit{autómata finito determinista}) es equivalente al conjunto de lenguajes reconocidos por un autómata finito con transiciones $\varepsilon$ (\textit{autómata finito no determinista}).

\subsection{Autómata de pila y Gramáticas libres del contexto}

Un autómata de pila \cite{authomataTheory} es un modelo matemático de computación que extiende los autómatas finitos al incluir una estructura de datos adicional: una pila. Este modelo es capaz de reconocer lenguajes libres de contexto,
proporcionando una conexión directa con las gramáticas libres de contexto \textit{CFG}, es decir dada una gramática libre del contexto se puede construir un autómata de pila que reconozca el lenguaje generado por la gramática y viceversa.

Formalmente, un autómata de pila se define como una 7-tupla
\[
      \mathcal{A} = (Q, \Sigma, \Gamma, \delta, q_0, Z_0, F),
\]
donde:

\begin{itemize}
      \item $Q$: Es un conjunto finito de \textbf{estados}.
      \item $\Sigma$: Es el \textbf{alfabeto} finito de entrada.
      \item $\Gamma$: Es el \textbf{alfabeto} finito de la pila (conjunto de símbolos que se pueden almacenar en la pila).
      \item $\delta$: Es la función de transición, $\delta: Q \times (\Sigma \cup \{\varepsilon\}) \times \Gamma \to \mathcal{P}(Q \times \Gamma^*)$, que describe cómo cambia el estado, el contenido de la pila y la posición en la entrada.
      \item $q_0 \in Q$: Es el \textbf{estado inicial} desde donde comienza la computación.
      \item $Z_0 \in \Gamma$: Es el símbolo inicial en la pila.
      \item $F \subseteq Q$: Es el conjunto de \textbf{estados de aceptación o estados finales}.
\end{itemize}

Un autómata de pila procesa una cadena de entrada desde el estado inicial $q_0$ y puede utilizar transiciones $\varepsilon$ (sin consumir entrada). En cada paso, la función $\delta$ determina el nuevo estado, los símbolos que se empujan o desapilan, y el avance en la entrada. Si, tras procesar toda la cadena, el autómata termina en un estado de aceptación o con una pila vacía (dependiendo del criterio de aceptación), la cadena es aceptada.


\subsection{Gramáticas Matriciales}

Una gramática matricial \cite{simpleMatrixLanguages} de grado $n$ \textit{$n$-MG} es una 4-tupla:

\[
      G_n = (V, P, S,\Sigma)
\]

donde:
\begin{itemize}
      \item \( V \) es un conjunto finito de \textbf{símbolos no terminales}.
      \item \( \Sigma \) es un conjunto finito de \textbf{símbolos terminales}, con \( V \cap \Sigma = \emptyset \).
      \item \( P \) es un conjunto finito de matrices. Cada matriz es una secuencia ordenada de \textbf{producciones} de la forma:
            \[
                  [P_1, P_2, \dots, P_k]
            \]
            donde cada \( P_i \) es una regla \( A \to \alpha \), con $1\leq k\leq n$, \( A \in N \) y \( \alpha \in (N \cup T)^* \).
      \item \( S  \) es el \textbf{símbolo inicial}.
\end{itemize}

Observe que das CFG son gramáticas matriciales de grado 1, es decir $1-MG$.

\subsubsection{Proceso de derivación}

El proceso de derivación en una gramática matricial se realiza de la siguiente manera:
\begin{enumerate}
      \item Se selecciona una matriz \( [P_1, P_2, \dots, P_k] \in P \).
      \item Las reglas \( P_1, P_2, \dots, P_k \) se aplican de manera secuencial a la cadena actual.
      \item La derivación continúa hasta que la cadena derivada contenga solo símbolos terminales, es decir, pertenezca a \( T^* \).
\end{enumerate}

\subsubsection{Gramáticas Matriciales Simples}

Una gramática matricial simple de grado $n$ \textit{$n$-SMG} es una ($n$+3)-tupla:
$$
      G_n=(V_1,V_2,\ldots,V_n,P,S,\Sigma)
$$
donde:
\begin{itemize}
      \item \( V_1, V_2, \ldots, V_n \) son conjuntos finitos de \textbf{símbolos no terminales} disjuntos 2 a 2.
      \item \( \Sigma \) es un conjunto finito de \textbf{símbolos terminales}, con \( V_i \cap \Sigma = \emptyset\,\forall\,1\leq i\leq n \).
      \item \( P \) es un conjunto finito de matrices. Cada matriz es una secuencia ordenada \textbf{producciones} que cumplan con una de las siguientes reglas:
            \begin{itemize}
                  \item $[S\to w]$, donde $w\in \Sigma ^*$.
                  \item $[S\to a_{11}A_{11}\ldots A_{1k}a_{21}A_{21}\ldots A_{2k}\ldots A_{n1}a_{n1}\ldots A_{nk}b]$,
                        donde $\forall i,j$ con $1\leq i\leq n\wedge 1\leq j\leq k$ se cumple que
                        $A_{ij}\in V_i$, $a_{ij}\in \Sigma ^*$ y $b\in \Sigma ^*$.
                  \item $[A_1\to w_1,\ldots, A_n\to w_n]$, donde $A_i\in V_i\wedge w_i\in \Sigma ^*$ $\forall i\, 1\leq i\leq n$.
                  \item $[A_1 \to a_{11}A_{11}\ldots a_{1k}A_{1k}b_1,\ldots,A_n \to a_{n1}A_{n1}\ldots a_{nk}A_{nk}b_n]$, donde $\forall i,j$
                        con $1\leq i\leq n\wedge 1\leq j\leq k$ se cumple que
                        $A_{ij}\in V_i$, $a_{ij}\in \Sigma ^*$ y $b_{i}\in \Sigma ^*$.
            \end{itemize}
      \item \( S \) es el \textbf{símbolo inicial}.
\end{itemize}


Observe que la restricción impuesta sobre las $n$-MG es cada matriz de producciones debe contener exactamente $n$ reglas de producción
donde cada regla de producción utiliza no terminales de conjuntos distintos o puede contener una única producción cuya secuencia de no terminales
esta compuesta por una secuencia de subsecuentes de terminales de conjuntos distintos.

\subsection{Gramáticas de Índice Global}

Una gramática de índice global \textit{GIG} \cite{globalIndexLanguages}, es una extensión de las CFL, que añaden un mecanismo 
de memoria al proceso de derivación, esta característica permite la generación de lenguajes más generales que los generados por las CFL.
El mecanismo de memorización consiste en una pila en la cual se pueden almacenar símbolos que pertenecen a un conjunto predeterminado,
en cada producción se puede realizar una operación de push o pop en la pila o dejarla en su estado actual.

Una GIG es una 6-tupla:
$$
      G = (N, \Sigma, I, S, \#, P) 
$$
donde:

\begin{itemize}
      \item $N$ es un conjunto finito de \textbf{símbolos no terminales}.
      \item \( \Sigma \) es un conjunto finito de \textbf{símbolos terminales}, $\Sigma \cap N=\emptyset$.
      \item $I$ es un conjunto finito de \textbf{índices de pila}, $\Sigma \cap I=\emptyset \wedge I \cap I=\emptyset$.
      \item $S\in N$ es el \textbf{símbolo inicial}.
      \item $\#$ es el \textbf{símbolo inicial de la pila}, $\# \notin \Sigma \cup N \cup I$.
      \item $P$ es un conjunto finito de \textbf{producciones} que tienen la siguiente forma, donde $x\in I\cup N\cup \Sigma$ y $y\in I\cup N$:
            \begin{itemize}
                  \item $A \underset{\varepsilon}{\to} \alpha$ o $A \to \alpha$ (reglas epsilon o reglas libres del contexto).
                  \item $A \underset{[y]}{\to}  \alpha$ o $[..]A \to [..]\alpha$ (reglas epsilon o reglas con restricciones).
                  \item $A \underset{x}{\to} a \beta$ o $[..]A \to  [x..]a\beta$ (reglas de push o apertura de paréntesis).
                  \item $A \underset{\overline{x}}{\to} \alpha$ o $[x..]A \to [..]\alpha$ (reglas de pop o cierre de paréntesis).
            \end{itemize}
\end{itemize}

Como se puede observar la primera regla de producción consiste en dejar la pila intacta y puede ser interpretada como una regla de derivación
libre del contexto, la segunda regla consiste en dejar la pila intacta pero solo se puede realizar si el caracter en el tope de la pila es el
especificado, la tercera regla consiste en añadir un caracter a la pila y la cuarta regla consiste en eliminar un caracter de la pila.
Una gramática GIG solo con producciones de la primera regla de producción es equivalente a una CFG.

\subsubsection{Proceso de derivación}

Como se mencionó anteriormente el proceso de derivación en las GIG es idéntico al proceso de derivación de las CFG, con la diferencia que
en cada paso de la derivación se puede realizar una operación de push o pop en la pila, además de las operaciones de sustitución de símbolos.
Otra restricción adicional es que el proceso de derivación en las GIG debe ser siempre de extrema izquierda.

Entonces una cadena es reconocida por una GIG si existe una secuencia de derivaciones desde $S$ que genere la cadena y que además la pila
termine vacía al final de la derivación (con el símbolo $\#$ en el tope de la pila). Luego se puede definir el lenguaje generado por una GIG, $G$
como $L(G)=\{w\,|\,\#S\overset{*}{\to}\#w \wedge w\in \Sigma^* \}$.

\subsection{Propiedades de las GIG}

A continuación se describen las principales propiedades de las GIG:
\begin{itemize}
      \item \textbf{Cerradura bajo unión:} Dadas dos GIG $G_1$ y $G_2$, la unión de los lenguajes reconocidos por $G_1$ y $G_2$ es reconocida por una GIG
            $$G=(N_1\cup N_2\cup \{S\},\Sigma_1\cup \Sigma_2,I_1\cup I_2,S,\#,P_1\cup P_2\cup \{S \underset{\varepsilon}{\to} S_1|S_2\})$$
      \item \textbf{Cerradas bajo concatenación:} Dadas 2 GIG $G_1$ y $G_2$, la concatenación de los lenguajes reconocidos por $G_1$ y $G_2$ es reconocida por una GIG
            $$G=(N_1\cup N_2\cup \{S\},\Sigma_1\cup \Sigma_2,I_1\cup I_2,S,\#,P_1\cup P_2\cup \{S \underset{\varepsilon}{\to} S_1S_2\})$$
      \item \textbf{Cerradas bajo clausura de Kleene:} Dada una GIG $G$, el lenguaje reconocido por $G$ elevado a la clausura de Kleene es reconocido por una GIG
            $$G=(N\cup \{S'\},\Sigma,I\cup \{S'\},S,\#,P\cup \{S'\underset{\varepsilon}{\to} S'S|\varepsilon\})$$
      \item  \textbf{Cerradas bajo homomorfismo} Dada una GIG $G$, el homomorfismo de un lenguaje reconocido por $G$ es un lenguaje de índice global \cite{globalIndexLanguages}.
      \item  \textbf{Cerradas bajo transducción finita:} Dada una GIG $G$, la transducción finita de un lenguaje reconocido por $G$ es un lenguaje de índice global \cite{globalIndexLanguages}.
\end{itemize}

\subsection{Gramáticas de Concatenación de Rango}

Las gramáticas de concatenación de rango (\textit{RCG}) \cite{mainRCGBib} son un formalismo de gramáticas desarrollado para describir lenguajes más generales que los generados por gramáticas libres del contexto.
Este formalismo extiende las capacidades descriptivas al incluir relaciones entre rangos de la cadena de una manera más flexible,
permitiendo la generación de lenguajes sensibles al contexto.

\subsubsection{Definiciones}

\paragraph{Rango:} un rango es una tupla $(i, j)$ que representa un intervalo de posiciones en la cadena, donde $i$ y $j$ son enteros no negativos tales que $i \leq j$.

\paragraph{Gramática de Concatenación de Rango Positiva:} una gramática de concatenación de rango positiva (\textit{PRCG}) se define como una 5-tupla:

\[
      G = (N, T, V, P, S),
\]
donde:

\begin{itemize}
      \item $N$: Es un conjunto finito de \textbf{predicados o símbolos no terminales}: Cada predicado tiene una \textbf{aridad}, que indica el número de argumentos que toma.
      \item $T$: Es un conjunto finito de \textbf{símbolos terminales}.
      \item $V$: Es un conjunto finito de \textbf{variables}.
      \item $P$: Es un conjunto finito de \textbf{cláusulas}, de la forma:
            \[
                  A(x_1, x_2, \ldots, x_k) \to B_1(y_{1,1}, y_{1,2}, \ldots, y_{1,m_1}) \ldots B_n(y_{n,1}, y_{n,2}, \ldots, y_{n,m_n}),
            \]
            donde $A, B_i \in N$, $x_i, y_{i,j} \in (V \cup T)^*$, y $k$ es la aridad de $A$.
      \item $S \in N$: Es el \textbf{predicado inicial} de la gramática.
\end{itemize}

\paragraph{Gramática de Concatenación de Rango Negativa:} una gramática de concatenación de rango negativa (\textit{NRCG}) es similar a una PRCG, pero predicados o no terminales negativos que se denotan de la siguiente manera: $\overline{A}$.

\paragraph{Gramática de Concatenación de Rango Simple:} las gramáticas de concatenación de rango simple (\textit{SRCG}) son un subconjunto de las RCG que restringen la forma de las cláusulas de producción.
Una RCG $G$ es \textbf{simple} si los argumentos en el lado derecho de una cláusula son variables distintas, y todas estas variables (y no otras) aparecen una sola vez en los argumentos del lado izquierdo.
Un resultado interesante es que para cada CFL existe una SRCG equivalente que genera el mismo lenguaje.

\paragraph{Sustiución de rango:} una sustitución de rango es un mecanismo que reemplaza una variable por un rango de la cadena.
Por ejemplo dado el predicado $A(Xa)$ donde $X \in V \wedge a \in T$, si se instancia la cadena $baa$ en $A$, $X$ puede
ser asociada con el rango $ba$ de la cadena original.

\subsubsection{Proceso de derivación}

La principal idea detrás de las RCG, para realizar una derivación, se basa en encontrar para cada argumento del predicado izquierdo de una cláusula todas las
posibles sustituciones en rango de la cadena, asociar los valores de las variables a los argumentos de los predicados derechos y continuar
el proceso de derivación en los predicados derechos.

Por ejemplo, dada la cláusula $A(X,aYb)\to B(aXb,Y)$ , donde $X$ y $Y$ son símbolos variables y $a$ y $b$
son símbolos terminales, la cadena predicado $A(a,abb)$ deriva como $B(aab,b)$, porque $A(a,abb)$
coincide con $A(X,aYb)$ cuando $ X=a \wedge Y=b$. De forma similar, si existiera una regla

Una secuencia de argumentos son reconocidos por un predicado si existe una secuencia de derivaciones que comienza
en dicho predicado y termina en la cadena vacía, si el predicado es negativo en el caso de las NRCG ocurre lo contrario
la secuencia de argumentos no es reconocida por el predicado. Una RCG reconoce una cadena si dicha cadena es reconocida
por el predicado inicial.

Ejemplo dada la siguiente RCG:

\[
      G = (N, T, V, P, S),
\]
donde:

\begin{itemize}
      \item  N=$\{A,S\}$.
      \item T=$\{a,b,c\}$.
      \item V=$\{X,Y,Z\}$.
      \item El conjunto de cláusulas $P$ es el siguiente:
            $$S(XYZ)\to A(X,Y,Z)$$
            $$A(aX,aY,aZ)\to A(X,Y,Z)$$
            $$A(bX,bY,bZ)\to A(X,Y,Z)$$
            $$A(cX,cY,cZ)\to A(X,Y,Z)$$
            $$A(\varepsilon,\varepsilon,\varepsilon)\to \varepsilon$$
      \item El símbolo inicial es $S$.
\end{itemize}
La cadena $aaabbbccc$ es reconocida por la RCG anterior, ya que se puede derivar de la siguiente manera:
$$S(abcabcabc)\to A(abc,abc,abc)\to A(bc,bc,bc)\to A(c,c,c)\to A(\varepsilon,\varepsilon,\varepsilon)\to \varepsilon$$

De manera general el lenguaje reconocido por la RCG anterior es $L=\{www\,|\,w\in \{a,b,c\}^*\}$.

\subsubsection{Propiedades de las RCG}

A continuación se describen las principales propiedades de las RCG:
\begin{itemize}
      \item \textbf{Cerradura bajo unión:} Dadas dos RCG $G_1$ y $G_2$, la unión de los lenguajes reconocidos por $G_1$ y $G_2$ es reconocida por una RCG
            $$G=(N_1\cup N_2\cup \{S\},T_1\cup T_2,V_1\cup V_2,P_1\cup P_2\cup \{S(X)\to S_1(X)|S_2(X)\},S)$$
      \item \textbf{Cerradas bajo intersección:} Dadas dos RCG $G_1$ y $G_2$, la intersección de los lenguajes reconocidos por $G_1$ y $G_2$ es reconocida por una RCG
            $$G=(N_1\cup N_2\cup \{S\},T_1\cup T_2,V_1\cup V_2,P_1\cup P_2\cup \{S(X)\to S_1(X)S_2(X)\},S)$$
      \item \textbf{Cerradas bajo complemento:} Dada una RCG $G$, el complemento del lenguaje reconocido por $G$ es reconocido por una RCG
            $$G'=(N\cup \{\overline{S}\},T,V,P\cup \{S'(X)\to \overline{S}(X)\},S')$$
      \item \textbf{Cerradas bajo concatenación:} Dadas dos RCG $G_1$ y $G_2$, la concatenación de los lenguajes reconocidos por $G_1$ y $G_2$ es reconocida por una RCG
            $$G=(N_1\cup N_2\cup \{S\},T_1\cup T_2,V_1\cup V_2,P_1\cup P_2\cup \{S(XY)\to S_1(X)S_2(Y)\},S)$$
      \item \textbf{Cerradas bajo clausura de Kleene:} Dada una RCG $G$, la clausura de Kleene del lenguaje reconocido por $G$ es reconocida por una RCG
            $$G'=(N\cup \{S'\},T,V,P\cup \{S'(XY)\to S(X)S'(Y)|\varepsilon\},S')$$
      \item  \textbf{No cerradas bajo homomorfismo:} Dada una RCG $G$, el homomorfismo de un lenguaje reconocido por $G$ no es necesariamente reconocido por una RCG \cite{propertiesRCGBib}.
      \item  \textbf{No cerradas bajo transducción finita:} Dada una RCG $G$, la transducción finita de un lenguaje reconocido por $G$ no es necesariamente reconocida por una RCG.
            Esto es una consecuencia de la no cerradura bajo homomorfismo.
\end{itemize}


\subsubsection{Problema de la palabra, problema del vacío y equivalencia de 2 RCG}

\paragraph{Problema de la palabra:} En general en la mayoría de los casos este problema es polinomial y pasa por
un algoritmo de memorización sobre las cadenas que son instanciadas en los rangos de los predicados de la RCG \cite{mainRCGBib} (como la cantidad
máxima de rangos de la cadena es $n^2$ y la máxima aridad de un predicado es constante, este proceso de memorización cuenta
con cantidad polinomial de estados), en
una complejidad de $O(|P|n^{2h(l+1)})$ donde $h$ es la máxima aridad en un predicado, $l$ es la máxima cantidad de predicados
en el lado derecho de una cláusula y $n$ es la longitud de la cadena a ser reconocida.

Pero existen casos en los que el problema de la palabra no
es polinomial, por ejemplo puede pasar que se instancien argumentos de en los predicados con rangos que no pertenezcan
a la propia cadena de entrada y sean generados durante el proceso de reconocimiento.

\paragraph{Problema del vacío:} El problema del vacío para una RCG es indecidible \cite{propertiesRCGBib}, la razón principal para esto es que como se mencionó anteriormente
para toda CFL existe una RCG equivalente y como las RCG son cerradas bajo intersección existen RCG que describen
la intersección de 2 lenguajes libres del contexto y determinar si dicha intersección es vacía es un problema indecidible.

En el caso de las SRCG este problema es polinomial \cite{mainRCGBib}.

\paragraph{Problema de la equivalencia:} El problema de la equivalencia para 2 RCG es indecidible, la demostración es muy sencilla dadas 2 RCG $G_1$ y $G_2$ el problema
de saber si $G_1$ es equivalente a $G_2$ es equivalente a saber si $G_1\cap \overline{G_2}=\emptyset$ y como se mencionó anteriormente el problema del vacío para una RCG
es indecidible.

\section{Problema de la satisfacibilidad booleana}

El problema de la satisfacibilidad booleana (\textit{SAT}), es un problema fundamental en la teoría de la computación y la lógica matemática. El objetivo principal del problema es determinar si existe una asignación de valores a las variables de una expresión booleana tal que la expresión sea verdadera.

\subsection{Variables booleanas}

Una variable booleana es una variable que puede tomar uno de dos valores posibles: \textit{true} (verdadero) o \textit{false} (falso). Estas variables se utilizan para construir expresiones lógicas.

\subsection{Literales}

Un literal es una variable booleana o su negación. Formalmente, si \( x \) es una variable booleana, entonces \( x \) y \( \neg x \) (la negación de \( x \)) son literales. Un literal puede tomar los valores \( true \) o \( false \) dependiendo de la asignación de valores a las variables.

\subsection{Cláusulas}

Una cláusula es una disyunción (operador \textbf{OR}) de uno o más literales. Por ejemplo, la cláusula \( (x \vee \neg y \vee z) \) es una disyunción de tres literales: \( x \), \( \neg y \) y \( z \). Una cláusula es verdadera si al menos uno de sus literales es verdadero. Si todos los literales son falsos, la cláusula será falsa.

\subsection{Fórmulas en forma normal conjuntiva}

Una fórmula booleana en forma normal conjuntiva (\textit{CNF}) es una conjunción (operador \textbf{AND}) de cláusulas. En otras palabras, es una expresión booleana que se puede escribir como una serie de cláusulas unidas por el operador \textbf{AND}. Por ejemplo:

\[
      (x \vee \neg y \vee z) \wedge (\neg x \vee y) \wedge (x \vee \neg z)
\]

\subsection{Fórmulas booleanas equivalentes}

Dos fórmulas booleanas se consideran equivalentes si, para cualquier asignación de valores a sus variables, ambas producen el mismo resultado lógico. Por ejemplo, las fórmulas \( x \vee (y \wedge z) \) y \( (x \vee y) \wedge (x \vee z) \) son equivalentes, ya que para cualquier combinación de valores \( x, y, z \), ambas tienen el mismo valor lógico.

Para cualquier fórmula booleana existe una fórmula booleana equivalente en CNF y el algoritmo para encontrarla es polinomial, por lo tanto
de aquí se puede asumir que toda fórmula booleana está en CNF.

\subsection{Definición del problema de la satisfacibilidad booleana}

El problema de la satisfacibilidad booleana, o SAT, consiste en determinar si existe una asignación de valores \( true \) o \( false \) a las variables de una fórmula booleana tal que la fórmula completa sea verdadera. En términos formales, dado un conjunto de cláusulas en CNF, el problema es encontrar una asignación de valores a las variables que haga que la conjunción de las cláusulas sea verdadera.

Formalmente, se dice que una fórmula booleana en CNF es satisfacible si existe una asignación de valores a las variables tal que todas las cláusulas de la fórmula sean verdaderas simultáneamente.

\begin{itemize}
      \item Si existe tal asignación, la fórmula es \textit{satisfacible}.
      \item Si no existe tal asignación, la fórmula es \textit{insatisfacible}.
\end{itemize}

Un SAT con exactamente $n$ variables distintas se denomina $n$-SAT.

\subsection{SAT como Problema NP-Completo}

El SAT es el primer problema demostrado como NP-Completo \cite{authomataTheory} y juega un rol central en la teoría de la complejidad computacional. Se define en la clase NP porque, dada una asignación de valores a las variables de la fórmula booleana, se puede verificar en tiempo polinómico si dicha asignación satisface la fórmula.

Además, la prueba de que SAT es NP-Completo fue una de las contribuciones principales de Stephen Cook en 1971, marcando el inicio de la teoría de la NP-completitud.

\subsection{Equivalencia entre SAT y 3-SAT}

Para el problema 2-SAT existe una solución polinomial que determina si la fórmula booleana es satisfacible o no, pero para el problema 3-SAT no se conoce ningún algoritmo que permita
determinar si una fórmula booleana es satisfacible o no.

Cualquier fórmula booleana del problema $n$-SAT puede ser reducida a una fórmula booleana equivalente del problema 3-SAT, por lo tanto, SAT es equivalente a 3-SAT en términos de complejidad computacional.

\paragraph{Transformación de SAT a 3-SAT:}

Dada una fórmula en CNF con cláusulas de \( k > 3 \) literales, se puede convertir a 3-CNF introduciendo variables adicionales. Por ejemplo, considere una cláusula de cuatro literales:

\[
      (a \vee b \vee c \vee d)
\]

Esta cláusula puede reescribirse como un conjunto de cláusulas en 3-CNF introduciendo una nueva variable \( x \):

\[
      (a \vee b \vee x) \wedge (\neg x \vee c \vee d)
\]

Este proceso se puede aplicar iterativamente para todas las cláusulas con más de tres literales, asegurando que la nueva fórmula sea satisfacible si y solo si la fórmula original también lo es.

\subsection{Problemas SAT solubles en tiempo polinomial}

Como se mencionó anteriormente no se conoce ningún algoritmo polinomial para resolver el problema SAT en general, pero
existen casos particulares del problema que sí pueden ser resueltos en tiempo polinomial. A continuación se presentan los
principales casos:

\begin{enumerate}
      \item \textbf{1-SAT:} El problema 1-SAT es una instancia particular de SAT donde cada cláusula tiene a lo sumo un literal.
            Este problema puede ser resuelto en tiempo polinomial mediante un algoritmo de asignación de valores de verdad.
      \item \textbf{2-SAT:} Como se mencionó anteriormente, el problema 2-SAT puede ser resuelto en tiempo polinomial mediante
            una modelación basada en grafos.
      \item \textbf{Horn-SAT:} El problema Horn-SAT es una generalización del problema 2-SAT, donde cada cláusula tiene a lo sumo
            un literal positivo. Este problema puede ser resuelto en tiempo polinomial mediante el algoritmo de resolución de Horn.
\end{enumerate}

\documentclass[12pt]{article}

\usepackage{amsmath}
\usepackage{multicol}
\usepackage{tikz}
\usetikzlibrary{automata, positioning}

\usepackage[lmargin=2cm, rmargin=5cm]{geometry}
%%%{{{ Comments and the like
\usepackage[textwidth=4cm]{todonotes}
\usepackage{soul}
\usepackage{xcolor}
\newcounter{todocounter}
\newcommand{\comment}[2]{\stepcounter{todocounter}
  {\color{green!50!blue}{(#1$^{{\color{black}\textbf{\thetodocounter}}}$)}}
  \todo[color=green,noline,size=\tiny]{\textbf{\thetodocounter:} #2

  }}
\newcommand{\quitaesto}[1]{{\color{red}(\st{#1})}}

\newcommand{\cambio}[2]{{\color{cyan}{{#2}}}{\color{red}{(\st{#1})}}}

\newcommand{\agregaesto}[1]{{\color{cyan}{{#1}}}}

\newcommand{\notaparaelautor}[1]{{\color{brown}{\textbf{#1}}}}

\newcommand{\errorortografico}[1]{{\fcolorbox{gray}{magenta}{\textcolor{yellow}{\bf #1}}}}
    
%%%}}}

\title{Solución del SAT usando Teoría de Lenguajes}
\author{Raudel Alejandro Gómez Molina}

\begin{document}

\maketitle

En este capitulo se presentará un enfoque distinto a los anteriores, que se basa en definir un lenguaje al cual
pertenecen todos los problemas SAT que son satisfacibles y al cual se le denominará $L_{S-SAT}$.

\section{Transformación de una fórmula booleana a una cadena}

Primeramente para definir $L_{SAT}$ se debe definir una transformación de una fórmula booleana a una cadena de 
símbolos.

Dada una fórmula booleana en $F$ CNF se puede definir la siguiente estructura:
$$F=X_1 \wedge X_2 \wedge \ldots \wedge X_n$$
donde cada cláusula $X_i$ es una disyunción de literales:
$$X_i=L_{i1} \vee L_{i2} \vee \ldots \vee L_{im}$$
y cada literal $L_{ij}$ es una variable booleana o su negación. En cada cláusula $X_i$ las variables que aparecen en $F$, puede tener cada una 3 estados posibles: $a$ si la variable aparece positiva, $b$ si la variable aparece negada y $c$ si la variable no pertenece a ninguno de los literales de la cláusula.

Por ejemplo la siguiente fórmula booleana en $CNF$:

$$F=(x_1 \vee x_2) \wedge (\neg x_1 \vee x_2 \vee x_3) \wedge (x_1 \vee \neg x_2 \vee x_3)$$

para la primera cláusula $x_1$ aparece positiva ($a$), $x_2$ aparece positiva ($a$) y $x_3$ no aparece $(c)$, para la segunda
$x_1$ aparece negada ($b$), $x_2$ aparece positiva ($a$) y $x_3$ aparece positiva $(a)$ y para la tercera 
$x_1$ aparece positiva ($a$), $x_2$ aparece negada ($b$) y $x_3$ aparece positiva $(a)$.

A partir de la afirmación anterior, se puede definir una cadena de símbolos $w$ que representa a la cláusula $X_i$ sobre una secuencia de variables $v_1,v_2,\ldots,v_p$ de la siguiente manera:

\begin{itemize}
    \item $w$ cuenta con exactamente $p$ símbolos.
    \item Si la variable $v_j$ aparece positiva en $X_i$, entonces el $j$-ésimo símbolo es $a$.
    \item Si la variable $v_j$ aparece negada en $X_i$, entonces el $j$-ésimo símbolo es $b$.
    \item Si la variable $v_j$ no aparece en $X_i$, entonces el $j$-ésimo símbolo es $c$.
\end{itemize}
Si se toma la secuencia de variables correspondiente a $F$, y se le aplica el procedimiento anterior a cada cláusula
se obtendrá una cadena de símbolos que representa a dicha cláusula en $F$.

Si ya se tiene una representación para cada cláusula de $F$ solo resta obtener una cadena de símbolos que represente a $F$,
esto se puede lograr concatenando las cadenas de símbolos de cada cláusula de $F$ en el orden que aparecen con un separador
en este caso se eligió el símbolo $d$.

Por ejemplo la siguiente fórmula booleana en \textit{CNF}:
$$F=(x_1 \vee x_2) \wedge (\neg x_1 \vee x_2 \vee x_3) \wedge (x_1 \vee \neg x_2 \vee x_3)$$
puede ser expresada como la cadena de símbolos:
$$w=aacdbaadabad$$
tomando como secuencia de variables $x_1, x_2, x_3$.

\agregaesto{TRANSICIÓN A LO QUE SIGUE.}

\subsection{Definición de $L_{SAT}$}

El lenguaje de todas las fórmulas booleanas en CNF que son satisfacibles se define como $L_{S-SAT}=\{w\,|\,w \in L_{FULL-SAT} \wedge f_{SAT}(w)\}$, 
donde $L_{FULL-SAT}$ representa el lenguaje de todas las fórmulas booleanas en CNF y $f_{SAT}(w)$ es una función que 
determina si $w$ es satisfacible.

En las próximas secciones se presentarán varios enfoques para definir $L_{S-SAT}$: el primer enfoque se basa en definir
el lenguaje mediante un transductor finito seleccionando un formalismo que cumpla ciertas restricciones y el segundo se
basa en definir el lenguaje mediante una gramática de concatenación de rango. 

\section{Transductor SAT}

La idea para definir $L_{S-SAT}$ es construir un transductor finito que acepte como entrada cadenas del lenguaje $L_{0,1}=\{wd\}^+$ donde $w\in \{0,1\}^*$
y tenga como salida cadenas que representan una fórmula booleana en \textit{CNF}  si y solo si la fórmula es satisfacible. 

Detrás de esta construcción se busca asociar cada carácter 0 ó 1 en la cadena de entrada al valor de la variable booleana correspondiente en la cadena de salida y verificar que para dichos
valores al evaluar la fórmula booleana se obtenga un valor de verdad. 

Mediante esta construcción se mantiene la invariante fundamental del SAT que a dos instancias
de la misma variable se les asocia el mismo valor de verdad, esto es posible por como está definido el formato de la cadena de entrada.

A continuación se define el transductor finito $T_{SAT}$ que sigue la construcción definida anteriormente, para ello se define
el transductor $T_{CLAUSE}$ (Figura \ref{fig:transducer}) que hace el proceso de transducción para los valores de verdad de una cláusula $w$, donde $w\in \{0,1\}$:

\[
    T_{CLAUSE} = (Q, {\Sigma}, \Gamma, \delta, q_{0}, F),
\]
donde:
\begin{itemize}
    \item \(Q\) = ${q_0,q_p,q_n}$.
    \item \(\Sigma\) = ${0,1}$.
    \item \(\Gamma\) = ${a,b,c}$.
    \item \(\delta: Q \times \Sigma \to Q \times \Gamma^*\) función de transición.
    \item \(q_{0} = q_0\) estado inicial.
    \item \(F={q_p}\) conjunto de estados finales.
\end{itemize}
se define la función de transición $\delta$ de la siguiente manera:

\begin{itemize}
    \item  transiciones para el estado $q_0$: representa el estado inicial. Si la entrada es un 1 el transductor
          puede escribir $a$, $b$ y $c$ si escriba $a$ pasa al estado positivo, si escribe $b$ pasa al estado negativo
          y si escribe $c$ permanece en el mismo estado. Por otro lado si la entrada es un 0 se intercambian los estados 
          cuando se escribe $a$ y $b$ y cuando se escribe $c$ permanece en el mismo estado.
          
          \begin{multicols}{2}
              \begin{itemize}
                  \item $\delta_{SAT}(q_0,1)=(q_p,a)$
                  \item $\delta_{SAT}(q_0,0)=(q_n,a)$
                  \item $\delta_{SAT}(q_0,1)=(q_n,b)$
                  \item $\delta_{SAT}(q_0,0)=(q_p,b)$
                  \item $\delta_{SAT}(q_0,1)=(q_0,c)$
                  \item $\delta_{SAT}(q_0,0)=(q_0,c)$
              \end{itemize}
          \end{multicols}
          
    \item  transiciones para el estado $q_p$ (estado positivo de $T_{CLAUSE}$): representa que para los valores de verdad de la cláusula obtiene un valor de verdad positivo.
          Como la fórmula se encuentra ya en un estado positivo lo que significa que al menos un literal se evalúa positivo no importa
          la entrada y lo que el transductor escriba se mantiene en el mismo estado.      
          
          \begin{multicols}{2}
              \begin{itemize}
                  \item $\delta_{SAT}(q_{p},1)=(q_{p},a)$
                  \item $\delta_{SAT}(q_{p},0)=(q_{p},a)$
                  \item $\delta_{SAT}(q_{p},1)=(q_{p},b)$
                  \item $\delta_{SAT}(q_{p},0)=(q_{p},b)$
                  \item $\delta_{SAT}(q_{p},1)=(q_{p},c)$
                  \item $\delta_{SAT}(q_{p},0)=(q_{p},c)$
              \end{itemize}
          \end{multicols}
          
    \item  transiciones para el estado $q_n$ (estado negativo de $T_{CLAUSE}$): representa que para los valores de verdad la cláusula obtiene un valor de verdad negativo.
          Aquí las transiciones son idénticas a las del estado inicial, reemplazando el estado inicial por el estado negativo en los posibles resultados
          de la función de transición.       
          
          \begin{multicols}{2}
              \begin{itemize}
                  \item $\delta_{SAT}(q_{n},1)=(q_{p},a)$
                  \item $\delta_{SAT}(q_{n},0)=(q_{n},a)$
                  \item $\delta_{SAT}(q_{n},1)=(q_{n},b)$
                  \item $\delta_{SAT}(q_{n},0)=(q_{p},b)$
                  \item $\delta_{SAT}(q_{n},1)=(q_{n},c)$
                  \item $\delta_{SAT}(q_{n},0)=(q_{n},c)$
              \end{itemize}
          \end{multicols}
\end{itemize}

\begin{figure}[h]
    \centering \begin{tikzpicture}[shorten >=1pt, node distance=3cm, on grid, auto]
        
        % Nodos
        \node[state, initial] (q0)   {$q_0$};
        \node[state] (qn) [above right=of q0] {$q_n$};
        \node[state, accepting] (qp) [below right=of q0] {$q_p$};
        
        % Transiciones
        \path[->]
        (q0) edge [bend left] node {0/a,1/b} (qn)
        (q0) edge [bend right] node {1/a,0/b} (qp)
        (q0) edge [loop right] node {0/c,1/c} (q0)
        
        (qn) edge [bend left] node {1/a,0/b} (qp)
        (qn) edge [loop above] node {0/a,1/b,0/c,1/c} (qn)
        
        (qp) edge [loop below] node {1/a,0/b,0/a,1/b,0/c,1/c} (qp);
        
    \end{tikzpicture}
    \caption{Transductor $T_{CLAUSE}$.}
    \label{fig:transducer} % Esto es para referenciar la figura en el texto
\end{figure}

Para definir el transductor $T_{SAT}$ se toman dos instancias del transductor $T_{CLAUSE}$ ($T_1$ y $T_2$ respectivamente) 
y se concatenan añadiendo una transición del estado $q_{p_1}$ (estado positivo de $T_1$) al estado $q_{0_2}$
(estado inicial de $T_2$) con el símbolo $d$ (tanto de lectura como de escritura) y además se agrega una clausura a $T_2$ con una transición del estado $q_{p_2}$ (estado positivo de $T_2$) al estado $q_{0_2}$ con el símbolo $d$ (tanto de lectura como de escritura). Entonces solo resta definir el estado inicial y el estado final de $T_{SAT}$, los cuales serían $q_{0_1}$ (estado inicial de $T_1$) y $q_{0_2}$ (estado inicial de $T_2$), respectivamente.

\agregaesto{TRANSICIÓN A LO QUE SIGUE.}

\subsection{Definición del $L_{S-SAT}$ usando transducción finita}

Finalmente se define $L_{S-SAT}$ como el lenguaje de todas las cadenas $e$ que son aceptadas por el transductor $T_{SAT}$, a partir del lenguaje
de cadenas de entrada $L_{0,1}=\{wd\}^+$ donde $w\in \{0,1\}^*$. 

$$L_{S-SAT} = \{e\,|\,\exists w \in L_{0,1} \wedge e \in T_{SAT}(w) \}$$

Luego $L_{S-SAT}$ contiene todas las fórmulas booleanas satisfacibles, pero este conjunto por si solo no sirve de mucho sin 
un formalismo que permita conocer si una cadena que representa una fórmula booleana pertenece al lenguaje o 
no. Para ello se necesita encontrar un formalismo que sea capaz de generar el lenguaje $L_{0,1}$ y al aplicarle el transductor 
$T_{SAT}$ a dicho formalismo se obtenga un formalismo que cuente con un algoritmo de reconocimiento para reconocer 
si una cadena pertenece a dicho formalismo o no.

Como se evidenció en esta sección encontrar un formalismo que genere el lenguaje $L_{0,1}$, el cual sea cerrado bajo transducción finita
es suficiente para generar el lenguaje $L_{S-SAT}$. Una pregunta interesante sería saber si la existencia de dicho formalismo
es una condición necesaria para definir el lenguaje $L_{S-SAT}$ otra pregunta interesante sería saber si existe un formalismo
que sea capaz de describir el lenguaje $L_{S-SAT}$ y el problema de la palabra en dicho formalismo sea polinomial (observe que de esta manera
se estaría resolviendo el problema \textbf{P vs NP}).

Dado el resultado anterior se puede demostrar que el problema de la palabra de cualquier formalismo que genere el lenguaje 
$L_{0,1}$ y sea cerrado bajo transducción finita es $NP-Duro$, ya que puede ser reducido al SAT y por tanto a cualquier problema
en NP. Para esto habría que demostrar que el lenguaje que genere $L_{S-SAT}$ tiene complejidad O(1).

\agregaesto{TRANSICIÓN A LO QUE SIGUE.}

\section{$L_{S-SAT}$ como lenguaje de índice global}

En esta sección se presentará una una forma de generar el lenguaje $L_{0,1}$ empleando una GIG. En \cite{globalIndexLanguages} se presenta
una GIG, $G_{ww^+}$ que genera el lenguaje $L(G_{ww^*})=\{ww^+\,|\,w\in\{a,b\}\}$: 
$$
    G_{ww^+} = (N, \Sigma, I, S, \#, P) 
$$
donde:

\begin{itemize}
    \item $N= \{S,R,A,B,C\}$.
    \item \( \Sigma=\{a,b\} \) .
    \item $I=\{i,j\}$.
    \item $S$ es el \textbf{símbolo inicial}.
    \item $\#$ es el \textbf{símbolo inicial de la pila}.
    \item $P$ es un conjunto finito de \textbf{producciones}:
          \begin{multicols}{2}
              \begin{itemize}
                  \item $S\underset{\varepsilon}{\to} AS\,|\,BS\,|\,C$
                  \item $C\underset{\varepsilon}{\to} RC\,|\,L$
                  \item $R\underset{\overline{i}}{\to} RA$
                  \item $R\underset{\overline{j}}{\to} RB$
                  \item $R\underset{[\#]}{\to} \varepsilon$
                  \item $A\underset{i}{\to} a$
                  \item $B\underset{j}{\to} b$
                  \item $L\underset{\overline{i}}{\to} La\,|\,a$
                  \item $L\underset{\overline{j}}{\to} Lb\,|\,b$
              \end{itemize}
          \end{multicols}
\end{itemize}

Observe que la pila en esta gramática es un mecanismo de memoria suficiente para generar \textit{Copy} (el lenguaje \textit{Copy} sobre un alfabeto
$\Sigma$ se define como $L_{copy}=\{ww^+\,|\,w\in Z^*\}$), a cada caracter de $\Sigma$ le corresponde un no terminal y un símbolo de la pila 
por el que se puede producir dicho caracter almacenando el caracter el símbolo de la pila correspondiente, el funcionamiento 
de la gramática lo compone además un mecanismo de recursión por el que se puede producir un no terminal asociado a un caracter
solo eliminando el símbolo de la pila asociado a dicho caracter por último el mecanismo de recursión produce la cadena vacía
solo si el símbolo inicial de la pila se encuentra en el tope.

Entonces dada esta gramática es relativamente realizar una modificación para generar el lenguaje $L_{0,1}$,
a esta nueva gramática se denominará $G_{0,1}$, que se define como:

$$
    G_{0,1} = (N, \Sigma, I, S, \#, P) 
$$
donde:

\begin{itemize}
    \item $N= \{S,R,A,B,C,D\}$.
    \item \( \Sigma=\{0,1,d\} \) .
    \item $I=\{i,j,k\}$.
    \item $S$ es el \textbf{símbolo inicial}.
    \item $\#$ es el \textbf{símbolo inicial de la pila}.
    \item $P$ es un conjunto finito de \textbf{producciones}:
          \begin{multicols}{2}
              \begin{itemize}
                  \item $S\underset{\varepsilon}{\to} AS\,|\,BS\,|\,DC$
                  \item $C\underset{\varepsilon}{\to} RC\,|\,L$
                  \item $R\underset{\overline{i}}{\to} RA$
                  \item $R\underset{\overline{j}}{\to} RB$
                  \item $R\underset{\overline{k}}{\to} RD$
                        
                  \item $R\underset{[\#]}{\to} \varepsilon$
                  \item $A\underset{i}{\to} a$
                  \item $B\underset{j}{\to} b$
                  \item $D\underset{k}{\to} d$
                  \item $L\underset{\overline{i}}{\to} L$
                  \item $L\underset{\overline{j}}{\to} L$
                  \item $L\underset{\overline{k}}{\to} L$
                  \item $L\underset{[\#]}{\to} \varepsilon$
              \end{itemize}
          \end{multicols}
\end{itemize}

Como modificaciones a la gramática anterior se ha introducido un nuevo terminal, 
un no terminal y un símbolo de la pila manteniendo la invariante de correspondencia 
que se mencionó anteriormente entre los elementos de estos 3 conjuntos. Por otro lado se modificaron 
las producciones del no terminal $L$ para que unicamente produzca la cadena vacía eliminando todos los elementos de la pila, 
con ello se puede generar el lenguaje $L_{0,1}$.

\notaparaelautor{¿Y qué pasó con esto? Supongo que hay que decir en algún lugar que los GIL son cerrados bajo transducción finita, ¿no?}

\agregaesto{TRANSICIÓN A LO QUE SIGUE.}

\section{$L_{S-SAT}$ como lenguaje de concatenación de rango}

\comment{En esta sección se presentará un enfoque para generar el lenguaje $L_{0,1}$ primeramente y luego para generar el lenguaje $L_{S-SAT}$.}{¿por qué queremos hacer eso? ¿El problema de la palabra es polinomial en este caso o el del vacío?}
Ambos enfoques usando gramáticas de concatenación de rango.

\subsection{$L_{0,1}$ como lenguaje de concatenación de rango}

Se define la gramática $G_{0,1}$ como sigue:
\[
    G_{0,1} = (N, T, V, P, S),
\]
donde:

\begin{itemize}
    \item $N=\{S,A,B,C,Eq\}$
    \item $T=\{0,1,d\}$.
    \item $V=\{X,Y,P\}$.
    \item El conjunto de cláusulas $P$ es el siguiente:
          \begin{itemize}
              \item  $S(X)\to A(X)$
              \item $A(YdX)\to B(X,Y)C(X)$
              \item $B(XdY,P)\to B(Y,P) C(X) Eq(X,P)$
              \item $B(\varepsilon,Y)\to \varepsilon$
              \item $C(0X)\to C(X)$
              \item $C(1X)\to C(X)$
              \item $C(\varepsilon)\to \varepsilon$
          \end{itemize}
    \item El \textbf{símbolo inicial} es $S$.
\end{itemize}

El predicado $Eq$ se define en \cite{mainRCGBib} y comprueba que dos cadenas sobre un alfabeto sean iguales, 
por otro lado el predicado $B$ se encarga de definir la sustitución en rango de la próxima cadena de 0 y 1 y 
comprobar que este sea igual al patrón inicial. De esta manera se pueden reconocer cadenas que pertenezcan al lenguaje $L_{0,1}$.

Como se mencionó anteriormente las RCG no son cerradas bajo transducción finita, por tanto no se puede realizar el mismo 
análisis que en las 2 secciones anteriores, \agregaesto{pero} \comment{esto no necesariamente impide que el lenguaje generado
por la transducción finita del lenguaje que representa la gramática anterior no pueda cumplir ciertas características que permitan 
generar lenguaje $L_{S-SAT}$.}{Esto está dificil de entender.}

\agregaesto{TRANSICIÓN A LO QUE SIGUE.}

\subsection{Otro enfoque para generar $L_{SAT}$}

A continuación se presentará una estrategia distinta para generar el lenguaje $L_{S-SAT}$, 
usando gramáticas de concatenación de rango que no emplea el transductor $T_{SAT}$.
En este caso la función $f_{SAT}$ se define dentro del funcionamiento de la gramática para ello se define la siguiente RCG:
\[
    G_{S-SAT} = (N, T, V, P, S),
\]
donde:

\begin{itemize}
    \item $N=\{S,A,B,C,P,N,Cp,Cn\}$
    \item $T=\{a,b,c,d\}$.
    \item $V=\{X,Y\}$.
    \item El \textbf{símbolo inicial} es $S$.
\end{itemize}

A continuación se desglosa el conjunto de \textbf{cláusulas} $P$ en varias fases agrupando las cláusulas
por funcionalidad:

\begin{itemize}
    \item $S(X)\to A(X)$
    \item \textbf{Primera fase:} El siguiente conjunto de cláusulas genera la cadena de 0 y 1 que que da valores a las variables de la
          fórmula booleana:
          \begin{multicols}{2}
              \begin{itemize}
                  \item $A(aX)\to P(X,1)$
                  \item $A(aX)\to N(X,0)$
                  \item $A(bX)\to N(X,1)$
                  \item $A(bX)\to P(X,0)$
                  \item $A(cX)\to N(X,1)$
                  \item $A(cX)\to N(X,0)$
                        
                  \item $P(aX,Y)\to P(X,Y1)$
                  \item $P(aX,Y)\to P(X,Y0)$
                  \item $P(bX,Y)\to P(X,Y1)$
                  \item $P(bX,Y)\to P(X,Y0)$
                  \item $P(cX,Y)\to P(X,Y1)$
                  \item $P(cX,Y)\to P(X,Y0)$
                  \item $P(dX,Y)\to B(X,Y)$
                        
                  \item $N(aX,Y)\to P(X,Y1)$
                  \item $N(aX,Y)\to N(X,Y0)$
                  \item $N(bX,Y)\to N(X,Y1)$
                  \item $N(bX,Y)\to P(X,Y0)$
                  \item $N(cX,Y)\to N(X,Y1)$
                  \item $N(cX,Y)\to N(X,Y0)$
              \end{itemize}
          \end{multicols}
          
          El predicado $A$ representa el primer predicado reconocimiento de este se deriva a los predicados $P$
          (representa que la cláusula de la fórmula booleana se encuentra en un estado de verdad positivo) 
          y $N$ (representa que la cláusula de la fórmula booleana se encuentra en un estado de verdad negativo)
          en dependencia del valor de la instancia de la variable correspondiente. El predicado $P$ deriva hacia
          sí mismo independientemente del símbolo, exceptuando el símbolo $d$, caso en el que se procede a la siguiente
          fase.
          El funcionamiento de esta fase es prácticamente el mismo que el del transductor $T_{SAT}$.
          
    \item \textbf{Segunda fase:} El siguiente conjunto de cláusulas se encarga de un mecanismo de clausura que le permite a la gramática
          reconocer si la asignación realizada en la fase anterior es válida para las restantes cláusulas de la fórmula
          booleana.
          \begin{itemize}
              \item $B(X_1dX_2,Y)\to C(X_1,Y) B(X_2,Y)$
              \item $B(\varepsilon,Y)\to\varepsilon$
          \end{itemize}
          
          El predicado $B$ permite realizar la clausura mientras que el predicado $C$ comprueba que la cláusula de la fórmula
          booleana actual sea satisfacible.
          
    \item \textbf{Tercera fase:} Solo resta definir el comportamiento de $C$:
          \begin{itemize}
              \begin{multicols}{2}
                  \item $C(X,Y)\to Cn(X,Y)$
                  
                  \item $Cn(aX,1Y) \to Cp(X,Y)$
                  \item $Cn(aX,0Y) \to Cn(X,Y)$
                  \item $Cn(bX,1Y) \to Cn(X,Y)$
                  \item $Cn(bX,0Y) \to Cp(X,Y)$
                  \item $Cn(cX,1Y) \to Cn(X,Y)$
                  \item $Cn(cX,0Y) \to Cn(X,Y)$
                  
                  \item $Cp(aX,1Y) \to Cp(X,Y)$
                  \item $Cp(aX,0Y) \to Cp(X,Y)$
                  \item $Cp(bX,1Y) \to Cp(X,Y)$
                  \item $Cp(bX,0Y) \to Cp(X,Y)$
                  \item $Cp(cX,1Y) \to Cp(X,Y)$
                  \item $Cp(cX,0Y) \to Cp(X,Y)$
                  \item $Cp(\varepsilon,\varepsilon)\to \varepsilon$
              \end{multicols}
          \end{itemize}
          
          Observe que este funcionamiento es exactamente igual al de la primera fase con un predicado que representa un estado
          positivo ($Cp$) y un predicado que representa un estado
          positivo ($Cn$) pero esta vez no se genera la cadena sino que se comprueba con un patrón predefinido en la primera
          fase.
          
          Como $G_{S-SAT}$ reconoce las fórmulas booleanas satisfacibles solo se debe analizar el problema de la
          palabra para determinar si una fórmula es satisfacible.
          
          \notaparaelautor{Ejemplito, por favor *cara de gatico de Shrek*}
          
          \agregaesto{TRANSICIÓN A LO QUE SIGUE.}
          
          \subsubsection{Análisis de la complejidad computacional}
          
          Como se mencionó anteriormente no todas las RCG tienen un algoritmo de parsing lineal y $G_{S-SAT}$ es un ejemplo de 
          ello, observe que en la primera fase se genera la cadena binaria que representa la asignación de valores a las variables
          booleanas y dicha cadena participa en los predicados de fases posteriores. Si se analiza el algoritmo de parsing descrito en 
          \cite{mainRCGBib} un factor en la complejidad del algoritmo de parsing es la cantidad de rangos posibles para una cadena 
          que debe ser reconocida por un predicado y en este caso la cadena que estamos analizando puede tomar $2^n$ valores distintos, donde
          $n$ es la cantidad de variables en la fórmula booleana por lo que la cantidad de rangos sería $n^22^n$, pero esta es una cota
          burda ya que una vez generada la cadena de asignación por la forma de la gramática solo se utiliza un solo rango que se va construyendo
          bajo demanda. El resto de las fases de la gramática tienen una complejidad de $m^2$ donde $m$ es la cantidad de caracteres
          en la cadena de entrada, por lo que la complejidad total sería $O(2^nm^2)$.
          
          Este es un resultado interesante ya que demuestra que no es necesario usar el transductor $T_{SAT}$ para definir el 
          lenguaje $L_{S-SAT}$, mediante formalismo de escritura regulada.
          
          
\end{itemize}

\begin{thebibliography}{99}
    
    \bibitem{mainRCGBib}
    Boullier, Pierre. 
    \textit{Proposal for a Natural Language Processing Syntactic Backbone}. 
    Research Report RR-3342, INRIA, 1998. 
    
    \bibitem{propertiesRCGBib}
    Boullier, Pierre. 
    \textit{A Cubic Time Extension of Context-Free Grammars}. 
    Research Report RR-3611, INRIA, 1999. 
    
    \bibitem{simpleMatrixLanguages}
    Ibarra, Oscar H. 
    \textit{Simple matrix languages}. 
    \textit{Information and Control}, Vol. 17, No. 4, pp. 359-394, 1970. 
    
    \bibitem{globalIndexLanguages}
    Castaño, José M. 
    \textit{Global Index Languages}. 
    Ph.D. Thesis, The Faculty of the Graduate School of Arts and Sciences, Brandeis University, 2004.
    
    \bibitem{authomataTheory}
    Hopcroft, John E., Motwani, Rajeev, y Ullman, Jeffrey D. 
    \textit{Introduction to Automata Theory, Languages, and Computation}. 
    3ª edición, Addison-Wesley, 2006. ISBN: 9780321455369.
    
    \bibitem{aCFSAT}
    Fernández Arias, Alina. 
    \textit{El problema de la satisfacibilidad booleana libre del contexto}. 
    Facultad de Matemática y Computación, Universidad de La Habana, 2007.
    
    \bibitem{aSRCSAT}
    Aguilera López, Manuel. 
    \textit{Problema de la Satisfacibilidad Booleana de Concatenación de Rango Simple}. 
    Facultad de Matemática y Computación, Universidad de La Habana, 2016.
    
    \bibitem{aSMSAT}
    Rodríguez Salgado, José Jorge. 
    \textit{Gramáticas Matriciales Simples. Primera aproximación para una solución al problema SAT}. 
    Facultad de Matemática y Computación, Universidad de La Habana, 2019.
    
\end{thebibliography}


\end{document}

\documentclass{article}

\usepackage{amsmath}
\usepackage{multicol}
\usepackage{tikz}
\usetikzlibrary{automata, positioning}

\title{FULL SAT}
\author{Raudel Alejandro Gómez Molina}

\begin{document}

\maketitle

\section{Lenguaje FULL-SAT}

En esta sección se presentará un nuevo enfoque distinto a los anteriores, el cual se basa en definir un lenguaje al cual
pertenecen todos los problemas SAT que son satisfacible y al cual se le denominará \textit{FULL-SAT}.

\section{Transformación de una fórmula booleana a una cadena}

Primeramente para definir FULL-SAT se debe definir una transformación de una fórmula booleana a una cadena de símbolos.

Dada una fórmula booleana en CNF:
$$F=X_1 \wedge X_2 \wedge \ldots \wedge X_n$$
donde cada cláusula $X_i$ es una disyunción de literales:
$$X_i=L_{i1} \vee L_{i2} \vee \ldots \vee L_{im}$$
y cada literal $L_{ij}$ es una variable booleana o su negación. En cada cláusula $X_i$ las variables que aparecen en $F$,
puede tener cada una 3 estados posibles: $a$ si la variable aparece positiva, $b$ si la variable aparece negada y $c$ si la variable
no pertenece a ninguno de los literales de la cláusula.

Ahora dada la afirmación anterior, se puede definir una cadena de símbolos $w$
que representa a la cláusula $X_i$ sobre una secuencia de variables $v_1,v_2,\ldots,v_p$ de la siguiente manera:

\begin{itemize}
    \item $w$ cuenta con exactamente $p$ símbolos.
    \item Si la variable $v_j$ aparece positiva en $X_i$, entonces el $j$-ésimo símbolo es $a$.
    \item Si la variable $v_j$ aparece negada en $X_i$, entonces el $j$-ésimo símbolo es $b$.
    \item Si la variable $v_j$ no aparece en $X_i$, entonces el $j$-ésimo símbolo es $c$.
\end{itemize}
Si se toma la secuencia de variables correspondiente a $F$, y se le aplica el procedimiento anterior a cada cláusula
se obtendrá una cadena de símbolos que representa a dicha cláusula en $F$.

Si ya se tiene una representación para cada cláusula de $F$ solo resta obtener una cadena de símbolos que represente a $F$,
esto se puede lograr concatenando las cadenas de símbolos de cada cláusula de $F$ en el orden que aparecen con un separador
en este caso se eligió el símbolo $d$.

Por ejemplo la siguiente fórmula booleana en \textit{CNF}:
$$F=(x_1 \vee x_2) \wedge (\neg x_1 \vee x_2 \vee x_3) \wedge (x_1 \vee \neg x_2 \vee x_3)$$
puede ser expresada como la cadena de símbolos:
$$w=aacdbaadabad$$
tomando como secuencia de variables $x_1, x_2, x_3$ como se describió anteriormente.

\subsection{Definición del lenguaje FULL-SAT}

El lenguaje FULL-SAT se define como $L_{FULL_SAT}=\{w\,|\,w \in L_{CNF} \wedge f_{SAT}(w)\}$, donde $L_{CNF}$ representa el lenguaje
de todas las fórmulas booleanas en CNF y $f_{SAT}(w)$ es una función que determinista si $w$ es satisfacible.

En las próximas secciones se presentarán varios enfoques para definir $f_{SAT}$.

\section{Transductor FULL-SAT}

En esta sección la idea para definir $f_{SAT}$ es construir un transductor finito que acepte como entrada cadenas del lenguaje $L_{0,1}=\{wd\}^+$ donde $w\in \{0,1\}^*$
y tenga como salida cadenas, donde cada cadena $e$ representa una fórmula booleana en \textit{CNF} y que acepte si y solo si la fórmula es satisfacible. Detrás
de esta construcción se busca asociar cada carácter 0 ó 1 en la cadena de entrada al valor de la variable booleana correspondiente en la cadena de salida y verificar que para dichos
valores al evaluar la fórmula booleana se obtenga un valor de verdad. Observe que mediante esta construcción se mantiene la invariante fundamental del SAT que a dos instancias
de la misma variable se les asocia el mismo valor de verdad, esto es posible por como está definido el formato de la cadena de entrada.

A continuación se define el transductor finito $T_{SAT}$ que sigue la construcción definida anteriormente, para ello se define
el transductor $T_{CLAUSE}$ (Figura \ref{fig:transducer}) que hace el proceso de transducción para los valores de verdad de una cláusula $w$, donde $w\in \{0,1\}$:

\[
    T_{CLAUSE} = (Q, {\Sigma}, \Gamma, \delta, q_{0}, F),
\]
donde:
\begin{itemize}
    \item \(Q\) = ${q_0,q_p,q_n}$.
    \item \(\Sigma\) = ${0,1}$.
    \item \(\Gamma\) = ${a,b,c}$.
    \item \(\delta: Q \times \Sigma \to Q \times \Gamma^*\) función de transición.
    \item \(q_{0} = q_0\) estado inicial.
    \item \(F={q_p}\) conjunto de estados finales.
\end{itemize}
se define la función de transición $\delta$ de la siguiente manera:

\begin{itemize}
    \item  transiciones para el estado $q_0$: representa el estado inicial.
          \begin{multicols}{2}
              \begin{itemize}
                  \item $\delta_{SAT}(q_0,1)=(q_p,a)$
                  \item $\delta_{SAT}(q_0,0)=(q_n,a)$
                  \item $\delta_{SAT}(q_0,1)=(q_n,b)$
                  \item $\delta_{SAT}(q_0,0)=(q_p,b)$
                  \item $\delta_{SAT}(q_0,1)=(q_0,c)$
                  \item $\delta_{SAT}(q_0,0)=(q_0,c)$
              \end{itemize}
          \end{multicols}
          
    \item  transiciones para el estado $q_p$ (estado positivo de $T_{CLAUSE}$): representa que para los valores de verdad de la cláusula obtiene un valor de verdad positivo.
          \begin{multicols}{2}
              \begin{itemize}
                  \item $\delta_{SAT}(q_{p},1)=(q_{p},a)$
                  \item $\delta_{SAT}(q_{p},0)=(q_{p},a)$
                  \item $\delta_{SAT}(q_{p},1)=(q_{p},b)$
                  \item $\delta_{SAT}(q_{p},0)=(q_{p},b)$
                  \item $\delta_{SAT}(q_{p},1)=(q_{p},c)$
                  \item $\delta_{SAT}(q_{p},0)=(q_{p},c)$
              \end{itemize}
          \end{multicols}
          
    \item  transiciones para el estado $q_n$ (estado negativo de $T_{CLAUSE}$): representa que para los valores de verdad la cláusula obtiene un valor de verdad negativo.
          \begin{multicols}{2}
              \begin{itemize}
                  \item $\delta_{SAT}(q_{n},1)=(q_{p},a)$
                  \item $\delta_{SAT}(q_{n},0)=(q_{n},a)$
                  \item $\delta_{SAT}(q_{n},1)=(q_{n},b)$
                  \item $\delta_{SAT}(q_{n},0)=(q_{p},b)$
                  \item $\delta_{SAT}(q_{n},1)=(q_{n},c)$
                  \item $\delta_{SAT}(q_{n},0)=(q_{n},c)$
              \end{itemize}
          \end{multicols}
\end{itemize}

\begin{figure}[h]
    \centering \begin{tikzpicture}[shorten >=1pt, node distance=3cm, on grid, auto]
        
        % Nodos
        \node[state, initial] (q0)   {$q_0$};
        \node[state] (qn) [above right=of q0] {$q_n$};
        \node[state, accepting] (qp) [below right=of q0] {$q_p$};
        
        % Transiciones
        \path[->]
        (q0) edge [bend left] node {0/a,1/b} (qn)
        (q0) edge [bend right] node {1/a,0/b} (qp)
        (q0) edge [loop right] node {0/c,1/c} (q0)
        
        (qn) edge [bend left] node {1/a,0/b} (qp)
        (qn) edge [loop above] node {0/a,1/b,0/c,1/c} (qn)
        
        (qp) edge [loop below] node {1/a,0/b,0/a,1/b,0/c,1/c} (qp);
        
    \end{tikzpicture}
    \caption{Transductor $T_{CLAUSE}$.}
    \label{fig:transducer} % Esto es para referenciar la figura en el texto
\end{figure}

Ahora para definir el transductor $T_{SAT}$ se toman dos instancias del transductor $T_{CLAUSE}$ ($T_1$ y $T_2$ respectivamente) y se concatenan
añadiendo una transición del estado $q_{p_1}$ (estado positivo de $T_1$) al estado $q_{0_2}$ (estado inicial de $T_2$) con el símbolo $d$ (tanto de
lectura como de escritura) y además se lenguaje agrega una clausura a $T_2$ con una transición del estado $q_{p_2}$
(estado positivo de $T_2$) al estado $q_{0_2}$ con el símbolo $d$ (tanto de lectura como de escritura). Entonces solo resta definir
el estado inicial y el estado final de $T_{SAT}$, los cuales serían $q_{0_1}$ (estado inicial de $T_1$) y $q_{0_2}$ (estado inicial de $T_2$),
respectivamente.

\subsection{Definición del lenguaje FULL-SAT usando transducción finita}

Finalmente se define el lenguaje FULL-SAT como el lenguaje de todas las cadenas $e$ que son aceptadas por el transductor $T_{SAT}$, a partir del lenguaje
de cadenas de entrada $L_{0,1}=\{wd\}^+$ donde $w\in \{0,1\}^*$. 

$$L_{FULL-SAT} = \{e\,|\,\exists w \in L_{0,1} \wedge e \in T_{SAT}(w) \}$$

Luego $L_{FULL-SAT}$ contiene todas las fórmulas booleanas satisfacibles, pero este conjunto por si solo no sirve de mucho sin un formalismo
que permita conocer si una cadena que representa una fórmula booleana pertenece al lenguaje o no, para ello se necesita encontrar un formalismo que sea capaz
de generar el lenguaje $L_{0,1}$ y al aplicarle el transductor $T_{SAT}$ a dicho formalismo se obtenga un formalismo que cuente con un algoritmo de parsing
para reconocer si una cadena pertenece a dicho formalismo o no.

Como se evidenció en esta sección encontrar un formalismo que genere el lenguaje $L_{0,1}$, el cual sea cerrado bajo transducción finita
es suficiente para generar el lenguaje $L_{FULL-SAT}$. Una pregunta interesante sería saber si la existencia de dicho formalismo
es una condición necesaria para definir el lenguaje $L_{FULL-SAT}$ otra pregunta interesante sería saber si existe un formalismo
que sea capaz de describir el lenguaje $L_{FULL-SAT}$ y el problema de la palabra en dicho formalismo sea polinomial (observe que de esta manera
se estaría resolviendo el problema \textbf{P vs NP}).

\section{FULL-SAT como lenguaje de índice global}

En esta sección se presentará una una forma de generar el lenguaje $L_{0,1}$ empleando una GIG. En \cite{globalIndexLanguages} se presenta
una GIG, $G_{ww^+}$ que genera el lenguaje $L(G_{ww^*})=\{ww^+\,|\,w\in\{a,b\}\}$: 
$$
    G_{ww^+} = (N, \Sigma, I, S, \#, P) 
$$
donde:

\begin{itemize}
    \item $N= \{S,R,A,B,C\}$.
    \item \( \Sigma=\{a,b\} \) .
    \item $I=\{i,j\}$.
    \item $S$ es el \textbf{símbolo inicial}.
    \item $\#$ es el \textbf{símbolo inicial de la pila}.
    \item $P$ es un conjunto finito de \textbf{producciones}:
          \begin{multicols}{2}
              \begin{itemize}
                  \item $S\underset{\varepsilon}{\to} AS\,|\,BS\,|\,C$
                  \item $C\underset{\varepsilon}{\to} RC\,|\,L$
                  \item $R\underset{\overline{i}}{\to} RA$
                  \item $R\underset{\overline{j}}{\to} RB$
                  \item $R\underset{[\#]}{\to} \varepsilon$
                  \item $A\underset{i}{\to} a$
                  \item $B\underset{j}{\to} b$
                  \item $L\underset{\overline{i}}{\to} La\,|\,a$
                  \item $L\underset{\overline{j}}{\to} Lb\,|\,b$
              \end{itemize}
          \end{multicols}
\end{itemize}

Observe que la pila en esta gramática es un mecanismo de memoria suficiente para generar el lenguaje \textit{Copy} como se
define en \cite{globalIndexLanguages}, a cada caracter de $\Sigma$ le corresponde un no terminal y un símbolo de la pila 
por el que se puede producir dicho caracter almacenando el caracter el símbolo de la pila correspondiente, el funcionamiento 
de la gramática lo compone además un mecanismo de recursión por el que se puede producir un no terminal asociado a un caracter
solo eliminando el símbolo de la pila asociado a dicho caracter por último el mecanismo de recursión produce la cadena vacía
solo si el símbolo inicial de la pila se encuentra en el tope.

Entonces dada esta gramática es relativamente realizar una modificación para generar el lenguaje $L_{0,1}$,
a esta nueva gramática se denominará $G_{0,1}$. Luego $G_{0,1}$ se define como:

$$
    G_{0,1} = (N, \Sigma, I, S, \#, P) 
$$
donde:

\begin{itemize}
    \item $N= \{S,R,A,B,C,D\}$.
    \item \( \Sigma=\{0,1,d\} \) .
    \item $I=\{i,j,k\}$.
    \item $S$ es el \textbf{símbolo inicial}.
    \item $\#$ es el \textbf{símbolo inicial de la pila}.
    \item $P$ es un conjunto finito de \textbf{producciones}:
          \begin{multicols}{2}
              \begin{itemize}
                  \item $S\underset{\varepsilon}{\to} AS\,|\,BS\,|\,DC$
                  \item $C\underset{\varepsilon}{\to} RC\,|\,L$
                  \item $R\underset{\overline{i}}{\to} RA$
                  \item $R\underset{\overline{j}}{\to} RB$
                  \item $R\underset{\overline{k}}{\to} RD$
                        
                  \item $R\underset{[\#]}{\to} \varepsilon$
                  \item $A\underset{i}{\to} a$
                  \item $B\underset{j}{\to} b$
                  \item $D\underset{k}{\to} d$
                  \item $L\underset{\overline{i}}{\to} L$
                  \item $L\underset{\overline{j}}{\to} L$
                  \item $L\underset{\overline{k}}{\to} L$
                  \item $L\underset{[\#]}{\to} \varepsilon$
              \end{itemize}
          \end{multicols}
\end{itemize}

Como modificaciones a la gramática anterior se ha introducido un nuevo terminal, un no terminal y un símbolo de la pila 
manteniendo la invariante de correspondencia que se mencionó anteriormente entre los elementos de estos 3 conjuntos. Por
otro se modificaron las producciones del no terminal $L$ para que unicamente produzca la cadena vacía eliminando todos Los
elementos de la pila, con ello se puede generar el lenguaje $L_{0,1}$.
\section{FULL-SAT como lenguaje de concatenación de rango}

En esta sección se presentará un enfoque para generar el lenguaje $L_{0,1}$ primeramente y luego para generar el lenguaje $L_{FULL-SAT}$.
Ambos enfoques usando gramáticas de concatenación de rango.

\subsection{$L_{0,1}$ como lenguaje de concatenación de rango}

Se define la gramática $G_{0,1}$ como sigue:
\[
    G_{0,1} = (N, T, V, P, S),
\]
donde:

\begin{itemize}
    \item $N=\{S,A,B,C,Eq\}$
    \item $=\{0,1,d\}$.
    \item $V=\{X,Y,P\}$.
    \item El conjunto de cláusulas $P$ es el siguiente:
          \begin{itemize}
              \item  $S(X)\to A(X)$
              \item $A(YdX)\to B(X,Y)C(X)$
              \item $B(XdY,P)\to B(Y,P) C(X) Eq(X,P)$
              \item $B(\varepsilon,Y)\to \varepsilon$
              \item $C(0X)\to C(X)$
              \item $C(1X)\to C(X)$
              \item $C(\varepsilon)\to \varepsilon$
          \end{itemize}
    \item El \textbf{símbolo inicial} es $S$.
\end{itemize}

El predicado $Eq$ se define en \cite{mainRCGBib} y comprueba que dos cadenas sobre un alfabeto sean iguales, por otro lado 
le predicado $B$ se encarga de definir la sustitución en rango de la próxima cadena de 0 y 1 y comprobar que este sea igual
al patrón inicial. De esta manera se pueden reconocer cadenas que pertenezcan al lenguaje $L_{0,1}$.

Como se mencionó anteriormente las RCG no son cerradas bajo transducción finita, por tanto no se puede realizar el mismo 
análisis que en las 2 secciones anteriores, esto no necesariamente impide que el lenguaje generado por la transducción finita del lenguaje que 
representa la gramática anterior no pueda cumplir ciertas características que permitan generar lenguaje $L_{FULL-SAT}$. 

\subsection{Otro enfoque para generar el lenguaje FULL-SAT}

A continuación se presentará una estrategia distinta para generar el lenguaje $L_{FULL-SAT}$, que no emplea el transductor $T_{SAT}$,
en este caso la función $f_{SAT}$ se define dentro del funcionamiento de la gramática para ello se define la siguiente RCG:
\[
    G_{FULL-SAT} = (N, T, V, P, S),
\]
donde:

\begin{itemize}
    \item $N=\{S,A,B,C,P,N,Cp,Cn\}$
    \item $=\{a,b,c,d\}$.
    \item $V=\{X,Y\}$.
    \item El \textbf{símbolo inicial} es $S$.
\end{itemize}

A continuación se desglosa el conjunto de \textbf{cláusulas} $P$ en varias fases agrupando las cláusulas
por funcionalidad:

\begin{itemize}
    \item $S(X)\to A(X)$
    \item \textbf{Primera fase:} El siguiente conjunto de cláusulas genera la cadena de 0 y 1 que que da valores a las variables de la
          fórmula booleana:
          \begin{multicols}{2}
              \begin{itemize}
                  \item $A(aX)\to P(X,1)$
                  \item $A(aX)\to N(X,0)$
                  \item $A(bX)\to N(X,1)$
                  \item $A(bX)\to P(X,0)$
                  \item $A(cX)\to N(X,1)$
                  \item $A(cX)\to N(X,0)$
                        
                  \item $P(aX,Y)\to P(X,Y1)$
                  \item $P(aX,Y)\to P(X,Y0)$
                  \item $P(bX,Y)\to P(X,Y1)$
                  \item $P(bX,Y)\to P(X,Y0)$
                  \item $P(cX,Y)\to P(X,Y1)$
                  \item $P(cX,Y)\to P(X,Y0)$
                  \item $P(dX,Y)\to B(X,Y)$
                        
                  \item $N(aX,Y)\to P(X,Y1)$
                  \item $N(aX,Y)\to N(X,Y0)$
                  \item $N(bX,Y)\to N(X,Y1)$
                  \item $N(bX,Y)\to P(X,Y0)$
                  \item $N(cX,Y)\to N(X,Y1)$
                  \item $N(cX,Y)\to N(X,Y0)$
              \end{itemize}
          \end{multicols}
          
          El predicado $A$ representa el primer predicado reconocimiento de este se deriva a los predicados $P$
          (representa que la cláusula de la fórmula booleana se encuentra en un estado de verdad positivo) 
          y $N$ (representa que la cláusula de la fórmula booleana se encuentra en un estado de verdad negativo)
          en dependencia del valor de la instancia de la variable correspondiente. El predicado $P$ deriva hacia
          sí mismo independientemente del símbolo, exceptuando el símbolo $d$, caso en el que se procede a la siguiente
          fase.
          El funcionamiento de esta face es prácticamente el mismo de descrito en el transductor $T_{SAT}$.
          
    \item \textbf{Segunda fase:} El siguiente conjunto de cláusulas se encarga de un mecanismo de clausura que le permite a la gramática
          reconocer si la asignación realiza en la fase anterior es válida para las restantes cláusulas de la fórmula
          booleana.
          \begin{itemize}
              \item $B(X_1dX_2,Y)\to C(X_1,Y) B(X_2,Y)$
              \item $B(\varepsilon,Y)\to\varepsilon$
          \end{itemize}
          
          El predicado $B$ permite realizar la clausura mientras que le predicado $C$ comprueba que la cláusula de la fórmula
          booleana actual sea satisfacible.
          
    \item \textbf{Tercera fase:} Solo resta definir el comportamiento de $C$:
          \begin{itemize}
              \begin{multicols}{2}
                  \item $C(X,Y)\to Cn(X,Y)$
                  
                  \item $Cn(aX,1Y) \to Cp(X,Y)$
                  \item $Cn(aX,0Y) \to Cn(X,Y)$
                  \item $Cn(bX,1Y) \to Cn(X,Y)$
                  \item $Cn(bX,0Y) \to Cp(X,Y)$
                  \item $Cn(cX,1Y) \to Cn(X,Y)$
                  \item $Cn(cX,0Y) \to Cn(X,Y)$
                  
                  \item $Cp(aX,1Y) \to Cp(X,Y)$
                  \item $Cp(aX,0Y) \to Cp(X,Y)$
                  \item $Cp(bX,1Y) \to Cp(X,Y)$
                  \item $Cp(bX,0Y) \to Cp(X,Y)$
                  \item $Cp(cX,1Y) \to Cp(X,Y)$
                  \item $Cp(cX,0Y) \to Cp(X,Y)$
                  \item $Cp(\varepsilon,\varepsilon)\to \varepsilon$
              \end{multicols}
          \end{itemize}
          
          Observe que este funcionamiento es exactamente igual al de la primera fase con un predicado que representa un estado
          positivo ($Cp$) y un predicado que representa un estado
          positivo ($Cn$) pero esta vez no se genera la cadena sino que se comprueba con un patrón predefinido en la primera
          fase.
          
          Ahora como $G_{FULL-SAT}$ reconoce las fórmulas booleanas satisfacibles solo se debe analizar el problema de la
          palabra para determinar si una fórmula es satisfacible.
          
          \subsubsection{Análisis de la complejidad computacional}
          
          Como se mencionó anteriormente no todas las RCG tienen un algoritmo de parsing lineal y $G_{FULL-SAT}$ es un ejemplo de 
          ello, observe que en la primera fase se genera la cadena binaria que representa la asignación de valores a las variables
          booleanas y dicha cadena participa en los predicados de fases posteriores. Si se analiza el algoritmo de parsing descrito en 
          \cite{mainRCGBib} un factor en la complejidad del algoritmo de parsing es la cantidad de rangos posibles para una cadena 
          que debe ser reconocida por un predicado y en este caso la cadena que estamos analizando puede tomar $2^n$ valores distintos, donde
          $n$ es la cantidad de variables en la fórmula booleana por lo que la cantidad de rangos sería $n^22^n$, pero esta es una cota
          burda ya que una vez generada la cadena de asignación por la forma de la gramática solo se utiliza un solo rango que se va construyendo
          bajo demanda. El resto de las fases de la gramática tienen una complejidad de $m^2$ donde $m$ es la cantidad de caracteres
          en la cadena de entrada, por lo que la complejidad total sería $O(2^nm^2)$.
          
          Este es un resultado interesante ya que demuestra que no es necesario usar el transductor $T_{SAT}$ para definir el 
          lenguaje $L_{FULL-SAT}$, mediante formalismo de escritura regulada.
          
          
\end{itemize}

\bibliographystyle{plain}
\bibliography{../Bibliography}


\end{document}


\backmatter

\begin{conclusions}

    En este trabajo se presentó una estrategia para resolver el SAT usando teoría de lenguajes, la cual se basa en definir
    una codificación de una fórmula booleana en una cadena y definir y construir el lenguaje de todas las fórmulas booleanas
    satisfacibles $L_{S-SAT}$. Luego para determinar si una fórmula booleana es satisfacible es necesario verificar si la cadena asociada
    a dicha fórmula pertenece a $L_{S-SAT}$.
    
    En el capítulo \ref{chap:LSATFT}, se construyó $L_{S-SAT}$ mediante el transductor finito $T_{SAT}$ que recibe
    cadenas del lenguaje $L_{0,1,d}$, las cuales representan todas las posibles interpretaciones de las fórmulas
    booleanas en CNF y genera cadenas del lenguaje $L_{FULL-SAT}$, tales que la fórmula booleana asociada a estas
    cadenas es satisfacible.
    
    El problema de la palabra para todo formalismo que genere
    el lenguaje $L_{0,1,d}$ y sea cerrado bajo transducción finita, es NP-Duro, teniendo en cuenta la conjetura
    de que cualquier formalismo que genere el lenguaje $L_{0,1,d}$, tiene tamaño $O(1)$ en su representación.
    
    En el capítulo \ref{chap:LSATRCG}, se presentó una RCG que reconoce el lenguaje $L_{0,1,d}$ y se argumentó por qué no es posible
    usar esta gramática para construir $L_{S-SAT}$ mediante transducción finita, ya que las RCG no son cerradas bajo transducción finita.
    
    Se construyó una RCG que reconoce el lenguaje $L_{S-SAT}$, lo que permitió demostrar
    que no es necesario construir $L_{S-SAT}$ mediante transducción finita. La gramática que se construyó tiene el problema
    de la palabra no polinomial, y constituye un ejemplo de una RCG donde el algoritmo de reconocimiento es no polinomial.
    Además al obtener una RCG que reconoce $L_{S-SAT}$, se demostró que las RCG cubren todos los problemas de la clase NP,
    ya que las RCG cubren todos los problemas en P \cite{mainRCGBib} y existe una reducción polinomial del SAT a todo problema en NP \cite{authomataTheory}.
    
    Las estrategias presentadas constituyen una vía diferente
    para resolver el SAT, y aunque el problema de la palabra para el formalismo que se construyó es no polinomial,
    este acercamiento puede contribuir a nuevas líneas de investigación para la búsqueda de algoritmos eficientes que permitan
    resolver el SAT.
    
\end{conclusions}

\documentclass[12pt]{article}

\usepackage[utf8]{inputenc} % Permite escribir caracteres especiales directamente
\usepackage[spanish]{babel} % Configura el idioma a español

\usepackage{amsmath}
\usepackage{tikz}
\usepackage{xcolor}
\usepackage[lmargin=2cm,rmargin=5cm]{geometry}

%%%{{{ Comments and the like
\usepackage[textwidth=4cm]{todonotes}
\usepackage{soul}
\usepackage{xcolor}
\newcounter{todocounter}
\newcommand{\comment}[2]{\stepcounter{todocounter}
  {\color{green!50!blue}{(#1$^{{\color{black}\textbf{\thetodocounter}}}$)}}
  \todo[color=green,noline,size=\tiny]{\textbf{\thetodocounter:} #2

  }}
\newcommand{\quitaesto}[1]{{\color{red}(\st{#1})}}

\newcommand{\cambio}[2]{{\color{cyan}{{#2}}}{\color{red}{(\st{#1})}}}

\newcommand{\agregaesto}[1]{{\color{cyan}{{#1}}}}

\newcommand{\notaparaelautor}[1]{{\color{brown}{\textbf{#1}}}}

\newcommand{\errorortografico}[1]{{\fcolorbox{gray}{magenta}{\textcolor{yellow}{\bf #1}}}}
    
%%%}}}


\title{Recomendaciones}
\author{Raudel Alejandro Gómez Molina}

\begin{document}

\maketitle

A partir del trabajo realizado se proponen como temas para investigaciones futuras los
siguientes:

\begin{itemize}
    \item Buscar un formalismo que sea capaz de generar el lenguaje $L_{0,1,d}$, el cual representa todas las interpretaciones
          de las fórmulas booleanas en CNF, que sea cerrado bajo transducción finita, y luego analizar el problema de la palabra para
          el formalismo que se obtiene después de aplicarle el transductor $T_{SAT}$.
    \item Demostrar que cualquier formalismo que genere $L_{0,1,d}$ tiene un tamaño $O(1)$ en su representación.
    \item Analizar qué tipo de formalismo se obtiene al aplicarle el transductor $T_{SAT}$ a la RCG que reconoce
          el $L_{0,1,d}$.
    \item  Analizar qué propiedades limitan que las RCG no sean cerradas bajo transducción finita, construir
          un formalismo basado en las RCG que sea cerrado bajo transducción finita y comprobar que este formalismo
          sea capaz de describir el lenguaje $L_{0,1,d}$.
          \item Construir una RCG que reconozca fórmulas booleanas satisfacibles, donde cada cláusula tiene a lo sumo dos literales (2-SAT),
          que tenga el problema de la palabra polinomial.
\end{itemize}




\begin{thebibliography}{99}

    \bibitem{mainRCGBib}
    Boullier, Pierre.
    \textit{Proposal for a Natural Language Processing Syntactic Backbone}.
    Research Report RR-3342, INRIA, 1998.

    \bibitem{propertiesRCGBib}
    Boullier, Pierre.
    \textit{A Cubic Time Extension of Context-Free Grammars}.
    Research Report RR-3611, INRIA, 1999.

    \bibitem{simpleMatrixLanguages}
    Ibarra, Oscar H.
    \textit{Simple matrix languages}.
    \textit{Information and Control}, Vol. 17, No. 4, pp. 359-394, 1970.

    \bibitem{globalIndexLanguages}
    Castaño, José M.
    \textit{Global Index Languages}.
    Ph.D. Thesis, The Faculty of the Graduate School of Arts and Sciences, Brandeis University, 2004.

    \bibitem{authomataTheory}
    Hopcroft, John E., Motwani, Rajeev, y Ullman, Jeffrey D.
    \textit{Introduction to Automata Theory, Languages, and Computation}.
    3ª edición, Addison-Wesley, 2006. ISBN: 9780321455369.

    \bibitem{aCFSAT}
    Fernández Arias, Alina.
    \textit{El problema de la satisfacibilidad booleana libre del contexto}.
    Facultad de Matemática y Computación, Universidad de La Habana, 2007.

    \bibitem{aSRCSAT}
    Aguilera López, Manuel.
    \textit{Problema de la Satisfacibilidad Booleana de Concatenación de Rango Simple}.
    Facultad de Matemática y Computación, Universidad de La Habana, 2016.

    \bibitem{aSMSAT}
    Rodríguez Salgado, José Jorge.
    \textit{Gramáticas Matriciales Simples. Primera aproximación para una solución al problema SAT}.
    Facultad de Matemática y Computación, Universidad de La Habana, 2019.

\end{thebibliography}


% Posibles conclusiones
% - teorica
% - como las gramáticas de concatenacion de rango constituyen un nuevo enfoque en la solucion del satisfacibilidad

% Posibles recomendaciones
% - por que via del transductor Full-SAT pueden desarrollarse nuevas investigaciones


\end{document}
\printbibliography[heading=bibintoc]


\end{document}