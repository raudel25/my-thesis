\begin{resumen}
	El problema de la satisfacibilidad booleana es un problema NP-Completo y consiste en determinar si existe
	alguna interpretación verdadera de una fórmula booleana dada. La teoría de lenguajes es una rama fundamental de la Ciencia de la Computación y la matemática que se enfoca 
	en el estudio de los lenguajes formales. El objetivo de este trabajo es vincular el problema de la satisfacibilidad
	booleana y la teoría de lenguajes, para construir el lenguaje de todas las fórmulas booleanas satisfacibles.
	Para esta construcción se emplean 2 estrategias, la primera utiliza una transducción de una variante del lenguaje
	\textit{Copy} para construir dicho lenguaje y la segunda utiliza una gramática de concatenación de rango 
	que reconoce este lenguaje. Como resultado de la primera estrategia se demuestra que el problema de la palabra 
	para todos los formalismos
	que generen la variante del lenguaje \textit{Copy} y sean cerrados bajo transducción finita, es NP-Duro.
	Por otro lado, la gramática de concatenación de rango que se obtiene en la segunda estrategia permite 
	demostrar que las gramáticas de concatenación de rango reconocen todos los problemas de la clase NP, en 
	su representación como lenguaje formal.
\end{resumen}

\begin{abstract}
	The boolean satisfiability problem is an NP-Complete problem and consists in determining whether there exists
	any true interpretation of a given boolean formula. Language theory is a fundamental branch of Computer Science and Mathematics that focuses
	on the study of formal languages. The objective of this work is to link the problem of boolean
	satisfiability and language theory, to build the language of all satisfiable boolean formulas.
	For this construction, 2 strategies are used, the first uses a transduction of a variant of the language
	\textit{Copy} to build said language and the second uses a range concatenation grammar
	that recognizes this language. As a result of the first strategy, it is shown that the word problem
	for all formalisms
	that generate the variant of the language \textit{Copy} and are closed under finite transduction, is NP-Hard.
	On the other hand, the rank concatenation grammar obtained in the second strategy allows
	to demonstrate that rank concatenation grammars recognize all the problems of the NP class, in
	their representation as a formal language.
\end{abstract}