\documentclass{beamer}

% Paquetes adicionales (opcional)
\usepackage[utf8]{inputenc} % Para caracteres especiales
\usepackage[spanish]{babel} % Idioma español
\usepackage{graphicx} % Para incluir imágenes

\usetheme{Madrid} % Puedes probar otros temas como Warsaw, AnnArbor, etc.

\usepackage{graphicx} % Permite incluir imágenes
\usepackage{hyperref} % Enlaces clicables
\usepackage{tikz} % Para diagramas
\usepackage{ragged2e}
\usepackage{amsmath}
\usepackage{xcolor}
\usetikzlibrary{shapes.geometric, arrows}

\tikzstyle{startstop} = [rectangle, rounded corners, minimum width=3cm, minimum height=1cm,text centered, draw=black, fill=red!30]
\tikzstyle{process} = [rectangle, minimum width=3cm, minimum height=1cm, text centered, draw=black, fill=blue!20]
\tikzstyle{decision} = [diamond, minimum width=3cm, minimum height=1cm, text centered, draw=black, fill=green!20]
\tikzstyle{arrow} = [thick,->,>=stealth]

\usepackage{cancel}

\usepackage{tikz}
\usetikzlibrary{automata, positioning}

% \AtBeginSection[]{
%     \begin{frame}{Contenido}
%         \tableofcontents[currentsection]
%     \end{frame}
% }


% Información del título
\title{Una aproximación al lenguaje de todas las fórmulas booleanas satisfacibles}
\author{Raudel Alejandro Gómez Molina}
\institute{Facultad de Matemática y Computación \\ Universidad de La Habana}
\date{\today}

\newcommand{\tutor}{MSc. Fernando Raul Rodriguez Flores} %
% Comando personalizado para la información de la portada


\begin{document}

% Portada
\begin{frame}
    \titlepage
    \vspace{1cm} % Espacio adicional
    \begin{center}
        Tutor: \tutor \\ % Muestra el nombre del tutor
        \smallskip
    \end{center}
\end{frame}

\begin{frame}
    \frametitle{Teoría de lenguajes (Conceptos)}

    \begin{block}{Alfabeto}
        Un alfabeto es un conjunto finito de símbolos
        $$\Sigma=\{0,1\}$$
    \end{block}

    \pause

    \begin{block}{Cadena}
        Una cadena es una sucesión finita de símbolos del alfabeto
        $$w=0001111$$
    \end{block}

    \pause

    \begin{block}{Lenguaje}
        Un lenguaje es un conjunto finito de cadenas
        $$L=\{0^n1^m\mid n,m\in \mathbb{N}\}$$
    \end{block}
\end{frame}

\begin{frame}
    \frametitle{Teoría de lenguajes (Problemas)}

    \begin{block}{Problema de la palabra}
        \begin{itemize}
            \item Determinar si una cadena pertenece a un lenguaje ¿$w$ pertenece a $L$?
                  \pause
            \item $10001\in \{1w\mid w\in\{0,1\}^*\}$
                  \pause
            \item $00001\notin \{1w\mid w\in\{0,1\}^*\}$
        \end{itemize}

    \end{block}

    \pause

    \begin{block}{Problema en Ciencia de la Computación}
        \begin{itemize}
            \item Todo problema se puede reducir a un problema de la palabra
                  \pause
            \item Todo problema se puede codificar como lenguaje formal
                  \pause
            \item Determinar si dos números son primos relativos
                  \pause
            \item Determinar si un arreglo está ordenado
        \end{itemize}
    \end{block}

\end{frame}

\begin{frame}
    \frametitle{Clases de problemas}

    \begin{itemize}
        \item Existen problemas para los cuales no se conoce una solución eficiente
              \pause
        \item Comprobar si una solución es válida es eficiente (clase NP)
              \pause
        \item El SAT es el primer problema demostrado como NP-Completo
              \pause
        \item Pertenece a la clase NP
              \pause
        \item Se puede reducir a cualquier problema en NP en tiempo polinomial
    \end{itemize}

\end{frame}

\begin{frame}
    \frametitle{Problema de la Satisfacibilidad booleana (SAT)}

    \begin{itemize}
        \item Consiste en determinar si una fórmula booleana es satisfacible
              \begin{Large}
                  $$x_1\vee x_2 \wedge \neg x_1 \wedge x_3$$
              \end{Large}
              \pause
        \item Existen instancias polinomiales 2-SAT, Horn-SAT y XOR-SAT
    \end{itemize}
\end{frame}

\begin{frame}
    \frametitle{Estructura de la presentación}

    Definir y construir el lenguaje de todas las fórmulas booleanas satisfacibles

    \pause

    \begin{block}{Definir $L_{S-SAT}$}
        \begin{itemize}
            \item Codificar una fórmula booleana
            \item Definir $L_{S-SAT}$
        \end{itemize}
    \end{block}

    \pause

    \begin{block}{Construir $L_{S-SAT}$ mediante transducción finita}
        \begin{itemize}
            \item Asignar valores a una fórmula booleana mediante una cadena
            \item Construir $L_{S-SAT}$ mediante un transductor finito
        \end{itemize}
    \end{block}

\end{frame}

% \begin{frame}
%     \frametitle{Estructura de la presentación}

%     \tableofcontents
% \end{frame}

\begin{frame}
    \frametitle{Estructura de la presentación}

    Definir y construir el lenguaje de todas las fórmulas booleanas satisfacibles

    \begin{block}{Definir $L_{S-SAT}$}
        \begin{itemize}
            \item Codificar una fórmula booleana
            \item Definir $L_{S-SAT}$
        \end{itemize}
    \end{block}

    \setbeamercolor{block title}{fg=white, bg=gray!50} % Color del título del bloque (gris al 50%)
    \setbeamercolor{block body}{bg=gray!20}

    \begin{block}{Construir $L_{S-SAT}$ mediante transducción finita}
        \begin{itemize}
            \item Asignar valores a una fórmula booleana mediante una cadena
            \item Construir $L_{S-SAT}$ mediante un transductor finito
        \end{itemize}
    \end{block}
\end{frame}

\section{Codificación de una fórmula booleana en una cadena}

\begin{frame}
    \frametitle{Fórmula normal conjuntiva (CNF)}

    \begin{itemize}
        \item Toda fórmula booleana tiene una fórmula equivalente en CNF
              \pause
        \item Asumimos que las fórmulas booleanas están en CNF
    \end{itemize}

\end{frame}

\begin{frame}

    \frametitle{Estados de una variable en una cláusula}

    \begin{Large}
        $$(x_1\vee  \neg x_3)$$
    \end{Large}

    \pause
    \vspace{1cm}
    \begin{itemize}
        \item<1-> $x_1$ está sin negar en la cláusula
        \item<3-> $x_2$ no está en la cláusula
        \item<2-> $x_3$ está negada en la cláusula
    \end{itemize}
\end{frame}

\begin{frame}
    \frametitle{Codificación de una cláusula en una cadena}

    \begin{itemize}
        \item $a$: la variable está sin negar en la cláusula
              \pause
        \item $b$: la variable está negada en la cláusula
              \pause
        \item $c$: la variable no está en la cláusula
    \end{itemize}

    \pause


    \begin{Large}
        $$x_1\vee \neg x_3$$

        \pause

        \begin{columns}
            \begin{column}{1cm}
                $$({x_1}$$$$a$$
            \end{column}
            \pause
            \begin{column}{1cm}
                $$\cancel{\vee}$$$$\ $$
            \end{column}
            \begin{column}{1cm}
                $$\cancel{x_2}$$$$c$$
            \end{column}
            \pause
            \begin{column}{1cm}
                $$\vee$$$$\ $$
            \end{column}
            \begin{column}{1cm}
                $$\neg{x_3})$$$$b$$
            \end{column}
        \end{columns}

        \pause

        $$acb$$
    \end{Large}


\end{frame}

\begin{frame}
    \frametitle{Codificación de una fórmula booleana en una cadena}

    \begin{itemize}
        \item Obtener la codificación de cada cláusula
              \pause
        \item Establecer un separador para delimitar cada cláusula $d$
              \pause
        \item Concatenar cada cláusula seguida del separador $d$
    \end{itemize}

    \pause

    \begin{Large}
        $$(x_1)\wedge (x_1\vee \neg x_2 \vee x_3)\wedge (\neg x_2\vee x_3)$$

        \pause

        \begin{columns}
            \begin{column}{1cm}
                $$(x_1)$$
                $$acc\mathbf{d}$$
            \end{column}
            \pause
            \begin{column}{0.2cm}
                $$\wedge$$
                $$\ $$
            \end{column}
            \begin{column}{3cm}
                $$(x_1\vee \neg x_2 \vee x_3)$$
                $$aba\mathbf{d}$$
            \end{column}
            \pause
            \begin{column}{0.2cm}
                $$\wedge$$
                $$\ $$
            \end{column}
            \begin{column}{2cm}
                $$(\neg x_2\vee x_3)$$
                $$cba\mathbf{d}$$
            \end{column}
        \end{columns}
        \pause

        $$acc\mathbf{d}aba\mathbf{d}cba\mathbf{d}$$
    \end{Large}


\end{frame}

\begin{frame}
    \frametitle{Lenguaje de todas las fórmulas booleanas satisfacibles}

    \begin{itemize}
        \item $L_{FULL-SAT}$ lenguaje de todas las fórmulas booleanas en CNF
              \pause
        \item  $L_{S-SAT}$ lenguaje de todas las fórmulas booleanas satisfacibles
              \pause
              \begin{Large}

                  $$x_1\wedge x_2 \wedge x_3$$
                  $$acc\mathbf{d}cac\mathbf{d}cca\mathbf{d}\in L_{S-SAT}$$

              \end{Large}

    \end{itemize}
\end{frame}

\section{Asignar valores a una fórmula booleana mediante una cadena}

\begin{frame}
    \frametitle{Estructura de la presentación}

    Definir y construir el lenguaje de todas las fórmulas booleanas satisfacibles

    \setbeamercolor{block title}{fg=white, bg=gray!50} % Color del título del bloque (gris al 50%)
    \setbeamercolor{block body}{bg=gray!20}

    \begin{block}{Definir $L_{S-SAT}$}
        \begin{itemize}
            \item Codificar una fórmula booleana
            \item Definir $L_{S-SAT}$
        \end{itemize}
    \end{block}

    \setbeamercolor{block title}{fg=white, bg=structure}
    \setbeamercolor{block body}{bg=structure!20}

    \begin{block}{Construir $L_{S-SAT}$ mediante transducción finita}
        \begin{itemize}
            \item Asignar valores a una fórmula booleana mediante una cadena
            \item Construir $L_{S-SAT}$ mediante un transductor finito
        \end{itemize}
    \end{block}

\end{frame}


\begin{frame}
    \frametitle{Asignar valores a una cláusula mediante una cadena binaria}

    \begin{itemize}
        \item<1-> Se tiene una cadena $q$ que representa una cláusula
        \item<2-> Se tiene una cadena binaria $w$ que representa una asignación
        \item<3-> Se debe cumplir que $|q|=|w|$
        \item<4-> Si el $i$-ésimo caracter de $q$ es 1, $x_i=true$; si es 0, $x_i=false$
              \vspace{0.5cm}
        \item<1-> $q=acb$
        \item<2-> $w=101$
        \item<4-> $x_1=true$, $x_2=false$ y $x_3=true$
    \end{itemize}

\end{frame}

\begin{frame}
    \frametitle{Asignar valores a una cláusula mediante una cadena binaria}

    \begin{center}
        \begin{Large}
            $w=101$ \hspace{1cm} $q=acb\ \Leftrightarrow\ C=({x_1}\cancel{\vee} \cancel{x_2}\vee \neg {x_3})$
        \end{Large}
    \end{center}

    \pause
    \vspace{1cm}

    \begin{itemize}
        \item $w_1=1\Rightarrow x_1=true$ $C$ se evalúa positiva
              \pause
        \item $w_2=0\Rightarrow x_2=false$ $C$ se mantiene positiva
              \pause
        \item $w_3=1\Rightarrow x_3=true$ $C$ se mantiene positiva

    \end{itemize}
\end{frame}

\begin{frame}
    \frametitle{Asignar valores a una fórmula mediante una cadena}

    \begin{itemize}
        \item<1->$w$ (cadena binaria) es la asignación de valores para una cláusula
        \item<2-> Si la fórmula booleana $F$ tiene $n$ cláusulas
        \item<3-> $(wd)^n$ representa la asignación de variables para $F$
        \item<4-> $L_{0,1,d}=\{(wd)^+\mid w\in\{0,1\}^+\}$ lenguaje de todas las interpretaciones
              \vspace{0.5cm}
        \item<1-> $w=101$
        \item<2-> $e= acc\mathbf{d}aba\mathbf{d}cba\mathbf{d}$
        \item<3-> $r=101\mathbf{d}101\mathbf{d}101\mathbf{d}$
              $$(true)\wedge(true\vee \neg false \vee true) \wedge (\neg false\vee true)=true$$

    \end{itemize}
\end{frame}

\section{Construcción de $L_{S-SAT}$ mediante una transducción finita}

\begin{frame}
    \frametitle{Transductor finito}

    \begin{itemize}
        \item<1-> Transductor finito
        \item<2-> En cada transición se lee y se escribe un símbolo
        \item<3-> Se reconoce y se escribe una cadena
    \end{itemize}
\end{frame}

\begin{frame}
    \frametitle{Transductor finito}

    \scalebox{0.9}{
        \begin{columns}
            \begin{column}{0.48\textwidth}
                \begin{figure}[h]
                    \centering  \begin{otherlanguage}{english}
                        \begin{tikzpicture}[shorten >=1pt, node distance=3cm, on grid, auto]

                            % Nodos
                            \node[state, initial] (q0)   {$q_0$};
                            \node[state] (q1) [above right=of q0] {$q_1$};
                            \node[state, accepting] (q2) [below right=of q0] {$q_2$};

                            % Transiciones
                            \path[->]
                            (q0) edge [bend left] node {0} (q1)
                            (q1) edge [loop above] node {0} (q1)
                            (q1) edge [bend left] node {1} (q2)
                            (q2) edge [loop below] node {1} (q2);
                        \end{tikzpicture}
                    \end{otherlanguage}
                \end{figure}
            \end{column}
            \begin{column}{0.48\textwidth}
                \begin{figure}[h]
                    \centering  \begin{otherlanguage}{english}
                        \begin{tikzpicture}[shorten >=1pt, node distance=3cm, on grid, auto]

                            % Nodos
                            \node[state, initial] (q0)   {$q_0$};
                            \node[state] (q1) [above right=of q0] {$q_1$};
                            \node[state, accepting] (q2) [below right=of q0] {$q_2$};

                            % Transiciones
                            \path[->]
                            (q0) edge [bend left] node {0/a} (q1)
                            (q1) edge [loop above] node {0/a} (q1)
                            (q1) edge [bend left] node {1/b} (q2)
                            (q2) edge [loop below] node {1/b} (q2);
                        \end{tikzpicture}
                    \end{otherlanguage}
                \end{figure}
            \end{column}
        \end{columns}}

\end{frame}

\begin{frame}

    \frametitle{Construcción de $L_{S-SAT}$ mediante una transducción finita}

    \begin{itemize}
        \item Cadena binaria $w$ generar todas las cláusulas satisfacibles por $w$
              \pause
        \item Si se lee un 1 y se escribe una $a$ (positiva)
              \pause
        \item Si se lee un 0 y se escribe una $a$ (negativa)
              \pause
        \item Si se lee un 1 y se escribe una $b$ (negativa)
              \pause
        \item Si se lee un 0 y se escribe una $b$ (positiva)
              \pause
        \item Si se lee cualquier cosa y se escribe una $c$ (negativa)
              \pause
        \item Transductor $T_{CLAUSE}$
    \end{itemize}

\end{frame}

\begin{frame}
    \frametitle{Transductor $T_{CLAUSE}$}

    \begin{columns}
        \begin{column}{0.38\textwidth}
            Entrada y Salida
            \begin{itemize}
                \item Cadena binaria $w$
                \item Todas las cláusulas satisfacibles por $w$
            \end{itemize}

            \pause
            \vspace{0.5cm}

            Estados
            \begin{itemize}
                \item $q_0$: estado inicial
                \item $q_p$: estado positivo (estado de aceptación)
                \item $q_n$: estado negativo
            \end{itemize}
            \pause
        \end{column}

        \begin{column}{0.6\textwidth}

            \begin{figure}[h]
                \centering  \begin{otherlanguage}{english}
                    \begin{tikzpicture}[shorten >=1pt, node distance=3cm, on grid, auto]

                        % Nodos
                        \node[state, initial] (q0)   {$q_0$};
                        \node[state] (qn) [above right=of q0] {$q_n$};
                        \node[state, accepting] (qp) [below right=of q0] {$q_p$};

                        % Transiciones
                        \path[->]
                        (q0) edge [bend left] node {0/a,1/b} (qn)
                        (q0) edge [bend right] node {1/a,0/b} (qp)
                        (q0) edge [loop right] node {0/c,1/c} (q0)

                        (qn) edge [bend left] node {1/a,0/b} (qp)
                        (qn) edge [loop above] node {0/a,1/b,0/c,1/c} (qn)

                        (qp) edge [loop below] node {1/a,0/b,0/a,1/b,0/c,1/c} (qp);

                    \end{tikzpicture}
                \end{otherlanguage}
            \end{figure}
        \end{column}

    \end{columns}


\end{frame}

\begin{frame}
    \frametitle{Construcción de $L_{S-SAT}$ mediante una transducción finita}

    \begin{itemize}
        \item<1-> Cadena $r\in L_{0,1,d}$ generar todas las cadenas $e\in L_{FULL-SAT}$
        \item<2-> $e$ representa una fórmula booleana satisfacible por $r$
        \item<3-> Transductor $T_{SAT}$
    \end{itemize}
\end{frame}


\begin{frame}
    \frametitle{Transductor $T_{SAT}$}

    \begin{columns}
        \begin{column}{0.38\textwidth}
            Entrada y Salida
            \begin{itemize}
                \item Cadena $r\in L_{0,1,d}$
                \item Todas las fórmulas satisfacibles por $r$
            \end{itemize}

            \pause
            \vspace{0.5cm}

            Estados
            \begin{itemize}
                \item $q_0$: estado inicial (estado de aceptación)
                \item $q_p$: estado positivo
                \item $q_n$: estado negativo
            \end{itemize}
            \pause
        \end{column}

        \begin{column}{0.6\textwidth}

            \begin{figure}[h]
                \centering  \begin{otherlanguage}{english}
                    \begin{tikzpicture}[shorten >=1pt, node distance=3cm, on grid, auto]

                        % Nodos
                        \node[state, initial, accepting] (q0)   {$q_0$};
                        \node[state] (qn) [above right=of q0] {$q_n$};
                        \node[state] (qp) [below right=of q0] {$q_p$};

                        % Transiciones
                        \path[->]
                        (q0) edge [bend left] node {0/a,1/b} (qn)
                        (q0) edge [bend right] node {1/a,0/b} (qp)
                        (q0) edge [loop right] node {0/c,1/c} (q0)

                        (qn) edge [bend left] node {1/a,0/b} (qp)
                        (qn) edge [loop above] node {0/a,1/b,0/c,1/c} (qn)

                        (qp) edge [loop below] node {1/a,0/b,0/a,1/b,0/c,1/c} (qp)

                        (qp) edge [bend left=75, color=green] node {d/d} (q0);

                    \end{tikzpicture}
                \end{otherlanguage}
            \end{figure}
        \end{column}

    \end{columns}


\end{frame}

\begin{frame}
    \frametitle{Transductor $T_{SAT}$}

    \begin{figure}[h]
        \scalebox{0.9}{
            \begin{otherlanguage}{english}
                \centering \begin{tikzpicture}[shorten >=1pt, node distance=3cm, on grid, auto]

                    % Nodos
                    \node[state, initial] (q01)   {$q_{0_1}$};
                    \node[state] (qn1) [above right=of q01] {$q_{n_1}$};
                    \node[state] (qp1) [below right=of q0] {$q_{p_1}$};
                    \node[state, accepting] (q02) [right=6cm of q01] {$q_{0_2}$};
                    \node[state] (qn2) [above right=of q02] {$q_{n_2}$};
                    \node[state] (qp2) [below right=of q02] {$q_{p_2}$};


                    % Transiciones
                    \path[->]
                    (q01) edge [bend left] node {0/a,1/b} (qn1)
                    (q01) edge [bend right] node {1/a,0/b} (qp1)
                    (q01) edge [loop right] node {0/c,1/c} (q01)

                    (qn1) edge [bend left] node {1/a,0/b} (qp1)
                    (qn1) edge [loop above] node {0/a,1/b,0/c,1/c} (qn1)

                    (qp1) edge [loop below] node {1/a,0/b,0/a,1/b,0/c,1/c} (qp1)

                    (q02) edge [bend left] node {0/a,1/b} (qn2)
                    (q02) edge [bend right] node {1/a,0/b} (qp2)
                    (q02) edge [loop right] node {0/c,1/c} (q02)

                    (qn2) edge [bend left] node {1/a,0/b} (qp2)
                    (qn2) edge [loop above] node {0/a,1/b,0/c,1/c} (qn2)

                    (qp2) edge [loop below] node {1/a,0/b,0/a,1/b,0/c,1/c} (qp2);
                \end{tikzpicture}
            \end{otherlanguage}}
    \end{figure}
\end{frame}


\begin{frame}
    \frametitle{Transductor $T_{SAT}$}

    \begin{figure}[h]
        \scalebox{0.9}{
            \begin{otherlanguage}{english}
                \centering \begin{tikzpicture}[shorten >=1pt, node distance=3cm, on grid, auto]

                    % Nodos
                    \node[state, initial] (q01)   {$q_{0_1}$};
                    \node[state] (qn1) [above right=of q01] {$q_{n_1}$};
                    \node[state] (qp1) [below right=of q0] {$q_{p_1}$};
                    \node[state, accepting] (q02) [right=6cm of q01] {$q_{0_2}$};
                    \node[state] (qn2) [above right=of q02] {$q_{n_2}$};
                    \node[state] (qp2) [below right=of q02] {$q_{p_2}$};


                    % Transiciones
                    \path[->]
                    (q01) edge [bend left] node {0/a,1/b} (qn1)
                    (q01) edge [bend right] node {1/a,0/b} (qp1)
                    (q01) edge [loop right] node {0/c,1/c} (q01)

                    (qn1) edge [bend left] node {1/a,0/b} (qp1)
                    (qn1) edge [loop above] node {0/a,1/b,0/c,1/c} (qn1)

                    (qp1) edge [loop below] node {1/a,0/b,0/a,1/b,0/c,1/c} (qp1)

                    (q02) edge [bend left] node {0/a,1/b} (qn2)
                    (q02) edge [bend right] node {1/a,0/b} (qp2)
                    (q02) edge [loop right] node {0/c,1/c} (q02)

                    (qn2) edge [bend left] node {1/a,0/b} (qp2)
                    (qn2) edge [loop above] node {0/a,1/b,0/c,1/c} (qn2)

                    (qp2) edge [loop below] node {1/a,0/b,0/a,1/b,0/c,1/c} (qp2)

                    (qp1) edge [bend right, color=green] node {d/d} (q02);

                \end{tikzpicture}
            \end{otherlanguage}}
    \end{figure}
\end{frame}


\begin{frame}
    \frametitle{Transductor $T_{SAT}$}

    \begin{figure}[h]
        \scalebox{0.9}{
            \begin{otherlanguage}{english}
                \centering \begin{tikzpicture}[shorten >=1pt, node distance=3cm, on grid, auto]

                    % Nodos
                    \node[state, initial] (q01)   {$q_{0_1}$};
                    \node[state] (qn1) [above right=of q01] {$q_{n_1}$};
                    \node[state] (qp1) [below right=of q0] {$q_{p_1}$};
                    \node[state, accepting] (q02) [right=6cm of q01] {$q_{0_2}$};
                    \node[state] (qn2) [above right=of q02] {$q_{n_2}$};
                    \node[state] (qp2) [below right=of q02] {$q_{p_2}$};


                    % Transiciones
                    \path[->]
                    (q01) edge [bend left] node {0/a,1/b} (qn1)
                    (q01) edge [bend right] node {1/a,0/b} (qp1)
                    (q01) edge [loop right] node {0/c,1/c} (q01)

                    (qn1) edge [bend left] node {1/a,0/b} (qp1)
                    (qn1) edge [loop above] node {0/a,1/b,0/c,1/c} (qn1)

                    (qp1) edge [loop below] node {1/a,0/b,0/a,1/b,0/c,1/c} (qp1)

                    (q02) edge [bend left] node {0/a,1/b} (qn2)
                    (q02) edge [bend right] node {1/a,0/b} (qp2)
                    (q02) edge [loop right] node {0/c,1/c} (q02)

                    (qn2) edge [bend left] node {1/a,0/b} (qp2)
                    (qn2) edge [loop above] node {0/a,1/b,0/c,1/c} (qn2)

                    (qp2) edge [loop below] node {1/a,0/b,0/a,1/b,0/c,1/c} (qp2)

                    (qp1) edge [bend right] node {d/d} (q02)
                    (qp2) edge [bend left=75, color=green] node {d/d} (q02);

                \end{tikzpicture}
            \end{otherlanguage}}
    \end{figure}
\end{frame}

\begin{frame}
    \frametitle{Resultados derivados de $T_{SAT}$}

    \begin{block}{Lenguaje de todas las fórmulas booleanas satisfacibles}
        $$L_{S-SAT}=\{e\mid \exists r\in L_{0,1,d} \text{ y } e\in T_{SAT}(r)\}$$
    \end{block}
    \pause
    \begin{block}{Construcción de $L_{S-SAT}$}
        \begin{itemize}
            \item Es necesario un formalismo que genere $L_{0,1,d}$
                  \pause
            \item Sea cerrado bajo transducción finita
        \end{itemize}
    \end{block}
    \pause
    \begin{block}{Resultados}
        \begin{itemize}
            \item Formalismo que genere $L_{0,1,d}$ ($G_{0,1,d}$)
                  \pause
            \item Si $G_{0,1,d}$ es cerrado bajo transducción finita
                  \pause
            \item El problema de la palabra de $G_{0,1,d}$ es NP-Duro
                  \pause
            \item Todo formalismo que genere $L_{0,1,d}$ tiene tamaño $O(1)$ (Conjetura)

        \end{itemize}
    \end{block}
\end{frame}


\begin{frame}
    \frametitle{Conclusiones}

    \begin{itemize}
        \item Formalismo que genere $L_{0,1,d}$ ($G_{0,1,d}$)
              \pause
        \item Si $G_{0,1,d}$ es cerrado bajo transducción finita
              \pause
        \item El problema de la palabra de $G_{0,1,d}$ es NP-Duro
              \pause
        \item Todo formalismo que genere $L_{0,1,d}$ tiene tamaño $O(1)$ (Conjetura)

    \end{itemize}
\end{frame}

\section{Recomendaciones}
\begin{frame}
    \frametitle{Recomendaciones}

    \begin{itemize}
        \item Otro formalismo que genere $L_{0,1,d}$ (cerrado bajo transducción finita)
              \pause
        \item Todo formalismo que genere $L_{0,1,d}$ tiene un tamaño $O(1)$
    \end{itemize}


\end{frame}

\begin{frame}
    \titlepage
    \vspace{1cm} % Espacio adicional
    \begin{center}
        Tutor: \tutor
    \end{center}
\end{frame}


\end{document}