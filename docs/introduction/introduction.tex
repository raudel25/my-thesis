\documentclass{article}

\begin{document}

El problema de satisfacibilidad booleana (\textit{SAT}, por sus siglas en inglés)
es uno de los problemas más fundamentales en la teoría de la computación.
Consiste en determinar si existe una asignación de valores de verdad que satisfaga una fórmula booleana arbitraria.
Fue introducido formalmente en el marco de la lógica proposicional y ha servido como base para numerosas investigaciones
en complejidad computacional.

Una de las primeras variantes del problema SAT que se estudió fue el \textit{2-SAT}, donde cada cláusula de la fórmula
tiene exactamente dos literales. Este problema puede resolverse en tiempo polinómico mediante algoritmos basados en grafos,
como la identificación de componentes fuertemente conexas en grafos dirigidos. A pesar de su simplicidad relativa,
el \textit{2-SAT} ha encontrado aplicaciones prácticas en áreas como el diseño de circuitos y la verificación formal.

Por otro lado, el \textit{3-SAT}, donde cada cláusula tiene exactamente tres literales,
marcó una diferencia significativa al ser demostrado NP-completo. Esta variante, más expresiva y compleja,
se utiliza comúnmente como caso representativo en la teoría de complejidad.


\end{document}