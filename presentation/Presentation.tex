\documentclass{beamer}

% Paquetes adicionales (opcional)
\usepackage[utf8]{inputenc} % Para caracteres especiales
\usepackage[spanish]{babel} % Idioma español
\usepackage{graphicx} % Para incluir imágenes

\usepackage{tikz}
\usetikzlibrary{automata, positioning}



% Información del título
\title{Una aproximación al lenguaje de todas las fórmulas booleanas satisfacibles}
\author{Raudel Alejando Gómez Molina}
\institute{Facultad de Matemática y Computación \\ Universidad de La Habana}
\date{\today}

\newcommand{\tutor}{MSc. Fernando Raul Rodriguez Flores} %
% Comando personalizado para la información de la portada


\begin{document}

% Portada
\begin{frame}
    \titlepage
    \vspace{1cm} % Espacio adicional
    \begin{center}
        Tutor: \tutor \\ % Muestra el nombre del tutor
        \smallskip
    \end{center}
\end{frame}

\begin{frame}
    \frametitle{Objetivos del trabajo}

    \begin{itemize}
        \item Objetivo general: Definir y construir el lenguaje de todas las fórmulas booleanas satisfacibles
        \item Objetivos específicos:
              \begin{itemize}
                  \item Estudiar el estado del arte
                  \item Codificación de una fórmula booleana en una cadena
                  \item Definir el lenguaje de todas las fórmulas booleanas satisfacibles $L_{S-SAT}$
                  \item Construir $L_{S-SAT}$ utilizando una transducción finita
                  \item Construir $L_{S-SAT}$ utilizando gramáticas de concatenación de rango y sin usar la transducción finita.
                  \item Analizar las implicaciones computacionales
              \end{itemize}
    \end{itemize}


\end{frame}

\begin{frame}
    \frametitle{Gramáticas de concatenación de rango (RCG)}

    \begin{itemize}
        \item Un rango es un intervalo de la cadena
        \item A los no terminales se les llama predicados
        \item Cada predicado tiene una secuencia de argumentos y recibe un vector de cadenas
        \item Cada argumento está formado por variables y no terminales
              $$A(aX,cbYZ)$$
        \item Una sustitución de rango asocia una variable a un rango
        \item A las producciones se les llama cláusulas
              \[
                  A(x_1, \ldots, x_k) \to B_1(y_{1,1}, \ldots, y_{1,m_1}) \ldots B_n(y_{n,1}, \ldots, y_{n,m_n}),
              \]


    \end{itemize}
\end{frame}

\begin{frame}
    \frametitle{Gramáticas de concatenación de rango (RCG)}

    \begin{itemize}


        \item Cuando se deriva en las RCG
              \begin{itemize}
                  \item Se asocia cada cadena del vector que recibe el predicado izquierdo a un argumento
                  \item Se hace la sustitución de rango para cada argumento obteniendo los valores de las variables
                  \item Con los valores de las variables se construyen los vectores con los que se evalúa
                        en los predicados derechos
                  \item Se repite el mismo proceso en cada uno de los predicados derechos
              \end{itemize}
        \item Un predicado reconoce un vector si existe una secuencia de derivaciones y sustituciones de rango, tales
              que se deriva en la cadena vacía
    \end{itemize}
\end{frame}


\begin{frame}
    \frametitle{Codificación de una fórmula booleana en una cadena}

    \begin{itemize}
        \item $a$: la variable está sin negar en la cláusula
        \item $b$: la variable está negada en la cláusula
        \item $c$: la variable no está en la cláusula
        \item $d$: separador para delimitar una cláusula
    \end{itemize}

    \vspace{0.5cm}

    \begin{center}
        $(x_1)\wedge(x_1\vee \neg x_2 \vee x_3) \wedge (\neg x_2\vee x_3)$ $\Leftrightarrow$ $acc\mathbf{d}aba\mathbf{d}cba\mathbf{d}$
    \end{center}

    \vspace{0.5cm}

    \begin{itemize}
        \item $L_{FULL-SAT}$ lenguaje de todas las fórmulas booleanas en forma normal conjuntiva
        \item $L_{SAT}$ lenguaje de todas las fórmulas booleanas satisfacibles
    \end{itemize}
\end{frame}

\begin{frame}
    \frametitle{Asignación de variables para una fórmula booleana}

    \begin{itemize}
        \item $L_{0,1,d}=\{(wd)^+\mid w\in\{0,1\}^+\}$
        \item $r\in L_{0,1,d}$ representa una asignación de valores y $e\in L_{FULL-SAT}$ representa una fórmula booleana:
              se debe cumplir que $|e|=|r|$ y que $e$ tenga la misma cantidad de caracteres $d$ que $r$
        \item $r=101\mathbf{d}101\mathbf{d}101\mathbf{d}$ y $e=abc\mathbf{d}cbb\mathbf{d}acc\mathbf{d}$,
              $$(x_1\vee\neg x_2)\wedge (\neg x_2 \vee \neg x_3)\wedge (x_1)$$
              $$(true\vee\neg false)\wedge (\neg false \vee \neg true)\wedge (true)=true$$
    \end{itemize}

\end{frame}

\begin{frame}
    \frametitle{Transductor $T_{CLAUSE}$}

    \begin{columns}
        \begin{column}{0.38\textwidth}
            Entrada: $r\in L_{0,1,d}$\\
            Salida: $e\in L_{FULL-SAT}$
            \vspace{1cm}

            Estados de $T_{CLAUSE}$
            \begin{itemize}
                \item $q_0$: estado inicial
                \item $q_p$: representa el estado positivo (estado de aceptación)
                \item $q_n$: representa el estado negativo
            \end{itemize}
        \end{column}

        \begin{column}{0.6\textwidth}

            \begin{figure}[h]
                \centering  \begin{otherlanguage}{english}
                    \begin{tikzpicture}[shorten >=1pt, node distance=3cm, on grid, auto]

                        % Nodos
                        \node[state, initial] (q0)   {$q_0$};
                        \node[state] (qn) [above right=of q0] {$q_n$};
                        \node[state, accepting] (qp) [below right=of q0] {$q_p$};

                        % Transiciones
                        \path[->]
                        (q0) edge [bend left] node {0/a,1/b} (qn)
                        (q0) edge [bend right] node {1/a,0/b} (qp)
                        (q0) edge [loop right] node {0/c,1/c} (q0)

                        (qn) edge [bend left] node {1/a,0/b} (qp)
                        (qn) edge [loop above] node {0/a,1/b,0/c,1/c} (qn)

                        (qp) edge [loop below] node {1/a,0/b,0/a,1/b,0/c,1/c} (qp);

                    \end{tikzpicture}
                \end{otherlanguage}
            \end{figure}
        \end{column}

    \end{columns}


\end{frame}

\begin{frame}
    \frametitle{Transductor $T_{SAT}$}

    \begin{figure}[h]
        \scalebox{0.9}{
            \begin{otherlanguage}{english}
                \centering \begin{tikzpicture}[shorten >=1pt, node distance=3cm, on grid, auto]

                    % Nodos
                    \node[state, initial] (q01)   {$q_{0_1}$};
                    \node[state] (qn1) [above right=of q01] {$q_{n_1}$};
                    \node[state] (qp1) [below right=of q0] {$q_{p_1}$};
                    \node[state, accepting] (q02) [right=6cm of q01] {$q_{0_2}$};
                    \node[state] (qn2) [above right=of q02] {$q_{n_2}$};
                    \node[state] (qp2) [below right=of q02] {$q_{p_2}$};


                    % Transiciones
                    \path[->]
                    (q01) edge [bend left] node {0/a,1/b} (qn1)
                    (q01) edge [bend right] node {1/a,0/b} (qp1)
                    (q01) edge [loop right] node {0/c,1/c} (q01)

                    (qn1) edge [bend left] node {1/a,0/b} (qp1)
                    (qn1) edge [loop above] node {0/a,1/b,0/c,1/c} (qn1)

                    (qp1) edge [loop below] node {1/a,0/b,0/a,1/b,0/c,1/c} (qp1)

                    (q02) edge [bend left] node {0/a,1/b} (qn2)
                    (q02) edge [bend right] node {1/a,0/b} (qp2)
                    (q02) edge [loop right] node {0/c,1/c} (q02)

                    (qn2) edge [bend left] node {1/a,0/b} (qp2)
                    (qn2) edge [loop above] node {0/a,1/b,0/c,1/c} (qn2)

                    (qp2) edge [loop below] node {1/a,0/b,0/a,1/b,0/c,1/c} (qp2)

                    (qp1) edge [bend right] node {d/d} (q02)
                    (qp2) edge [bend left=75] node {d/d} (q02);

                \end{tikzpicture}
            \end{otherlanguage}}
    \end{figure}
\end{frame}

\begin{frame}
    \frametitle{$L_{0,1,d}$ como lenguaje de concatenación de rango}

    Producciones de la gramática $G_{0,1,d}$
    \begin{itemize}
        \item $S(X)\to A(X)$
        \item $A(XdY)\to B(Y,X)C(X)$
        \item $B(XdY,P)\to B(Y,P) C(X) Eq(X,P)$
        \item $B(\varepsilon,P)\to \varepsilon$
    \end{itemize}
    \vspace{0.5cm}
    \begin{itemize}
        \item $C$ comprueba que la cadena está formada por 0 y 1
        \item $Eq$ comprueba que 2 cadenas sean iguales
    \end{itemize}
\end{frame}

\begin{frame}
    \frametitle{Construcción de $L_{SAT}$ mediante una RCG}

    Las producciones de la gramática $G_{S-SAT}$ se agrupan en 4 grupos (fases):

    \begin{enumerate}
        \item Derivación inicial de la gramática
        \item Mientras se reconoce la cadena se generan todas las posibles interpretaciones
              que hacen verdadera la primera cláusula
        \item Comprobar que la interpretación generada en la primera fase satisface el resto de las cláusulas
        \item Comprobar si la interpretación generada en la primera fase satisface una cláusula
    \end{enumerate}

\end{frame}

\begin{frame}
    \frametitle{Producciones de $G_{S-SAT}$}

    Producciones agrupadas por fases
    \begin{enumerate}
        \item $S(X)\to A(X)$
        \item Compuesta por los predicados $A$, $P$ (estado positivo) y $N$ (estado negativo)
              $$H(\_X,Y)\to J(X,Y\_)$$
        \item $B(X_1dX_2,Y)\to C(X_1,Y) B(X_2,Y)$ y $B(\varepsilon,Y)\to\varepsilon$
        \item Compuesta por los predicados $C$, $Cp$ (estado positivo) y $Cn$ estado negativo\\
              $$H(\_X,\_Y)\to J(X,Y)$$

    \end{enumerate}

\end{frame}

\begin{frame}
    \frametitle{Resultados derivados de $T_{SAT}$ y $G_{S-SAT}$}

    Resultados derivados de $T_{SAT}$
    \begin{itemize}
        \item El problema de la palabra para todo formalismo que genere $L_{0,1,d}$ y sea cerrado bajo transducción finita, es NP-Duro
        \item Se conjetura que todo formalismo que genere $L_{0,1,d}$ tiene tamaño $O(1)$ en su representación
    \end{itemize}
    \vspace{0.5cm}
    Resultados derivados de $G_{S-SAT}$
    \begin{itemize}
        \item No es necesaria la transducción finita de $T_{SAT}$ para construir $L_{S-SAT}$
        \item Las RCG reconocen todos los problemas de la clase NP-Completo en su representación como lenguaje formal
    \end{itemize}
\end{frame}

\begin{frame}
    \frametitle{Problemas propuestos}

    Instancias polinomiales de SAT
    \begin{itemize}
        \item Encontrar una RCG que permita reconocer el 2-SAT y cuyo problema de la palabra sea polinomial

    \end{itemize}
    \vspace{0.5cm}
    Nuevo formalismo basado en las RCG
    \begin{itemize}
        \item Investigar qué propiedades de las RCG limitan que estas no sean cerradas bajo transducción finita
        \item Construir un nuevo formalismo basado en las RCG que sea cerrado bajo transducción finita
        \item Comprobar si este formalismo es capaz de describir el lenguaje $L_{0,1,d}$
    \end{itemize}
\end{frame}

\begin{frame}
    \titlepage
    \vspace{1cm} % Espacio adicional
    \begin{center}
        Tutor: \tutor\\
        \vspace{1cm}
        {\LARGE Muchas gracias}
        \smallskip
    \end{center}
\end{frame}


\end{document}