\documentclass{beamer}

% Additional packages (optional)
\usepackage[utf8]{inputenc} % For special characters
\usepackage[english]{babel} % English language
\usepackage{graphicx} % For images

\usetheme{Madrid} % You can try other themes like Warsaw, AnnArbor, etc.

\usepackage{graphicx} % Allows including images
\usepackage{hyperref} % Clickable links
\usepackage{tikz} % For diagrams
\usepackage{ragged2e}
\usepackage{amsmath}
\usepackage{xcolor}
\usetikzlibrary{shapes.geometric, arrows}

\tikzstyle{startstop} = [rectangle, rounded corners, minimum width=3cm, minimum height=1cm,text centered, draw=black, fill=red!30]
\tikzstyle{process} = [rectangle, minimum width=3cm, minimum height=1cm, text centered, draw=black, fill=blue!20]
\tikzstyle{decision} = [diamond, minimum width=3cm, minimum height=1cm, text centered, draw=black, fill=green!20]
\tikzstyle{arrow} = [thick,->,>=stealth]

\usepackage{cancel}

\usepackage{tikz}
\usetikzlibrary{automata, positioning}

% \AtBeginSection[]{
%     \begin{frame}{Contents}
%         \tableofcontents[currentsection]
%     \end{frame}
% }

% Title information
\title{An Approach to the Language of All Satisfiable Boolean Formulas Using Range Concatenation Grammars}
\author{Raudel Alejandro Gómez Molina}
\institute{Faculty of Mathematics and Computer Science \\ University of Havana}
\date{\today}

\newcommand{\tutor}{MSc. Fernando Raul Rodriguez Flores} %
% Custom command for cover information

\begin{document}

% Cover page
\begin{frame}
    \titlepage
    \vspace{1cm} % Additional space
    \begin{center}
        Advisor: \tutor \\ % Shows the advisor's name
        \smallskip
    \end{center}
\end{frame}

\begin{frame}
    \frametitle{Language Theory (Concepts)}

    \begin{block}{Alphabet}
        An alphabet is a finite set of symbols
        $$\Sigma=\{0,1\}$$
    \end{block}

    \pause

    \begin{block}{String}
        A string is a finite sequence of symbols from the alphabet
        $$w=0001111$$
    \end{block}

    \pause

    \begin{block}{Language}
        A language is a finite set of strings
        $$L=\{0^n1^m\mid n,m\in \mathbb{N}\}$$
    \end{block}
\end{frame}

\begin{frame}
    \frametitle{Language Theory (Problems)}

    \begin{block}{Word Problem}
        \begin{itemize}
            \item Determine if a string belongs to a language. Does $w$ belong to $L$?
                  \pause
            \item $10001\in \{1w\mid w\in\{0,1\}^*\}$
                  \pause
            \item $00001\notin \{1w\mid w\in\{0,1\}^*\}$
        \end{itemize}
    \end{block}

    \pause

    \begin{block}{Problem in Computer Science}
        \begin{itemize}
            \item Every problem can be reduced to a word problem
                  \pause
            \item Every problem can be encoded as a formal language
                  \pause
            \item Determine if two numbers are coprime
                  \pause
            \item Determine if an array is sorted
        \end{itemize}
    \end{block}
\end{frame}

\begin{frame}
    \frametitle{Problem Classes}

    \begin{itemize}
        \item There are problems for which no efficient solution is known
              \pause
        \item Verifying a solution is efficient (class NP)
              \pause
        \item SAT was the first problem proven to be NP-Complete
              \pause
        \item It belongs to the NP class
              \pause
        \item Can be reduced to any problem in NP
    \end{itemize}
\end{frame}

\begin{frame}
    \frametitle{Boolean Satisfiability Problem (SAT)}

    \begin{itemize}
        \item It consists of determining if a Boolean formula is satisfiable
              \begin{Large}
                  $$x_1\vee x_2 \wedge \neg x_1 \wedge x_3$$
              \end{Large}
              \pause
        \item Polynomial instances exist: 2-SAT, Horn-SAT, and XOR-SAT
    \end{itemize}
\end{frame}

\begin{frame}
    \frametitle{Presentation Structure}

    Define and construct the language of all satisfiable Boolean formulas

    \pause

    \begin{block}{Define $L_{S-SAT}$}
        \begin{itemize}
            \item Encode a Boolean formula
            \item Define $L_{S-SAT}$
        \end{itemize}
    \end{block}

    \pause

    \begin{block}{Construct $L_{S-SAT}$ using a Range Concatenation Grammar}
        \begin{itemize}
            \item Introduce Range Concatenation Grammars (RCG)
            \item RCG that recognizes $L_{S-SAT}$
        \end{itemize}
    \end{block}
\end{frame}

\begin{frame}
    \frametitle{Presentation Structure}

    Define and construct the language of all satisfiable Boolean formulas

    \begin{block}{Define $L_{S-SAT}$}
        \begin{itemize}
            \item Encode a Boolean formula
            \item Define $L_{S-SAT}$
        \end{itemize}
    \end{block}

    \setbeamercolor{block title}{fg=white, bg=gray!50} % Block title color (50% gray)
    \setbeamercolor{block body}{bg=gray!20}

    \begin{block}{Construct $L_{S-SAT}$ using a Range Concatenation Grammar}
        \begin{itemize}
            \item Introduce Range Concatenation Grammars (RCG)
            \item RCG that recognizes $L_{S-SAT}$
        \end{itemize}
    \end{block}
\end{frame}

\section{Encoding a Boolean Formula into a String}

\begin{frame}
    \frametitle{Conjunctive Normal Form (CNF)}

    \begin{itemize}
        \item Every Boolean formula has an equivalent CNF formula
              \pause
        \item We assume Boolean formulas are in CNF
    \end{itemize}
\end{frame}

\begin{frame}
    \frametitle{Variable States in a Clause}

    \begin{Large}
        $$(x_1\vee \neg x_3)$$
    \end{Large}

    \pause
    \vspace{1cm}
    \begin{itemize}
        \item<1-> $x_1$ appears unnegated in the clause
        \item<3-> $x_2$ does not appear in the clause
        \item<2-> $x_3$ appears negated in the clause
    \end{itemize}
\end{frame}

\begin{frame}
    \frametitle{Encoding a Clause into a String}

    \begin{itemize}
        \item $a$: the variable appears unnegated in the clause
              \pause
        \item $b$: the variable appears negated in the clause
              \pause
        \item $c$: the variable does not appear in the clause
    \end{itemize}

    \pause

    \begin{Large}
        $$x_1\vee \neg x_3$$

        \pause

        \begin{columns}
            \begin{column}{1cm}
                $$({x_1}$$$$a$$
            \end{column}
            \pause
            \begin{column}{1cm}
                $$\cancel{\vee}$$$$\ $$
            \end{column}
            \begin{column}{1cm}
                $$\cancel{x_2}$$$$c$$
            \end{column}
            \pause
            \begin{column}{1cm}
                $$\vee$$$$\ $$
            \end{column}
            \begin{column}{1cm}
                $$\neg{x_3})$$$$b$$
            \end{column}
        \end{columns}

        \pause

        $$acb$$
    \end{Large}
\end{frame}

\begin{frame}
    \frametitle{Encoding a Boolean Formula into a String}

    \begin{itemize}
        \item Obtain the encoding of each clause
              \pause
        \item Set a separator to delimit clauses ($d$)
              \pause
        \item Concatenate each clause followed by the separator $d$
    \end{itemize}

    \pause

    \begin{Large}
        $$(x_1)\wedge (x_1\vee \neg x_2 \vee x_3)\wedge (\neg x_2\vee x_3)$$

        \pause

        \begin{columns}
            \begin{column}{1cm}
                $$(x_1)$$
                $$acc\mathbf{d}$$
            \end{column}
            \pause
            \begin{column}{0.2cm}
                $$\wedge$$
                $$\ $$
            \end{column}
            \begin{column}{3cm}
                $$(x_1\vee \neg x_2 \vee x_3)$$
                $$aba\mathbf{d}$$
            \end{column}
            \pause
            \begin{column}{0.2cm}
                $$\wedge$$
                $$\ $$
            \end{column}
            \begin{column}{2cm}
                $$(\neg x_2\vee x_3)$$
                $$cba\mathbf{d}$$
            \end{column}
        \end{columns}
        \pause

        $$acc\mathbf{d}aba\mathbf{d}cba\mathbf{d}$$
    \end{Large}
\end{frame}

\begin{frame}
    \frametitle{Language of All Satisfiable Boolean Formulas}

    \begin{itemize}
        \item $L_{FULL-SAT}$: language of all Boolean formulas in CNF
              \pause
        \item $L_{S-SAT}$: language of all satisfiable Boolean formulas
              \pause
              \begin{Large}
                  $$x_1\wedge x_2 \wedge x_3$$
                  $$acc\mathbf{d}cac\mathbf{d}cca\mathbf{d}\in L_{S-SAT}$$
              \end{Large}
    \end{itemize}
\end{frame}

\section{Range Concatenation Grammars (RCG)}

\begin{frame}
    \frametitle{Presentation Structure}

    Define and construct the language of all satisfiable Boolean formulas

    \setbeamercolor{block title}{fg=white, bg=gray!50} % Block title color (50% gray)
    \setbeamercolor{block body}{bg=gray!20}

    \begin{block}{Define $L_{S-SAT}$}
        \begin{itemize}
            \item Encode a Boolean formula
            \item Define $L_{S-SAT}$
        \end{itemize}
    \end{block}

    \setbeamercolor{block title}{fg=white, bg=structure}
    \setbeamercolor{block body}{bg=structure!20}

    \begin{block}{Construct $L_{S-SAT}$ using a Range Concatenation Grammar}
        \begin{itemize}
            \item Introduce Range Concatenation Grammars (RCG)
            \item RCG that recognizes $L_{S-SAT}$
        \end{itemize}
    \end{block}
\end{frame}

\begin{frame}
    \frametitle{Range Concatenation Grammars (RCG)}

    \begin{itemize}
        \item Developed in 1988 by Pierre Boullier
              \pause
        \item More general and expressive model than context-free grammars
              \pause
        \item Analyze properties of natural language
              \pause
        \item Chinese numbers and random order of German words
    \end{itemize}
\end{frame}

\begin{frame}
    \frametitle{Range Concatenation Grammars (RCG)}

    \begin{itemize}
        \item A range is an interval of the string
              $$a\textcolor{red}{bc}d$$
              \pause
        \item Non-terminals are called predicates
              \pause
        \item Each predicate has a sequence of arguments
              \pause
        \item Each argument consists of variables and terminals
              $$A(aX,cbYZ)$$
    \end{itemize}
\end{frame}

\begin{frame}
    \frametitle{Range Substitution}

    \begin{itemize}
        \item<1-> Each predicate receives a vector of strings
        \item<2-> Each argument is assigned a string
        \item<3-> The predicate $A(aX,cbYZ)$ receives the vector $[aaa,cbff]$
            \only<3>{\begin{Large}
                    $$A(aX,cbYZ)$$
                    $$A(aaa,cbff)$$
                \end{Large}}
            \only<4>{
                \begin{Large}
                    $$A(\textcolor{red}{aX},cbYZ)$$
                    $$A(\textcolor{red}{aaa},cbff)$$
                \end{Large}}
            \only<5>{
                \begin{Large}
                    $$A(aX,\textcolor{red}{cbYZ})$$
                    $$A(aaa,\textcolor{red}{cbff})$$
                \end{Large}}
            \only<6>{
                \begin{Large}
                    $$A(\textcolor{green}{a}X,cbYZ)$$
                    $$A(\textcolor{green}{a}aa,cbff)$$
                \end{Large}}
            \only<7>{
                \begin{Large}
                    $$A(aX,\textcolor{green}{c}bYZ)$$
                    $$A(aaa,\textcolor{green}{c}bff)$$
                \end{Large}}
            \only<8>{
                \begin{Large}
                    $$A(aX,c\textcolor{green}{b}YZ)$$
                    $$A(aaa,c\textcolor{green}{b}ff)$$
                \end{Large}}
            \only<9>{
                \begin{Large}
                    $$A(a\textcolor{red}{X},cbYZ)$$
                    $$A(a\textcolor{red}{aa},cbff)$$
                    \begin{center} $X=aa$\end{center}
                \end{Large}}
            \only<10>{
                \begin{Large}
                    $$A(aX,cb\textcolor{red}{Y}Z)$$
                    $$A(aaa,cb\textcolor{red}{ff})$$
                    \begin{center} $X=aa$, $Y=ff$\end{center}
                \end{Large}}
            \only<11>{
                \begin{Large}
                    $$A(aX,cbY\textcolor{red}{Z})$$
                    $$A(aaa,cbff)$$
                    \begin{center} $X=aa$, $Y=ff$, $Z=\varepsilon$\end{center}
                \end{Large}}
            \only<12>{
                \begin{Large}
                    $$A(aX,cb\textcolor{red}{Y}Z)$$
                    $$A(aaa,cb\textcolor{red}{f}f)$$
                    \begin{center} $X=aa$, $Y=f$\end{center}
                \end{Large}}
            \only<13>{
                \begin{Large}
                    $$A(aX,cbY\textcolor{red}{Z})$$
                    $$A(aaa,cbf\textcolor{red}{f})$$
                    \begin{center} $X=aa$, $Y=f$, $Z=f$\end{center}
                \end{Large}}
            \only<14>{\begin{Large}
                    $$A(aX,cb\textcolor{red}{Y}Z)$$
                    $$A(aaa,cbff)$$
                    \begin{center} $X=aa$, $Y=\varepsilon$\end{center}
                \end{Large}}
            \only<15>{\begin{Large}
                    $$A(aX,cbY\textcolor{red}{Z})$$
                    $$A(aaa,cb\textcolor{red}{ff})$$
                    \begin{center} $X=aa$, $Y=\varepsilon$, $Z=ff$\end{center}
                \end{Large}}
            \only<16>{\begin{Large}
                    $$A(aX,cbYZ)$$
                    $$A(aaa,cbff)$$
                \end{Large}}
        \item<16-> Range substitution
    \end{itemize}
\end{frame}

\begin{frame}
    \frametitle{RCG Productions}

    \begin{itemize}
        \item Productions are called clauses
              \[
                  A(x_1, \ldots, x_k) \to B_1(y_{1,1}, \ldots, y_{1,m_1}) \ldots B_n(y_{n,1}, \ldots, y_{n,m_n})
              \]
              \pause
        \item $A(aX,cbYZ)\to B(X,Y,Z)$
              \pause
        \item $A$ receives the vector $[aaa,cbff]$
              \pause
        \item $X=aa$, $Y=f$, and $Z=f$
              \pause
        \item The vector $[aa,f,f]$ is constructed
              \pause
        \item Evaluated in $B$
    \end{itemize}
\end{frame}

\begin{frame}
    \frametitle{RCG Productions}

    \begin{itemize}
        \item Another type of production:
              $$A(x_1, \ldots, x_k) \to \varepsilon$$
              \pause
        \item $B(aa,f,f)\to \varepsilon$
              \pause
        \item A predicate recognizes a vector
              \pause
        \item If there exists a sequence of derivations and range substitutions
              \pause
        \item That derives the empty string
              \pause
        \item $B$ recognizes the vector $[aa,f,f]$
    \end{itemize}
\end{frame}

\begin{frame}
    \frametitle{RCG Properties}

    \begin{itemize}
        \item RCGs do not generate; instead, they recognize strings
              \pause
        \item For most RCGs, the word problem is polynomial
              \pause
        \item Ambiguous RCGs whose word problem is not polynomial
              \pause
        \item RCGs recognize all problems in class P
    \end{itemize}
\end{frame}

\section{Constructing $L_{S-SAT}$ Using an RCG}

\begin{frame}
    \frametitle{Constructing $L_{SAT}$ Using an RCG}

    The productions of the grammar $G_{S-SAT}$ are grouped into 4 phases:

    \begin{enumerate}
        \item Initial derivation of the grammar
              \pause
        \item All possible interpretations (first clause is true)
              \pause
        \item The generated interpretation satisfies the remaining clauses
              \pause
        \item The generated interpretation satisfies a clause
    \end{enumerate}
\end{frame}

\begin{frame}
    \frametitle{Productions of $G_{S-SAT}$}

    \begin{block}{First phase}
        $S(X)\to A(X)$
        \pause
        $$S(abcd)\to A(abcd)$$
    \end{block}

    \pause
    \begin{block}{Second phase}
        Predicates $A$, $P$ (positive state), and $N$ (negative state)
        \begin{columns}
            \begin{column}{0.45\textwidth}
                \begin{itemize}
                    \item $A(aX)\to P(X,1)$
                          \pause
                    \item $A(aX)\to N(X,0)$
                          \pause
                    \item $N(bX,Y)\to P(X,Y0)$
                \end{itemize}
            \end{column}
            \pause
            \begin{column}{0.45\textwidth}
                \begin{itemize}
                    \item $N(bX,Y)\to N(X,Y1)$
                          \pause
                    \item $P(\_X,Y)\to P(X,Y\_)$
                          \pause
                    \item $P(dX,Y)\to B(X,Y)$
                \end{itemize}
            \end{column}
        \end{columns}
    \end{block}
\end{frame}

\begin{frame}
    \frametitle{Productions of $G_{S-SAT}$}

    \begin{block}{Third phase}
        \begin{itemize}
            \item $B(X_1dX_2,Y)\to C(X_1,Y) B(X_2,Y)$
                  \pause
            \item $B(\varepsilon,Y)\to\varepsilon$
                  \pause
                  $$B(abcdaacd,100)\to C(abc,100) B(aacd,100)$$
        \end{itemize}
    \end{block}
    \pause
    \begin{block}{Fourth phase}
        Predicates $C$, $Cp$ (positive state), and $Cn$ (negative state)\\
        \begin{columns}
            \begin{column}{0.45\textwidth}
                \begin{itemize}
                    \item $C(X,Y)\to Cn(X,Y)$
                          \pause
                    \item $Cn(aX,1Y)\to Cp(X,Y)$
                          \pause
                    \item $Cn(bX,1Y)\to Cn(X,Y)$
                \end{itemize}
            \end{column}
            \pause
            \begin{column}{0.45\textwidth}
                \begin{itemize}
                    \item $Cn(cX,\_Y)\to Cn(X,Y)$
                          \pause
                    \item $Cp(\_X,\_Y)\to Cp(X,Y)$
                          \pause
                    \item $Cp(\varepsilon,\varepsilon)\to \varepsilon$
                \end{itemize}
            \end{column}
        \end{columns}
    \end{block}
\end{frame}

\begin{frame}
    \frametitle{Grammar $G_{S-SAT}$ Definition}

    \[
        G_{S-SAT} = (N, T, V, P, S),
    \]
    where:

    \begin{itemize}
        \item $N=\{S,A,B,C,P,N,Cp,Cn\}$
        \item $T=\{a,b,c,d\}$.
        \item $V=\{X,Y,X_1,X_2\}$.
        \item The \textbf{initial symbol} is $S$.
    \end{itemize}
\end{frame}

\begin{frame}
    \frametitle{Grammar $G_{S-SAT}$ Productions}

    \begin{columns}
        \begin{column}{0.48\textwidth}
            \begin{itemize}
                \item $S(X)\to A(X)$.

                \item $A(aX)\to P(X,1)$
                \item $A(aX)\to N(X,0)$
                \item $A(bX)\to N(X,1)$
                \item $A(bX)\to P(X,0)$
                \item $A(cX)\to N(X,1)$
                \item $A(cX)\to N(X,0)$

                \item $P(aX,Y)\to P(X,Y1)$
                \item $P(aX,Y)\to P(X,Y0)$
                \item $P(bX,Y)\to P(X,Y1)$
            \end{itemize}
        \end{column}
        \begin{column}{0.48\textwidth}
            \begin{itemize}
                \item $P(bX,Y)\to P(X,Y0)$

                \item $P(cX,Y)\to P(X,Y1)$
                \item $P(cX,Y)\to P(X,Y0)$
                \item $P(dX,Y)\to B(X,Y)$

                \item $N(aX,Y)\to P(X,Y1)$
                \item $N(aX,Y)\to N(X,Y0)$
                \item $N(bX,Y)\to N(X,Y1)$
                \item $N(bX,Y)\to P(X,Y0)$
                \item $N(cX,Y)\to N(X,Y1)$
                \item $N(cX,Y)\to N(X,Y0)$
            \end{itemize}
        \end{column}
    \end{columns}
\end{frame}

\begin{frame}
    \frametitle{Grammar $G_{S-SAT}$ Productions}

    \begin{columns}
        \begin{column}{0.5\textwidth}
            \begin{itemize}
                \item $B(X_1dX_2,Y)\to C(X_1,Y) B(X_2,Y)$
                \item $B(\varepsilon,Y)\to\varepsilon$

                \item $C(X,Y)\to Cn(X,Y)$

                \item $Cn(aX,1Y) \to Cp(X,Y)$
                \item $Cn(aX,0Y) \to Cn(X,Y)$
                \item $Cn(bX,1Y) \to Cn(X,Y)$
                \item $Cn(bX,0Y) \to Cp(X,Y)$
                \item $Cn(cX,1Y) \to Cn(X,Y)$

            \end{itemize}
        \end{column}
        \begin{column}{0.5\textwidth}
            \begin{itemize}
                \item $Cn(cX,0Y) \to Cn(X,Y)$
                \item $Cp(aX,1Y) \to Cp(X,Y)$
                \item $Cp(aX,0Y) \to Cp(X,Y)$
                \item $Cp(bX,1Y) \to Cp(X,Y)$
                \item $Cp(bX,0Y) \to Cp(X,Y)$
                \item $Cp(cX,1Y) \to Cp(X,Y)$
                \item $Cp(cX,0Y) \to Cp(X,Y)$
                \item $Cp(\varepsilon,\varepsilon)\to \varepsilon$
            \end{itemize}
        \end{column}
    \end{columns}
\end{frame}

\begin{frame}
    \frametitle{Results Derived from $G_{S-SAT}$}

    \begin{block}{Language of All Satisfiable Boolean Formulas}
        $$L_{S-SAT} = L_{G_{S-SAT}}$$
    \end{block}
    \pause
    \begin{block}{Summary}
        \begin{itemize}
            \item 8 predicates, 4 terminals, 4 variables, and 36 productions
                  \pause
            \item Word problem for $G_{S-SAT}$ is not polynomial (memoization algorithm)
        \end{itemize}
    \end{block}
    \pause
    \begin{block}{Results}
        \begin{itemize}
            \item RCGs recognize all problems in class NP
                  \pause
            \item The word problem for RCGs is NP-Hard
        \end{itemize}
    \end{block}
\end{frame}

\begin{frame}
    \frametitle{Presentation Structure}

    Define and construct the language of all satisfiable Boolean formulas

    \setbeamercolor{block title}{fg=white, bg=gray!50} % Block title color (50% gray)
    \setbeamercolor{block body}{bg=gray!20}

    \begin{block}{Define $L_{S-SAT}$}
        \begin{itemize}
            \item Encode a Boolean formula
            \item Define $L_{S-SAT}$
        \end{itemize}
    \end{block}

    \begin{block}{Construct $L_{S-SAT}$ using a Range Concatenation Grammar}
        \begin{itemize}
            \item Introduce Range Concatenation Grammars (RCG)
            \item RCG that recognizes $L_{S-SAT}$
        \end{itemize}
    \end{block}
\end{frame}

\begin{frame}
    \frametitle{Conclusions}

    \begin{itemize}
        \item It is possible to describe $L_{S-SAT}$ using a regulated rewriting formalism
        \item RCGs recognize all problems in class NP
              \pause
        \item The word problem for RCGs is NP-Hard
    \end{itemize}
\end{frame}

\section{Recommendations}
\begin{frame}
    \frametitle{Recommendations}

    \begin{itemize}
        \item Investigate other algorithms for the RCG word problem?
              \pause
        \item RCG that recognizes polynomial-time solvable SAT (2-SAT)
              \pause
        \item Explore other formalisms capable of describing $L_{S-SAT}$
    \end{itemize}
\end{frame}


\begin{frame}
    \titlepage
    \vspace{1cm} % Espacio adicional
    \begin{center}
        Tutor: \tutor
    \end{center}
\end{frame}

\end{document}