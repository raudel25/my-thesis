\documentclass{beamer}

% Paquetes adicionales (opcional)
\usepackage[utf8]{inputenc} % Para caracteres especiales
\usepackage[spanish]{babel} % Idioma español
\usepackage{graphicx} % Para incluir imágenes

\usetheme{Madrid} % Puedes probar otros temas como Warsaw, AnnArbor, etc.

\usepackage{graphicx} % Permite incluir imágenes
\usepackage{hyperref} % Enlaces clicables
\usepackage{tikz} % Para diagramas
\usepackage{ragged2e}
\usepackage{amsmath}
\usepackage{xcolor}
\usetikzlibrary{shapes.geometric, arrows}

\tikzstyle{startstop} = [rectangle, rounded corners, minimum width=3cm, minimum height=1cm,text centered, draw=black, fill=red!30]
\tikzstyle{process} = [rectangle, minimum width=3cm, minimum height=1cm, text centered, draw=black, fill=blue!20]
\tikzstyle{decision} = [diamond, minimum width=3cm, minimum height=1cm, text centered, draw=black, fill=green!20]
\tikzstyle{arrow} = [thick,->,>=stealth]

\usepackage{cancel}

\usepackage{tikz}
\usetikzlibrary{automata, positioning}

% \AtBeginSection[]{
%     \begin{frame}{Contenido}
%         \tableofcontents[currentsection]
%     \end{frame}
% }


% Información del título
\title{Una aproximación al lenguaje de todas las fórmulas booleanas satisfacibles, usando gramáticas de concatenación de rango}
\author{Raudel Alejandro Gómez Molina}
\institute{Facultad de Matemática y Computación \\ Universidad de La Habana}
\date{\today}

\newcommand{\tutor}{MSc. Fernando Raul Rodriguez Flores} %
% Comando personalizado para la información de la portada


\begin{document}

% Portada
\begin{frame}
    \titlepage
    \vspace{1cm} % Espacio adicional
    \begin{center}
        Tutor: \tutor \\ % Muestra el nombre del tutor
        \smallskip
    \end{center}
\end{frame}

\begin{frame}
    \frametitle{Teoría de lenguajes (Conceptos)}

    \begin{block}{Alfabeto}
        Un alfabeto es un conjunto finito de símbolos
        $$\Sigma=\{0,1\}$$
    \end{block}

    \pause

    \begin{block}{Cadena}
        Una cadena es una sucesión finita de símbolos del alfabeto
        $$w=0001111$$
    \end{block}

    \pause

    \begin{block}{Lenguaje}
        Un lenguaje es un conjunto finito de cadenas
        $$L=\{0^n1^m\mid n,m\in \mathbb{N}\}$$
    \end{block}
\end{frame}

\begin{frame}
    \frametitle{Teoría de lenguajes (Problemas)}

    \begin{block}{Problema de la palabra}
        \begin{itemize}
            \item Determinar si una cadena pertenece a un lenguaje ¿$w$ pertenece a $L$?
                  \pause
            \item $10001\in \{1w\mid w\in\{0,1\}^*\}$
                  \pause
            \item $00001\notin \{1w\mid w\in\{0,1\}^*\}$
        \end{itemize}

    \end{block}

    \pause

    \begin{block}{Problema en Ciencia de la Computación}
        \begin{itemize}
            \item Todo problema se puede reducir a un problema de la palabra
                  \pause
            \item Todo problema se puede codificar como lenguaje formal
                  \pause
            \item Determinar si dos números son primos relativos
                  \pause
            \item Determinar si un arreglo está ordenado
        \end{itemize}
    \end{block}

\end{frame}

\begin{frame}
    \frametitle{Clases de problemas}

    \begin{itemize}
        \item Existen problemas para los cuales no se conoce una solución eficiente
              \pause
        \item Comprobar si una solución es válida es eficiente (clase NP)
              \pause
        \item El SAT es el primer problema demostrado como NP-Completo
              \pause
        \item Pertenece a la clase NP
              \pause
        \item Se puede reducir a cualquier problema en NP en tiempo polinomial
    \end{itemize}

\end{frame}

\begin{frame}
    \frametitle{Problema de la Satisfacibilidad booleana (SAT)}

    \begin{itemize}
        \item Consiste en determinar si una fórmula booleana es satisfacible
              \begin{Large}
                  $$x_1\vee x_2 \wedge \neg x_1 \wedge x_3$$
              \end{Large}
              \pause
        \item Existen instancias polinomiales 2-SAT, Horn-SAT y XOR-SAT
    \end{itemize}
\end{frame}

\begin{frame}
    \frametitle{Estructura de la presentación}

    Definir y construir el lenguaje de todas las fórmulas booleanas satisfacibles

    \pause

    \begin{block}{Definir $L_{S-SAT}$}
        \begin{itemize}
            \item Codificar una fórmula booleana
            \item Definir $L_{S-SAT}$
        \end{itemize}
    \end{block}

    \pause

    \begin{block}{Construir $L_{S-SAT}$ mediante una gramática de concatenación de rango}
        \begin{itemize}
            \item Introducir las gramáticas de concatenación de rango
            \item Gramática de concatenación de rango que reconoce $L_{S-SAT}$
        \end{itemize}
    \end{block}
\end{frame}

% \begin{frame}
%     \frametitle{Estructura de la presentación}

%     \tableofcontents
% \end{frame}

\begin{frame}
    \frametitle{Estructura de la presentación}

    Definir y construir el lenguaje de todas las fórmulas booleanas satisfacibles

    \begin{block}{Definir $L_{S-SAT}$}
        \begin{itemize}
            \item Codificar una fórmula booleana
            \item Definir $L_{S-SAT}$
        \end{itemize}
    \end{block}

    \setbeamercolor{block title}{fg=white, bg=gray!50} % Color del título del bloque (gris al 50%)
    \setbeamercolor{block body}{bg=gray!20}

    \begin{block}{Construir $L_{S-SAT}$ mediante una gramática de concatenación de rango}
        \begin{itemize}
            \item Introducir las gramáticas de concatenación de rango
            \item Gramática de concatenación de rango que reconoce $L_{S-SAT}$
        \end{itemize}
    \end{block}
\end{frame}

\section{Codificación de una fórmula booleana en una cadena}

\begin{frame}
    \frametitle{Fórmula normal conjuntiva (CNF)}

    \begin{itemize}
        \item Toda fórmula booleana tiene una fórmula equivalente en CNF
              \pause
        \item Asumimos que las fórmulas booleanas están en CNF
    \end{itemize}

\end{frame}

\begin{frame}

    \frametitle{Estados de una variable en una cláusula}

    \begin{Large}
        $$(x_1\vee  \neg x_3)$$
    \end{Large}

    \pause
    \vspace{1cm}
    \begin{itemize}
        \item<1-> $x_1$ está sin negar en la cláusula
        \item<3-> $x_2$ no está en la cláusula
        \item<2-> $x_3$ está negada en la cláusula
    \end{itemize}
\end{frame}

\begin{frame}
    \frametitle{Codificación de una cláusula en una cadena}

    \begin{itemize}
        \item $a$: la variable está sin negar en la cláusula
              \pause
        \item $b$: la variable está negada en la cláusula
              \pause
        \item $c$: la variable no está en la cláusula
    \end{itemize}

    \pause


    \begin{Large}
        $$x_1\vee \neg x_3$$

        \pause

        \begin{columns}
            \begin{column}{1cm}
                $$({x_1}$$$$a$$
            \end{column}
            \pause
            \begin{column}{1cm}
                $$\cancel{\vee}$$$$\ $$
            \end{column}
            \begin{column}{1cm}
                $$\cancel{x_2}$$$$c$$
            \end{column}
            \pause
            \begin{column}{1cm}
                $$\vee$$$$\ $$
            \end{column}
            \begin{column}{1cm}
                $$\neg{x_3})$$$$b$$
            \end{column}
        \end{columns}

        \pause

        $$acb$$
    \end{Large}


\end{frame}

\begin{frame}
    \frametitle{Codificación de una fórmula booleana en una cadena}

    \begin{itemize}
        \item Obtener la codificación de cada cláusula
              \pause
        \item Establecer un separador para delimitar cada cláusula $d$
              \pause
        \item Concatenar cada cláusula seguida del separador $d$
    \end{itemize}

    \pause

    \begin{Large}
        $$(x_1)\wedge (x_1\vee \neg x_2 \vee x_3)\wedge (\neg x_2\vee x_3)$$

        \pause

        \begin{columns}
            \begin{column}{1cm}
                $$(x_1)$$
                $$acc\mathbf{d}$$
            \end{column}
            \pause
            \begin{column}{0.2cm}
                $$\wedge$$
                $$\ $$
            \end{column}
            \begin{column}{3cm}
                $$(x_1\vee \neg x_2 \vee x_3)$$
                $$aba\mathbf{d}$$
            \end{column}
            \pause
            \begin{column}{0.2cm}
                $$\wedge$$
                $$\ $$
            \end{column}
            \begin{column}{2cm}
                $$(\neg x_2\vee x_3)$$
                $$cba\mathbf{d}$$
            \end{column}
        \end{columns}
        \pause

        $$acc\mathbf{d}aba\mathbf{d}cba\mathbf{d}$$
    \end{Large}


\end{frame}

\begin{frame}
    \frametitle{Lenguaje de todas las fórmulas booleanas satisfacibles}

    \begin{itemize}
        \item $L_{FULL-SAT}$ lenguaje de todas las fórmulas booleanas en CNF
              \pause
        \item  $L_{S-SAT}$ lenguaje de todas las fórmulas booleanas satisfacibles
              \pause
              \begin{Large}

                  $$x_1\wedge x_2 \wedge x_3$$
                  $$acc\mathbf{d}cac\mathbf{d}cca\mathbf{d}\in L_{S-SAT}$$

              \end{Large}

    \end{itemize}
\end{frame}


\section{Gramáticas de concatenación de rango}

\begin{frame}
    \frametitle{Estructura de la presentación}

    Definir y construir el lenguaje de todas las fórmulas booleanas satisfacibles

    \setbeamercolor{block title}{fg=white, bg=gray!50} % Color del título del bloque (gris al 50%)
    \setbeamercolor{block body}{bg=gray!20}

    \begin{block}{Definir $L_{S-SAT}$}
        \begin{itemize}
            \item Codificar una fórmula booleana
            \item Definir $L_{S-SAT}$
        \end{itemize}
    \end{block}

    \setbeamercolor{block title}{fg=white, bg=structure}
    \setbeamercolor{block body}{bg=structure!20}

    \begin{block}{Construir $L_{S-SAT}$ mediante una gramática de concatenación de rango}
        \begin{itemize}
            \item Introducir las gramáticas de concatenación de rango
            \item Gramática de concatenación de rango que reconoce $L_{S-SAT}$
        \end{itemize}
    \end{block}
\end{frame}

\begin{frame}
    \frametitle{Gramáticas de concatenación de rango (RCG)}

    \begin{itemize}
        \item Desarrolladas  en 1988 como una propuesta de Pierre Boullier
              \pause
        \item Modelo más general y expresivo que las gramáticas libres del contexto
              \pause
        \item Analizar propiedades del lenguaje natural
              \pause
        \item Números chinos y el orden aleatorio de algunas palabras alemanas
    \end{itemize}
\end{frame}

\begin{frame}
    \frametitle{Gramáticas de concatenación de rango (RCG)}

    \begin{itemize}
        \item Un rango es un intervalo de la cadena
              $$a\textcolor{red}{bc}d$$
              \pause
        \item A los no terminales se les llama predicados
              \pause
        \item Cada predicado tiene una secuencia de argumentos
              \pause
        \item Cada argumento está formado por variables y terminales
              $$A(aX,cbYZ)$$
    \end{itemize}
\end{frame}

\begin{frame}
    \frametitle{Sustitución de rango}

    \begin{itemize}
        \item<1-> Cada predicado recibe un vector de cadenas
        \item<2-> A cada argumento se le asocia una cadena
        \item<3-> El predicado $A(aX,cbYZ)$ recibe el vector $[aaa,cbff]$
            \only<3>{\begin{Large}
                    $$A(aX,cbYZ)$$
                    $$A(aaa,cbff)$$
                \end{Large}}
            \only<4>{
                \begin{Large}
                    $$A(\textcolor{red}{aX},cbYZ)$$
                    $$A(\textcolor{red}{aaa},cbff)$$
                \end{Large}}
            \only<5>{
                \begin{Large}
                    $$A(aX,\textcolor{red}{cbYZ})$$
                    $$A(aaa,\textcolor{red}{cbff})$$
                \end{Large}}
            \only<6>{
                \begin{Large}
                    $$A(\textcolor{green}{a}X,cbYZ)$$
                    $$A(\textcolor{green}{a}aa,cbff)$$
                \end{Large}}
            \only<7>{
                \begin{Large}
                    $$A(aX,\textcolor{green}{c}bYZ)$$
                    $$A(aaa,\textcolor{green}{c}bff)$$
                \end{Large}}
            \only<8>{
                \begin{Large}
                    $$A(aX,c\textcolor{green}{b}YZ)$$
                    $$A(aaa,c\textcolor{green}{b}ff)$$
                \end{Large}}
            \only<9>{
                \begin{Large}
                    $$A(a\textcolor{red}{X},cbYZ)$$
                    $$A(a\textcolor{red}{aa},cbff)$$
                    \begin{center} $X=aa$\end{center}
                \end{Large}}
            \only<10>{
                \begin{Large}
                    $$A(aX,cb\textcolor{red}{Y}Z)$$
                    $$A(aaa,cb\textcolor{red}{ff})$$
                    \begin{center} $X=aa$, $Y=ff$\end{center}
                \end{Large}}
            \only<11>{
                \begin{Large}
                    $$A(aX,cbY\textcolor{red}{Z})$$
                    $$A(aaa,cbff)$$
                    \begin{center} $X=aa$, $Y=ff$, $Z=\varepsilon$\end{center}
                \end{Large}}
            \only<12>{
                \begin{Large}
                    $$A(aX,cb\textcolor{red}{Y}Z)$$
                    $$A(aaa,cb\textcolor{red}{f}f)$$
                    \begin{center} $X=aa$, $Y=f$\end{center}
                \end{Large}}
            \only<13>{
                \begin{Large}
                    $$A(aX,cbY\textcolor{red}{Z})$$
                    $$A(aaa,cbf\textcolor{red}{f})$$
                    \begin{center} $X=aa$, $Y=f$, $Z=f$\end{center}
                \end{Large}}
            \only<14>{\begin{Large}
                    $$A(aX,cb\textcolor{red}{Y}Z)$$
                    $$A(aaa,cbff)$$
                    \begin{center} $X=aa$, $Y=\varepsilon$\end{center}
                \end{Large}}
            \only<15>{\begin{Large}
                    $$A(aX,cbY\textcolor{red}{Z})$$
                    $$A(aaa,cb\textcolor{red}{ff})$$
                    \begin{center} $X=aa$, $Y=\varepsilon$, $Z=ff$\end{center}
                \end{Large}}
            \only<16>{\begin{Large}
                    $$A(aX,cbYZ)$$
                    $$A(aaa,cbff)$$
                \end{Large}}
        \item<16-> Sustitución de rango
    \end{itemize}
\end{frame}

\begin{frame}
    \frametitle{Producciones de las RCG}

    \begin{itemize}
        \item A las producciones se les llama cláusulas
              \[
                  A(x_1, \ldots, x_k) \to B_1(y_{1,1}, \ldots, y_{1,m_1}) \ldots B_n(y_{n,1}, \ldots, y_{n,m_n})
              \]
              \pause
        \item $A(aX,cbYZ)\to B(X,Y,Z)$
              \pause
        \item $A$ recibe el vector $[aaa,cbff]$
              \pause
        \item $X=aa$, $Y=f$ y $Z=f$
              \pause
        \item Se construye el vector $[aa,f,f]$
              \pause
        \item Se evalúa en $B$
    \end{itemize}
\end{frame}

\begin{frame}
    \frametitle{Producciones de las RCG}

    \begin{itemize}
        \item Existen otro tipo de producciones de la forma
              $$A(x_1, \ldots, x_k) \to \varepsilon$$
              \pause
        \item $B(aa,f,f)\to \varepsilon$
              \pause
        \item Un predicado reconoce un vector
              \pause
        \item Si existe una secuencia de derivaciones y sustituciones de rango
              \pause
        \item Tales que se deriva en la cadena vacía
              \pause
        \item $B$ reconoce el vector $[aa,f,f]$
    \end{itemize}
\end{frame}


\begin{frame}
    \frametitle{Propiedades de las RCG}

    \begin{itemize}
        \item Las RCG no generan, por el contrario, reconocen cadenas
              \pause
        \item Para la mayoría de las RCG el problema de la palabra es polinomial
              \pause
        \item RCG ambiguas cuyo problema de la palabra no es polinomial
              \pause
        \item Las RCG reconocen todos los problemas de la clase P

    \end{itemize}
\end{frame}

\section{Construcción $L_{S-SAT}$ utilizando una RCG}
% \begin{frame}
%     \frametitle{$L_{0,1,d}$ como lenguaje de concatenación de rango}

%     Producciones de la gramática $G_{0,1,d}$
%     \begin{itemize}
%         \item $S(X)\to A(X)$
%               \pause
%         \item $A(XdY)\to B(Y,X)C(X)$
%               \pause
%         \item $B(XdY,P)\to B(Y,P) C(X) Eq(X,P)$
%               \pause
%         \item $B(\varepsilon,P)\to \varepsilon$
%     \end{itemize}
%     \vspace{0.5cm}
%     \begin{itemize}
%         \item $C$ comprueba que la cadena está formada por 0 y 1
%               \pause
%         \item $Eq$ comprueba que 2 cadenas sean iguales
%     \end{itemize}
% \end{frame}

% \begin{frame}
%     \frametitle{$L_{0,1,d}$ como lenguaje de concatenación de rango}

%     \begin{itemize}
%         \item Cuando se le aplica $T_{SAT}$ a $G_{0,1,d}$
%               \pause
%         \item El formalismo resultante no es necesariamente una RCG
%               \pause
%         \item Porque las RCG no son cerradas bajo transducción finita
%     \end{itemize}
% \end{frame}

\begin{frame}
    \frametitle{Construcción de $L_{SAT}$ mediante una RCG}

    Las producciones de la gramática $G_{S-SAT}$ se agrupan en 4 grupos (fases):

    \begin{enumerate}
        \item Derivación inicial de la gramática
              \pause
        \item Todas las posibles interpretaciones (verdadera la primera cláusula)
              \pause
        \item La interpretación generada satisface el resto de las cláusulas
              \pause
        \item La interpretación generada satisface una cláusula
    \end{enumerate}

\end{frame}

\begin{frame}
    \frametitle{Producciones de $G_{S-SAT}$}

    \begin{block}{Primera fase}
        $S(X)\to A(X)$
        \pause
        $$S(abcd)\to A(abcd)$$
    \end{block}

    \pause
    \begin{block}{Segunda fase}
        Predicados $A$, $P$ (estado positivo) y $N$ (estado negativo)
        \begin{columns}
            \begin{column}{0.45\textwidth}
                \begin{itemize}
                    \item $A(aX)\to P(X,1)$
                          \pause
                    \item $A(aX)\to N(X,0)$
                          \pause
                    \item $N(bX,Y)\to P(X,Y0)$
                \end{itemize}
            \end{column}
            \pause
            \begin{column}{0.45\textwidth}
                \begin{itemize}
                    \item $N(bX,Y)\to N(X,Y1)$
                          \pause
                    \item $P(\_X,Y)\to P(X,Y\_)$
                          \pause
                    \item $P(dX,Y)\to B(X,Y)$
                \end{itemize}
            \end{column}
        \end{columns}

    \end{block}
\end{frame}

\begin{frame}
    \frametitle{Producciones de $G_{S-SAT}$}

    \begin{block}{Tercera fase}
        \begin{itemize}
            \item $B(X_1dX_2,Y)\to C(X_1,Y) B(X_2,Y)$
                  \pause
            \item $B(\varepsilon,Y)\to\varepsilon$
                  \pause
                  $$B(abcdaacd,100)\to C(abc,100) B(aacd,100)$$
        \end{itemize}

    \end{block}
    \pause
    \begin{block}{Cuarta fase}
        Predicados $C$, $Cp$ (estado positivo) y $Cn$ (estado negativo)\\
        \begin{columns}
            \begin{column}{0.45\textwidth}
                \begin{itemize}
                    \item $C(X,Y)\to Cn(X,Y)$
                          \pause
                    \item $Cn(aX,1Y)\to Cp(X,Y)$
                          \pause
                    \item $Cn(bX,1Y)\to Cn(X,Y)$
                \end{itemize}
            \end{column}
            \pause
            \begin{column}{0.45\textwidth}
                \begin{itemize}
                    \item $Cn(cX,\_Y)\to Cn(X,Y)$
                          \pause
                    \item $Cp(\_X,\_Y)\to Cp(X,Y)$
                          \pause
                    \item $Cp(\varepsilon,\varepsilon)\to \varepsilon$
                \end{itemize}
            \end{column}
        \end{columns}

    \end{block}
\end{frame}


\begin{frame}
    \frametitle{Gramática $G_{S-SAT}$ definición}

    \[
        G_{S-SAT} = (N, T, V, P, S),
    \]
    donde:

    \begin{itemize}
        \item $N=\{S,A,B,C,P,N,Cp,Cn\}$
        \item $T=\{a,b,c,d\}$.
        \item $V=\{X,Y,X_1,X_2\}$.
        \item El \textbf{símbolo inicial} es $S$.
    \end{itemize}
\end{frame}

\begin{frame}
    \frametitle{Gramática $G_{S-SAT}$ producciones}

    \begin{columns}
        \begin{column}{0.48\textwidth}
            \begin{itemize}
                \item $S(X)\to A(X)$.

                \item $A(aX)\to P(X,1)$
                \item $A(aX)\to N(X,0)$
                \item $A(bX)\to N(X,1)$
                \item $A(bX)\to P(X,0)$
                \item $A(cX)\to N(X,1)$
                \item $A(cX)\to N(X,0)$

                \item $P(aX,Y)\to P(X,Y1)$
                \item $P(aX,Y)\to P(X,Y0)$
                \item $P(bX,Y)\to P(X,Y1)$
            \end{itemize}
        \end{column}
        \begin{column}{0.48\textwidth}
            \begin{itemize}
                \item $P(bX,Y)\to P(X,Y0)$

                \item $P(cX,Y)\to P(X,Y1)$
                \item $P(cX,Y)\to P(X,Y0)$
                \item $P(dX,Y)\to B(X,Y)$

                \item $N(aX,Y)\to P(X,Y1)$
                \item $N(aX,Y)\to N(X,Y0)$
                \item $N(bX,Y)\to N(X,Y1)$
                \item $N(bX,Y)\to P(X,Y0)$
                \item $N(cX,Y)\to N(X,Y1)$
                \item $N(cX,Y)\to N(X,Y0)$
            \end{itemize}
        \end{column}

    \end{columns}
\end{frame}

\begin{frame}
    \frametitle{Gramática $G_{S-SAT}$ producciones}

    \begin{columns}
        \begin{column}{0.5\textwidth}
            \begin{itemize}
                \item $B(X_1dX_2,Y)\to C(X_1,Y) B(X_2,Y)$
                \item $B(\varepsilon,Y)\to\varepsilon$

                \item $C(X,Y)\to Cn(X,Y)$

                \item $Cn(aX,1Y) \to Cp(X,Y)$
                \item $Cn(aX,0Y) \to Cn(X,Y)$
                \item $Cn(bX,1Y) \to Cn(X,Y)$
                \item $Cn(bX,0Y) \to Cp(X,Y)$
                \item $Cn(cX,1Y) \to Cn(X,Y)$

            \end{itemize}
        \end{column}
        \begin{column}{0.5\textwidth}
            \begin{itemize}
                \item $Cn(cX,0Y) \to Cn(X,Y)$
                \item $Cp(aX,1Y) \to Cp(X,Y)$
                \item $Cp(aX,0Y) \to Cp(X,Y)$
                \item $Cp(bX,1Y) \to Cp(X,Y)$
                \item $Cp(bX,0Y) \to Cp(X,Y)$
                \item $Cp(cX,1Y) \to Cp(X,Y)$
                \item $Cp(cX,0Y) \to Cp(X,Y)$
                \item $Cp(\varepsilon,\varepsilon)\to \varepsilon$
            \end{itemize}
        \end{column}

    \end{columns}

\end{frame}

\begin{frame}
    \frametitle{Resultados derivados de $G_{S-SAT}$}

    \begin{block}{Lenguaje de todas las fórmulas booleanas satisfacibles}
        $$L_{S-SAT} = L_{G_{S-SAT}}$$
    \end{block}
    \pause
    \begin{block}{Resumen}
        \begin{itemize}
            \item 8 predicados, 4 terminales, 4 variables y 36 producciones
                  \pause
            \item Problema de la palabra $G_{S-SAT}$ no es polinomial (algoritmo memorización)
        \end{itemize}
    \end{block}
    \pause
    \begin{block}{Resultados}
        \begin{itemize}
            \item Las RCG reconocen todos los problemas de la clase NP
                  \pause
            \item EL problema de la palabra para las RCG es NP-Duro
        \end{itemize}
    \end{block}

\end{frame}

\begin{frame}
    \frametitle{Estructura de la presentación}

    Definir y construir el lenguaje de todas las fórmulas booleanas satisfacibles

    \setbeamercolor{block title}{fg=white, bg=gray!50} % Color del título del bloque (gris al 50%)
    \setbeamercolor{block body}{bg=gray!20}

    \begin{block}{Definir $L_{S-SAT}$}
        \begin{itemize}
            \item Codificar una fórmula booleana
            \item Definir $L_{S-SAT}$
        \end{itemize}
    \end{block}

    \begin{block}{Construir $L_{S-SAT}$ mediante una gramática de concatenación de rango}
        \begin{itemize}
            \item Introducir las gramáticas de concatenación de rango
            \item Gramática de concatenación de rango que reconoce $L_{S-SAT}$
        \end{itemize}
    \end{block}
\end{frame}

\begin{frame}
    \frametitle{Conclusiones}

    \begin{itemize}
        \item Es posible describir el lenguaje $L_{S-SAT}$ mediante un formalismo de escritura regulada
              \pause
        \item Las RCG reconocen todos los problemas de la clase NP
              \pause
        \item El problema de la palabra de las RCG es NP-Duro
    \end{itemize}
\end{frame}

\section{Recomendaciones}
\begin{frame}
    \frametitle{Recomendaciones}

    \begin{itemize}
        \item Investigar otro tipo de algoritmos para el problema de la palabra de las RCG?
              \pause
        \item RCG que reconozca los SAT solubles en tiempo polinomial (2-SAT)
              \pause
        \item Buscar otros formalismos capaces de describir el lenguaje $L_{S-SAT}$
    \end{itemize}


\end{frame}

\begin{frame}
    \titlepage
    \vspace{1cm} % Espacio adicional
    \begin{center}
        Tutor: \tutor
    \end{center}
\end{frame}


\end{document}